% Options for packages loaded elsewhere
\PassOptionsToPackage{unicode}{hyperref}
\PassOptionsToPackage{hyphens}{url}
\PassOptionsToPackage{dvipsnames,svgnames,x11names}{xcolor}
%
\documentclass[
  letterpaper,
]{scrbook}

\usepackage{amsmath,amssymb}
\usepackage{iftex}
\ifPDFTeX
  \usepackage[T1]{fontenc}
  \usepackage[utf8]{inputenc}
  \usepackage{textcomp} % provide euro and other symbols
\else % if luatex or xetex
  \usepackage{unicode-math}
  \defaultfontfeatures{Scale=MatchLowercase}
  \defaultfontfeatures[\rmfamily]{Ligatures=TeX,Scale=1}
\fi
\usepackage{lmodern}
\ifPDFTeX\else  
    % xetex/luatex font selection
    \setmainfont[]{Times New Roman}
\fi
% Use upquote if available, for straight quotes in verbatim environments
\IfFileExists{upquote.sty}{\usepackage{upquote}}{}
\IfFileExists{microtype.sty}{% use microtype if available
  \usepackage[]{microtype}
  \UseMicrotypeSet[protrusion]{basicmath} % disable protrusion for tt fonts
}{}
\makeatletter
\@ifundefined{KOMAClassName}{% if non-KOMA class
  \IfFileExists{parskip.sty}{%
    \usepackage{parskip}
  }{% else
    \setlength{\parindent}{0pt}
    \setlength{\parskip}{6pt plus 2pt minus 1pt}}
}{% if KOMA class
  \KOMAoptions{parskip=half}}
\makeatother
\usepackage{xcolor}
\usepackage[top=30mm,left=20mm]{geometry}
\setlength{\emergencystretch}{3em} % prevent overfull lines
\setcounter{secnumdepth}{5}
% Make \paragraph and \subparagraph free-standing
\makeatletter
\ifx\paragraph\undefined\else
  \let\oldparagraph\paragraph
  \renewcommand{\paragraph}{
    \@ifstar
      \xxxParagraphStar
      \xxxParagraphNoStar
  }
  \newcommand{\xxxParagraphStar}[1]{\oldparagraph*{#1}\mbox{}}
  \newcommand{\xxxParagraphNoStar}[1]{\oldparagraph{#1}\mbox{}}
\fi
\ifx\subparagraph\undefined\else
  \let\oldsubparagraph\subparagraph
  \renewcommand{\subparagraph}{
    \@ifstar
      \xxxSubParagraphStar
      \xxxSubParagraphNoStar
  }
  \newcommand{\xxxSubParagraphStar}[1]{\oldsubparagraph*{#1}\mbox{}}
  \newcommand{\xxxSubParagraphNoStar}[1]{\oldsubparagraph{#1}\mbox{}}
\fi
\makeatother

\usepackage{color}
\usepackage{fancyvrb}
\newcommand{\VerbBar}{|}
\newcommand{\VERB}{\Verb[commandchars=\\\{\}]}
\DefineVerbatimEnvironment{Highlighting}{Verbatim}{commandchars=\\\{\}}
% Add ',fontsize=\small' for more characters per line
\usepackage{framed}
\definecolor{shadecolor}{RGB}{241,243,245}
\newenvironment{Shaded}{\begin{snugshade}}{\end{snugshade}}
\newcommand{\AlertTok}[1]{\textcolor[rgb]{0.68,0.00,0.00}{#1}}
\newcommand{\AnnotationTok}[1]{\textcolor[rgb]{0.37,0.37,0.37}{#1}}
\newcommand{\AttributeTok}[1]{\textcolor[rgb]{0.40,0.45,0.13}{#1}}
\newcommand{\BaseNTok}[1]{\textcolor[rgb]{0.68,0.00,0.00}{#1}}
\newcommand{\BuiltInTok}[1]{\textcolor[rgb]{0.00,0.23,0.31}{#1}}
\newcommand{\CharTok}[1]{\textcolor[rgb]{0.13,0.47,0.30}{#1}}
\newcommand{\CommentTok}[1]{\textcolor[rgb]{0.37,0.37,0.37}{#1}}
\newcommand{\CommentVarTok}[1]{\textcolor[rgb]{0.37,0.37,0.37}{\textit{#1}}}
\newcommand{\ConstantTok}[1]{\textcolor[rgb]{0.56,0.35,0.01}{#1}}
\newcommand{\ControlFlowTok}[1]{\textcolor[rgb]{0.00,0.23,0.31}{\textbf{#1}}}
\newcommand{\DataTypeTok}[1]{\textcolor[rgb]{0.68,0.00,0.00}{#1}}
\newcommand{\DecValTok}[1]{\textcolor[rgb]{0.68,0.00,0.00}{#1}}
\newcommand{\DocumentationTok}[1]{\textcolor[rgb]{0.37,0.37,0.37}{\textit{#1}}}
\newcommand{\ErrorTok}[1]{\textcolor[rgb]{0.68,0.00,0.00}{#1}}
\newcommand{\ExtensionTok}[1]{\textcolor[rgb]{0.00,0.23,0.31}{#1}}
\newcommand{\FloatTok}[1]{\textcolor[rgb]{0.68,0.00,0.00}{#1}}
\newcommand{\FunctionTok}[1]{\textcolor[rgb]{0.28,0.35,0.67}{#1}}
\newcommand{\ImportTok}[1]{\textcolor[rgb]{0.00,0.46,0.62}{#1}}
\newcommand{\InformationTok}[1]{\textcolor[rgb]{0.37,0.37,0.37}{#1}}
\newcommand{\KeywordTok}[1]{\textcolor[rgb]{0.00,0.23,0.31}{\textbf{#1}}}
\newcommand{\NormalTok}[1]{\textcolor[rgb]{0.00,0.23,0.31}{#1}}
\newcommand{\OperatorTok}[1]{\textcolor[rgb]{0.37,0.37,0.37}{#1}}
\newcommand{\OtherTok}[1]{\textcolor[rgb]{0.00,0.23,0.31}{#1}}
\newcommand{\PreprocessorTok}[1]{\textcolor[rgb]{0.68,0.00,0.00}{#1}}
\newcommand{\RegionMarkerTok}[1]{\textcolor[rgb]{0.00,0.23,0.31}{#1}}
\newcommand{\SpecialCharTok}[1]{\textcolor[rgb]{0.37,0.37,0.37}{#1}}
\newcommand{\SpecialStringTok}[1]{\textcolor[rgb]{0.13,0.47,0.30}{#1}}
\newcommand{\StringTok}[1]{\textcolor[rgb]{0.13,0.47,0.30}{#1}}
\newcommand{\VariableTok}[1]{\textcolor[rgb]{0.07,0.07,0.07}{#1}}
\newcommand{\VerbatimStringTok}[1]{\textcolor[rgb]{0.13,0.47,0.30}{#1}}
\newcommand{\WarningTok}[1]{\textcolor[rgb]{0.37,0.37,0.37}{\textit{#1}}}

\providecommand{\tightlist}{%
  \setlength{\itemsep}{0pt}\setlength{\parskip}{0pt}}\usepackage{longtable,booktabs,array}
\usepackage{calc} % for calculating minipage widths
% Correct order of tables after \paragraph or \subparagraph
\usepackage{etoolbox}
\makeatletter
\patchcmd\longtable{\par}{\if@noskipsec\mbox{}\fi\par}{}{}
\makeatother
% Allow footnotes in longtable head/foot
\IfFileExists{footnotehyper.sty}{\usepackage{footnotehyper}}{\usepackage{footnote}}
\makesavenoteenv{longtable}
\usepackage{graphicx}
\makeatletter
\def\maxwidth{\ifdim\Gin@nat@width>\linewidth\linewidth\else\Gin@nat@width\fi}
\def\maxheight{\ifdim\Gin@nat@height>\textheight\textheight\else\Gin@nat@height\fi}
\makeatother
% Scale images if necessary, so that they will not overflow the page
% margins by default, and it is still possible to overwrite the defaults
% using explicit options in \includegraphics[width, height, ...]{}
\setkeys{Gin}{width=\maxwidth,height=\maxheight,keepaspectratio}
% Set default figure placement to htbp
\makeatletter
\def\fps@figure{htbp}
\makeatother

\makeatletter
\@ifpackageloaded{bookmark}{}{\usepackage{bookmark}}
\makeatother
\makeatletter
\@ifpackageloaded{caption}{}{\usepackage{caption}}
\AtBeginDocument{%
\ifdefined\contentsname
  \renewcommand*\contentsname{Table of contents}
\else
  \newcommand\contentsname{Table of contents}
\fi
\ifdefined\listfigurename
  \renewcommand*\listfigurename{List of Figures}
\else
  \newcommand\listfigurename{List of Figures}
\fi
\ifdefined\listtablename
  \renewcommand*\listtablename{List of Tables}
\else
  \newcommand\listtablename{List of Tables}
\fi
\ifdefined\figurename
  \renewcommand*\figurename{Figure}
\else
  \newcommand\figurename{Figure}
\fi
\ifdefined\tablename
  \renewcommand*\tablename{Table}
\else
  \newcommand\tablename{Table}
\fi
}
\@ifpackageloaded{float}{}{\usepackage{float}}
\floatstyle{ruled}
\@ifundefined{c@chapter}{\newfloat{codelisting}{h}{lop}}{\newfloat{codelisting}{h}{lop}[chapter]}
\floatname{codelisting}{Listing}
\newcommand*\listoflistings{\listof{codelisting}{List of Listings}}
\makeatother
\makeatletter
\makeatother
\makeatletter
\@ifpackageloaded{caption}{}{\usepackage{caption}}
\@ifpackageloaded{subcaption}{}{\usepackage{subcaption}}
\makeatother

\ifLuaTeX
  \usepackage{selnolig}  % disable illegal ligatures
\fi
\usepackage{bookmark}

\IfFileExists{xurl.sty}{\usepackage{xurl}}{} % add URL line breaks if available
\urlstyle{same} % disable monospaced font for URLs
\hypersetup{
  pdftitle={Statistical Methods for Research},
  pdfauthor={Longhai Li},
  colorlinks=true,
  linkcolor={Maroon},
  filecolor={Maroon},
  citecolor={Blue},
  urlcolor={Blue},
  pdfcreator={LaTeX via pandoc}}


\title{Statistical Methods for Research}
\author{Longhai Li}
\date{2025-11-09}

\begin{document}
\frontmatter
\maketitle

\renewcommand*\contentsname{Table of contents}
{
\hypersetup{linkcolor=}
\setcounter{tocdepth}{2}
\tableofcontents
}

\mainmatter
\bookmarksetup{startatroot}

\chapter{Introduction to Statistical Methods for
Research}\label{introduction-to-statistical-methods-for-research}

\section*{Welcome}\label{welcome}
\addcontentsline{toc}{section}{Welcome}

\markright{Welcome}

This book contains lecture notes for \textbf{STAT 845: Statistical
Methods for Research} at the \textbf{University of Saskatchewan}.

\begin{center}\rule{0.5\linewidth}{0.5pt}\end{center}

\subsection*{Table of Contents}\label{table-of-contents}
\addcontentsline{toc}{subsection}{Table of Contents}

\phantomsection\label{book-toc}

\bookmarksetup{startatroot}

\chapter{R for Data Analysis}\label{r-for-data-analysis}

\section{Basic R Objects and
Operations}\label{basic-r-objects-and-operations}

\begin{Shaded}
\begin{Highlighting}[]
\DocumentationTok{\#\# create a vector}
\NormalTok{x }\OtherTok{\textless{}{-}} \DecValTok{1}\SpecialCharTok{:}\DecValTok{10}
\NormalTok{x }\OtherTok{\textless{}{-}} \FunctionTok{seq}\NormalTok{ (}\DecValTok{30}\NormalTok{,}\DecValTok{3}\NormalTok{, }\AttributeTok{by =} \SpecialCharTok{{-}}\DecValTok{2}\NormalTok{)}
\NormalTok{a }\OtherTok{\textless{}{-}} \FunctionTok{c}\NormalTok{(}\FloatTok{66.32}\NormalTok{, }\FloatTok{69.87}\NormalTok{, }\FloatTok{70.12}\NormalTok{, }\FloatTok{90.37}\NormalTok{, }\FloatTok{50.08}\NormalTok{, }\FloatTok{61.20}\NormalTok{, }\FloatTok{65.00}\NormalTok{, }\FloatTok{57.65}\NormalTok{)}
\NormalTok{d }\OtherTok{\textless{}{-}}\NormalTok{ a [}\DecValTok{1}\NormalTok{]}
\NormalTok{a [}\DecValTok{1}\NormalTok{] }\OtherTok{\textless{}{-}} \FloatTok{85.34}

\FunctionTok{mean}\NormalTok{ (a)}
\end{Highlighting}
\end{Shaded}

\begin{verbatim}
[1] 68.70375
\end{verbatim}

\begin{Shaded}
\begin{Highlighting}[]
\NormalTok{ma }\OtherTok{\textless{}{-}} \FunctionTok{mean}\NormalTok{ (a)}
\DocumentationTok{\#\# read a vector of numbers from a file}
\NormalTok{x }\OtherTok{\textless{}{-}} \FunctionTok{scan}\NormalTok{(}\StringTok{"numbers.txt"}\NormalTok{)}
\NormalTok{x2 }\OtherTok{\textless{}{-}} \FunctionTok{scan}\NormalTok{(}\StringTok{"number2.txt"}\NormalTok{)}

\DocumentationTok{\#\# one can also read number withoug saving to a file}
\NormalTok{y }\OtherTok{\textless{}{-}} \FunctionTok{scan}\NormalTok{(}\AttributeTok{text =} \StringTok{"7  8  9 10 11 12 13 13 14 17 17 45"}\NormalTok{)}

\DocumentationTok{\#\# create a matrix}
\NormalTok{A }\OtherTok{\textless{}{-}} \FunctionTok{matrix}\NormalTok{ (}\DecValTok{0}\NormalTok{, }\DecValTok{4}\NormalTok{, }\DecValTok{2}\NormalTok{)}

\NormalTok{A }\OtherTok{\textless{}{-}} \FunctionTok{matrix}\NormalTok{ (}\DecValTok{1}\SpecialCharTok{:}\DecValTok{8}\NormalTok{, }\DecValTok{4}\NormalTok{,}\DecValTok{2}\NormalTok{)}

\NormalTok{A}
\end{Highlighting}
\end{Shaded}

\begin{verbatim}
     [,1] [,2]
[1,]    1    5
[2,]    2    6
[3,]    3    7
[4,]    4    8
\end{verbatim}

\begin{Shaded}
\begin{Highlighting}[]
\NormalTok{D }\OtherTok{\textless{}{-}} \FunctionTok{matrix}\NormalTok{ (a, }\DecValTok{4}\NormalTok{, }\DecValTok{2}\NormalTok{, }\AttributeTok{byrow=}\NormalTok{T)}

\NormalTok{D }\OtherTok{\textless{}{-}} \FunctionTok{matrix}\NormalTok{(}\DecValTok{1}\SpecialCharTok{:}\DecValTok{8}\NormalTok{, }\DecValTok{2}\NormalTok{, }\DecValTok{4}\NormalTok{)}
\NormalTok{D}
\end{Highlighting}
\end{Shaded}

\begin{verbatim}
     [,1] [,2] [,3] [,4]
[1,]    1    3    5    7
[2,]    2    4    6    8
\end{verbatim}

\begin{Shaded}
\begin{Highlighting}[]
\DocumentationTok{\#\# create another matrix with all entry 0}
\NormalTok{B }\OtherTok{\textless{}{-}} \FunctionTok{matrix}\NormalTok{ (}\DecValTok{1}\SpecialCharTok{:}\DecValTok{5000}\NormalTok{, }\DecValTok{100}\NormalTok{, }\DecValTok{50}\NormalTok{)}

\DocumentationTok{\#\# assign a number to B}
\NormalTok{B[}\DecValTok{2}\NormalTok{,}\DecValTok{4}\NormalTok{] }\OtherTok{\textless{}{-}} \DecValTok{45}
\NormalTok{B[}\DecValTok{1}\NormalTok{,]}
\end{Highlighting}
\end{Shaded}

\begin{verbatim}
 [1]    1  101  201  301  401  501  601  701  801  901 1001 1101 1201 1301 1401
[16] 1501 1601 1701 1801 1901 2001 2101 2201 2301 2401 2501 2601 2701 2801 2901
[31] 3001 3101 3201 3301 3401 3501 3601 3701 3801 3901 4001 4101 4201 4301 4401
[46] 4501 4601 4701 4801 4901
\end{verbatim}

\begin{Shaded}
\begin{Highlighting}[]
\NormalTok{B[,}\DecValTok{1}\NormalTok{]}
\end{Highlighting}
\end{Shaded}

\begin{verbatim}
  [1]   1   2   3   4   5   6   7   8   9  10  11  12  13  14  15  16  17  18
 [19]  19  20  21  22  23  24  25  26  27  28  29  30  31  32  33  34  35  36
 [37]  37  38  39  40  41  42  43  44  45  46  47  48  49  50  51  52  53  54
 [55]  55  56  57  58  59  60  61  62  63  64  65  66  67  68  69  70  71  72
 [73]  73  74  75  76  77  78  79  80  81  82  83  84  85  86  87  88  89  90
 [91]  91  92  93  94  95  96  97  98  99 100
\end{verbatim}

\begin{Shaded}
\begin{Highlighting}[]
\NormalTok{B[}\DecValTok{1}\NormalTok{,] }\OtherTok{\textless{}{-}} \DecValTok{1}\SpecialCharTok{:}\DecValTok{50}


\DocumentationTok{\#\# create a list}
\NormalTok{E }\OtherTok{\textless{}{-}} \FunctionTok{list}\NormalTok{ (}\AttributeTok{newa =}\NormalTok{ a, }\AttributeTok{newA =}\NormalTok{ A)}
\DocumentationTok{\#\# list the names of components}
\FunctionTok{names}\NormalTok{ (E)}
\end{Highlighting}
\end{Shaded}

\begin{verbatim}
[1] "newa" "newA"
\end{verbatim}

\begin{Shaded}
\begin{Highlighting}[]
\DocumentationTok{\#\# to look at the component of E}
\NormalTok{E}\SpecialCharTok{$}\NormalTok{newA }
\end{Highlighting}
\end{Shaded}

\begin{verbatim}
     [,1] [,2]
[1,]    1    5
[2,]    2    6
[3,]    3    7
[4,]    4    8
\end{verbatim}

\begin{Shaded}
\begin{Highlighting}[]
\NormalTok{E}\SpecialCharTok{$}\NormalTok{newa }\OtherTok{\textless{}{-}} \DecValTok{10}\SpecialCharTok{:}\DecValTok{17}

\DocumentationTok{\#\# create a dataframe}
\NormalTok{scores }\OtherTok{\textless{}{-}} \FunctionTok{c}\NormalTok{ (}\DecValTok{30}\NormalTok{, }\DecValTok{45}\NormalTok{, }\DecValTok{50}\NormalTok{)}
\NormalTok{names }\OtherTok{\textless{}{-}} \FunctionTok{c}\NormalTok{(}\StringTok{"Peter"}\NormalTok{, }\StringTok{"John"}\NormalTok{, }\StringTok{"Alice"}\NormalTok{)}
\NormalTok{stat245\_scores }\OtherTok{\textless{}{-}} \FunctionTok{data.frame}\NormalTok{ (names, scores)}
\NormalTok{stat245\_scores}
\end{Highlighting}
\end{Shaded}

\begin{verbatim}
  names scores
1 Peter     30
2  John     45
3 Alice     50
\end{verbatim}

\begin{Shaded}
\begin{Highlighting}[]
\NormalTok{stat245\_scores}\SpecialCharTok{$}\NormalTok{names}
\end{Highlighting}
\end{Shaded}

\begin{verbatim}
[1] "Peter" "John"  "Alice"
\end{verbatim}

\begin{Shaded}
\begin{Highlighting}[]
\NormalTok{stat245\_scores}\SpecialCharTok{$}\NormalTok{scores [}\DecValTok{1}\NormalTok{] }\OtherTok{\textless{}{-}} \DecValTok{40}
\NormalTok{stat245\_scores}
\end{Highlighting}
\end{Shaded}

\begin{verbatim}
  names scores
1 Peter     40
2  John     45
3 Alice     50
\end{verbatim}

\begin{Shaded}
\begin{Highlighting}[]
\NormalTok{stat245\_scores}\SpecialCharTok{$}\NormalTok{perc }\OtherTok{\textless{}{-}}\NormalTok{ stat245\_scores}\SpecialCharTok{$}\NormalTok{scores}\SpecialCharTok{/}\DecValTok{50} \SpecialCharTok{*} \DecValTok{100}
\NormalTok{stat245\_scores}
\end{Highlighting}
\end{Shaded}

\begin{verbatim}
  names scores perc
1 Peter     40   80
2  John     45   90
3 Alice     50  100
\end{verbatim}

\begin{Shaded}
\begin{Highlighting}[]
\NormalTok{stat245\_scores}\SpecialCharTok{$}\NormalTok{adj }\OtherTok{\textless{}{-}}\NormalTok{ stat245\_scores}\SpecialCharTok{$}\NormalTok{perc }\SpecialCharTok{+} \DecValTok{10}
\NormalTok{stat245\_scores}
\end{Highlighting}
\end{Shaded}

\begin{verbatim}
  names scores perc adj
1 Peter     40   80  90
2  John     45   90 100
3 Alice     50  100 110
\end{verbatim}

\begin{Shaded}
\begin{Highlighting}[]
\DocumentationTok{\#\#\#\#\#\#\#\#\#\#\#\#\#\#\#\#\#\#\#\#\#\#\#\#\#\#\#\#\#\#\#\#\#\#\#\#\#\#\#\#\#\#\#\#\#\#\#\#\#\#\#\#\#\#\#\#\#\#\#\#\#\#\#\#\#\#\#\#\#\#\#\#\#\#\#\#\#\#\#}
\end{Highlighting}
\end{Shaded}

\section{Import a dataset into R environment and Simple
Operation}\label{import-a-dataset-into-r-environment-and-simple-operation}

\begin{Shaded}
\begin{Highlighting}[]
\DocumentationTok{\#\#\#\#\#\#\#\#\#\#\#\#\#\#\#\#\#\#\#\#\#\#\#\#\#\#\#\#\#\#\#\#\#\#\#\#\#\#\#\#\#\#\#\#\#\#\#\#\#\#\#\#\#\#\#\#\#\#\#\#\#\#\#\#\#\#\#\#\#\#\#\#\#\#\#\#\#\#\#}

\DocumentationTok{\#\# import myagpop.csv into an R data frame called \textquotesingle{}myagpop\textquotesingle{}}
\NormalTok{agpop }\OtherTok{\textless{}{-}} \FunctionTok{read.csv}\NormalTok{(}\StringTok{"agpop.csv"}\NormalTok{)}

\DocumentationTok{\#\# Now, we can use the data:}

\DocumentationTok{\#\# preview agpop}
\FunctionTok{head}\NormalTok{ (agpop)}
\end{Highlighting}
\end{Shaded}

\begin{verbatim}
                 county state acres92 acres87 acres82 farms92 farms87 farms82
1 ALEUTIAN ISLANDS AREA    AK  683533  726596  764514      26      27      28
2        ANCHORAGE AREA    AK   47146   59297  256709     217     245     223
3        FAIRBANKS AREA    AK  141338  154913  204568     168     175     170
4           JUNEAU AREA    AK     210     214     127       8       8      12
5  KENAI PENINSULA AREA    AK   50810   85712   98035      93     119     137
6        AUTAUGA COUNTY    AL  107259  116050  145044     322     388     453
  largef92 largef87 largef82 smallf92 smallf87 smallf82 region
1       14       16       20        6        4        1      W
2        9       10       11       41       52       38      W
3       25       28       21       12       18       25      W
4        0        0        0        5        4        8      W
5        9       18       17       12       18       19      W
6       25       32       32        8       19       17      S
\end{verbatim}

\begin{Shaded}
\begin{Highlighting}[]
\DocumentationTok{\#\# look at the variable name}
\FunctionTok{colnames}\NormalTok{ (agpop) }
\end{Highlighting}
\end{Shaded}

\begin{verbatim}
 [1] "county"   "state"    "acres92"  "acres87"  "acres82"  "farms92" 
 [7] "farms87"  "farms82"  "largef92" "largef87" "largef82" "smallf92"
[13] "smallf87" "smallf82" "region"  
\end{verbatim}

\begin{Shaded}
\begin{Highlighting}[]
\DocumentationTok{\#\# find number of cols}
\FunctionTok{ncol}\NormalTok{ (agpop) }
\end{Highlighting}
\end{Shaded}

\begin{verbatim}
[1] 15
\end{verbatim}

\begin{Shaded}
\begin{Highlighting}[]
\DocumentationTok{\#\# find number of rows}
\FunctionTok{nrow}\NormalTok{ (agpop) }
\end{Highlighting}
\end{Shaded}

\begin{verbatim}
[1] 3078
\end{verbatim}

\begin{Shaded}
\begin{Highlighting}[]
\DocumentationTok{\#\# access a certain row }
\NormalTok{agpop [}\DecValTok{2}\NormalTok{, ]}
\end{Highlighting}
\end{Shaded}

\begin{verbatim}
          county state acres92 acres87 acres82 farms92 farms87 farms82 largef92
2 ANCHORAGE AREA    AK   47146   59297  256709     217     245     223        9
  largef87 largef82 smallf92 smallf87 smallf82 region
2       10       11       41       52       38      W
\end{verbatim}

\begin{Shaded}
\begin{Highlighting}[]
\DocumentationTok{\#\# access a certain column}
\NormalTok{agpop [}\DecValTok{1}\SpecialCharTok{:}\DecValTok{20}\NormalTok{, }\StringTok{"acres92"}\NormalTok{] }\DocumentationTok{\#\# equivalent to }
\end{Highlighting}
\end{Shaded}

\begin{verbatim}
 [1] 683533  47146 141338    210  50810 107259 167832 177189  48022 137426
[11] 144799  96427  73841 109555 121504  99466  67950  61426  68478  47200
\end{verbatim}

\begin{Shaded}
\begin{Highlighting}[]
\NormalTok{agpop}\SpecialCharTok{$}\NormalTok{acres92[}\DecValTok{1}\SpecialCharTok{:}\DecValTok{20}\NormalTok{]}
\end{Highlighting}
\end{Shaded}

\begin{verbatim}
 [1] 683533  47146 141338    210  50810 107259 167832 177189  48022 137426
[11] 144799  96427  73841 109555 121504  99466  67950  61426  68478  47200
\end{verbatim}

\begin{Shaded}
\begin{Highlighting}[]
\NormalTok{agpop}\SpecialCharTok{$}\NormalTok{largef92[}\DecValTok{1}\SpecialCharTok{:}\DecValTok{20}\NormalTok{]}
\end{Highlighting}
\end{Shaded}

\begin{verbatim}
 [1] 14  9 25  0  9 25 24 40  6  9 29 18  4 22 24  8  9 13  4  5
\end{verbatim}

\begin{Shaded}
\begin{Highlighting}[]
\DocumentationTok{\#\# find mean of acres92}
\FunctionTok{mean}\NormalTok{ (agpop }\SpecialCharTok{$}\NormalTok{acres92)}
\end{Highlighting}
\end{Shaded}

\begin{verbatim}
[1] 306677
\end{verbatim}

\begin{Shaded}
\begin{Highlighting}[]
\DocumentationTok{\#\# find sd of acres92}
\FunctionTok{sd}\NormalTok{ (agpop }\SpecialCharTok{$}\NormalTok{acres92)}
\end{Highlighting}
\end{Shaded}

\begin{verbatim}
[1] 424686.7
\end{verbatim}

\begin{Shaded}
\begin{Highlighting}[]
\NormalTok{agpop\_AK  }\OtherTok{\textless{}{-}}\NormalTok{ agpop [agpop}\SpecialCharTok{$}\NormalTok{state }\SpecialCharTok{==} \StringTok{"AK"}\NormalTok{, ]}

\NormalTok{agpop\_AK }\OtherTok{\textless{}{-}} \FunctionTok{subset}\NormalTok{ (agpop, state }\SpecialCharTok{==} \StringTok{"AK"}\NormalTok{)}

\NormalTok{agpop\_W }\OtherTok{\textless{}{-}} \FunctionTok{subset}\NormalTok{ (agpop, region }\SpecialCharTok{==} \StringTok{"W"}\NormalTok{)}

\NormalTok{agpop\_largefarm }\OtherTok{\textless{}{-}} \FunctionTok{subset}\NormalTok{ (agpop, largef92 }\SpecialCharTok{\textgreater{}} \DecValTok{10}\NormalTok{)}


\FunctionTok{hist}\NormalTok{ (agpop}\SpecialCharTok{$}\NormalTok{acres92)}
\end{Highlighting}
\end{Shaded}

\includegraphics{unit1-introR/introR_files/figure-pdf/unnamed-chunk-2-1.pdf}

Produce Plots

\begin{Shaded}
\begin{Highlighting}[]
\CommentTok{\#pdf ("hist\_acres92.pdf") \#\# use this command and dev.off to save the output to a file}
\FunctionTok{hist}\NormalTok{ (agpop}\SpecialCharTok{$}\NormalTok{acres92)}
\end{Highlighting}
\end{Shaded}

\includegraphics{unit1-introR/introR_files/figure-pdf/unnamed-chunk-3-1.pdf}

\begin{Shaded}
\begin{Highlighting}[]
\CommentTok{\#dev.off()}

\CommentTok{\#jpeg ("agpop\_acres\_87v92.jpg")}

\FunctionTok{plot}\NormalTok{ (agpop}\SpecialCharTok{$}\NormalTok{acres87, agpop}\SpecialCharTok{$}\NormalTok{acres92)}
\FunctionTok{abline}\NormalTok{ (}\AttributeTok{a =} \DecValTok{0}\NormalTok{, }\AttributeTok{b =} \DecValTok{1}\NormalTok{)}
\end{Highlighting}
\end{Shaded}

\includegraphics{unit1-introR/introR_files/figure-pdf/unnamed-chunk-3-2.pdf}

\begin{Shaded}
\begin{Highlighting}[]
\CommentTok{\#dev.off()\#\# this is used to close the jpeg file}
\end{Highlighting}
\end{Shaded}

\section{Create your own function}\label{create-your-own-function}

\begin{Shaded}
\begin{Highlighting}[]
\DocumentationTok{\#\#\# data is a matrix or data.frame}
\NormalTok{means\_col }\OtherTok{\textless{}{-}} \ControlFlowTok{function}\NormalTok{ (data)}
\NormalTok{\{}
\NormalTok{    n }\OtherTok{\textless{}{-}} \FunctionTok{ncol}\NormalTok{ (data)}
\NormalTok{    cmeans }\OtherTok{\textless{}{-}} \FunctionTok{rep}\NormalTok{ (}\ConstantTok{NA}\NormalTok{, n)}
    \ControlFlowTok{for}\NormalTok{ (j }\ControlFlowTok{in} \DecValTok{1}\SpecialCharTok{:}\NormalTok{n)}
\NormalTok{    \{}
\NormalTok{        cmeans[j] }\OtherTok{\textless{}{-}} \FunctionTok{mean}\NormalTok{ (data[,j])}
        
\NormalTok{    \}}
\NormalTok{    cmeans}
\NormalTok{\}}

\DocumentationTok{\#\#\# apply function}
\FunctionTok{means\_col}\NormalTok{ (agpop[, }\DecValTok{3}\SpecialCharTok{:}\DecValTok{13}\NormalTok{])}
\end{Highlighting}
\end{Shaded}

\begin{verbatim}
 [1] 306676.97141 313016.37817 320193.69298    625.50357    678.28428
 [6]    728.06238     56.17674     54.86160     52.62248     54.09227
[11]     59.53769
\end{verbatim}

\begin{Shaded}
\begin{Highlighting}[]
\DocumentationTok{\#\#\# R built{-}in function}
\FunctionTok{colMeans}\NormalTok{ (agpop[, }\DecValTok{3}\SpecialCharTok{:}\DecValTok{13}\NormalTok{])}
\end{Highlighting}
\end{Shaded}

\begin{verbatim}
     acres92      acres87      acres82      farms92      farms87      farms82 
306676.97141 313016.37817 320193.69298    625.50357    678.28428    728.06238 
    largef92     largef87     largef82     smallf92     smallf87 
    56.17674     54.86160     52.62248     54.09227     59.53769 
\end{verbatim}

\section{Include Images Saved in An External
File}\label{include-images-saved-in-an-external-file}

Using the following R code to include your images saved in an external
file.

\begin{Shaded}
\begin{Highlighting}[]
\NormalTok{knitr}\SpecialCharTok{::}\FunctionTok{include\_graphics}\NormalTok{(}\StringTok{"handwriting.png"}\NormalTok{)}
\end{Highlighting}
\end{Shaded}

\includegraphics[width=2.04in,height=\textheight]{unit1-introR/handwriting.png}

You can hide the above R code by setting ``echo=FALSE'' for the r chunk.
For example, I will include the image once again as follows:

\begin{figure}

\centering{

\includegraphics[width=2.04in,height=\textheight]{unit1-introR/handwriting.png}

}

\caption{\label{fig-handwriting}This is a figure inserted from the file
called ``handwriting.png''}

\end{figure}%

\bookmarksetup{startatroot}

\chapter{Simple Linear Regression}\label{simple-linear-regression}

A Simulation Illustration with R

\hfill\break

\begin{Shaded}
\begin{Highlighting}[]
\FunctionTok{require}\NormalTok{(}\StringTok{"knitr"}\NormalTok{)}
\NormalTok{knitr}\SpecialCharTok{::}\NormalTok{opts\_chunk}\SpecialCharTok{$}\FunctionTok{set}\NormalTok{(}
  \AttributeTok{comment =} \StringTok{"\#"}\NormalTok{,}
  \AttributeTok{fig.width =} \DecValTok{6}\NormalTok{,}
  \AttributeTok{fig.height =} \DecValTok{6}\NormalTok{,}
  \AttributeTok{cache =} \ConstantTok{TRUE}
\NormalTok{)}
\FunctionTok{set.seed}\NormalTok{(}\DecValTok{47}\NormalTok{)}

\FunctionTok{options}\NormalTok{(}\AttributeTok{sim\_rebuild=}\ConstantTok{FALSE}\NormalTok{)}
\end{Highlighting}
\end{Shaded}

\section{Overview of Simple Linear
Regression}\label{overview-of-simple-linear-regression}

To make the simple linear regression model concrete, let's first
visualize a simulated dataset that follows \[
Y_i = \beta_0 + \beta_1 X_i + \varepsilon_i, \qquad
\varepsilon_i \sim \mathcal N(0, \sigma^2).
\]

Here, \(\beta_0\) is the intercept, \(\beta_1\) is the slope, and
\(\varepsilon_i\) represents random noise.

\begin{Shaded}
\begin{Highlighting}[]
\FunctionTok{set.seed}\NormalTok{(}\DecValTok{2025}\NormalTok{)}
\NormalTok{n }\OtherTok{\textless{}{-}} \DecValTok{40}
\NormalTok{beta0 }\OtherTok{\textless{}{-}} \DecValTok{2}\NormalTok{; beta1 }\OtherTok{\textless{}{-}} \FloatTok{1.5}\NormalTok{; sigma }\OtherTok{\textless{}{-}} \DecValTok{2}
\NormalTok{x }\OtherTok{\textless{}{-}} \FunctionTok{runif}\NormalTok{(n, }\DecValTok{0}\NormalTok{, }\DecValTok{10}\NormalTok{)}
\NormalTok{y }\OtherTok{\textless{}{-}}\NormalTok{ beta0 }\SpecialCharTok{+}\NormalTok{ beta1 }\SpecialCharTok{*}\NormalTok{ x }\SpecialCharTok{+} \FunctionTok{rnorm}\NormalTok{(n, }\DecValTok{0}\NormalTok{, sigma)}
\NormalTok{dat }\OtherTok{\textless{}{-}} \FunctionTok{data.frame}\NormalTok{(x, y)}

\NormalTok{fit }\OtherTok{\textless{}{-}} \FunctionTok{lm}\NormalTok{(y }\SpecialCharTok{\textasciitilde{}}\NormalTok{ x, }\AttributeTok{data =}\NormalTok{ dat)}

\FunctionTok{plot}\NormalTok{(x, y, }\AttributeTok{pch =} \DecValTok{19}\NormalTok{, }\AttributeTok{col =} \StringTok{"steelblue"}\NormalTok{,}
     \AttributeTok{xlab =} \StringTok{"Predictor X"}\NormalTok{, }\AttributeTok{ylab =} \StringTok{"Response Y"}\NormalTok{,}
     \AttributeTok{main =} \StringTok{"Simulated Data with Fitted Linear Regression Line"}\NormalTok{)}
\FunctionTok{abline}\NormalTok{(fit, }\AttributeTok{col =} \StringTok{"red"}\NormalTok{, }\AttributeTok{lwd =} \DecValTok{2}\NormalTok{)}
\FunctionTok{legend}\NormalTok{(}\StringTok{"topleft"}\NormalTok{, }\AttributeTok{legend =} \FunctionTok{c}\NormalTok{(}\StringTok{"Observed data"}\NormalTok{, }\StringTok{"Fitted line"}\NormalTok{),}
       \AttributeTok{pch =} \FunctionTok{c}\NormalTok{(}\DecValTok{19}\NormalTok{, }\ConstantTok{NA}\NormalTok{), }\AttributeTok{lty =} \FunctionTok{c}\NormalTok{(}\ConstantTok{NA}\NormalTok{, }\DecValTok{1}\NormalTok{), }\AttributeTok{col =} \FunctionTok{c}\NormalTok{(}\StringTok{"steelblue"}\NormalTok{, }\StringTok{"red"}\NormalTok{), }\AttributeTok{bty =} \StringTok{"n"}\NormalTok{)}
\end{Highlighting}
\end{Shaded}

\includegraphics{unit2-slr/slr_files/figure-pdf/unnamed-chunk-2-1.pdf}

The scatterplot shows data points scattered around a line --- the red
line is the fitted regression model.

\begin{center}\rule{0.5\linewidth}{0.5pt}\end{center}

\subsection{Least Squares Estimation}\label{least-squares-estimation}

\textbf{Goal:} Find \(\hat\beta_0\) and \(\hat\beta_1\) that minimize \[
\mathrm{SSE} = \sum_{i=1}^n (y_i - \hat\beta_0 - \hat\beta_1 x_i)^2.
\]

\textbf{Solutions:} \[
\hat\beta_1 = \frac{\sum_i (x_i - \bar x)(y_i - \bar y)}{\sum_i (x_i - \bar x)^2}
= \frac{S_{xy}}{S_{xx}},
\qquad
\hat\beta_0 = \bar y - \hat\beta_1\,\bar x.
\]

Here \[
S_{xy} = \sum_i (x_i - \bar x)(y_i - \bar y),
\qquad
S_{xx} = \sum_i (x_i - \bar x)^2.
\]

\textbf{Shortcut (computational) formulas:} \[
S_{xy} = \sum_i x_i y_i - n\,\bar x\,\bar y,
\qquad
S_{xx} = \sum_i x_i^2 - n\,\bar x^2.
\]

\textbf{Interpretation:}\\
- \(\hat\beta_1\) measures the estimated change in \(Y\) for each unit
increase in \(X\).\\
- \(\hat\beta_0\) represents the fitted value of \(Y\) when \(X=0\).

\begin{center}\rule{0.5\linewidth}{0.5pt}\end{center}

\subsection{Residual and Sum of Squares
Definitions}\label{residual-and-sum-of-squares-definitions}

Let \(\hat y_i = \hat\beta_0 + \hat\beta_1 x_i\) and
\(e_i = y_i - \hat y_i\).

\begin{longtable}[]{@{}
  >{\raggedright\arraybackslash}p{(\columnwidth - 4\tabcolsep) * \real{0.1169}}
  >{\raggedright\arraybackslash}p{(\columnwidth - 4\tabcolsep) * \real{0.1688}}
  >{\raggedright\arraybackslash}p{(\columnwidth - 4\tabcolsep) * \real{0.7143}}@{}}
\toprule\noalign{}
\begin{minipage}[b]{\linewidth}\raggedright
Symbol
\end{minipage} & \begin{minipage}[b]{\linewidth}\raggedright
Definition
\end{minipage} & \begin{minipage}[b]{\linewidth}\raggedright
Computing Formula (in terms of \(S_{xx}, S_{xy}\), etc.)
\end{minipage} \\
\midrule\noalign{}
\endhead
\bottomrule\noalign{}
\endlastfoot
\textbf{SST} & Total Sum of Squares &
\(\displaystyle \sum_i (y_i - \bar y)^2 = S_{yy} = \sum_i y_i^2 - n\,\bar y^2\) \\
\textbf{SSR} & Regression Sum of Squares &
\(\displaystyle \sum_i (\hat y_i - \bar y)^2 = \hat\beta_1^2 S_{xx} = \frac{S_{xy}^2}{S_{xx}}\) \\
\textbf{SSE} & Error (Residual) Sum of Squares &
\(\displaystyle \sum_i (y_i - \hat y_i)^2 = S_{yy} - \frac{S_{xy}^2}{S_{xx}}\) \\
\end{longtable}

\textbf{Identity:} \[
\mathrm{SST} = \mathrm{SSR} + \mathrm{SSE}.
\]

Here, \[
S_{xx} = \sum_i (x_i - \bar x)^2 = \sum_i x_i^2 - n\bar x^2, \qquad
S_{yy} = \sum_i (y_i - \bar y)^2 = \sum_i y_i^2 - n\bar y^2, \qquad
S_{xy} = \sum_i (x_i - \bar x)(y_i - \bar y) = \sum_i x_i y_i - n\bar x \bar y.
\]

\begin{center}\rule{0.5\linewidth}{0.5pt}\end{center}

\subsection{\texorpdfstring{Coefficient of Determination
(\(R^2\))}{Coefficient of Determination (R\^{}2)}}\label{coefficient-of-determination-r2}

Measures the proportion of total variation in \(Y\) explained by \(X\):
\[
R^2 = \frac{\mathrm{SSR}}{\mathrm{SST}}
= 1 - \frac{\mathrm{SSE}}{\mathrm{SST}}.
\]

\textbf{Interpretation:}

\begin{itemize}
\tightlist
\item
  \(R^2 = 1\) means perfect linear fit;
\item
  \(R^2 = 0\) means the model explains none of the variation.
\end{itemize}

\begin{center}\rule{0.5\linewidth}{0.5pt}\end{center}

\subsection{F-test for Overall
Significance}\label{f-test-for-overall-significance}

Tests whether \(X\) is linearly related to \(Y\).

\textbf{Hypotheses:} \[
H_0: \beta_1 = 0
\quad \text{vs.} \quad
H_A: \beta_1 \ne 0.
\]

\textbf{Test Statistic:} \[
F = \frac{\text{MSR}}{\text{MSE}}
= \frac{\text{SSR}/1}{\text{SSE}/(n-2)}
\sim F_{1,n-2}\quad (H_0).
\]

\textbf{p-value approach for observe \(F^{\mathrm{obs}}\):}

Given the observed statistic \(F^{\text{obs}}\) with \((1,\,n-2)\) df,
\[
p-\text{value} \;=\; \Pr\!\big(F_{1,\,n-2} \ge F^{\text{obs}}\big)
\;=\; \mathrm{pf}\!\big(F^{\text{obs}},\,1,\,n-2,\ \text{lower.tail}= \mathrm{FALSE}\big).
\]

\begin{Shaded}
\begin{Highlighting}[]
\DocumentationTok{\#\# {-}{-} Inputs (provide these from your analysis context) {-}{-}{-}{-}{-}{-}{-}{-}{-}{-}{-}{-}{-}{-}{-}{-}{-}{-}{-}{-}{-}{-}{-}{-}{-}}
\DocumentationTok{\#\# n   \textless{}{-} ...   \# sample size}
\DocumentationTok{\#\# SSR \textless{}{-} ...   \# regression sum of squares}
\DocumentationTok{\#\# SSE \textless{}{-} ...   \# error sum of squares}
\NormalTok{n   }\OtherTok{\textless{}{-}} \DecValTok{20}
\NormalTok{SSR }\OtherTok{\textless{}{-}} \DecValTok{5}
\NormalTok{SSE }\OtherTok{\textless{}{-}} \DecValTok{40}




\NormalTok{df1  }\OtherTok{\textless{}{-}} \DecValTok{1}
\NormalTok{df2  }\OtherTok{\textless{}{-}}\NormalTok{ n }\SpecialCharTok{{-}} \DecValTok{2}
\NormalTok{Fobs }\OtherTok{\textless{}{-}}\NormalTok{ (SSR}\SpecialCharTok{/}\NormalTok{df1) }\SpecialCharTok{/}\NormalTok{ (SSE}\SpecialCharTok{/}\NormalTok{df2)         }\CommentTok{\# observed F}
\NormalTok{pval }\OtherTok{\textless{}{-}} \FunctionTok{pf}\NormalTok{(Fobs, }\AttributeTok{df1 =}\NormalTok{ df1, }\AttributeTok{df2 =}\NormalTok{ df2, }\AttributeTok{lower.tail =} \ConstantTok{FALSE}\NormalTok{)}
\NormalTok{pval}
\end{Highlighting}
\end{Shaded}

\begin{verbatim}
[1] 0.1509505
\end{verbatim}

\begin{Shaded}
\begin{Highlighting}[]
\DocumentationTok{\#\# {-}{-} Plot F density and shade the p{-}value tail (with proper annotations) {-}{-}{-}{-}{-}{-}{-}}
\NormalTok{xmax }\OtherTok{\textless{}{-}} \FunctionTok{max}\NormalTok{(}\FunctionTok{qf}\NormalTok{(}\FloatTok{0.995}\NormalTok{, df1, df2), Fobs }\SpecialCharTok{*} \FloatTok{1.2}\NormalTok{)  }\CommentTok{\# extra space for labels}
\NormalTok{peak }\OtherTok{\textless{}{-}} \FunctionTok{max}\NormalTok{(}\FunctionTok{df}\NormalTok{(}\FunctionTok{seq}\NormalTok{(}\DecValTok{0}\NormalTok{, xmax, }\AttributeTok{length.out =} \DecValTok{500}\NormalTok{), df1, df2))}

\DocumentationTok{\#\# Density curve}
\FunctionTok{curve}\NormalTok{(}\FunctionTok{df}\NormalTok{(x, df1, df2), }\AttributeTok{from =} \DecValTok{0}\NormalTok{, }\AttributeTok{to =}\NormalTok{ xmax,}
      \AttributeTok{xlab =} \StringTok{"F"}\NormalTok{, }\AttributeTok{ylab =} \StringTok{"Density"}\NormalTok{,}
      \AttributeTok{main =} \FunctionTok{sprintf}\NormalTok{(}\StringTok{"F(\%d, \%d) density  |  observed F = \%.3f"}\NormalTok{, df1, df2, Fobs))}

\DocumentationTok{\#\# Shade right tail (p{-}value region)}
\NormalTok{xs }\OtherTok{\textless{}{-}} \FunctionTok{seq}\NormalTok{(Fobs, xmax, }\AttributeTok{length.out =} \DecValTok{300}\NormalTok{)}
\NormalTok{ys }\OtherTok{\textless{}{-}} \FunctionTok{df}\NormalTok{(xs, df1, df2)}
\FunctionTok{polygon}\NormalTok{(}\FunctionTok{c}\NormalTok{(Fobs, xs, xmax), }\FunctionTok{c}\NormalTok{(}\DecValTok{0}\NormalTok{, ys, }\DecValTok{0}\NormalTok{),}
        \AttributeTok{col =} \FunctionTok{rgb}\NormalTok{(}\DecValTok{0}\NormalTok{, }\DecValTok{0}\NormalTok{, }\DecValTok{0}\NormalTok{, }\FloatTok{0.18}\NormalTok{), }\AttributeTok{border =} \ConstantTok{NA}\NormalTok{)}

\DocumentationTok{\#\# Vertical line at Fobs (optional visual aid)}
\FunctionTok{abline}\NormalTok{(}\AttributeTok{v =}\NormalTok{ Fobs, }\AttributeTok{lwd =} \DecValTok{2}\NormalTok{)}

\DocumentationTok{\#\# {-}{-}{-}{-} Annotation for F\^{}obs pointing to the x{-}axis value (Fobs, 0) {-}{-}{-}{-}{-}{-}{-}{-}{-}{-}{-}{-}{-}}
\NormalTok{x\_txt\_F }\OtherTok{\textless{}{-}}\NormalTok{ Fobs }\SpecialCharTok{+} \FloatTok{0.06} \SpecialCharTok{*}\NormalTok{ xmax}
\NormalTok{y\_txt\_F }\OtherTok{\textless{}{-}} \FloatTok{0.45} \SpecialCharTok{*}\NormalTok{ peak}
\FunctionTok{arrows}\NormalTok{(}\AttributeTok{x0 =}\NormalTok{ x\_txt\_F, }\AttributeTok{y0 =}\NormalTok{ y\_txt\_F, }\AttributeTok{x1 =}\NormalTok{ Fobs, }\AttributeTok{y1 =} \DecValTok{0}\NormalTok{,}
       \AttributeTok{length =} \FloatTok{0.08}\NormalTok{, }\AttributeTok{lwd =} \FloatTok{1.5}\NormalTok{)}
\FunctionTok{text}\NormalTok{(x\_txt\_F, y\_txt\_F,}
     \AttributeTok{labels =} \FunctionTok{bquote}\NormalTok{(F}\SpecialCharTok{\^{}}\NormalTok{\{obs\} }\SpecialCharTok{==}\NormalTok{ .(}\FunctionTok{format}\NormalTok{(Fobs, }\AttributeTok{digits =} \DecValTok{3}\NormalTok{))),}
     \AttributeTok{pos =} \DecValTok{4}\NormalTok{)}

\DocumentationTok{\#\# {-}{-}{-}{-} Annotation for p{-}value pointing into the shaded tail {-}{-}{-}{-}{-}{-}{-}{-}{-}{-}{-}{-}{-}{-}{-}{-}{-}{-}{-}{-}}
\NormalTok{x\_tip\_p }\OtherTok{\textless{}{-}}\NormalTok{ (Fobs }\SpecialCharTok{+}\NormalTok{ xmax) }\SpecialCharTok{/} \FloatTok{1.7}
\NormalTok{y\_tip\_p }\OtherTok{\textless{}{-}} \FunctionTok{df}\NormalTok{(x\_tip\_p, df1, df2)}
\NormalTok{x\_txt\_p }\OtherTok{\textless{}{-}}\NormalTok{ Fobs }\SpecialCharTok{+} \FloatTok{0.08} \SpecialCharTok{*}\NormalTok{ xmax}
\NormalTok{y\_txt\_p }\OtherTok{\textless{}{-}} \FloatTok{0.80} \SpecialCharTok{*}\NormalTok{ peak}
\FunctionTok{arrows}\NormalTok{(}\AttributeTok{x0 =}\NormalTok{ x\_txt\_p, }\AttributeTok{y0 =}\NormalTok{ y\_txt\_p, }\AttributeTok{x1 =}\NormalTok{ x\_tip\_p, }\AttributeTok{y1 =}\NormalTok{ y\_tip\_p,}
       \AttributeTok{length =} \FloatTok{0.08}\NormalTok{, }\AttributeTok{lwd =} \FloatTok{1.5}\NormalTok{)}
\FunctionTok{text}\NormalTok{(x\_txt\_p, y\_txt\_p,}
     \AttributeTok{labels =} \FunctionTok{bquote}\NormalTok{(p }\SpecialCharTok{==}\NormalTok{ .(}\FunctionTok{format}\NormalTok{(pval, }\AttributeTok{digits =} \DecValTok{4}\NormalTok{, }\AttributeTok{scientific =} \ConstantTok{TRUE}\NormalTok{))),}
     \AttributeTok{pos =} \DecValTok{4}\NormalTok{)}
\end{Highlighting}
\end{Shaded}

\includegraphics{unit2-slr/slr_files/figure-pdf/unnamed-chunk-3-1.pdf}

\begin{center}\rule{0.5\linewidth}{0.5pt}\end{center}

\subsection{\texorpdfstring{t-test for the Slope
\(\beta_1\)}{t-test for the Slope \textbackslash beta\_1}}\label{t-test-for-the-slope-beta_1}

Equivalent to the \(F\)-test in simple regression since \(t^2 = F\).

\textbf{Formula:} \[
t = \frac{\hat\beta_1}{\operatorname{SE}(\hat\beta_1)},
\qquad
\operatorname{SE}(\hat\beta_1) = \sqrt{\frac{\hat\sigma^2}{\sum_i (x_i - \bar x)^2}},
\qquad
\hat\sigma^2 = \frac{\mathrm{SSE}}{n-2}.
\]

\textbf{Distribution:} \[
t \sim t_{n-2}\quad (H_0:\beta_1=0).
\]

\begin{center}\rule{0.5\linewidth}{0.5pt}\end{center}

\subsection{\texorpdfstring{Prediction for a New Case
\(x_0\)}{Prediction for a New Case x\_0}}\label{prediction-for-a-new-case-x_0}

\textbf{Predicted mean response:} \[
\hat y(x_0) = \hat\beta_0 + \hat\beta_1 x_0.
\]

\textbf{95\% Confidence interval for mean response:} \[
\hat y(x_0) \pm t_{1-\alpha/2,,n-2},
\hat\sigma,\sqrt{\frac{1}{n} + \frac{(x_0 - \bar x)^2}{\sum_i (x_i - \bar x)^2}}.
\]

\textbf{95\% Prediction interval for a new observation:} \[
\hat y(x_0) \pm t_{1-\alpha/2,,n-2},
\hat\sigma,\sqrt{1 + \frac{1}{n} + \frac{(x_0 - \bar x)^2}{\sum_i (x_i - \bar x)^2}}.
\]

\begin{center}\rule{0.5\linewidth}{0.5pt}\end{center}

\textbf{Summary Cheat Sheet}

\begin{longtable}[]{@{}
  >{\raggedright\arraybackslash}p{(\columnwidth - 2\tabcolsep) * \real{0.1512}}
  >{\raggedright\arraybackslash}p{(\columnwidth - 2\tabcolsep) * \real{0.8488}}@{}}
\toprule\noalign{}
\begin{minipage}[b]{\linewidth}\raggedright
Concept
\end{minipage} & \begin{minipage}[b]{\linewidth}\raggedright
Key Formula
\end{minipage} \\
\midrule\noalign{}
\endhead
\bottomrule\noalign{}
\endlastfoot
Model & \(Y_i = \beta_0 + \beta_1 X_i + \varepsilon_i\) \\
LS Estimates & \(\hat\beta_1 = S_{xy}/S_{xx}\),
\(\hat\beta_0 = \bar y - \hat\beta_1\bar x\) \\
Decomposition & \(\mathrm{SST} = \mathrm{SSR} + \mathrm{SSE}\) \\
\(R^2\) & \(R^2 = 1 - \mathrm{SSE}/\mathrm{SST}\) \\
\(F\)-test & \(F = (\mathrm{SSR}/1)/(\mathrm{SSE}/(n-2))\) \\
\(t\)-test & \(t = \hat\beta_1 / \operatorname{SE}(\hat\beta_1)\) \\
Prediction & \(\hat y(x_0) = \hat\beta_0 + \hat\beta_1 x_0\) \\
\end{longtable}

\section{Example 1: Vehicle Insurance Premium
(warm-up)}\label{example-1-vehicle-insurance-premium-warm-up}

We examine premiums \(y\) vs.~driving amount \(x\). The scatterplot
hints at a \textbf{downward} trend.

\subsection{Input data}\label{input-data}

\begin{Shaded}
\begin{Highlighting}[]
\NormalTok{issu }\OtherTok{\textless{}{-}} \FunctionTok{data.frame}\NormalTok{(}
  \AttributeTok{driving =} \FunctionTok{c}\NormalTok{(}\DecValTok{5}\NormalTok{, }\DecValTok{2}\NormalTok{, }\DecValTok{12}\NormalTok{, }\DecValTok{9}\NormalTok{, }\DecValTok{15}\NormalTok{, }\DecValTok{6}\NormalTok{, }\DecValTok{25}\NormalTok{, }\DecValTok{16}\NormalTok{),}
  \AttributeTok{premium =} \FunctionTok{c}\NormalTok{(}\DecValTok{64}\NormalTok{, }\DecValTok{87}\NormalTok{, }\DecValTok{50}\NormalTok{, }\DecValTok{71}\NormalTok{, }\DecValTok{44}\NormalTok{, }\DecValTok{56}\NormalTok{, }\DecValTok{42}\NormalTok{, }\DecValTok{60}\NormalTok{)}
\NormalTok{)}

\NormalTok{y }\OtherTok{\textless{}{-}}\NormalTok{ issu}\SpecialCharTok{$}\NormalTok{premium}
\NormalTok{x }\OtherTok{\textless{}{-}}\NormalTok{ issu}\SpecialCharTok{$}\NormalTok{driving}
\NormalTok{xbar }\OtherTok{\textless{}{-}} \FunctionTok{mean}\NormalTok{(x); ybar }\OtherTok{\textless{}{-}} \FunctionTok{mean}\NormalTok{(y); n }\OtherTok{\textless{}{-}} \FunctionTok{length}\NormalTok{(y)}

\FunctionTok{plot}\NormalTok{(x, y, }\AttributeTok{xlab =} \StringTok{"Driving"}\NormalTok{, }\AttributeTok{ylab =} \StringTok{"Premium"}\NormalTok{,}
     \AttributeTok{main =} \StringTok{"Vehicle Insurance: Premium vs. Driving"}\NormalTok{)}
\FunctionTok{abline}\NormalTok{(}\AttributeTok{h =}\NormalTok{ ybar, }\AttributeTok{lty =} \DecValTok{3}\NormalTok{)}
\end{Highlighting}
\end{Shaded}

\includegraphics{unit2-slr/slr_files/figure-pdf/unnamed-chunk-4-1.pdf}

\textbf{Narrative.} The horizontal line at \(\bar y\) represents the
intercept-only model. Any fitted line that tilts away from this must
earn its keep by reducing residual variation enough to offset the loss
of one degree of freedom.

\subsection{Estimating regression
coefficients}\label{estimating-regression-coefficients}

\begin{Shaded}
\begin{Highlighting}[]
\NormalTok{fit.issu }\OtherTok{\textless{}{-}} \FunctionTok{lm}\NormalTok{(y }\SpecialCharTok{\textasciitilde{}}\NormalTok{ x)}
\FunctionTok{plot}\NormalTok{(x, y, }\AttributeTok{xlab =} \StringTok{"Driving"}\NormalTok{, }\AttributeTok{ylab =} \StringTok{"Premium"}\NormalTok{,}
     \AttributeTok{main =} \StringTok{"Fitted Simple Linear Regression"}\NormalTok{)}
\FunctionTok{abline}\NormalTok{(fit.issu, }\AttributeTok{lwd =} \DecValTok{2}\NormalTok{)}
\end{Highlighting}
\end{Shaded}

\includegraphics{unit2-slr/slr_files/figure-pdf/unnamed-chunk-5-1.pdf}

The slope estimate \(\hat\beta_1\) captures the \textbf{marginal change
in premium per unit of driving} (units of \(y\) per unit of \(x\)).
Inference on \(\beta_1\) tells us whether the pattern rises above noise.

\subsection{Residuals and fitted values (geometry
picture)}\label{residuals-and-fitted-values-geometry-picture}

Let \(\hat y_i=\hat\beta_0+\hat\beta_1 x_i\) and \(\tilde y_i=\bar y\).
Residuals are \(e_i=y_i-\hat y_i\) (model) and \(y_i-\bar y\) (null).
Visualizing all three clarifies the ANOVA identity.

\begin{Shaded}
\begin{Highlighting}[]
\NormalTok{beta0 }\OtherTok{\textless{}{-}} \FunctionTok{coef}\NormalTok{(fit.issu)[}\DecValTok{1}\NormalTok{]}
\NormalTok{beta1 }\OtherTok{\textless{}{-}} \FunctionTok{coef}\NormalTok{(fit.issu)[}\DecValTok{2}\NormalTok{]}
\NormalTok{fitted1 }\OtherTok{\textless{}{-}}\NormalTok{ beta0 }\SpecialCharTok{+}\NormalTok{ beta1 }\SpecialCharTok{*}\NormalTok{ x}
\NormalTok{fitted0 }\OtherTok{\textless{}{-}} \FunctionTok{rep}\NormalTok{(ybar, n)}
\NormalTok{residual1 }\OtherTok{\textless{}{-}}\NormalTok{ y }\SpecialCharTok{{-}}\NormalTok{ fitted1}
\NormalTok{residual0 }\OtherTok{\textless{}{-}}\NormalTok{ y }\SpecialCharTok{{-}}\NormalTok{ fitted0}

\FunctionTok{data.frame}\NormalTok{(y, fitted0, residual0, fitted1, residual1,}
           \AttributeTok{diff.fitted =}\NormalTok{ fitted1 }\SpecialCharTok{{-}}\NormalTok{ fitted0)}
\end{Highlighting}
\end{Shaded}

\begin{verbatim}
   y fitted0 residual0  fitted1  residual1 diff.fitted
1 64   59.25      4.75 68.92243  -4.922425    9.672425
2 87   59.25     27.75 73.56519  13.434811   14.315189
3 50   59.25     -9.25 58.08931  -8.089309   -1.160691
4 71   59.25     11.75 62.73207   8.267927    3.482073
5 44   59.25    -15.25 53.44654  -9.446545   -5.803455
6 56   59.25     -3.25 67.37484 -11.374837    8.124837
7 42   59.25    -17.25 37.97066   4.029335  -21.279335
8 60   59.25      0.75 51.89896   8.101043   -7.351043
\end{verbatim}

\includegraphics{unit2-slr/slr_files/figure-pdf/unnamed-chunk-7-1.pdf}

\subsection{SST, SSR, SSE and their
meanings}\label{sst-ssr-sse-and-their-meanings}

\begin{itemize}
\tightlist
\item
  \(\text{SST}=\sum (y_i-\bar y)^2\) quantifies \textbf{total}
  variability around the mean.
\item
  \(\text{SSR}=\sum (\hat y_i-\bar y)^2\) is the part \textbf{explained
  by \(x\)}.
\item
  \(\text{SSE}=\sum (y_i-\hat y_i)^2\) is the \textbf{leftover}
  (unexplained) variability.
\end{itemize}

\begin{Shaded}
\begin{Highlighting}[]
\NormalTok{SST }\OtherTok{\textless{}{-}} \FunctionTok{sum}\NormalTok{((y }\SpecialCharTok{{-}}\NormalTok{ fitted0)}\SpecialCharTok{\^{}}\DecValTok{2}\NormalTok{); SST}
\end{Highlighting}
\end{Shaded}

\begin{verbatim}
[1] 1557.5
\end{verbatim}

\begin{Shaded}
\begin{Highlighting}[]
\NormalTok{SSE }\OtherTok{\textless{}{-}} \FunctionTok{sum}\NormalTok{((y }\SpecialCharTok{{-}}\NormalTok{ fitted1)}\SpecialCharTok{\^{}}\DecValTok{2}\NormalTok{); SSE}
\end{Highlighting}
\end{Shaded}

\begin{verbatim}
[1] 639.0065
\end{verbatim}

\begin{Shaded}
\begin{Highlighting}[]
\NormalTok{SSR }\OtherTok{\textless{}{-}}\NormalTok{ SST }\SpecialCharTok{{-}}\NormalTok{ SSE; SSR}
\end{Highlighting}
\end{Shaded}

\begin{verbatim}
[1] 918.4935
\end{verbatim}

Direct check: \(\text{SSR}=\sum(\hat y_i-\bar y)^2\).

\begin{Shaded}
\begin{Highlighting}[]
\FunctionTok{sum}\NormalTok{((fitted1 }\SpecialCharTok{{-}}\NormalTok{ fitted0)}\SpecialCharTok{\^{}}\DecValTok{2}\NormalTok{)}
\end{Highlighting}
\end{Shaded}

\begin{verbatim}
[1] 918.4935
\end{verbatim}

\subsection{Visual ANOVA on an RSS
plot}\label{visual-anova-on-an-rss-plot}

We place the \textbf{residual sum of squares} against model dimension to
show the trade-off between fit and df.

\begin{Shaded}
\begin{Highlighting}[]
\DocumentationTok{\#\# Recompute cleanly}
\NormalTok{SST }\OtherTok{\textless{}{-}} \FunctionTok{sum}\NormalTok{((y }\SpecialCharTok{{-}} \FunctionTok{mean}\NormalTok{(y))}\SpecialCharTok{\^{}}\DecValTok{2}\NormalTok{)}
\NormalTok{SSE }\OtherTok{\textless{}{-}} \FunctionTok{sum}\NormalTok{(}\FunctionTok{resid}\NormalTok{(fit.issu)}\SpecialCharTok{\^{}}\DecValTok{2}\NormalTok{)}
\NormalTok{SSR }\OtherTok{\textless{}{-}}\NormalTok{ SST }\SpecialCharTok{{-}}\NormalTok{ SSE}
\NormalTok{df\_SSR }\OtherTok{\textless{}{-}} \DecValTok{1}
\NormalTok{df\_SSE }\OtherTok{\textless{}{-}}\NormalTok{ n }\SpecialCharTok{{-}} \DecValTok{2}

\FunctionTok{par}\NormalTok{(}\AttributeTok{mar =} \FunctionTok{c}\NormalTok{(}\DecValTok{6}\NormalTok{, }\DecValTok{4}\NormalTok{, }\DecValTok{4}\NormalTok{, }\DecValTok{2}\NormalTok{) }\SpecialCharTok{+} \FloatTok{0.1}\NormalTok{)}
\FunctionTok{plot}\NormalTok{(}\FunctionTok{c}\NormalTok{(}\DecValTok{1}\NormalTok{, }\DecValTok{2}\NormalTok{, n), }\FunctionTok{c}\NormalTok{(SST, SSE, }\DecValTok{0}\NormalTok{), }\AttributeTok{type =} \StringTok{"b"}\NormalTok{, }\AttributeTok{pch =} \DecValTok{19}\NormalTok{,}
     \AttributeTok{xlab =} \StringTok{"Number of Parameters in Model"}\NormalTok{,}
     \AttributeTok{ylab =} \StringTok{"Residual Sum of Squares (RSS)"}\NormalTok{,}
     \AttributeTok{main =} \StringTok{"ANOVA Geometry on RSS vs. Model Size"}\NormalTok{,}
     \AttributeTok{xlim =} \FunctionTok{c}\NormalTok{(}\DecValTok{0}\NormalTok{, }\DecValTok{14}\NormalTok{), }\AttributeTok{ylim =} \FunctionTok{c}\NormalTok{(}\SpecialCharTok{{-}}\DecValTok{400}\NormalTok{, SST }\SpecialCharTok{*} \FloatTok{1.1}\NormalTok{), }\AttributeTok{xaxt =} \StringTok{"n"}\NormalTok{)}
\FunctionTok{axis}\NormalTok{(}\DecValTok{1}\NormalTok{, }\AttributeTok{at =} \FunctionTok{c}\NormalTok{(}\DecValTok{1}\NormalTok{, }\DecValTok{2}\NormalTok{, n), }\AttributeTok{labels =} \FunctionTok{c}\NormalTok{(}\StringTok{"1 (Intercept)"}\NormalTok{, }\StringTok{"2 (+Slope)"}\NormalTok{, }\FunctionTok{paste}\NormalTok{(n, }\StringTok{"(Saturated)"}\NormalTok{)))}
\FunctionTok{abline}\NormalTok{(}\AttributeTok{h =} \FunctionTok{seq}\NormalTok{(}\DecValTok{0}\NormalTok{, }\DecValTok{2000}\NormalTok{, }\AttributeTok{by =} \DecValTok{100}\NormalTok{), }\AttributeTok{lty =} \DecValTok{3}\NormalTok{, }\AttributeTok{col =} \StringTok{"grey"}\NormalTok{)}

\FunctionTok{par}\NormalTok{(}\AttributeTok{xpd =} \ConstantTok{TRUE}\NormalTok{)}
\FunctionTok{arrows}\NormalTok{(}\DecValTok{9}\NormalTok{, }\DecValTok{0}\NormalTok{, }\DecValTok{9}\NormalTok{, SSE, }\AttributeTok{col =} \StringTok{"blue"}\NormalTok{, }\AttributeTok{code =} \DecValTok{3}\NormalTok{, }\AttributeTok{angle =} \DecValTok{90}\NormalTok{, }\AttributeTok{length =} \FloatTok{0.1}\NormalTok{, }\AttributeTok{lwd =} \DecValTok{2}\NormalTok{)}
\FunctionTok{text}\NormalTok{(}\DecValTok{9}\NormalTok{, SSE}\SpecialCharTok{/}\DecValTok{2}\NormalTok{, }\StringTok{"SSE"}\NormalTok{, }\AttributeTok{col =} \StringTok{"blue"}\NormalTok{, }\AttributeTok{pos =} \DecValTok{4}\NormalTok{, }\AttributeTok{font =} \DecValTok{2}\NormalTok{, }\AttributeTok{cex =} \FloatTok{1.2}\NormalTok{)}

\FunctionTok{arrows}\NormalTok{(}\DecValTok{9}\NormalTok{, SSE, }\DecValTok{9}\NormalTok{, SST, }\AttributeTok{col =} \StringTok{"red"}\NormalTok{, }\AttributeTok{code =} \DecValTok{3}\NormalTok{, }\AttributeTok{angle =} \DecValTok{90}\NormalTok{, }\AttributeTok{length =} \FloatTok{0.1}\NormalTok{, }\AttributeTok{lwd =} \DecValTok{2}\NormalTok{)}
\FunctionTok{text}\NormalTok{(}\DecValTok{9}\NormalTok{, (SST }\SpecialCharTok{+}\NormalTok{ SSE)}\SpecialCharTok{/}\DecValTok{2}\NormalTok{, }\StringTok{"SSR"}\NormalTok{, }\AttributeTok{col =} \StringTok{"red"}\NormalTok{, }\AttributeTok{pos =} \DecValTok{4}\NormalTok{, }\AttributeTok{font =} \DecValTok{2}\NormalTok{, }\AttributeTok{cex =} \FloatTok{1.2}\NormalTok{)}

\FunctionTok{arrows}\NormalTok{(}\DecValTok{2}\NormalTok{, }\SpecialCharTok{{-}}\DecValTok{200}\NormalTok{, n, }\SpecialCharTok{{-}}\DecValTok{200}\NormalTok{, }\AttributeTok{col =} \StringTok{"blue"}\NormalTok{, }\AttributeTok{code =} \DecValTok{3}\NormalTok{, }\AttributeTok{angle =} \DecValTok{90}\NormalTok{, }\AttributeTok{length =} \FloatTok{0.1}\NormalTok{, }\AttributeTok{lwd =} \DecValTok{2}\NormalTok{)}
\FunctionTok{text}\NormalTok{((}\DecValTok{2} \SpecialCharTok{+}\NormalTok{ n)}\SpecialCharTok{/}\DecValTok{2}\NormalTok{, }\SpecialCharTok{{-}}\DecValTok{250}\NormalTok{, }\FunctionTok{paste}\NormalTok{(}\StringTok{"df\_SSE ="}\NormalTok{, df\_SSE), }\AttributeTok{col =} \StringTok{"blue"}\NormalTok{, }\AttributeTok{font =} \DecValTok{2}\NormalTok{)}

\FunctionTok{arrows}\NormalTok{(}\DecValTok{1}\NormalTok{, }\SpecialCharTok{{-}}\DecValTok{200}\NormalTok{, }\DecValTok{2}\NormalTok{, }\SpecialCharTok{{-}}\DecValTok{200}\NormalTok{, }\AttributeTok{col =} \StringTok{"red"}\NormalTok{, }\AttributeTok{code =} \DecValTok{3}\NormalTok{, }\AttributeTok{angle =} \DecValTok{90}\NormalTok{, }\AttributeTok{length =} \FloatTok{0.1}\NormalTok{, }\AttributeTok{lwd =} \DecValTok{2}\NormalTok{)}
\FunctionTok{text}\NormalTok{(}\FloatTok{1.5}\NormalTok{, }\SpecialCharTok{{-}}\DecValTok{250}\NormalTok{, }\FunctionTok{paste}\NormalTok{(}\StringTok{"df\_SSR ="}\NormalTok{, df\_SSR), }\AttributeTok{col =} \StringTok{"red"}\NormalTok{, }\AttributeTok{font =} \DecValTok{2}\NormalTok{)}
\FunctionTok{par}\NormalTok{(}\AttributeTok{xpd =} \ConstantTok{FALSE}\NormalTok{)}

\NormalTok{f\_value }\OtherTok{\textless{}{-}}\NormalTok{ (SSR}\SpecialCharTok{/}\NormalTok{df\_SSR) }\SpecialCharTok{/}\NormalTok{ (SSE}\SpecialCharTok{/}\NormalTok{df\_SSE)}
\NormalTok{p\_value }\OtherTok{\textless{}{-}} \FunctionTok{pf}\NormalTok{(f\_value, }\AttributeTok{df1 =}\NormalTok{ df\_SSR, }\AttributeTok{df2 =}\NormalTok{ df\_SSE, }\AttributeTok{lower.tail =} \ConstantTok{FALSE}\NormalTok{)}
\FunctionTok{legend}\NormalTok{(}\StringTok{"topright"}\NormalTok{,}
       \AttributeTok{legend =} \FunctionTok{c}\NormalTok{(}\FunctionTok{sprintf}\NormalTok{(}\StringTok{"F{-}statistic: \%.2f"}\NormalTok{, f\_value),}
                  \FunctionTok{sprintf}\NormalTok{(}\StringTok{"p{-}value: \%.3f"}\NormalTok{, p\_value)),}
       \AttributeTok{title =} \StringTok{"ANOVA Results"}\NormalTok{, }\AttributeTok{bty =} \StringTok{"o"}\NormalTok{, }\AttributeTok{cex =} \FloatTok{0.9}\NormalTok{)}
\end{Highlighting}
\end{Shaded}

\includegraphics{unit2-slr/slr_files/figure-pdf/unnamed-chunk-10-1.pdf}

\subsection{\texorpdfstring{\(R^2\), \(F\) and a compact ANOVA
table}{R\^{}2, F and a compact ANOVA table}}\label{r2-f-and-a-compact-anova-table}

\begin{Shaded}
\begin{Highlighting}[]
\NormalTok{R2 }\OtherTok{\textless{}{-}}\NormalTok{ SSR }\SpecialCharTok{/}\NormalTok{ SST; R2}
\end{Highlighting}
\end{Shaded}

\begin{verbatim}
[1] 0.5897229
\end{verbatim}

\begin{Shaded}
\begin{Highlighting}[]
\NormalTok{f  }\OtherTok{\textless{}{-}}\NormalTok{ (SSR}\SpecialCharTok{/}\DecValTok{1}\NormalTok{) }\SpecialCharTok{/}\NormalTok{ (SSE}\SpecialCharTok{/}\NormalTok{(n}\DecValTok{{-}2}\NormalTok{)); f}
\end{Highlighting}
\end{Shaded}

\begin{verbatim}
[1] 8.624264
\end{verbatim}

\begin{Shaded}
\begin{Highlighting}[]
\NormalTok{pvf }\OtherTok{\textless{}{-}} \FunctionTok{pf}\NormalTok{(f, }\AttributeTok{df1 =} \DecValTok{1}\NormalTok{, }\AttributeTok{df2 =}\NormalTok{ n}\DecValTok{{-}2}\NormalTok{, }\AttributeTok{lower.tail =} \ConstantTok{FALSE}\NormalTok{); pvf}
\end{Highlighting}
\end{Shaded}

\begin{verbatim}
[1] 0.0260588
\end{verbatim}

\begin{Shaded}
\begin{Highlighting}[]
\NormalTok{Ftable }\OtherTok{\textless{}{-}} \FunctionTok{data.frame}\NormalTok{(}
  \AttributeTok{Source =} \FunctionTok{c}\NormalTok{(}\StringTok{"Regression"}\NormalTok{, }\StringTok{"Error"}\NormalTok{),}
  \AttributeTok{df     =} \FunctionTok{c}\NormalTok{(}\DecValTok{1}\NormalTok{, n }\SpecialCharTok{{-}} \DecValTok{2}\NormalTok{),}
  \AttributeTok{SS     =} \FunctionTok{c}\NormalTok{(SSR, SSE),}
  \AttributeTok{MS     =} \FunctionTok{c}\NormalTok{(SSR}\SpecialCharTok{/}\DecValTok{1}\NormalTok{, SSE}\SpecialCharTok{/}\NormalTok{(n}\DecValTok{{-}2}\NormalTok{)),}
  \AttributeTok{F      =} \FunctionTok{c}\NormalTok{(f, }\ConstantTok{NA}\NormalTok{),}
  \AttributeTok{pvalue =} \FunctionTok{c}\NormalTok{(pvf, }\ConstantTok{NA}\NormalTok{),}
  \AttributeTok{R2part =} \FunctionTok{c}\NormalTok{(SSR, SSE) }\SpecialCharTok{/}\NormalTok{ SST}
\NormalTok{)}
\NormalTok{Ftable}
\end{Highlighting}
\end{Shaded}

\begin{verbatim}
      Source df       SS       MS        F    pvalue    R2part
1 Regression  1 918.4935 918.4935 8.624264 0.0260588 0.5897229
2      Error  6 639.0065 106.5011       NA        NA 0.4102771
\end{verbatim}

A call to \texttt{anova()} reproduces the same test:

\begin{Shaded}
\begin{Highlighting}[]
\FunctionTok{anova}\NormalTok{(fit.issu)}
\end{Highlighting}
\end{Shaded}

\begin{verbatim}
Analysis of Variance Table

Response: y
          Df Sum Sq Mean Sq F value  Pr(>F)  
x          1 918.49  918.49  8.6243 0.02606 *
Residuals  6 639.01  106.50                  
---
Signif. codes:  0 '***' 0.001 '**' 0.01 '*' 0.05 '.' 0.1 ' ' 1
\end{verbatim}

\subsection{Sampling distributions via
animation}\label{sampling-distributions-via-animation}

Under \(H_0:\beta_1=0\), \(F\) follows \(F_{1,n-2}\). Under \(H_A\), the
distribution shifts right (noncentral \(F\)).

\subsubsection{\texorpdfstring{Null world (\(H_0\)
true)}{Null world (H\_0 true)}}\label{null-world-h_0-true}

\begin{figure}

\centering{

\includegraphics{unit2-slr/figs/sim_h0.png}

}

\caption{\label{fig-simulation}Simulation under H0: animated GIF (HTML)
and static PNG (PDF).}

\end{figure}%

\subsubsection{\texorpdfstring{Alternative world (\(H_1\)
true)}{Alternative world (H\_1 true)}}\label{alternative-world-h_1-true}

\begin{figure}

\centering{

\includegraphics{unit2-slr/figs/sim_HA.png}

}

\caption{\label{fig-simulation-HA}Simulation under HA (slope = -2):
animated GIF for HTML, static PNG for PDF.}

\end{figure}%

\begin{center}\rule{0.5\linewidth}{0.5pt}\end{center}

\section{Example 2: Oxygen Purity
Data}\label{example-2-oxygen-purity-data}

We model oxygen purity \(y\) as a function of hydrocarbon level \(x\)
and report both \textbf{mean response} and \textbf{prediction}
uncertainty.

\subsection{Data}\label{data}

\begin{Shaded}
\begin{Highlighting}[]
\NormalTok{x }\OtherTok{\textless{}{-}} \FunctionTok{c}\NormalTok{(}\FloatTok{0.99}\NormalTok{, }\FloatTok{1.02}\NormalTok{, }\FloatTok{1.15}\NormalTok{, }\FloatTok{1.29}\NormalTok{, }\FloatTok{1.46}\NormalTok{, }\FloatTok{1.36}\NormalTok{, }\FloatTok{0.87}\NormalTok{, }\FloatTok{1.23}\NormalTok{, }\FloatTok{1.55}\NormalTok{, }\FloatTok{1.40}\NormalTok{, }\FloatTok{1.19}\NormalTok{,}
       \FloatTok{1.15}\NormalTok{, }\FloatTok{0.98}\NormalTok{, }\FloatTok{1.01}\NormalTok{, }\FloatTok{1.11}\NormalTok{, }\FloatTok{1.20}\NormalTok{, }\FloatTok{1.26}\NormalTok{, }\FloatTok{1.32}\NormalTok{, }\FloatTok{1.43}\NormalTok{, }\FloatTok{0.95}\NormalTok{)}
\NormalTok{y }\OtherTok{\textless{}{-}} \FunctionTok{c}\NormalTok{(}\FloatTok{90.01}\NormalTok{, }\FloatTok{89.05}\NormalTok{, }\FloatTok{91.43}\NormalTok{, }\FloatTok{93.74}\NormalTok{, }\FloatTok{96.73}\NormalTok{, }\FloatTok{94.45}\NormalTok{, }\FloatTok{87.59}\NormalTok{, }\FloatTok{91.77}\NormalTok{, }\FloatTok{99.42}\NormalTok{, }\FloatTok{93.65}\NormalTok{,}
       \FloatTok{93.54}\NormalTok{, }\FloatTok{92.52}\NormalTok{, }\FloatTok{90.56}\NormalTok{, }\FloatTok{89.54}\NormalTok{, }\FloatTok{89.85}\NormalTok{, }\FloatTok{90.39}\NormalTok{, }\FloatTok{93.25}\NormalTok{, }\FloatTok{93.41}\NormalTok{, }\FloatTok{94.98}\NormalTok{, }\FloatTok{87.33}\NormalTok{)}
\NormalTok{n }\OtherTok{\textless{}{-}} \FunctionTok{length}\NormalTok{(x); n}
\end{Highlighting}
\end{Shaded}

\begin{verbatim}
[1] 20
\end{verbatim}

\begin{Shaded}
\begin{Highlighting}[]
\NormalTok{purity.data }\OtherTok{\textless{}{-}} \FunctionTok{data.frame}\NormalTok{(}\AttributeTok{x =}\NormalTok{ x, }\AttributeTok{y =}\NormalTok{ y)}
\FunctionTok{head}\NormalTok{(purity.data)}
\end{Highlighting}
\end{Shaded}

\begin{verbatim}
     x     y
1 0.99 90.01
2 1.02 89.05
3 1.15 91.43
4 1.29 93.74
5 1.46 96.73
6 1.36 94.45
\end{verbatim}

\subsection{Fit and quick summary}\label{fit-and-quick-summary}

\begin{Shaded}
\begin{Highlighting}[]
\NormalTok{fit }\OtherTok{\textless{}{-}} \FunctionTok{lm}\NormalTok{(y }\SpecialCharTok{\textasciitilde{}}\NormalTok{ x, }\AttributeTok{data =}\NormalTok{ purity.data)}
\FunctionTok{summary}\NormalTok{(fit)}
\end{Highlighting}
\end{Shaded}

\begin{verbatim}

Call:
lm(formula = y ~ x, data = purity.data)

Residuals:
     Min       1Q   Median       3Q      Max 
-1.83029 -0.73334  0.04497  0.69969  1.96809 

Coefficients:
            Estimate Std. Error t value Pr(>|t|)    
(Intercept)   74.283      1.593   46.62  < 2e-16 ***
x             14.947      1.317   11.35 1.23e-09 ***
---
Signif. codes:  0 '***' 0.001 '**' 0.01 '*' 0.05 '.' 0.1 ' ' 1

Residual standard error: 1.087 on 18 degrees of freedom
Multiple R-squared:  0.8774,    Adjusted R-squared:  0.8706 
F-statistic: 128.9 on 1 and 18 DF,  p-value: 1.227e-09
\end{verbatim}

\textbf{Interpretation.} The slope's sign gives the direction of
association; its \(t\) test (or \(F\) with 1 df) assesses evidence for a
trend. Look at \(\hat\sigma\) for noise scale and \(R^2\) for variance
explained.

\subsection{Scatter with fitted line}\label{scatter-with-fitted-line}

\begin{Shaded}
\begin{Highlighting}[]
\FunctionTok{plot}\NormalTok{(purity.data}\SpecialCharTok{$}\NormalTok{x, purity.data}\SpecialCharTok{$}\NormalTok{y,}
     \AttributeTok{xlab =} \StringTok{"Hydrocarbon level (x)"}\NormalTok{, }\AttributeTok{ylab =} \StringTok{"Purity (y)"}\NormalTok{,}
     \AttributeTok{main =} \StringTok{"Oxygen Purity vs Hydrocarbon Level"}\NormalTok{)}
\FunctionTok{abline}\NormalTok{(fit, }\AttributeTok{col =} \StringTok{"red"}\NormalTok{, }\AttributeTok{lwd =} \DecValTok{2}\NormalTok{)}
\end{Highlighting}
\end{Shaded}

\includegraphics{unit2-slr/slr_files/figure-pdf/plot-fit-1.pdf}

\subsection{Coefficient CIs and ANOVA}\label{coefficient-cis-and-anova}

\begin{Shaded}
\begin{Highlighting}[]
\FunctionTok{confint}\NormalTok{(fit, }\AttributeTok{level =} \FloatTok{0.95}\NormalTok{)}
\end{Highlighting}
\end{Shaded}

\begin{verbatim}
               2.5 %   97.5 %
(Intercept) 70.93555 77.63108
x           12.18107 17.71389
\end{verbatim}

\begin{Shaded}
\begin{Highlighting}[]
\FunctionTok{anova}\NormalTok{(fit)}
\end{Highlighting}
\end{Shaded}

\begin{verbatim}
Analysis of Variance Table

Response: y
          Df Sum Sq Mean Sq F value    Pr(>F)    
x          1 152.13 152.127  128.86 1.227e-09 ***
Residuals 18  21.25   1.181                      
---
Signif. codes:  0 '***' 0.001 '**' 0.01 '*' 0.05 '.' 0.1 ' ' 1
\end{verbatim}

\subsection{Mean-response and prediction
bands}\label{mean-response-and-prediction-bands}

The \textbf{mean-response CI} narrows near \(\bar x\) and widens at the
extremes; the \textbf{prediction band} is wider by the irreducible noise
term.

\begin{Shaded}
\begin{Highlighting}[]
\NormalTok{x0 }\OtherTok{\textless{}{-}} \FunctionTok{seq}\NormalTok{(}\FunctionTok{min}\NormalTok{(purity.data}\SpecialCharTok{$}\NormalTok{x), }\FunctionTok{max}\NormalTok{(purity.data}\SpecialCharTok{$}\NormalTok{x), }\AttributeTok{length =} \DecValTok{50}\NormalTok{)}
\NormalTok{newdata }\OtherTok{\textless{}{-}} \FunctionTok{data.frame}\NormalTok{(}\AttributeTok{x =}\NormalTok{ x0)}

\NormalTok{est.mean }\OtherTok{\textless{}{-}} \FunctionTok{predict}\NormalTok{(fit, }\AttributeTok{newdata =}\NormalTok{ newdata, }\AttributeTok{interval =} \StringTok{"confidence"}\NormalTok{, }\AttributeTok{level =} \FloatTok{0.95}\NormalTok{)}
\NormalTok{pred.new }\OtherTok{\textless{}{-}} \FunctionTok{predict}\NormalTok{(fit, }\AttributeTok{newdata =}\NormalTok{ newdata, }\AttributeTok{interval =} \StringTok{"prediction"}\NormalTok{, }\AttributeTok{level =} \FloatTok{0.95}\NormalTok{)}
\end{Highlighting}
\end{Shaded}

\begin{Shaded}
\begin{Highlighting}[]
\FunctionTok{plot}\NormalTok{(purity.data}\SpecialCharTok{$}\NormalTok{x, purity.data}\SpecialCharTok{$}\NormalTok{y,}
     \AttributeTok{xlab =} \StringTok{"Hydrocarbon level (x)"}\NormalTok{, }\AttributeTok{ylab =} \StringTok{"Purity (y)"}\NormalTok{,}
     \AttributeTok{main =} \StringTok{"Regression Line with Confidence and Prediction Bands"}\NormalTok{)}
\FunctionTok{abline}\NormalTok{(fit)}
\FunctionTok{matlines}\NormalTok{(x0, est.mean[, }\DecValTok{2}\SpecialCharTok{:}\DecValTok{3}\NormalTok{], }\AttributeTok{col =} \StringTok{"blue"}\NormalTok{, }\AttributeTok{lty =} \DecValTok{2}\NormalTok{, }\AttributeTok{lwd =} \DecValTok{2}\NormalTok{)}
\FunctionTok{matlines}\NormalTok{(x0, pred.new[, }\DecValTok{2}\SpecialCharTok{:}\DecValTok{3}\NormalTok{], }\AttributeTok{col =} \StringTok{"red"}\NormalTok{,  }\AttributeTok{lty =} \DecValTok{3}\NormalTok{, }\AttributeTok{lwd =} \DecValTok{2}\NormalTok{)}
\FunctionTok{legend}\NormalTok{(}\StringTok{"topleft"}\NormalTok{, }\FunctionTok{c}\NormalTok{(}\StringTok{"Confidence Bands (mean)"}\NormalTok{, }\StringTok{"Prediction Bands (new y)"}\NormalTok{),}
       \AttributeTok{col =} \FunctionTok{c}\NormalTok{(}\StringTok{"blue"}\NormalTok{, }\StringTok{"red"}\NormalTok{), }\AttributeTok{lty =} \DecValTok{2}\SpecialCharTok{:}\DecValTok{3}\NormalTok{, }\AttributeTok{bty =} \StringTok{"n"}\NormalTok{)}
\end{Highlighting}
\end{Shaded}

\includegraphics{unit2-slr/slr_files/figure-pdf/plot-bands-1.pdf}

\subsection{Residual diagnostics (assumptions
check)}\label{residual-diagnostics-assumptions-check}

We look for \textbf{no pattern} in residuals vs.~fits and
\textbf{approximate straightness} in the Q--Q plot.

\begin{Shaded}
\begin{Highlighting}[]
\NormalTok{pred }\OtherTok{\textless{}{-}} \FunctionTok{fitted.values}\NormalTok{(fit)}
\NormalTok{e }\OtherTok{\textless{}{-}} \FunctionTok{resid}\NormalTok{(fit)}
\NormalTok{d }\OtherTok{\textless{}{-}}\NormalTok{ e }\SpecialCharTok{/} \FunctionTok{summary}\NormalTok{(fit)}\SpecialCharTok{$}\NormalTok{sigma}

\FunctionTok{par}\NormalTok{(}\AttributeTok{mfrow =} \FunctionTok{c}\NormalTok{(}\DecValTok{2}\NormalTok{,}\DecValTok{2}\NormalTok{))}
\FunctionTok{plot}\NormalTok{(purity.data}\SpecialCharTok{$}\NormalTok{x, purity.data}\SpecialCharTok{$}\NormalTok{y, }\AttributeTok{xlab =} \StringTok{"x"}\NormalTok{, }\AttributeTok{ylab =} \StringTok{"y"}\NormalTok{); }\FunctionTok{abline}\NormalTok{(fit)}
\FunctionTok{qqnorm}\NormalTok{(d, }\AttributeTok{main =} \StringTok{"Normal Q–Q"}\NormalTok{); }\FunctionTok{qqline}\NormalTok{(d)}
\FunctionTok{plot}\NormalTok{(pred, d, }\AttributeTok{xlab =} \StringTok{"Fitted"}\NormalTok{, }\AttributeTok{ylab =} \StringTok{"Std. residuals"}\NormalTok{, }\AttributeTok{main =} \StringTok{"Residuals vs Fits"}\NormalTok{); }\FunctionTok{abline}\NormalTok{(}\AttributeTok{h =} \DecValTok{0}\NormalTok{, }\AttributeTok{lty =} \DecValTok{2}\NormalTok{)}
\FunctionTok{plot}\NormalTok{(}\DecValTok{1}\SpecialCharTok{:}\NormalTok{n, d, }\AttributeTok{xlab =} \StringTok{"Order"}\NormalTok{, }\AttributeTok{ylab =} \StringTok{"Std. residuals"}\NormalTok{, }\AttributeTok{main =} \StringTok{"Residuals vs Order"}\NormalTok{); }\FunctionTok{abline}\NormalTok{(}\AttributeTok{h =} \DecValTok{0}\NormalTok{, }\AttributeTok{lty =} \DecValTok{2}\NormalTok{)}
\end{Highlighting}
\end{Shaded}

\includegraphics{unit2-slr/slr_files/figure-pdf/residual-analysis-1.pdf}

\begin{Shaded}
\begin{Highlighting}[]
\FunctionTok{par}\NormalTok{(}\AttributeTok{mfrow =} \FunctionTok{c}\NormalTok{(}\DecValTok{1}\NormalTok{,}\DecValTok{1}\NormalTok{))}
\end{Highlighting}
\end{Shaded}

\begin{center}\rule{0.5\linewidth}{0.5pt}\end{center}

\section{Correlation analysis (for comparison, not
causation)}\label{correlation-analysis-for-comparison-not-causation}

Correlation summarizes linear association without fitting a line or
making model assumptions.

\subsection{Data and scatter}\label{data-and-scatter}

\begin{Shaded}
\begin{Highlighting}[]
\NormalTok{strength }\OtherTok{\textless{}{-}} \FunctionTok{c}\NormalTok{(}\FloatTok{9.95}\NormalTok{,}\FloatTok{24.45}\NormalTok{,}\FloatTok{31.75}\NormalTok{,}\FloatTok{35.00}\NormalTok{,}\FloatTok{25.02}\NormalTok{,}\FloatTok{16.86}\NormalTok{,}\FloatTok{14.38}\NormalTok{,}\FloatTok{9.60}\NormalTok{,}\FloatTok{24.35}\NormalTok{,}
              \FloatTok{27.50}\NormalTok{,}\FloatTok{17.08}\NormalTok{,}\FloatTok{37.00}\NormalTok{,}\FloatTok{41.95}\NormalTok{,}\FloatTok{11.66}\NormalTok{,}\FloatTok{21.65}\NormalTok{,}\FloatTok{17.89}\NormalTok{,}\FloatTok{69.00}\NormalTok{,}\FloatTok{10.30}\NormalTok{,}
              \FloatTok{34.93}\NormalTok{,}\FloatTok{46.59}\NormalTok{,}\FloatTok{44.88}\NormalTok{,}\FloatTok{54.12}\NormalTok{,}\FloatTok{56.63}\NormalTok{,}\FloatTok{22.13}\NormalTok{,}\FloatTok{21.15}\NormalTok{)}
\NormalTok{length }\OtherTok{\textless{}{-}} \FunctionTok{c}\NormalTok{(}\DecValTok{2}\NormalTok{,}\DecValTok{8}\NormalTok{,}\DecValTok{11}\NormalTok{,}\DecValTok{10}\NormalTok{,}\DecValTok{8}\NormalTok{,}\DecValTok{4}\NormalTok{,}\DecValTok{2}\NormalTok{,}\DecValTok{2}\NormalTok{,}\DecValTok{9}\NormalTok{,}\DecValTok{8}\NormalTok{,}\DecValTok{4}\NormalTok{,}\DecValTok{11}\NormalTok{,}\DecValTok{12}\NormalTok{,}\DecValTok{2}\NormalTok{,}\DecValTok{4}\NormalTok{,}\DecValTok{4}\NormalTok{,}\DecValTok{20}\NormalTok{,}\DecValTok{1}\NormalTok{,}\DecValTok{10}\NormalTok{,}
            \DecValTok{15}\NormalTok{,}\DecValTok{15}\NormalTok{,}\DecValTok{16}\NormalTok{,}\DecValTok{17}\NormalTok{,}\DecValTok{6}\NormalTok{,}\DecValTok{5}\NormalTok{)}
\FunctionTok{plot}\NormalTok{(length, strength, }\AttributeTok{xlab =} \StringTok{"Length"}\NormalTok{, }\AttributeTok{ylab =} \StringTok{"Strength"}\NormalTok{,}
     \AttributeTok{main =} \StringTok{"Strength vs Length (scatter)"}\NormalTok{)}
\end{Highlighting}
\end{Shaded}

\includegraphics{unit2-slr/slr_files/figure-pdf/correlation-data-1.pdf}

\subsection{Pearson correlation and
test}\label{pearson-correlation-and-test}

\begin{Shaded}
\begin{Highlighting}[]
\FunctionTok{cor}\NormalTok{(strength, length)}
\end{Highlighting}
\end{Shaded}

\begin{verbatim}
[1] 0.9818118
\end{verbatim}

\begin{Shaded}
\begin{Highlighting}[]
\FunctionTok{cor.test}\NormalTok{(strength, length)}
\end{Highlighting}
\end{Shaded}

\begin{verbatim}

    Pearson's product-moment correlation

data:  strength and length
t = 24.801, df = 23, p-value < 2.2e-16
alternative hypothesis: true correlation is not equal to 0
95 percent confidence interval:
 0.9585414 0.9920735
sample estimates:
      cor 
0.9818118 
\end{verbatim}

\textbf{Note.} A large \(|r|\) and small \(p\) indicate linear
association; regression further quantifies the slope and supports
prediction, with diagnostics to check assumptions.

\begin{center}\rule{0.5\linewidth}{0.5pt}\end{center}

\section{What to report (checklist)}\label{what-to-report-checklist}

\begin{itemize}
\tightlist
\item
  Estimated line \(\hat y=\hat\beta_0+\hat\beta_1 x\) with units.
\item
  \(t\)/\(F\) test for slope, \(p\)-value and CI for \(\beta_1\).
\item
  \(R^2\) and \(\hat\sigma\) (RMSE) for fit quality.
\item
  Mean-response and prediction intervals at substantively relevant
  \(x_0\).
\item
  Residual diagnostics and any remedies (transformations, robust
  methods) if needed.
\end{itemize}

\bookmarksetup{startatroot}

\chapter{Multiple Linear Regression}\label{multiple-linear-regression}

\section{An Example: Wire Bond Strength
Dataset}\label{an-example-wire-bond-strength-dataset}

\subsection{Loading Data and
Visualization}\label{loading-data-and-visualization}

\textbf{Note:} You must change the file paths in the \texttt{read.csv()}
functions below to match the location of the files on your computer (for
example
\texttt{C:\textbackslash{}\textbackslash{}Users\textbackslash{}\textbackslash{}\textless{}YourUsername\textgreater{}\textbackslash{}\textbackslash{}Documents}
on Windows).

\begin{Shaded}
\begin{Highlighting}[]
\DocumentationTok{\#\# Read data. Change the path as necessary.}
\DocumentationTok{\#\# Example: bond.data \textless{}{-} read.csv("wire{-}bond.csv")}
\NormalTok{bond.data }\OtherTok{\textless{}{-}} \FunctionTok{read.csv}\NormalTok{(}\StringTok{"wire{-}bond.csv"}\NormalTok{)}

\DocumentationTok{\#\# This will now be automatically rendered as a paged table}
\NormalTok{bond.data}
\end{Highlighting}
\end{Shaded}

\begin{verbatim}
   strength length height
1      9.95      2     50
2     24.45      8    110
3     31.75     11    120
4     35.00     10    550
5     25.02      8    295
6     16.86      4    200
7     14.38      2    375
8      9.60      2     52
9     24.35      9    100
10    27.50      8    300
11    17.08      4    412
12    37.00     11    400
13    41.95     12    500
14    11.66      2    360
15    21.65      4    205
16    17.89      4    400
17    69.00     20    600
18    10.30      1    585
19    34.93     10    540
20    46.59     15    250
21    44.88     15    290
22    54.12     16    510
23    56.63     17    590
24    22.13      6    100
25    21.15      5    400
\end{verbatim}

\textbf{2D Visualization}

\begin{Shaded}
\begin{Highlighting}[]
\FunctionTok{par}\NormalTok{(}\AttributeTok{mfrow =} \FunctionTok{c}\NormalTok{(}\DecValTok{1}\NormalTok{, }\DecValTok{3}\NormalTok{), }\AttributeTok{mar =} \FunctionTok{c}\NormalTok{(}\DecValTok{5}\NormalTok{, }\DecValTok{4}\NormalTok{, }\DecValTok{2}\NormalTok{, }\DecValTok{1}\NormalTok{))}

\DocumentationTok{\#\# 1) length vs strength}
\NormalTok{i1 }\OtherTok{\textless{}{-}} \FunctionTok{which}\NormalTok{(}\SpecialCharTok{!}\FunctionTok{is.na}\NormalTok{(bond.data}\SpecialCharTok{$}\NormalTok{length) }\SpecialCharTok{\&} \SpecialCharTok{!}\FunctionTok{is.na}\NormalTok{(bond.data}\SpecialCharTok{$}\NormalTok{strength))}
\FunctionTok{plot}\NormalTok{(bond.data}\SpecialCharTok{$}\NormalTok{length[i1], bond.data}\SpecialCharTok{$}\NormalTok{strength[i1],}
     \AttributeTok{xlab =} \StringTok{"Wire Length"}\NormalTok{, }\AttributeTok{ylab =} \StringTok{"Pull strength"}\NormalTok{, }\AttributeTok{pch =} \DecValTok{19}\NormalTok{)}
\FunctionTok{text}\NormalTok{(bond.data}\SpecialCharTok{$}\NormalTok{length[i1], bond.data}\SpecialCharTok{$}\NormalTok{strength[i1],}
     \AttributeTok{labels =}\NormalTok{ i1, }\AttributeTok{pos =} \DecValTok{1}\NormalTok{, }\AttributeTok{offset =} \FloatTok{0.4}\NormalTok{, }\AttributeTok{cex =} \FloatTok{0.75}\NormalTok{)}

\DocumentationTok{\#\# 2) height vs strength}
\NormalTok{i2 }\OtherTok{\textless{}{-}} \FunctionTok{which}\NormalTok{(}\SpecialCharTok{!}\FunctionTok{is.na}\NormalTok{(bond.data}\SpecialCharTok{$}\NormalTok{height) }\SpecialCharTok{\&} \SpecialCharTok{!}\FunctionTok{is.na}\NormalTok{(bond.data}\SpecialCharTok{$}\NormalTok{strength))}
\FunctionTok{plot}\NormalTok{(bond.data}\SpecialCharTok{$}\NormalTok{height[i2], bond.data}\SpecialCharTok{$}\NormalTok{strength[i2],}
     \AttributeTok{xlab =} \StringTok{"Die height"}\NormalTok{, }\AttributeTok{ylab =} \StringTok{"Pull strength"}\NormalTok{, }\AttributeTok{pch =} \DecValTok{19}\NormalTok{)}
\FunctionTok{text}\NormalTok{(bond.data}\SpecialCharTok{$}\NormalTok{height[i2], bond.data}\SpecialCharTok{$}\NormalTok{strength[i2],}
     \AttributeTok{labels =}\NormalTok{ i2, }\AttributeTok{pos =} \DecValTok{1}\NormalTok{, }\AttributeTok{offset =} \FloatTok{0.4}\NormalTok{, }\AttributeTok{cex =} \FloatTok{0.75}\NormalTok{)}

\DocumentationTok{\#\# 3) height vs length}
\NormalTok{i3 }\OtherTok{\textless{}{-}} \FunctionTok{which}\NormalTok{(}\SpecialCharTok{!}\FunctionTok{is.na}\NormalTok{(bond.data}\SpecialCharTok{$}\NormalTok{height) }\SpecialCharTok{\&} \SpecialCharTok{!}\FunctionTok{is.na}\NormalTok{(bond.data}\SpecialCharTok{$}\NormalTok{length))}
\FunctionTok{plot}\NormalTok{(bond.data}\SpecialCharTok{$}\NormalTok{height[i3], bond.data}\SpecialCharTok{$}\NormalTok{length[i3],}
     \AttributeTok{xlab =} \StringTok{"Die height"}\NormalTok{, }\AttributeTok{ylab =} \StringTok{"Length"}\NormalTok{, }\AttributeTok{pch =} \DecValTok{19}\NormalTok{)}
\FunctionTok{text}\NormalTok{(bond.data}\SpecialCharTok{$}\NormalTok{height[i3], bond.data}\SpecialCharTok{$}\NormalTok{length[i3],}
     \AttributeTok{labels =}\NormalTok{ i3, }\AttributeTok{pos =} \DecValTok{1}\NormalTok{, }\AttributeTok{offset =} \FloatTok{0.4}\NormalTok{, }\AttributeTok{cex =} \FloatTok{0.75}\NormalTok{)}
\end{Highlighting}
\end{Shaded}

\includegraphics{unit3-mlr/mlr_files/figure-pdf/unnamed-chunk-4-1.pdf}

\textbf{3D Visualize}

\begin{Shaded}
\begin{Highlighting}[]
\FunctionTok{library}\NormalTok{(scatterplot3d)}

\FunctionTok{par}\NormalTok{(}\AttributeTok{mfrow =} \FunctionTok{c}\NormalTok{(}\DecValTok{1}\NormalTok{,}\DecValTok{1}\NormalTok{))}
\NormalTok{s3d }\OtherTok{\textless{}{-}} \FunctionTok{with}\NormalTok{(bond.data, }\FunctionTok{scatterplot3d}\NormalTok{(}
  \AttributeTok{x =}\NormalTok{ length,}
  \AttributeTok{y =}\NormalTok{ height,}
  \AttributeTok{z =}\NormalTok{ strength,}
  \AttributeTok{pch =} \DecValTok{19}\NormalTok{,}
  \AttributeTok{color =} \StringTok{"steelblue"}\NormalTok{,}
  \AttributeTok{main =} \StringTok{"3D Scatterplot: Strength vs. Length and Height"}\NormalTok{,}
  \AttributeTok{xlab =} \StringTok{"Length"}\NormalTok{,}
  \AttributeTok{ylab =} \StringTok{"Height"}\NormalTok{,}
  \AttributeTok{zlab =} \StringTok{"Strength"}\NormalTok{,}
  \AttributeTok{angle =} \DecValTok{60}
\NormalTok{))}

\NormalTok{fit }\OtherTok{\textless{}{-}} \FunctionTok{lm}\NormalTok{(strength }\SpecialCharTok{\textasciitilde{}}\NormalTok{ length }\SpecialCharTok{+}\NormalTok{ height, }\AttributeTok{data =}\NormalTok{ bond.data)}
\NormalTok{s3d}\SpecialCharTok{$}\FunctionTok{plane3d}\NormalTok{(fit, }\AttributeTok{lty.box =} \StringTok{"solid"}\NormalTok{)}
\end{Highlighting}
\end{Shaded}

\includegraphics{unit3-mlr/mlr_files/figure-pdf/unnamed-chunk-5-1.pdf}

\subsection{Model Fitting and Summary}\label{model-fitting-and-summary}

We fit a multiple linear regression model with \texttt{strength} as the
response variable and \texttt{length} and \texttt{height} as predictors.

\begin{Shaded}
\begin{Highlighting}[]
\NormalTok{fit }\OtherTok{\textless{}{-}} \FunctionTok{lm}\NormalTok{(strength }\SpecialCharTok{\textasciitilde{}}\NormalTok{ length }\SpecialCharTok{+}\NormalTok{ height, }\AttributeTok{data =}\NormalTok{ bond.data)}
\FunctionTok{summary}\NormalTok{(fit)}
\end{Highlighting}
\end{Shaded}

\begin{verbatim}

Call:
lm(formula = strength ~ length + height, data = bond.data)

Residuals:
   Min     1Q Median     3Q    Max 
-3.865 -1.542 -0.362  1.196  5.841 

Coefficients:
            Estimate Std. Error t value Pr(>|t|)    
(Intercept) 2.263791   1.060066   2.136 0.044099 *  
length      2.744270   0.093524  29.343  < 2e-16 ***
height      0.012528   0.002798   4.477 0.000188 ***
---
Signif. codes:  0 '***' 0.001 '**' 0.01 '*' 0.05 '.' 0.1 ' ' 1

Residual standard error: 2.288 on 22 degrees of freedom
Multiple R-squared:  0.9811,    Adjusted R-squared:  0.9794 
F-statistic: 572.2 on 2 and 22 DF,  p-value: < 2.2e-16
\end{verbatim}

The summary provides the ANOVA F-test for overall significance, \(R^2\),
adjusted \(R^2\), and t-tests for individual coefficients.

\subsection{Confidence Intervals and Model
Components}\label{confidence-intervals-and-model-components}

\begin{Shaded}
\begin{Highlighting}[]
\DocumentationTok{\#\# Confidence intervals}
\FunctionTok{confint}\NormalTok{(fit)}
\end{Highlighting}
\end{Shaded}

\begin{verbatim}
                  2.5 %     97.5 %
(Intercept) 0.065348613 4.46223426
length      2.550313061 2.93822623
height      0.006724246 0.01833138
\end{verbatim}

\begin{Shaded}
\begin{Highlighting}[]
\DocumentationTok{\#\# Fitted values and residuals}
\NormalTok{pred }\OtherTok{\textless{}{-}} \FunctionTok{fitted.values}\NormalTok{(fit)}
\NormalTok{e }\OtherTok{\textless{}{-}} \FunctionTok{resid}\NormalTok{(fit)}
\FunctionTok{data.frame}\NormalTok{(}\AttributeTok{y =}\NormalTok{ bond.data}\SpecialCharTok{$}\NormalTok{strength, }\AttributeTok{y.hat =}\NormalTok{ pred, }\AttributeTok{e =}\NormalTok{ e)}
\end{Highlighting}
\end{Shaded}

\begin{verbatim}
       y     y.hat           e
1   9.95  8.378721  1.57127871
2  24.45 25.596008 -1.14600783
3  31.75 33.954095 -2.20409488
4  35.00 36.596784 -1.59678413
5  25.02 27.913653 -2.89365294
6  16.86 15.746432  1.11356772
7  14.38 12.450260  1.92974001
8   9.60  8.403777  1.19622309
9  24.35 28.214999 -3.86499936
10 27.50 27.976292 -0.47629200
11 17.08 18.402328 -1.32232830
12 37.00 37.461882 -0.46188206
13 41.95 41.458933  0.49106715
14 11.66 12.262343 -0.60234282
15 21.65 15.809071  5.84092866
16 17.89 18.251995 -0.36199456
17 69.00 64.665871  4.33412887
18 10.30 12.336831 -2.03683074
19 34.93 36.471506 -1.54150602
20 46.59 46.559789  0.03021107
21 44.88 47.060901 -2.18090138
22 54.12 52.561290  1.55871047
23 56.63 56.307784  0.32221591
24 22.13 19.982190  2.14780957
25 21.15 20.996264  0.15373580
\end{verbatim}

\begin{Shaded}
\begin{Highlighting}[]
\DocumentationTok{\#\# Covariance matrix and standard errors}
\NormalTok{cov.mat }\OtherTok{\textless{}{-}} \FunctionTok{vcov}\NormalTok{(fit)}
\NormalTok{cov.mat}
\end{Highlighting}
\end{Shaded}

\begin{verbatim}
             (Intercept)        length        height
(Intercept)  1.123740429 -3.921612e-02 -1.781991e-03
length      -0.039216122  8.746709e-03 -9.903775e-05
height      -0.001781991 -9.903775e-05  7.831149e-06
\end{verbatim}

\begin{Shaded}
\begin{Highlighting}[]
\FunctionTok{data.frame}\NormalTok{(}\AttributeTok{std.error =} \FunctionTok{sqrt}\NormalTok{(}\FunctionTok{diag}\NormalTok{(cov.mat)))}
\end{Highlighting}
\end{Shaded}

\begin{verbatim}
              std.error
(Intercept) 1.060066238
length      0.093523844
height      0.002798419
\end{verbatim}

\section{\texorpdfstring{RSS-based Inference: F-test, and adjusted
\(R^2\)}{RSS-based Inference: F-test, and adjusted R\^{}2}}\label{rss-based-inference-f-test-and-adjusted-r2}

\textbf{The General Linear Model}

The general linear model is:

\[y = X\beta + \epsilon\]

\begin{itemize}
\tightlist
\item
  \(y\): \(n \times 1\) vector of responses
\item
  \(X\): \(n \times p\) design matrix (first column often ones)
\item
  \(\beta\): \(p \times 1\) parameter vector, where \(p=k+1\)
\item
  \(\epsilon\): \(n \times 1\) error vector
\end{itemize}

\subsection{RSS-Based Quantities}\label{rss-based-quantities}

\subsubsection{RSS-Based Quantities}\label{rss-based-quantities-1}

\begin{longtable}[]{@{}
  >{\raggedright\arraybackslash}p{(\columnwidth - 16\tabcolsep) * \real{0.1111}}
  >{\raggedright\arraybackslash}p{(\columnwidth - 16\tabcolsep) * \real{0.1111}}
  >{\raggedright\arraybackslash}p{(\columnwidth - 16\tabcolsep) * \real{0.1111}}
  >{\centering\arraybackslash}p{(\columnwidth - 16\tabcolsep) * \real{0.1111}}
  >{\raggedright\arraybackslash}p{(\columnwidth - 16\tabcolsep) * \real{0.1111}}
  >{\raggedright\arraybackslash}p{(\columnwidth - 16\tabcolsep) * \real{0.1111}}
  >{\raggedright\arraybackslash}p{(\columnwidth - 16\tabcolsep) * \real{0.1111}}
  >{\raggedright\arraybackslash}p{(\columnwidth - 16\tabcolsep) * \real{0.1111}}
  >{\raggedright\arraybackslash}p{(\columnwidth - 16\tabcolsep) * \real{0.1111}}@{}}
\toprule\noalign{}
\begin{minipage}[b]{\linewidth}\raggedright
Source
\end{minipage} & \begin{minipage}[b]{\linewidth}\raggedright
Sum of Squares
\end{minipage} & \begin{minipage}[b]{\linewidth}\raggedright
\(R^2\)
\end{minipage} & \begin{minipage}[b]{\linewidth}\centering
df
\end{minipage} & \begin{minipage}[b]{\linewidth}\raggedright
Mean Squares
\end{minipage} & \begin{minipage}[b]{\linewidth}\raggedright
\(F\)
\end{minipage} & \begin{minipage}[b]{\linewidth}\raggedright
SS\(_\mathrm{adj}\)
\end{minipage} & \begin{minipage}[b]{\linewidth}\raggedright
\(\hat{\sigma}^2\)
\end{minipage} & \begin{minipage}[b]{\linewidth}\raggedright
\(R^2_{\mathrm{adj}}\)
\end{minipage} \\
\midrule\noalign{}
\endhead
\bottomrule\noalign{}
\endlastfoot
\(x^\top\beta\) &
\(\mathrm{SSR} = \displaystyle \sum_{i=1}^n (\hat y_i - \bar y)^2\) &
\(\displaystyle \frac{\mathrm{SSR}}{\mathrm{SST}}\) & \(k\) &
\(\displaystyle \mathrm{MSR} = \frac{\mathrm{SSR}}{k}\) &
\(\displaystyle \frac{\mathrm{MSR}}{\mathrm{MSE}}\) &
\(\mathrm{SSR}_{\mathrm{adj}}\) &
\(\displaystyle \hat{\sigma}^2_{x^\top\beta} = \frac{\mathrm{SSR}_{\mathrm{adj}}}{n-1}\)
&
\(\displaystyle \frac{\mathrm{SSR}_{\mathrm{adj}}}{\mathrm{SST}} = 1 - \frac{\mathrm{MSE}}{\mathrm{MST}}\) \\
\(\epsilon\) &
\(\mathrm{SSE} = \displaystyle \sum_{i=1}^n (y_i - \hat y_i)^2\) & --- &
\(n-p\) & \(\displaystyle \mathrm{MSE} = \frac{\mathrm{SSE}}{n-p}\) &
--- & \(\mathrm{SSE}\) &
\(\displaystyle \hat{\sigma}^2_{\epsilon} = \mathrm{MSE}\) & --- \\
\(y\) & \(\mathrm{SST} = \displaystyle \sum_{i=1}^n (y_i - \bar y)^2\) &
--- & \(n-1\) &
\(\displaystyle \mathrm{MST} = \frac{\mathrm{SST}}{n-1}\) & --- &
\(\mathrm{SST}\) & \(\displaystyle \hat{\sigma}^2_{y} = \mathrm{MST}\) &
--- \\
\end{longtable}

\begin{center}\rule{0.5\linewidth}{0.5pt}\end{center}

\textbf{Interpretation of the} \(\hat{\sigma}^2\) Column

The \(\hat{\sigma}^2\) column highlights how each sum of squares
corresponds to an estimated variance.\\
This view makes the adjusted coefficient of determination clear:

\[
R^2_{\mathrm{adj}} 
= 1 - \frac{\hat{\sigma}^2_\epsilon}{\hat{\sigma}^2_y}
= \frac{\hat{\sigma}^2_{x^\top\beta}}{\hat{\sigma}^2_y}.
\]

Hence, the adjusted \(R^2\) simply expresses the \textbf{proportion of
total estimated variance} attributable to the fitted model \(X\beta\)
rather than the residual noise \(\epsilon\).

\subsection{Remarks}\label{remarks}

\subsubsection{Fundamental Identities}\label{fundamental-identities}

\[
\begin{aligned}
\mathrm{SST} &= \mathrm{SSR} + \mathrm{SSE}, \\
\mathrm{MST} &= \mathrm{MSE} + \frac{\mathrm{SSR}_{\mathrm{adj}}}{n-1}.
\end{aligned}
\]

where

\[
\mathrm{SSR}_{\mathrm{adj}} = (n-1)MST-(n-p+k)\mathrm{MSE} = \mathrm{SST}-\mathrm{SSE} - k\,\mathrm{MSE} = \mathrm{SSR} - k\,\mathrm{MSE}.
\]

\begin{center}\rule{0.5\linewidth}{0.5pt}\end{center}

\subsubsection{\texorpdfstring{Difference of \(\hat{\sigma}^2\) and Mean
Squares}{Difference of \textbackslash hat\{\textbackslash sigma\}\^{}2 and Mean Squares}}\label{difference-of-hatsigma2-and-mean-squares}

The quantity \(\hat{\sigma}^2\) represents the \textbf{estimated
variance} associated with each component of the model. MSE and MST are
the estimated variances of the \(\epsilon\) and \(y\) itself. However,
the MSR, although called \textbf{Mean Square for Regression (MSR)} is
\emph{NOT} an estimate of the variance or sample variance of
\(x^\top \beta\). The name of ``mean'' here is used to indicate a
different thing. Its name ``Mean Square'' reflects that it is also an
estimate estimate of noise variance \(\sigma^2\) under
\(H_0\!:\,\beta = 0\):

\[
E[\mathrm{MSR} \mid H_0] = \sigma^2,
\qquad 
E[\mathrm{MSR} \mid H_1] > \sigma^2.
\]

Hence the F-statistic

\[
F = \frac{\mathrm{MSR}}{\mathrm{MSE}}
\] is approximately equal to 1 subject to the variability as
characterized with F-distribution with degree freedoms of \(k\) and
\(n-p\). This test is to test whether any regression coefficients are
not equal to 0.

\begin{center}\rule{0.5\linewidth}{0.5pt}\end{center}

\subsubsection{\texorpdfstring{\(\hat \sigma^2_{x^\top\beta}=\frac{\mathrm{SSR}_{\text{adj}}}{n-1}\)}{\textbackslash hat \textbackslash sigma\^{}2\_\{x\^{}\textbackslash top\textbackslash beta\}=\textbackslash frac\{\textbackslash mathrm\{SSR\}\_\{\textbackslash text\{adj\}\}\}\{n-1\}}}\label{hat-sigma2_xtopbetafracmathrmssr_textadjn-1}

\(\hat \sigma^2_{x^\top\beta}\) is an unbiased estimator of the variance
of linear signal when \(x\) is a regarded as a random variable. This can
be seen from the following equations: \[
E[\mathrm{SSR}] = k\,\sigma^2 + \beta^\top X^\top (I - J/n)\,X\,\beta,
\qquad
E[\mathrm{MSE}] = \sigma^2.
\] Hence, \[
\begin{aligned}
E[\mathrm{SSR}_{\mathrm{adj}}]
&= E[\mathrm{SSR}] - k\,E[\mathrm{MSE}] \\
&= \beta^\top X^\top (I - J/n)\,X\,\beta \\
&= \sum_{i=1}^n (\mu_i - \bar\mu)^2,
\end{aligned}
\]

where \[
\begin{aligned}
\mu_i &= x_i^\top \beta \\
\bar\mu &= \tfrac{1}{n}\sum_{i=1}^n \mu_i
\end{aligned}
\] For fixed \(X\), \(\mathrm{SSR}_{\text{adj}}/(n-1)\) equals the
\textbf{sample variance} of the true means \(\{\mu_i\}\) over the
observed design points. If the rows of \(X\) are independently sampled
with covariance matrix \(\Sigma_X\) (the random-\(X\) model), then

\[
\mathbb{E}_X\!\left[\frac{\mathrm{SSR}_{\text{adj}}}{n-1}\right]
= \beta^\top \Sigma_X \beta
= \mathrm{Var}(x^\top \beta),
\]

\subsubsection{Connection to Rao-Blackwell
Formula}\label{connection-to-rao-blackwell-formula}

The decomposition of \(\hat{\sigma}^2\) is consistent with the
\textbf{Rao--Blackwell formula} for total variance:

\[
\mathrm{Var}(y) = \mathrm{Var}\!\big(E[y \mid x]\big) + E\!\big(\mathrm{Var}[y \mid x]\big).
\]

Here,

\begin{itemize}
\tightlist
\item
  \(\mathrm{Var}\!\big(E[y \mid x]\big)\) corresponds to the
  \textbf{explained variation} due to the regression component
  \(x^\top\beta\), and\\
\item
  \(E\!\big(\mathrm{Var}[y \mid x]\big)\) corresponds to the
  \textbf{residual variation} due to \(\epsilon\).
\end{itemize}

\subsection{A Simulation Study to Understand the Distributions of
RSS}\label{a-simulation-study-to-understand-the-distributions-of-rss}

\textbf{Data Generating Model}

For \(n=30\) and \(p_{max}=20\), simulate with either
\(H_0:\beta=\mathbf 0\) or \(H_1\) where only \(\beta_1\neq 0\);
\(\epsilon_i\sim N(0,1)\).

\textbf{Sequence of Fitted Models}

\begin{longtable}[]{@{}
  >{\raggedright\arraybackslash}p{(\columnwidth - 6\tabcolsep) * \real{0.2297}}
  >{\centering\arraybackslash}p{(\columnwidth - 6\tabcolsep) * \real{0.2432}}
  >{\centering\arraybackslash}p{(\columnwidth - 6\tabcolsep) * \real{0.2432}}
  >{\raggedright\arraybackslash}p{(\columnwidth - 6\tabcolsep) * \real{0.2838}}@{}}
\toprule\noalign{}
\begin{minipage}[b]{\linewidth}\raggedright
Model Name
\end{minipage} & \begin{minipage}[b]{\linewidth}\centering
\# of Predictors (k)
\end{minipage} & \begin{minipage}[b]{\linewidth}\centering
\# of Parameters (p)
\end{minipage} & \begin{minipage}[b]{\linewidth}\raggedright
R Formula
\end{minipage} \\
\midrule\noalign{}
\endhead
\bottomrule\noalign{}
\endlastfoot
Model 0 & 0 & 1 & \texttt{y\ \textasciitilde{}\ 1} \\
Model 1 & 2 & 3 & \texttt{y\ \textasciitilde{}\ x\_1\ +\ x\_2} \\
\ldots{} & \ldots{} & \ldots{} & \ldots{} \\
Final Model & 20 & 21 &
\texttt{y\ \textasciitilde{}\ x\_1\ +\ ...\ +\ x\_20} \\
\end{longtable}

\subsubsection{\texorpdfstring{When \(H_0\) is
true}{When H\_0 is true}}\label{when-h_0-is-true}

\begin{figure}[H]

{\centering \includegraphics[width=8in,height=\textheight]{unit3-mlr/figs/rss-h0.png}

}

\caption{When \(H_0\) is true: MP4 animation (HTML) or a representative
static frame (PDF).}

\end{figure}%

\subsubsection{\texorpdfstring{When \(H_1\) is
true}{When H\_1 is true}}\label{when-h_1-is-true}

\begin{figure}[H]

{\centering \includegraphics[width=8in,height=\textheight]{unit3-mlr/figs/rss-h1.png}

}

\caption{When \(H_1\) is true: MP4 animation (HTML) or a representative
static frame (PDF).}

\end{figure}%

\subsection{Example: Modelling Children Weight with Height and
Age}\label{example-modelling-children-weight-with-height-and-age}

\begin{Shaded}
\begin{Highlighting}[]
\DocumentationTok{\#\# Data: Weight, height and age of children}
\NormalTok{wgt }\OtherTok{\textless{}{-}} \FunctionTok{c}\NormalTok{(}\DecValTok{64}\NormalTok{, }\DecValTok{71}\NormalTok{, }\DecValTok{53}\NormalTok{, }\DecValTok{67}\NormalTok{, }\DecValTok{55}\NormalTok{, }\DecValTok{58}\NormalTok{, }\DecValTok{77}\NormalTok{, }\DecValTok{57}\NormalTok{, }\DecValTok{56}\NormalTok{, }\DecValTok{51}\NormalTok{, }\DecValTok{76}\NormalTok{, }\DecValTok{68}\NormalTok{)}
\NormalTok{hgt }\OtherTok{\textless{}{-}} \FunctionTok{c}\NormalTok{(}\DecValTok{57}\NormalTok{, }\DecValTok{59}\NormalTok{, }\DecValTok{49}\NormalTok{, }\DecValTok{62}\NormalTok{, }\DecValTok{51}\NormalTok{, }\DecValTok{50}\NormalTok{, }\DecValTok{55}\NormalTok{, }\DecValTok{48}\NormalTok{, }\DecValTok{42}\NormalTok{, }\DecValTok{42}\NormalTok{, }\DecValTok{61}\NormalTok{, }\DecValTok{57}\NormalTok{)}
\NormalTok{age }\OtherTok{\textless{}{-}} \FunctionTok{c}\NormalTok{(}\DecValTok{8}\NormalTok{, }\DecValTok{10}\NormalTok{, }\DecValTok{6}\NormalTok{, }\DecValTok{11}\NormalTok{, }\DecValTok{8}\NormalTok{, }\DecValTok{7}\NormalTok{, }\DecValTok{10}\NormalTok{, }\DecValTok{9}\NormalTok{, }\DecValTok{10}\NormalTok{, }\DecValTok{6}\NormalTok{, }\DecValTok{12}\NormalTok{, }\DecValTok{9}\NormalTok{)}
\NormalTok{child.data }\OtherTok{\textless{}{-}} \FunctionTok{data.frame}\NormalTok{(wgt, hgt, age)}
\end{Highlighting}
\end{Shaded}

\subsubsection{Problem 1: Height then
Age}\label{problem-1-height-then-age}

\begin{Shaded}
\begin{Highlighting}[]
\NormalTok{fit\_hgt\_age }\OtherTok{\textless{}{-}} \FunctionTok{lm}\NormalTok{(wgt }\SpecialCharTok{\textasciitilde{}}\NormalTok{ hgt }\SpecialCharTok{+}\NormalTok{ age, }\AttributeTok{data =}\NormalTok{ child.data)}
\FunctionTok{summary}\NormalTok{(fit\_hgt\_age)}
\end{Highlighting}
\end{Shaded}

\begin{verbatim}

Call:
lm(formula = wgt ~ hgt + age, data = child.data)

Residuals:
    Min      1Q  Median      3Q     Max 
-6.8708 -1.7004  0.3454  1.4642 10.2336 

Coefficients:
            Estimate Std. Error t value Pr(>|t|)  
(Intercept)   6.5530    10.9448   0.599   0.5641  
hgt           0.7220     0.2608   2.768   0.0218 *
age           2.0501     0.9372   2.187   0.0565 .
---
Signif. codes:  0 '***' 0.001 '**' 0.01 '*' 0.05 '.' 0.1 ' ' 1

Residual standard error: 4.66 on 9 degrees of freedom
Multiple R-squared:   0.78, Adjusted R-squared:  0.7311 
F-statistic: 15.95 on 2 and 9 DF,  p-value: 0.001099
\end{verbatim}

\begin{Shaded}
\begin{Highlighting}[]
\NormalTok{fit\_hgt }\OtherTok{\textless{}{-}} \FunctionTok{lm}\NormalTok{(wgt }\SpecialCharTok{\textasciitilde{}}\NormalTok{ hgt, }\AttributeTok{data =}\NormalTok{ child.data)}
\FunctionTok{summary}\NormalTok{(fit\_hgt)}
\end{Highlighting}
\end{Shaded}

\begin{verbatim}

Call:
lm(formula = wgt ~ hgt, data = child.data)

Residuals:
    Min      1Q  Median      3Q     Max 
-5.8736 -3.8973 -0.4402  2.2624 11.8375 

Coefficients:
            Estimate Std. Error t value Pr(>|t|)   
(Intercept)   6.1898    12.8487   0.482  0.64035   
hgt           1.0722     0.2417   4.436  0.00126 **
---
Signif. codes:  0 '***' 0.001 '**' 0.01 '*' 0.05 '.' 0.1 ' ' 1

Residual standard error: 5.471 on 10 degrees of freedom
Multiple R-squared:  0.663, Adjusted R-squared:  0.6293 
F-statistic: 19.67 on 1 and 10 DF,  p-value: 0.001263
\end{verbatim}

\begin{Shaded}
\begin{Highlighting}[]
\FunctionTok{anova}\NormalTok{(fit\_hgt, fit\_hgt\_age)}
\end{Highlighting}
\end{Shaded}

\begin{verbatim}
Analysis of Variance Table

Model 1: wgt ~ hgt
Model 2: wgt ~ hgt + age
  Res.Df    RSS Df Sum of Sq      F  Pr(>F)  
1     10 299.33                              
2      9 195.43  1     103.9 4.7849 0.05649 .
---
Signif. codes:  0 '***' 0.001 '**' 0.01 '*' 0.05 '.' 0.1 ' ' 1
\end{verbatim}

\begin{Shaded}
\begin{Highlighting}[]
\FunctionTok{anova}\NormalTok{(fit\_hgt\_age)}
\end{Highlighting}
\end{Shaded}

\begin{verbatim}
Analysis of Variance Table

Response: wgt
          Df Sum Sq Mean Sq F value    Pr(>F)    
hgt        1 588.92  588.92 27.1216 0.0005582 ***
age        1 103.90  103.90  4.7849 0.0564853 .  
Residuals  9 195.43   21.71                      
---
Signif. codes:  0 '***' 0.001 '**' 0.01 '*' 0.05 '.' 0.1 ' ' 1
\end{verbatim}

\subsubsection{Problem 2: Age then
Height}\label{problem-2-age-then-height}

\begin{Shaded}
\begin{Highlighting}[]
\NormalTok{fit\_age }\OtherTok{\textless{}{-}} \FunctionTok{lm}\NormalTok{(wgt }\SpecialCharTok{\textasciitilde{}}\NormalTok{ age, }\AttributeTok{data =}\NormalTok{ child.data)}
\FunctionTok{summary}\NormalTok{(fit\_age)}
\end{Highlighting}
\end{Shaded}

\begin{verbatim}

Call:
lm(formula = wgt ~ age, data = child.data)

Residuals:
    Min      1Q  Median      3Q     Max 
-11.000  -3.911   1.143   4.071  10.000 

Coefficients:
            Estimate Std. Error t value Pr(>|t|)   
(Intercept)  30.5714     8.6137   3.549  0.00528 **
age           3.6429     0.9551   3.814  0.00341 **
---
Signif. codes:  0 '***' 0.001 '**' 0.01 '*' 0.05 '.' 0.1 ' ' 1

Residual standard error: 6.015 on 10 degrees of freedom
Multiple R-squared:  0.5926,    Adjusted R-squared:  0.5519 
F-statistic: 14.55 on 1 and 10 DF,  p-value: 0.003407
\end{verbatim}

\begin{Shaded}
\begin{Highlighting}[]
\NormalTok{fit\_age\_hgt }\OtherTok{\textless{}{-}} \FunctionTok{lm}\NormalTok{(wgt }\SpecialCharTok{\textasciitilde{}}\NormalTok{ age }\SpecialCharTok{+}\NormalTok{ hgt, }\AttributeTok{data =}\NormalTok{ child.data)}
\FunctionTok{summary}\NormalTok{(fit\_age\_hgt)}
\end{Highlighting}
\end{Shaded}

\begin{verbatim}

Call:
lm(formula = wgt ~ age + hgt, data = child.data)

Residuals:
    Min      1Q  Median      3Q     Max 
-6.8708 -1.7004  0.3454  1.4642 10.2336 

Coefficients:
            Estimate Std. Error t value Pr(>|t|)  
(Intercept)   6.5530    10.9448   0.599   0.5641  
age           2.0501     0.9372   2.187   0.0565 .
hgt           0.7220     0.2608   2.768   0.0218 *
---
Signif. codes:  0 '***' 0.001 '**' 0.01 '*' 0.05 '.' 0.1 ' ' 1

Residual standard error: 4.66 on 9 degrees of freedom
Multiple R-squared:   0.78, Adjusted R-squared:  0.7311 
F-statistic: 15.95 on 2 and 9 DF,  p-value: 0.001099
\end{verbatim}

\begin{Shaded}
\begin{Highlighting}[]
\FunctionTok{anova}\NormalTok{(fit\_age, fit\_age\_hgt)}
\end{Highlighting}
\end{Shaded}

\begin{verbatim}
Analysis of Variance Table

Model 1: wgt ~ age
Model 2: wgt ~ age + hgt
  Res.Df    RSS Df Sum of Sq      F  Pr(>F)  
1     10 361.86                              
2      9 195.43  1    166.43 7.6646 0.02181 *
---
Signif. codes:  0 '***' 0.001 '**' 0.01 '*' 0.05 '.' 0.1 ' ' 1
\end{verbatim}

\begin{Shaded}
\begin{Highlighting}[]
\FunctionTok{anova}\NormalTok{(fit\_age\_hgt)}
\end{Highlighting}
\end{Shaded}

\begin{verbatim}
Analysis of Variance Table

Response: wgt
          Df Sum Sq Mean Sq F value    Pr(>F)    
age        1 526.39  526.39 24.2419 0.0008205 ***
hgt        1 166.43  166.43  7.6646 0.0218070 *  
Residuals  9 195.43   21.71                      
---
Signif. codes:  0 '***' 0.001 '**' 0.01 '*' 0.05 '.' 0.1 ' ' 1
\end{verbatim}

\subsection{Example: Wire bond
strength}\label{example-wire-bond-strength}

\begin{Shaded}
\begin{Highlighting}[]
\NormalTok{fit\_len\_hgt }\OtherTok{\textless{}{-}}  \FunctionTok{lm}\NormalTok{(strength }\SpecialCharTok{\textasciitilde{}}\NormalTok{ length }\SpecialCharTok{+}\NormalTok{ height, }\AttributeTok{data =}\NormalTok{ bond.data)}
\NormalTok{fit\_hgt\_len }\OtherTok{\textless{}{-}}  \FunctionTok{lm}\NormalTok{(strength }\SpecialCharTok{\textasciitilde{}}\NormalTok{ height}\SpecialCharTok{+}\NormalTok{length, }\AttributeTok{data =}\NormalTok{ bond.data)}
\FunctionTok{anova}\NormalTok{(fit\_len\_hgt)}
\end{Highlighting}
\end{Shaded}

\begin{verbatim}
Analysis of Variance Table

Response: strength
          Df Sum Sq Mean Sq  F value    Pr(>F)    
length     1 5885.9  5885.9 1124.293 < 2.2e-16 ***
height     1  104.9   104.9   20.041 0.0001883 ***
Residuals 22  115.2     5.2                       
---
Signif. codes:  0 '***' 0.001 '**' 0.01 '*' 0.05 '.' 0.1 ' ' 1
\end{verbatim}

\begin{Shaded}
\begin{Highlighting}[]
\FunctionTok{anova}\NormalTok{(fit\_hgt\_len)}
\end{Highlighting}
\end{Shaded}

\begin{verbatim}
Analysis of Variance Table

Response: strength
          Df Sum Sq Mean Sq F value    Pr(>F)    
height     1 1483.2  1483.2  283.32 4.731e-14 ***
length     1 4507.5  4507.5  861.01 < 2.2e-16 ***
Residuals 22  115.2     5.2                      
---
Signif. codes:  0 '***' 0.001 '**' 0.01 '*' 0.05 '.' 0.1 ' ' 1
\end{verbatim}

\begin{Shaded}
\begin{Highlighting}[]
\FunctionTok{summary}\NormalTok{(fit\_hgt\_len)}
\end{Highlighting}
\end{Shaded}

\begin{verbatim}

Call:
lm(formula = strength ~ height + length, data = bond.data)

Residuals:
   Min     1Q Median     3Q    Max 
-3.865 -1.542 -0.362  1.196  5.841 

Coefficients:
            Estimate Std. Error t value Pr(>|t|)    
(Intercept) 2.263791   1.060066   2.136 0.044099 *  
height      0.012528   0.002798   4.477 0.000188 ***
length      2.744270   0.093524  29.343  < 2e-16 ***
---
Signif. codes:  0 '***' 0.001 '**' 0.01 '*' 0.05 '.' 0.1 ' ' 1

Residual standard error: 2.288 on 22 degrees of freedom
Multiple R-squared:  0.9811,    Adjusted R-squared:  0.9794 
F-statistic: 572.2 on 2 and 22 DF,  p-value: < 2.2e-16
\end{verbatim}

\begin{Shaded}
\begin{Highlighting}[]
\FunctionTok{summary}\NormalTok{(fit\_len\_hgt)}
\end{Highlighting}
\end{Shaded}

\begin{verbatim}

Call:
lm(formula = strength ~ length + height, data = bond.data)

Residuals:
   Min     1Q Median     3Q    Max 
-3.865 -1.542 -0.362  1.196  5.841 

Coefficients:
            Estimate Std. Error t value Pr(>|t|)    
(Intercept) 2.263791   1.060066   2.136 0.044099 *  
length      2.744270   0.093524  29.343  < 2e-16 ***
height      0.012528   0.002798   4.477 0.000188 ***
---
Signif. codes:  0 '***' 0.001 '**' 0.01 '*' 0.05 '.' 0.1 ' ' 1

Residual standard error: 2.288 on 22 degrees of freedom
Multiple R-squared:  0.9811,    Adjusted R-squared:  0.9794 
F-statistic: 572.2 on 2 and 22 DF,  p-value: < 2.2e-16
\end{verbatim}

\subsection{Relationship between t-test and partial
F-test}\label{relationship-between-t-test-and-partial-f-test}

\begin{itemize}
\tightlist
\item
  A t-test for a single coefficient is a special case of the partial
  F-test; the relationship is \(F = t^2\) for 1 df in the numerator.
\item
  The p-value from t-test (output of \emph{summary(lm())}) is the same
  as anova test for: \(H_0: \beta_j = 0\) vs \(H_1\): all covaraites
  have non-zero effects.
\end{itemize}

\section{Predictions for Mean Response and a Future
Observation}\label{predictions-for-mean-response-and-a-future-observation}

\subsection{Confidence Interval for Mean
Response}\label{confidence-interval-for-mean-response}

\begin{Shaded}
\begin{Highlighting}[]
\FunctionTok{predict}\NormalTok{(fit, }\AttributeTok{newdata =} \FunctionTok{data.frame}\NormalTok{(}\AttributeTok{length =} \DecValTok{8}\NormalTok{, }\AttributeTok{height =} \DecValTok{275}\NormalTok{),}
        \AttributeTok{interval =} \StringTok{"confidence"}\NormalTok{, }\AttributeTok{level =} \FloatTok{0.95}\NormalTok{)}
\end{Highlighting}
\end{Shaded}

\begin{verbatim}
      fit      lwr      upr
1 27.6631 26.66324 28.66296
\end{verbatim}

\subsection{Prediction Interval for a New
Observation}\label{prediction-interval-for-a-new-observation}

\begin{Shaded}
\begin{Highlighting}[]
\FunctionTok{predict}\NormalTok{(fit, }\AttributeTok{newdata =} \FunctionTok{data.frame}\NormalTok{(}\AttributeTok{length =} \DecValTok{8}\NormalTok{, }\AttributeTok{height =} \DecValTok{275}\NormalTok{),}
        \AttributeTok{interval =} \StringTok{"prediction"}\NormalTok{, }\AttributeTok{level =} \FloatTok{0.95}\NormalTok{)}
\end{Highlighting}
\end{Shaded}

\begin{verbatim}
      fit      lwr      upr
1 27.6631 22.81378 32.51241
\end{verbatim}

\section{Model Diagnostics}\label{model-diagnostics}

\subsection{Residual Calculations}\label{residual-calculations}

\begin{Shaded}
\begin{Highlighting}[]
\NormalTok{residuals\_df }\OtherTok{\textless{}{-}} \FunctionTok{data.frame}\NormalTok{(}
  \AttributeTok{hat\_values =} \FunctionTok{hatvalues}\NormalTok{(fit),}
  \AttributeTok{ordinary\_resid =} \FunctionTok{resid}\NormalTok{(fit),}
  \AttributeTok{standardized\_resid =} \FunctionTok{resid}\NormalTok{(fit) }\SpecialCharTok{/} \FunctionTok{sigma}\NormalTok{(fit),}
  \AttributeTok{studentized\_internal =} \FunctionTok{rstandard}\NormalTok{(fit),}
  \AttributeTok{studentized\_external =} \FunctionTok{rstudent}\NormalTok{(fit)}
\NormalTok{)}
\NormalTok{residuals\_df}
\end{Highlighting}
\end{Shaded}

\begin{verbatim}
   hat_values ordinary_resid standardized_resid studentized_internal
1  0.15728923     1.57127871         0.68673363           0.74808172
2  0.11164598    -1.14600783        -0.50086730          -0.53140990
3  0.14191905    -2.20409488        -0.96330846          -1.03992315
4  0.10188923    -1.59678413        -0.69788088          -0.73640435
5  0.04178381    -2.89365294        -1.26468257          -1.29196212
6  0.07486842     1.11356772         0.48668921           0.50599936
7  0.11806106     1.92974001         0.84340057           0.89807919
8  0.15608149     1.19622309         0.52281407           0.56911105
9  0.12797685    -3.86499936        -1.68921340          -1.80892479
10 0.04131672    -0.47629200        -0.20816532          -0.21260369
11 0.09253979    -1.32232830        -0.57792886          -0.60668127
12 0.05256700    -0.46188206        -0.20186740          -0.20739197
13 0.08202675     0.49106715         0.21462286           0.22400668
14 0.11291577    -0.60234282        -0.26325633          -0.27950939
15 0.07373697     5.84092866         2.55280118           2.65246601
16 0.08794942    -0.36199456        -0.15821117          -0.16566382
17 0.25934228     4.33412887         1.89424832           2.20104100
18 0.29287870    -2.03683074        -0.89020500          -1.05862725
19 0.09617553    -1.54150602        -0.67372136          -0.70866056
20 0.14726101     0.03021107         0.01320387           0.01429859
21 0.12963943    -2.18090138        -0.95317165          -1.02169558
22 0.13580052     1.55871047         0.68124063           0.73281364
23 0.18237610     0.32221591         0.14082575           0.15574183
24 0.10908869     2.14780957         0.93870874           0.99452024
25 0.07287021     0.15373580         0.06719084           0.06978142
   studentized_external
1            0.74035927
2           -0.52255660
3           -1.04194550
4           -0.72850799
5           -1.31305171
6            0.49726770
7            0.89397096
8            0.56016499
9           -1.91552083
10          -0.20792931
11          -0.59775404
12          -0.20282206
13           0.21910643
14          -0.27356920
15           3.14216850
16          -0.16195600
17           2.43521394
18          -1.06168251
19          -0.70040768
20           0.01396991
21          -1.02276424
22           0.72486668
23           0.15224503
24           0.99426154
25           0.06818458
\end{verbatim}

\subsection{Residual Plots}\label{residual-plots}

\begin{Shaded}
\begin{Highlighting}[]
\NormalTok{n }\OtherTok{\textless{}{-}} \FunctionTok{nrow}\NormalTok{(bond.data)}
\NormalTok{r }\OtherTok{\textless{}{-}} \FunctionTok{rstudent}\NormalTok{(fit) }
\NormalTok{y.hat }\OtherTok{\textless{}{-}} \FunctionTok{fitted.values}\NormalTok{(fit)}

\FunctionTok{par}\NormalTok{(}\AttributeTok{mfrow =} \FunctionTok{c}\NormalTok{(}\DecValTok{2}\NormalTok{, }\DecValTok{3}\NormalTok{), }\AttributeTok{mar =} \FunctionTok{c}\NormalTok{(}\DecValTok{4}\NormalTok{, }\DecValTok{4}\NormalTok{, }\DecValTok{2}\NormalTok{, }\DecValTok{1}\NormalTok{))}
\FunctionTok{qqnorm}\NormalTok{(r, }\AttributeTok{main =} \StringTok{"Normal Q{-}Q Plot"}\NormalTok{); }\FunctionTok{qqline}\NormalTok{(r)}
\FunctionTok{plot}\NormalTok{(y.hat, r, }\AttributeTok{xlab =} \StringTok{"Fitted values"}\NormalTok{, }\AttributeTok{ylab =} \StringTok{"Studentized Residuals"}\NormalTok{); }\FunctionTok{abline}\NormalTok{(}\AttributeTok{h =} \DecValTok{0}\NormalTok{)}
\FunctionTok{plot}\NormalTok{(}\DecValTok{1}\SpecialCharTok{:}\NormalTok{n, r, }\AttributeTok{xlab =} \StringTok{"Observation Number"}\NormalTok{, }\AttributeTok{ylab =} \StringTok{"Studentized Residuals"}\NormalTok{); }\FunctionTok{abline}\NormalTok{(}\AttributeTok{h =} \DecValTok{0}\NormalTok{)}
\FunctionTok{plot}\NormalTok{(bond.data}\SpecialCharTok{$}\NormalTok{length, r, }\AttributeTok{xlab =} \StringTok{"Wire Length"}\NormalTok{, }\AttributeTok{ylab =} \StringTok{"Studentized Residuals"}\NormalTok{); }\FunctionTok{abline}\NormalTok{(}\AttributeTok{h =} \DecValTok{0}\NormalTok{)}
\FunctionTok{plot}\NormalTok{(bond.data}\SpecialCharTok{$}\NormalTok{height, r, }\AttributeTok{xlab =} \StringTok{"Die Height"}\NormalTok{, }\AttributeTok{ylab =} \StringTok{"Studentized Residuals"}\NormalTok{); }\FunctionTok{abline}\NormalTok{(}\AttributeTok{h =} \DecValTok{0}\NormalTok{)}
\end{Highlighting}
\end{Shaded}

\includegraphics{unit3-mlr/mlr_files/figure-pdf/unnamed-chunk-18-1.pdf}

\section{Influential Observations}\label{influential-observations}

\begin{Shaded}
\begin{Highlighting}[]
\NormalTok{influence\_df }\OtherTok{\textless{}{-}} \FunctionTok{data.frame}\NormalTok{(}\AttributeTok{dffits =} \FunctionTok{dffits}\NormalTok{(fit),}
                           \AttributeTok{cook.D =} \FunctionTok{cooks.distance}\NormalTok{(fit),}
                           \FunctionTok{dfbetas}\NormalTok{(fit))}
\NormalTok{influence\_df}
\end{Highlighting}
\end{Shaded}

\begin{verbatim}
         dffits       cook.D  X.Intercept.       length       height
1   0.319854702 3.481748e-02  0.3179493921 -0.100534181 -0.200085326
2  -0.185251548 1.183028e-02 -0.1403477437 -0.051464370  0.148315370
3  -0.423741713 5.962023e-02 -0.2219151046 -0.237135616  0.339340552
4  -0.245376811 2.050736e-02  0.0787635526  0.022343842 -0.184260891
5  -0.274191565 2.426179e-02 -0.1572410603 -0.009662357  0.055328303
6   0.141461363 6.906752e-03  0.1301249135 -0.058073567 -0.049408295
7   0.327082639 3.598953e-02  0.1479853099 -0.261848970  0.142220906
8   0.240902557 1.996750e-02  0.2394962591 -0.076575387 -0.149787102
9  -0.733818383 1.600749e-01 -0.5011686139 -0.283749099  0.605559181
10 -0.043165927 6.493384e-04 -0.0241138520 -0.001038287  0.007460760
11 -0.190885522 1.251125e-02 -0.0602003847  0.132053762 -0.102733745
12 -0.047774650 7.954763e-04  0.0017554145 -0.016926531 -0.008495572
13  0.065496465 1.494604e-03 -0.0224255162  0.017340091  0.033756253
14 -0.097602753 3.314830e-03 -0.0483552961  0.077880727 -0.038066705
15  0.886553487 1.866936e-01  0.8097463528 -0.374290156 -0.292048214
16 -0.050292599 8.821617e-04 -0.0177647675  0.034746758 -0.025275682
17  1.441003392 5.654455e-01 -0.8513738015  1.008880052  0.413618783
18 -0.683268805 1.547244e-01 -0.0218935465  0.521608456 -0.532432956
19 -0.228476293 1.781295e-02  0.0700729860  0.018004228 -0.167581999
20  0.005805362 1.176892e-05  0.0005613509  0.004752581 -0.003094588
21 -0.394724862 5.182743e-02 -0.0084618169 -0.324109965  0.170622396
22  0.287343813 2.812893e-02 -0.1326208213  0.183002776  0.076391058
23  0.071903545 1.803448e-03 -0.0412553376  0.040164093  0.030365108
24  0.347915085 4.036930e-02  0.3084561584  0.016541769 -0.262089402
25  0.019115733 1.275757e-04  0.0062730782 -0.011614674  0.009459029
\end{verbatim}

\subsection{\texorpdfstring{Plotting with the \texttt{olsrr}
Package}{Plotting with the olsrr Package}}\label{plotting-with-the-olsrr-package}

\begin{Shaded}
\begin{Highlighting}[]
\DocumentationTok{\#\# install.packages("olsrr") \# Run once if needed}
\FunctionTok{library}\NormalTok{(olsrr)}

\FunctionTok{ols\_plot\_cooksd\_chart}\NormalTok{(fit)}
\end{Highlighting}
\end{Shaded}

\includegraphics{unit3-mlr/mlr_files/figure-pdf/unnamed-chunk-20-1.pdf}

\begin{Shaded}
\begin{Highlighting}[]
\FunctionTok{ols\_plot\_dffits}\NormalTok{(fit)}
\end{Highlighting}
\end{Shaded}

\includegraphics{unit3-mlr/mlr_files/figure-pdf/unnamed-chunk-20-2.pdf}

\begin{Shaded}
\begin{Highlighting}[]
\FunctionTok{ols\_plot\_dfbetas}\NormalTok{(fit)}
\end{Highlighting}
\end{Shaded}

\includegraphics{unit3-mlr/mlr_files/figure-pdf/unnamed-chunk-20-3.pdf}

\section{Polynomial Regression}\label{polynomial-regression}

\begin{Shaded}
\begin{Highlighting}[]
\NormalTok{y }\OtherTok{\textless{}{-}} \FunctionTok{c}\NormalTok{(}\FloatTok{1.81}\NormalTok{, }\FloatTok{1.70}\NormalTok{, }\FloatTok{1.65}\NormalTok{, }\FloatTok{1.55}\NormalTok{, }\FloatTok{1.48}\NormalTok{, }\FloatTok{1.40}\NormalTok{, }\FloatTok{1.30}\NormalTok{, }\FloatTok{1.26}\NormalTok{, }\FloatTok{1.24}\NormalTok{, }\FloatTok{1.21}\NormalTok{, }\FloatTok{1.20}\NormalTok{, }\FloatTok{1.18}\NormalTok{)}
\NormalTok{x }\OtherTok{\textless{}{-}} \FunctionTok{c}\NormalTok{(}\DecValTok{20}\NormalTok{, }\DecValTok{25}\NormalTok{, }\DecValTok{30}\NormalTok{, }\DecValTok{35}\NormalTok{, }\DecValTok{40}\NormalTok{, }\DecValTok{50}\NormalTok{, }\DecValTok{60}\NormalTok{, }\DecValTok{65}\NormalTok{, }\DecValTok{70}\NormalTok{, }\DecValTok{75}\NormalTok{, }\DecValTok{80}\NormalTok{, }\DecValTok{90}\NormalTok{)}
\NormalTok{fit\_poly }\OtherTok{\textless{}{-}} \FunctionTok{lm}\NormalTok{(y }\SpecialCharTok{\textasciitilde{}}\NormalTok{ x }\SpecialCharTok{+} \FunctionTok{I}\NormalTok{(x}\SpecialCharTok{\^{}}\DecValTok{2}\NormalTok{))}
\FunctionTok{summary}\NormalTok{(fit\_poly)}
\end{Highlighting}
\end{Shaded}

\begin{verbatim}

Call:
lm(formula = y ~ x + I(x^2))

Residuals:
       Min         1Q     Median         3Q        Max 
-0.0174763 -0.0065087  0.0001297  0.0071482  0.0151887 

Coefficients:
              Estimate Std. Error t value Pr(>|t|)    
(Intercept)  2.198e+00  2.255e-02   97.48 6.38e-15 ***
x           -2.252e-02  9.424e-04  -23.90 1.88e-09 ***
I(x^2)       1.251e-04  8.658e-06   14.45 1.56e-07 ***
---
Signif. codes:  0 '***' 0.001 '**' 0.01 '*' 0.05 '.' 0.1 ' ' 1

Residual standard error: 0.01219 on 9 degrees of freedom
Multiple R-squared:  0.9975,    Adjusted R-squared:  0.9969 
F-statistic:  1767 on 2 and 9 DF,  p-value: 2.096e-12
\end{verbatim}

\begin{Shaded}
\begin{Highlighting}[]
\FunctionTok{plot}\NormalTok{(x, y, }\AttributeTok{xlab =} \StringTok{"Lot size, x"}\NormalTok{, }\AttributeTok{ylab =} \StringTok{"Average cost per unit, y"}\NormalTok{)}
\FunctionTok{lines}\NormalTok{(x, }\FunctionTok{predict}\NormalTok{(fit\_poly, }\AttributeTok{newdata =} \FunctionTok{data.frame}\NormalTok{(}\AttributeTok{x =}\NormalTok{ x)), }\AttributeTok{type =} \StringTok{"l"}\NormalTok{)}
\end{Highlighting}
\end{Shaded}

\includegraphics{unit3-mlr/mlr_files/figure-pdf/unnamed-chunk-22-1.pdf}

\begin{Shaded}
\begin{Highlighting}[]
\NormalTok{fit1 }\OtherTok{\textless{}{-}} \FunctionTok{lm}\NormalTok{(y }\SpecialCharTok{\textasciitilde{}}\NormalTok{ x)}
\FunctionTok{anova}\NormalTok{(fit1, fit\_poly)}
\end{Highlighting}
\end{Shaded}

\begin{verbatim}
Analysis of Variance Table

Model 1: y ~ x
Model 2: y ~ x + I(x^2)
  Res.Df      RSS Df Sum of Sq      F    Pr(>F)    
1     10 0.032340                                  
2      9 0.001337  1  0.031002 208.67 1.564e-07 ***
---
Signif. codes:  0 '***' 0.001 '**' 0.01 '*' 0.05 '.' 0.1 ' ' 1
\end{verbatim}

\section{Handling Categorical Variables with Dummy
Variables}\label{handling-categorical-variables-with-dummy-variables}

Investigate the common observation that males tend to have higher blood
pressure than females of similar age.

\begin{Shaded}
\begin{Highlighting}[]
\DocumentationTok{\#\# Note: Update this path to your local file location}
\NormalTok{sbpdata }\OtherTok{\textless{}{-}} \FunctionTok{read.csv}\NormalTok{(}\StringTok{"sbpdata.csv"}\NormalTok{)}
\NormalTok{sbpdata}
\end{Highlighting}
\end{Shaded}

\begin{verbatim}
   sex sbp age
1    0 144  39
2    0 138  45
3    0 145  47
4    0 162  65
5    0 142  46
6    0 170  67
7    0 124  42
8    0 158  67
9    0 154  56
10   0 162  64
11   0 150  56
12   0 140  59
13   0 110  34
14   0 128  42
15   0 130  48
16   0 135  45
17   0 114  17
18   0 116  20
19   0 124  19
20   0 136  36
21   0 142  50
22   0 120  39
23   0 120  21
24   0 160  44
25   0 158  53
26   0 144  63
27   0 130  29
28   0 125  25
29   0 175  69
30   1 158  41
31   1 185  60
32   1 152  41
33   1 159  47
34   1 176  66
35   1 156  47
36   1 184  68
37   1 138  43
38   1 172  68
39   1 168  57
40   1 176  65
41   1 164  57
42   1 154  61
43   1 124  36
44   1 142  44
45   1 144  50
46   1 149  47
47   1 128  19
48   1 130  22
49   1 138  21
50   1 150  38
51   1 156  52
52   1 134  41
53   1 134  18
54   1 174  51
55   1 174  55
56   1 158  65
57   1 144  33
58   1 139  23
59   1 180  70
60   1 165  56
61   1 172  62
62   1 160  51
63   1 157  48
64   1 170  59
65   1 153  40
66   1 148  35
67   1 140  33
68   1 132  26
69   1 169  61
\end{verbatim}

\subsection{Four Models Involving
``sex''}\label{four-models-involving-sex}

\subsubsection{Coincidence Model (Age
Only)}\label{coincidence-model-age-only}

\begin{Shaded}
\begin{Highlighting}[]
\DocumentationTok{\#\# Ensure sex is a factor (labels will appear in the legend)}
\NormalTok{sbpdata}\SpecialCharTok{$}\NormalTok{sex }\OtherTok{\textless{}{-}} \FunctionTok{as.factor}\NormalTok{(sbpdata}\SpecialCharTok{$}\NormalTok{sex)}

\DocumentationTok{\#\# Fit (you already have this)}
\NormalTok{fit.age }\OtherTok{\textless{}{-}} \FunctionTok{lm}\NormalTok{(sbp }\SpecialCharTok{\textasciitilde{}}\NormalTok{ age, }\AttributeTok{data =}\NormalTok{ sbpdata)}

\DocumentationTok{\#\# Generate predictions over the observed age range}
\NormalTok{new\_age }\OtherTok{\textless{}{-}} \FunctionTok{seq}\NormalTok{(}\FunctionTok{min}\NormalTok{(sbpdata}\SpecialCharTok{$}\NormalTok{age, }\AttributeTok{na.rm =} \ConstantTok{TRUE}\NormalTok{),}
               \FunctionTok{max}\NormalTok{(sbpdata}\SpecialCharTok{$}\NormalTok{age, }\AttributeTok{na.rm =} \ConstantTok{TRUE}\NormalTok{),}
               \AttributeTok{length.out =} \DecValTok{200}\NormalTok{)}
\NormalTok{pred }\OtherTok{\textless{}{-}} \FunctionTok{predict}\NormalTok{(fit.age, }\AttributeTok{newdata =} \FunctionTok{data.frame}\NormalTok{(}\AttributeTok{age =}\NormalTok{ new\_age))}

\DocumentationTok{\#\# Simple palette for the sex levels (works for 1–3 levels; expand if needed)}
\NormalTok{lev  }\OtherTok{\textless{}{-}} \FunctionTok{levels}\NormalTok{(sbpdata}\SpecialCharTok{$}\NormalTok{sex)}
\NormalTok{cols }\OtherTok{\textless{}{-}} \FunctionTok{setNames}\NormalTok{(}\FunctionTok{c}\NormalTok{(}\StringTok{"steelblue3"}\NormalTok{, }\StringTok{"tomato3"}\NormalTok{, }\StringTok{"darkorchid3"}\NormalTok{)[}\FunctionTok{seq\_along}\NormalTok{(lev)], lev)}

\DocumentationTok{\#\# Scatter plot with colored points by sex}
\FunctionTok{plot}\NormalTok{(sbp }\SpecialCharTok{\textasciitilde{}}\NormalTok{ age, }\AttributeTok{data =}\NormalTok{ sbpdata,}
     \AttributeTok{col =}\NormalTok{ cols[sbpdata}\SpecialCharTok{$}\NormalTok{sex], }\AttributeTok{pch =} \DecValTok{16}\NormalTok{,}
     \AttributeTok{xlab =} \StringTok{"Age"}\NormalTok{, }\AttributeTok{ylab =} \StringTok{"Systolic BP"}\NormalTok{)}

\DocumentationTok{\#\# Add predicted line}
\FunctionTok{lines}\NormalTok{(new\_age, pred, }\AttributeTok{lwd =} \DecValTok{2}\NormalTok{)}

\DocumentationTok{\#\# Legend}
\FunctionTok{legend}\NormalTok{(}\StringTok{"topleft"}\NormalTok{, }\AttributeTok{legend =}\NormalTok{ lev, }\AttributeTok{col =}\NormalTok{ cols[lev], }\AttributeTok{pch =} \DecValTok{16}\NormalTok{, }\AttributeTok{bty =} \StringTok{"n"}\NormalTok{, }\AttributeTok{title =} \StringTok{"Sex"}\NormalTok{)}
\end{Highlighting}
\end{Shaded}

\includegraphics{unit3-mlr/mlr_files/figure-pdf/unnamed-chunk-25-1.pdf}

\begin{Shaded}
\begin{Highlighting}[]
\FunctionTok{data.frame}\NormalTok{(}\FunctionTok{model.matrix}\NormalTok{(fit.age)) }
\end{Highlighting}
\end{Shaded}

\begin{verbatim}
   X.Intercept. age
1             1  39
2             1  45
3             1  47
4             1  65
5             1  46
6             1  67
7             1  42
8             1  67
9             1  56
10            1  64
11            1  56
12            1  59
13            1  34
14            1  42
15            1  48
16            1  45
17            1  17
18            1  20
19            1  19
20            1  36
21            1  50
22            1  39
23            1  21
24            1  44
25            1  53
26            1  63
27            1  29
28            1  25
29            1  69
30            1  41
31            1  60
32            1  41
33            1  47
34            1  66
35            1  47
36            1  68
37            1  43
38            1  68
39            1  57
40            1  65
41            1  57
42            1  61
43            1  36
44            1  44
45            1  50
46            1  47
47            1  19
48            1  22
49            1  21
50            1  38
51            1  52
52            1  41
53            1  18
54            1  51
55            1  55
56            1  65
57            1  33
58            1  23
59            1  70
60            1  56
61            1  62
62            1  51
63            1  48
64            1  59
65            1  40
66            1  35
67            1  33
68            1  26
69            1  61
\end{verbatim}

\begin{Shaded}
\begin{Highlighting}[]
\FunctionTok{print}\NormalTok{(}\FunctionTok{anova}\NormalTok{(fit.age))}
\end{Highlighting}
\end{Shaded}

\begin{verbatim}
Analysis of Variance Table

Response: sbp
          Df  Sum Sq Mean Sq F value    Pr(>F)    
age        1 14951.3 14951.3  121.27 < 2.2e-16 ***
Residuals 67  8260.5   123.3                      
---
Signif. codes:  0 '***' 0.001 '**' 0.01 '*' 0.05 '.' 0.1 ' ' 1
\end{verbatim}

\subsubsection{Additive Effect Model (Age +
Sex)}\label{additive-effect-model-age-sex}

\begin{Shaded}
\begin{Highlighting}[]
\DocumentationTok{\#\# Parallelism: H0: beta3=0 (Sex has additive effect)}
\NormalTok{fit.agePLUSsex }\OtherTok{\textless{}{-}} \FunctionTok{lm}\NormalTok{(sbp }\SpecialCharTok{\textasciitilde{}}\NormalTok{ age }\SpecialCharTok{+}\NormalTok{ sex, }\AttributeTok{data =}\NormalTok{ sbpdata)}

\DocumentationTok{\#\# Ensure sex is a factor for labeling/colors}
\NormalTok{sbpdata}\SpecialCharTok{$}\NormalTok{sex }\OtherTok{\textless{}{-}} \FunctionTok{factor}\NormalTok{(sbpdata}\SpecialCharTok{$}\NormalTok{sex)}

\DocumentationTok{\#\# Fit (additive: parallelism)}
\NormalTok{fit.agePLUSsex }\OtherTok{\textless{}{-}} \FunctionTok{lm}\NormalTok{(sbp }\SpecialCharTok{\textasciitilde{}}\NormalTok{ age }\SpecialCharTok{+}\NormalTok{ sex, }\AttributeTok{data =}\NormalTok{ sbpdata)}

\DocumentationTok{\#\# X{-}range and palette}
\NormalTok{ages }\OtherTok{\textless{}{-}} \FunctionTok{seq}\NormalTok{(}\FunctionTok{min}\NormalTok{(sbpdata}\SpecialCharTok{$}\NormalTok{age, }\AttributeTok{na.rm =} \ConstantTok{TRUE}\NormalTok{),}
            \FunctionTok{max}\NormalTok{(sbpdata}\SpecialCharTok{$}\NormalTok{age, }\AttributeTok{na.rm =} \ConstantTok{TRUE}\NormalTok{),}
            \AttributeTok{length.out =} \DecValTok{200}\NormalTok{)}
\NormalTok{lev  }\OtherTok{\textless{}{-}} \FunctionTok{levels}\NormalTok{(sbpdata}\SpecialCharTok{$}\NormalTok{sex)}
\NormalTok{cols }\OtherTok{\textless{}{-}} \FunctionTok{setNames}\NormalTok{(}\FunctionTok{c}\NormalTok{(}\StringTok{"steelblue3"}\NormalTok{, }\StringTok{"tomato3"}\NormalTok{, }\StringTok{"darkorchid3"}\NormalTok{)[}\FunctionTok{seq\_along}\NormalTok{(lev)], lev)}

\DocumentationTok{\#\# Scatter with colored points by sex}
\FunctionTok{plot}\NormalTok{(sbp }\SpecialCharTok{\textasciitilde{}}\NormalTok{ age, }\AttributeTok{data =}\NormalTok{ sbpdata,}
     \AttributeTok{col =}\NormalTok{ cols[sbpdata}\SpecialCharTok{$}\NormalTok{sex], }\AttributeTok{pch =} \DecValTok{16}\NormalTok{,}
     \AttributeTok{xlab =} \StringTok{"Age"}\NormalTok{, }\AttributeTok{ylab =} \StringTok{"Systolic BP"}\NormalTok{)}

\DocumentationTok{\#\# Parallel fitted lines: one per sex (same slope, different intercepts)}
\ControlFlowTok{for}\NormalTok{ (sx }\ControlFlowTok{in}\NormalTok{ lev) \{}
\NormalTok{  nd }\OtherTok{\textless{}{-}} \FunctionTok{data.frame}\NormalTok{(}\AttributeTok{age =}\NormalTok{ ages, }\AttributeTok{sex =} \FunctionTok{factor}\NormalTok{(sx, }\AttributeTok{levels =}\NormalTok{ lev))}
\NormalTok{  yhat }\OtherTok{\textless{}{-}} \FunctionTok{predict}\NormalTok{(fit.agePLUSsex, }\AttributeTok{newdata =}\NormalTok{ nd)}
  \FunctionTok{lines}\NormalTok{(ages, yhat, }\AttributeTok{col =}\NormalTok{ cols[sx], }\AttributeTok{lwd =} \DecValTok{2}\NormalTok{)}
\NormalTok{\}}

\DocumentationTok{\#\# Legend}
\FunctionTok{legend}\NormalTok{(}\StringTok{"topleft"}\NormalTok{, }\AttributeTok{legend =}\NormalTok{ lev, }\AttributeTok{col =}\NormalTok{ cols[lev], }\AttributeTok{pch =} \DecValTok{16}\NormalTok{, }\AttributeTok{lwd =} \DecValTok{2}\NormalTok{, }\AttributeTok{bty =} \StringTok{"n"}\NormalTok{, }\AttributeTok{title =} \StringTok{"Sex"}\NormalTok{)}
\end{Highlighting}
\end{Shaded}

\includegraphics{unit3-mlr/mlr_files/figure-pdf/unnamed-chunk-27-1.pdf}

\begin{Shaded}
\begin{Highlighting}[]
\FunctionTok{data.frame}\NormalTok{(}\FunctionTok{model.matrix}\NormalTok{(fit.agePLUSsex))}
\end{Highlighting}
\end{Shaded}

\begin{verbatim}
   X.Intercept. age sex1
1             1  39    0
2             1  45    0
3             1  47    0
4             1  65    0
5             1  46    0
6             1  67    0
7             1  42    0
8             1  67    0
9             1  56    0
10            1  64    0
11            1  56    0
12            1  59    0
13            1  34    0
14            1  42    0
15            1  48    0
16            1  45    0
17            1  17    0
18            1  20    0
19            1  19    0
20            1  36    0
21            1  50    0
22            1  39    0
23            1  21    0
24            1  44    0
25            1  53    0
26            1  63    0
27            1  29    0
28            1  25    0
29            1  69    0
30            1  41    1
31            1  60    1
32            1  41    1
33            1  47    1
34            1  66    1
35            1  47    1
36            1  68    1
37            1  43    1
38            1  68    1
39            1  57    1
40            1  65    1
41            1  57    1
42            1  61    1
43            1  36    1
44            1  44    1
45            1  50    1
46            1  47    1
47            1  19    1
48            1  22    1
49            1  21    1
50            1  38    1
51            1  52    1
52            1  41    1
53            1  18    1
54            1  51    1
55            1  55    1
56            1  65    1
57            1  33    1
58            1  23    1
59            1  70    1
60            1  56    1
61            1  62    1
62            1  51    1
63            1  48    1
64            1  59    1
65            1  40    1
66            1  35    1
67            1  33    1
68            1  26    1
69            1  61    1
\end{verbatim}

\begin{Shaded}
\begin{Highlighting}[]
\FunctionTok{print}\NormalTok{(}\FunctionTok{anova}\NormalTok{(fit.age, fit.agePLUSsex))}
\end{Highlighting}
\end{Shaded}

\begin{verbatim}
Analysis of Variance Table

Model 1: sbp ~ age
Model 2: sbp ~ age + sex
  Res.Df    RSS Df Sum of Sq      F    Pr(>F)    
1     67 8260.5                                  
2     66 5202.0  1    3058.5 38.805 3.701e-08 ***
---
Signif. codes:  0 '***' 0.001 '**' 0.01 '*' 0.05 '.' 0.1 ' ' 1
\end{verbatim}

\subsubsection{Varying Intercept and Varying Slope Model (Age + Sex +
Age:Sex)}\label{varying-intercept-and-varying-slope-model-age-sex-agesex}

\begin{Shaded}
\begin{Highlighting}[]
\DocumentationTok{\#\# Make sure sex is a factor (for colors/legend)}
\NormalTok{sbpdata}\SpecialCharTok{$}\NormalTok{sex }\OtherTok{\textless{}{-}} \FunctionTok{factor}\NormalTok{(sbpdata}\SpecialCharTok{$}\NormalTok{sex)}

\DocumentationTok{\#\# Fit (interaction: different slopes by sex)}
\NormalTok{fit.age.TIMES.sex }\OtherTok{\textless{}{-}} \FunctionTok{lm}\NormalTok{(sbp }\SpecialCharTok{\textasciitilde{}}\NormalTok{ age }\SpecialCharTok{+}\NormalTok{ sex }\SpecialCharTok{+}\NormalTok{ age}\SpecialCharTok{:}\NormalTok{sex, }\AttributeTok{data =}\NormalTok{ sbpdata)}

\DocumentationTok{\#\# Age grid and palette}
\NormalTok{ages }\OtherTok{\textless{}{-}} \FunctionTok{seq}\NormalTok{(}\FunctionTok{min}\NormalTok{(sbpdata}\SpecialCharTok{$}\NormalTok{age, }\AttributeTok{na.rm =} \ConstantTok{TRUE}\NormalTok{),}
            \FunctionTok{max}\NormalTok{(sbpdata}\SpecialCharTok{$}\NormalTok{age, }\AttributeTok{na.rm =} \ConstantTok{TRUE}\NormalTok{),}
            \AttributeTok{length.out =} \DecValTok{200}\NormalTok{)}
\NormalTok{lev  }\OtherTok{\textless{}{-}} \FunctionTok{levels}\NormalTok{(sbpdata}\SpecialCharTok{$}\NormalTok{sex)}
\NormalTok{cols }\OtherTok{\textless{}{-}} \FunctionTok{setNames}\NormalTok{(}\FunctionTok{c}\NormalTok{(}\StringTok{"steelblue3"}\NormalTok{, }\StringTok{"tomato3"}\NormalTok{, }\StringTok{"darkorchid3"}\NormalTok{)[}\FunctionTok{seq\_along}\NormalTok{(lev)], lev)}

\DocumentationTok{\#\# Scatter: color points by sex}
\FunctionTok{plot}\NormalTok{(sbp }\SpecialCharTok{\textasciitilde{}}\NormalTok{ age, }\AttributeTok{data =}\NormalTok{ sbpdata,}
     \AttributeTok{col =}\NormalTok{ cols[sbpdata}\SpecialCharTok{$}\NormalTok{sex], }\AttributeTok{pch =} \DecValTok{16}\NormalTok{,}
     \AttributeTok{xlab =} \StringTok{"Age"}\NormalTok{, }\AttributeTok{ylab =} \StringTok{"Systolic BP"}\NormalTok{)}

\DocumentationTok{\#\# Fitted lines: one per sex (different slopes allowed)}
\ControlFlowTok{for}\NormalTok{ (sx }\ControlFlowTok{in}\NormalTok{ lev) \{}
\NormalTok{  nd }\OtherTok{\textless{}{-}} \FunctionTok{data.frame}\NormalTok{(}\AttributeTok{age =}\NormalTok{ ages, }\AttributeTok{sex =} \FunctionTok{factor}\NormalTok{(sx, }\AttributeTok{levels =}\NormalTok{ lev))}
\NormalTok{  yhat }\OtherTok{\textless{}{-}} \FunctionTok{predict}\NormalTok{(fit.age.TIMES.sex, }\AttributeTok{newdata =}\NormalTok{ nd)}
  \FunctionTok{lines}\NormalTok{(ages, yhat, }\AttributeTok{col =}\NormalTok{ cols[sx], }\AttributeTok{lwd =} \DecValTok{2}\NormalTok{)}
\NormalTok{\}}

\DocumentationTok{\#\# Legend}
\FunctionTok{legend}\NormalTok{(}\StringTok{"topleft"}\NormalTok{, }\AttributeTok{legend =}\NormalTok{ lev, }\AttributeTok{col =}\NormalTok{ cols[lev], }\AttributeTok{pch =} \DecValTok{16}\NormalTok{, }\AttributeTok{lwd =} \DecValTok{2}\NormalTok{, }\AttributeTok{bty =} \StringTok{"n"}\NormalTok{, }\AttributeTok{title =} \StringTok{"Sex"}\NormalTok{)}
\end{Highlighting}
\end{Shaded}

\includegraphics{unit3-mlr/mlr_files/figure-pdf/unnamed-chunk-29-1.pdf}

\textbf{Model Matrix and ANOVA}

\begin{Shaded}
\begin{Highlighting}[]
\FunctionTok{data.frame}\NormalTok{(}\FunctionTok{model.matrix}\NormalTok{(fit.age.TIMES.sex))}
\end{Highlighting}
\end{Shaded}

\begin{verbatim}
   X.Intercept. age sex1 age.sex1
1             1  39    0        0
2             1  45    0        0
3             1  47    0        0
4             1  65    0        0
5             1  46    0        0
6             1  67    0        0
7             1  42    0        0
8             1  67    0        0
9             1  56    0        0
10            1  64    0        0
11            1  56    0        0
12            1  59    0        0
13            1  34    0        0
14            1  42    0        0
15            1  48    0        0
16            1  45    0        0
17            1  17    0        0
18            1  20    0        0
19            1  19    0        0
20            1  36    0        0
21            1  50    0        0
22            1  39    0        0
23            1  21    0        0
24            1  44    0        0
25            1  53    0        0
26            1  63    0        0
27            1  29    0        0
28            1  25    0        0
29            1  69    0        0
30            1  41    1       41
31            1  60    1       60
32            1  41    1       41
33            1  47    1       47
34            1  66    1       66
35            1  47    1       47
36            1  68    1       68
37            1  43    1       43
38            1  68    1       68
39            1  57    1       57
40            1  65    1       65
41            1  57    1       57
42            1  61    1       61
43            1  36    1       36
44            1  44    1       44
45            1  50    1       50
46            1  47    1       47
47            1  19    1       19
48            1  22    1       22
49            1  21    1       21
50            1  38    1       38
51            1  52    1       52
52            1  41    1       41
53            1  18    1       18
54            1  51    1       51
55            1  55    1       55
56            1  65    1       65
57            1  33    1       33
58            1  23    1       23
59            1  70    1       70
60            1  56    1       56
61            1  62    1       62
62            1  51    1       51
63            1  48    1       48
64            1  59    1       59
65            1  40    1       40
66            1  35    1       35
67            1  33    1       33
68            1  26    1       26
69            1  61    1       61
\end{verbatim}

\begin{Shaded}
\begin{Highlighting}[]
\FunctionTok{summary}\NormalTok{(fit.age.TIMES.sex)}
\end{Highlighting}
\end{Shaded}

\begin{verbatim}

Call:
lm(formula = sbp ~ age + sex + age:sex, data = sbpdata)

Residuals:
    Min      1Q  Median      3Q     Max 
-20.647  -3.410   1.254   4.314  21.153 

Coefficients:
            Estimate Std. Error t value Pr(>|t|)    
(Intercept) 97.07708    5.17046  18.775  < 2e-16 ***
age          0.94932    0.10864   8.738 1.43e-12 ***
sex1        12.96144    7.01172   1.849   0.0691 .  
age:sex1     0.01203    0.14519   0.083   0.9342    
---
Signif. codes:  0 '***' 0.001 '**' 0.01 '*' 0.05 '.' 0.1 ' ' 1

Residual standard error: 8.946 on 65 degrees of freedom
Multiple R-squared:  0.7759,    Adjusted R-squared:  0.7656 
F-statistic: 75.02 on 3 and 65 DF,  p-value: < 2.2e-16
\end{verbatim}

\begin{Shaded}
\begin{Highlighting}[]
\FunctionTok{print}\NormalTok{(}\FunctionTok{anova}\NormalTok{(fit.age,fit.agePLUSsex,fit.age.TIMES.sex))}
\end{Highlighting}
\end{Shaded}

\begin{verbatim}
Analysis of Variance Table

Model 1: sbp ~ age
Model 2: sbp ~ age + sex
Model 3: sbp ~ age + sex + age:sex
  Res.Df    RSS Df Sum of Sq       F    Pr(>F)    
1     67 8260.5                                   
2     66 5202.0  1   3058.52 38.2210 4.692e-08 ***
3     65 5201.4  1      0.55  0.0069    0.9342    
---
Signif. codes:  0 '***' 0.001 '**' 0.01 '*' 0.05 '.' 0.1 ' ' 1
\end{verbatim}

\subsubsection{Varying Slope, Equal Intercept Model (Age +
Age:Sex)}\label{varying-slope-equal-intercept-model-age-agesex}

\begin{Shaded}
\begin{Highlighting}[]
\DocumentationTok{\#\# Make sure sex is a factor (for colors/legend)}
\NormalTok{sbpdata}\SpecialCharTok{$}\NormalTok{sex }\OtherTok{\textless{}{-}} \FunctionTok{factor}\NormalTok{(sbpdata}\SpecialCharTok{$}\NormalTok{sex)}

\DocumentationTok{\#\# Fit (interaction: different slopes by sex)}
\NormalTok{fit.equal.intercept }\OtherTok{\textless{}{-}} \FunctionTok{lm}\NormalTok{(sbp }\SpecialCharTok{\textasciitilde{}}\NormalTok{ age }\SpecialCharTok{+}\NormalTok{ age}\SpecialCharTok{:}\NormalTok{sex, }\AttributeTok{data =}\NormalTok{ sbpdata)}


\DocumentationTok{\#\# Age grid and palette}
\NormalTok{ages }\OtherTok{\textless{}{-}} \FunctionTok{seq}\NormalTok{(}\FunctionTok{min}\NormalTok{(sbpdata}\SpecialCharTok{$}\NormalTok{age, }\AttributeTok{na.rm =} \ConstantTok{TRUE}\NormalTok{),}
            \FunctionTok{max}\NormalTok{(sbpdata}\SpecialCharTok{$}\NormalTok{age, }\AttributeTok{na.rm =} \ConstantTok{TRUE}\NormalTok{),}
            \AttributeTok{length.out =} \DecValTok{200}\NormalTok{)}
\NormalTok{lev  }\OtherTok{\textless{}{-}} \FunctionTok{levels}\NormalTok{(sbpdata}\SpecialCharTok{$}\NormalTok{sex)}
\NormalTok{cols }\OtherTok{\textless{}{-}} \FunctionTok{setNames}\NormalTok{(}\FunctionTok{c}\NormalTok{(}\StringTok{"steelblue3"}\NormalTok{, }\StringTok{"tomato3"}\NormalTok{, }\StringTok{"darkorchid3"}\NormalTok{)[}\FunctionTok{seq\_along}\NormalTok{(lev)], lev)}

\DocumentationTok{\#\# Scatter: color points by sex}
\FunctionTok{plot}\NormalTok{(sbp }\SpecialCharTok{\textasciitilde{}}\NormalTok{ age, }\AttributeTok{data =}\NormalTok{ sbpdata,}
     \AttributeTok{col =}\NormalTok{ cols[sbpdata}\SpecialCharTok{$}\NormalTok{sex], }\AttributeTok{pch =} \DecValTok{16}\NormalTok{,}
     \AttributeTok{xlab =} \StringTok{"Age"}\NormalTok{, }\AttributeTok{ylab =} \StringTok{"Systolic BP"}\NormalTok{)}

\DocumentationTok{\#\# Fitted lines: one per sex (different slopes allowed)}
\ControlFlowTok{for}\NormalTok{ (sx }\ControlFlowTok{in}\NormalTok{ lev) \{}
\NormalTok{  nd }\OtherTok{\textless{}{-}} \FunctionTok{data.frame}\NormalTok{(}\AttributeTok{age =}\NormalTok{ ages, }\AttributeTok{sex =} \FunctionTok{factor}\NormalTok{(sx, }\AttributeTok{levels =}\NormalTok{ lev))}
\NormalTok{  yhat }\OtherTok{\textless{}{-}} \FunctionTok{predict}\NormalTok{(fit.equal.intercept, }\AttributeTok{newdata =}\NormalTok{ nd)}
  \FunctionTok{lines}\NormalTok{(ages, yhat, }\AttributeTok{col =}\NormalTok{ cols[sx], }\AttributeTok{lwd =} \DecValTok{2}\NormalTok{)}
\NormalTok{\}}

\DocumentationTok{\#\# Legend}
\FunctionTok{legend}\NormalTok{(}\StringTok{"topleft"}\NormalTok{, }\AttributeTok{legend =}\NormalTok{ lev, }\AttributeTok{col =}\NormalTok{ cols[lev], }\AttributeTok{pch =} \DecValTok{16}\NormalTok{, }\AttributeTok{lwd =} \DecValTok{2}\NormalTok{, }\AttributeTok{bty =} \StringTok{"n"}\NormalTok{, }\AttributeTok{title =} \StringTok{"Sex"}\NormalTok{)}
\end{Highlighting}
\end{Shaded}

\includegraphics{unit3-mlr/mlr_files/figure-pdf/unnamed-chunk-31-1.pdf}

\subsection{Orders of Terms Matters in ANOVA and Warnings in
Interpreting t-test
Tables}\label{orders-of-terms-matters-in-anova-and-warnings-in-interpreting-t-test-tables}

\begin{Shaded}
\begin{Highlighting}[]
\NormalTok{fit.int }\OtherTok{\textless{}{-}} \FunctionTok{lm}\NormalTok{(sbp }\SpecialCharTok{\textasciitilde{}} \DecValTok{1}\NormalTok{, }\AttributeTok{data =}\NormalTok{ sbpdata)}
\NormalTok{fit.sex }\OtherTok{\textless{}{-}} \FunctionTok{lm}\NormalTok{(sbp }\SpecialCharTok{\textasciitilde{}}\NormalTok{ sex, }\AttributeTok{data =}\NormalTok{ sbpdata)}

\FunctionTok{print}\NormalTok{(}\FunctionTok{anova}\NormalTok{(fit.int,fit.age,fit.agePLUSsex, fit.age.TIMES.sex))}
\end{Highlighting}
\end{Shaded}

\begin{verbatim}
Analysis of Variance Table

Model 1: sbp ~ 1
Model 2: sbp ~ age
Model 3: sbp ~ age + sex
Model 4: sbp ~ age + sex + age:sex
  Res.Df     RSS Df Sum of Sq        F    Pr(>F)    
1     68 23211.8                                    
2     67  8260.5  1   14951.3 186.8390 < 2.2e-16 ***
3     66  5202.0  1    3058.5  38.2210 4.692e-08 ***
4     65  5201.4  1       0.5   0.0069    0.9342    
---
Signif. codes:  0 '***' 0.001 '**' 0.01 '*' 0.05 '.' 0.1 ' ' 1
\end{verbatim}

\begin{Shaded}
\begin{Highlighting}[]
\FunctionTok{print}\NormalTok{(}\FunctionTok{anova}\NormalTok{(fit.int,fit.age,fit.equal.intercept, fit.age.TIMES.sex))}
\end{Highlighting}
\end{Shaded}

\begin{verbatim}
Analysis of Variance Table

Model 1: sbp ~ 1
Model 2: sbp ~ age
Model 3: sbp ~ age + age:sex
Model 4: sbp ~ age + sex + age:sex
  Res.Df     RSS Df Sum of Sq        F    Pr(>F)    
1     68 23211.8                                    
2     67  8260.5  1   14951.3 186.8390 < 2.2e-16 ***
3     66  5474.9  1    2785.6  34.8107 1.437e-07 ***
4     65  5201.4  1     273.4   3.4171   0.06907 .  
---
Signif. codes:  0 '***' 0.001 '**' 0.01 '*' 0.05 '.' 0.1 ' ' 1
\end{verbatim}

\begin{Shaded}
\begin{Highlighting}[]
\FunctionTok{print}\NormalTok{(}\FunctionTok{anova}\NormalTok{(fit.int,fit.sex,fit.agePLUSsex, fit.age.TIMES.sex))}
\end{Highlighting}
\end{Shaded}

\begin{verbatim}
Analysis of Variance Table

Model 1: sbp ~ 1
Model 2: sbp ~ sex
Model 3: sbp ~ age + sex
Model 4: sbp ~ age + sex + age:sex
  Res.Df     RSS Df Sum of Sq        F    Pr(>F)    
1     68 23211.8                                    
2     67 19282.5  1    3929.2  49.1017 1.684e-09 ***
3     66  5202.0  1   14080.6 175.9583 < 2.2e-16 ***
4     65  5201.4  1       0.5   0.0069    0.9342    
---
Signif. codes:  0 '***' 0.001 '**' 0.01 '*' 0.05 '.' 0.1 ' ' 1
\end{verbatim}

\begin{Shaded}
\begin{Highlighting}[]
\FunctionTok{summary}\NormalTok{(fit.age)}
\end{Highlighting}
\end{Shaded}

\begin{verbatim}

Call:
lm(formula = sbp ~ age, data = sbpdata)

Residuals:
    Min      1Q  Median      3Q     Max 
-26.782  -7.632   1.968   8.201  22.651 

Coefficients:
             Estimate Std. Error t value Pr(>|t|)    
(Intercept) 103.34905    4.33190   23.86   <2e-16 ***
age           0.98333    0.08929   11.01   <2e-16 ***
---
Signif. codes:  0 '***' 0.001 '**' 0.01 '*' 0.05 '.' 0.1 ' ' 1

Residual standard error: 11.1 on 67 degrees of freedom
Multiple R-squared:  0.6441,    Adjusted R-squared:  0.6388 
F-statistic: 121.3 on 1 and 67 DF,  p-value: < 2.2e-16
\end{verbatim}

\begin{Shaded}
\begin{Highlighting}[]
\FunctionTok{summary}\NormalTok{(fit.equal.intercept)}
\end{Highlighting}
\end{Shaded}

\begin{verbatim}

Call:
lm(formula = sbp ~ age + age:sex, data = sbpdata)

Residuals:
     Min       1Q   Median       3Q      Max 
-21.6338  -4.3067   0.9922   4.9819  20.2753 

Coefficients:
             Estimate Std. Error t value Pr(>|t|)    
(Intercept) 104.12501    3.55578  29.283  < 2e-16 ***
age           0.80908    0.07918  10.219 3.14e-15 ***
age:sex1      0.26705    0.04608   5.795 2.09e-07 ***
---
Signif. codes:  0 '***' 0.001 '**' 0.01 '*' 0.05 '.' 0.1 ' ' 1

Residual standard error: 9.108 on 66 degrees of freedom
Multiple R-squared:  0.7641,    Adjusted R-squared:  0.757 
F-statistic: 106.9 on 2 and 66 DF,  p-value: < 2.2e-16
\end{verbatim}

\begin{Shaded}
\begin{Highlighting}[]
\FunctionTok{summary}\NormalTok{(fit.agePLUSsex)}
\end{Highlighting}
\end{Shaded}

\begin{verbatim}

Call:
lm(formula = sbp ~ age + sex, data = sbpdata)

Residuals:
    Min      1Q  Median      3Q     Max 
-20.705  -3.299   1.248   4.325  21.160 

Coefficients:
            Estimate Std. Error t value Pr(>|t|)    
(Intercept) 96.77353    3.62085  26.727  < 2e-16 ***
age          0.95606    0.07153  13.366  < 2e-16 ***
sex1        13.51345    2.16932   6.229  3.7e-08 ***
---
Signif. codes:  0 '***' 0.001 '**' 0.01 '*' 0.05 '.' 0.1 ' ' 1

Residual standard error: 8.878 on 66 degrees of freedom
Multiple R-squared:  0.7759,    Adjusted R-squared:  0.7691 
F-statistic: 114.2 on 2 and 66 DF,  p-value: < 2.2e-16
\end{verbatim}

\begin{Shaded}
\begin{Highlighting}[]
\FunctionTok{summary}\NormalTok{(fit.age.TIMES.sex)}
\end{Highlighting}
\end{Shaded}

\begin{verbatim}

Call:
lm(formula = sbp ~ age + sex + age:sex, data = sbpdata)

Residuals:
    Min      1Q  Median      3Q     Max 
-20.647  -3.410   1.254   4.314  21.153 

Coefficients:
            Estimate Std. Error t value Pr(>|t|)    
(Intercept) 97.07708    5.17046  18.775  < 2e-16 ***
age          0.94932    0.10864   8.738 1.43e-12 ***
sex1        12.96144    7.01172   1.849   0.0691 .  
age:sex1     0.01203    0.14519   0.083   0.9342    
---
Signif. codes:  0 '***' 0.001 '**' 0.01 '*' 0.05 '.' 0.1 ' ' 1

Residual standard error: 8.946 on 65 degrees of freedom
Multiple R-squared:  0.7759,    Adjusted R-squared:  0.7656 
F-statistic: 75.02 on 3 and 65 DF,  p-value: < 2.2e-16
\end{verbatim}

\section{Model Building}\label{model-building}

\begin{Shaded}
\begin{Highlighting}[]
\FunctionTok{library}\NormalTok{(olsrr)}
\DocumentationTok{\#\# Note: Update this path to your local file location}
\NormalTok{wine }\OtherTok{\textless{}{-}} \FunctionTok{read.csv}\NormalTok{(}\StringTok{"wine.csv"}\NormalTok{)}

\NormalTok{model.wine }\OtherTok{\textless{}{-}} \FunctionTok{lm}\NormalTok{(quality }\SpecialCharTok{\textasciitilde{}}\NormalTok{ ., }\AttributeTok{data =}\NormalTok{ wine)}
\end{Highlighting}
\end{Shaded}

\subsection{All Possible Regression}\label{all-possible-regression}

\begin{Shaded}
\begin{Highlighting}[]
\FunctionTok{ols\_step\_best\_subset}\NormalTok{(model.wine)}
\end{Highlighting}
\end{Shaded}

\begin{verbatim}
             Best Subsets Regression             
-------------------------------------------------
Model Index    Predictors
-------------------------------------------------
     1         flavor                             
     2         flavor oakiness                    
     3         aroma flavor oakiness              
     4         clarity aroma flavor oakiness      
     5         clarity aroma body flavor oakiness 
-------------------------------------------------

                                                  Subsets Regression Summary                                                   
-------------------------------------------------------------------------------------------------------------------------------
                       Adj.        Pred                                                                                         
Model    R-Square    R-Square    R-Square     C(p)       AIC        SBIC        SBC        MSEP       FPE       HSP       APC  
-------------------------------------------------------------------------------------------------------------------------------
  1        0.6242      0.6137      0.5868    9.0436    130.0214    21.6859    134.9341    61.4102    1.7010    0.0462    0.4176 
  2        0.6611      0.6417      0.6058    6.8132    128.0901    20.1242    134.6404    57.0033    1.6171    0.0441    0.3970 
  3        0.7038      0.6776      0.6379    3.9278    124.9781    18.0702    133.1661    51.3383    1.4906    0.0409    0.3659 
  4        0.7147      0.6801      0.6102    4.6747    125.5480    19.2854    135.3736    50.9872    1.5143    0.0418    0.3717 
  5        0.7206      0.6769       0.587    6.0000    126.7552    21.0956    138.2183    51.5452    1.5649    0.0436    0.3842 
-------------------------------------------------------------------------------------------------------------------------------
AIC: Akaike Information Criteria 
 SBIC: Sawa's Bayesian Information Criteria 
 SBC: Schwarz Bayesian Criteria 
 MSEP: Estimated error of prediction, assuming multivariate normality 
 FPE: Final Prediction Error 
 HSP: Hocking's Sp 
 APC: Amemiya Prediction Criteria 
\end{verbatim}

\subsection{Automated Stepwise
Procedures}\label{automated-stepwise-procedures}

\begin{Shaded}
\begin{Highlighting}[]
\DocumentationTok{\#\# Backward Elimination (alpha\_out = 0.1)}
\FunctionTok{ols\_step\_backward\_p}\NormalTok{(model.wine, }\AttributeTok{p\_val =} \FloatTok{0.1}\NormalTok{)}
\end{Highlighting}
\end{Shaded}

\begin{verbatim}

                             Stepwise Summary                             
------------------------------------------------------------------------
Step    Variable        AIC        SBC       SBIC       R2       Adj. R2 
------------------------------------------------------------------------
 0      Full Model    126.755    138.218    21.096    0.72060    0.67694 
 1      body          125.548    135.374    19.285    0.71471    0.68013 
 2      clarity       124.978    133.166    18.070    0.70377    0.67763 
------------------------------------------------------------------------

Final Model Output 
------------------

                         Model Summary                          
---------------------------------------------------------------
R                       0.839       RMSE                 1.098 
R-Squared               0.704       MSE                  1.207 
Adj. R-Squared          0.678       Coef. Var            9.338 
Pred R-Squared          0.638       AIC                124.978 
MAE                     0.868       SBC                133.166 
---------------------------------------------------------------
 RMSE: Root Mean Square Error 
 MSE: Mean Square Error 
 MAE: Mean Absolute Error 
 AIC: Akaike Information Criteria 
 SBC: Schwarz Bayesian Criteria 

                               ANOVA                                
-------------------------------------------------------------------
               Sum of                                              
              Squares        DF    Mean Square      F         Sig. 
-------------------------------------------------------------------
Regression    108.935         3         36.312    26.925    0.0000 
Residual       45.853        34          1.349                     
Total         154.788        37                                    
-------------------------------------------------------------------

                                  Parameter Estimates                                    
----------------------------------------------------------------------------------------
      model      Beta    Std. Error    Std. Beta      t        Sig      lower     upper 
----------------------------------------------------------------------------------------
(Intercept)     6.467         1.333                  4.852    0.000     3.759     9.176 
      aroma     0.580         0.262        0.307     2.213    0.034     0.047     1.113 
     flavor     1.200         0.275        0.603     4.364    0.000     0.641     1.758 
   oakiness    -0.602         0.264       -0.217    -2.278    0.029    -1.140    -0.065 
----------------------------------------------------------------------------------------
\end{verbatim}

\begin{Shaded}
\begin{Highlighting}[]
\DocumentationTok{\#\# Forward Selection (alpha\_in = 0.1)}
\FunctionTok{ols\_step\_forward\_p}\NormalTok{(model.wine, }\AttributeTok{p\_val =} \FloatTok{0.1}\NormalTok{)}
\end{Highlighting}
\end{Shaded}

\begin{verbatim}

                             Stepwise Summary                             
------------------------------------------------------------------------
Step    Variable        AIC        SBC       SBIC       R2       Adj. R2 
------------------------------------------------------------------------
 0      Base Model    165.209    168.484    55.141    0.00000    0.00000 
 1      flavor        130.021    134.934    21.686    0.62417    0.61373 
 2      oakiness      128.090    134.640    20.124    0.66111    0.64175 
 3      aroma         124.978    133.166    18.070    0.70377    0.67763 
------------------------------------------------------------------------

Final Model Output 
------------------

                         Model Summary                          
---------------------------------------------------------------
R                       0.839       RMSE                 1.098 
R-Squared               0.704       MSE                  1.207 
Adj. R-Squared          0.678       Coef. Var            9.338 
Pred R-Squared          0.638       AIC                124.978 
MAE                     0.868       SBC                133.166 
---------------------------------------------------------------
 RMSE: Root Mean Square Error 
 MSE: Mean Square Error 
 MAE: Mean Absolute Error 
 AIC: Akaike Information Criteria 
 SBC: Schwarz Bayesian Criteria 

                               ANOVA                                
-------------------------------------------------------------------
               Sum of                                              
              Squares        DF    Mean Square      F         Sig. 
-------------------------------------------------------------------
Regression    108.935         3         36.312    26.925    0.0000 
Residual       45.853        34          1.349                     
Total         154.788        37                                    
-------------------------------------------------------------------

                                  Parameter Estimates                                    
----------------------------------------------------------------------------------------
      model      Beta    Std. Error    Std. Beta      t        Sig      lower     upper 
----------------------------------------------------------------------------------------
(Intercept)     6.467         1.333                  4.852    0.000     3.759     9.176 
     flavor     1.200         0.275        0.603     4.364    0.000     0.641     1.758 
   oakiness    -0.602         0.264       -0.217    -2.278    0.029    -1.140    -0.065 
      aroma     0.580         0.262        0.307     2.213    0.034     0.047     1.113 
----------------------------------------------------------------------------------------
\end{verbatim}

\begin{Shaded}
\begin{Highlighting}[]
\DocumentationTok{\#\# Stepwise Regression (alpha\_in = 0.1, alpha\_out = 0.1)}
\FunctionTok{ols\_step\_both\_p}\NormalTok{(model.wine, }\AttributeTok{p\_enter =} \FloatTok{0.1}\NormalTok{, }\AttributeTok{p\_remove =} \FloatTok{0.1}\NormalTok{)}
\end{Highlighting}
\end{Shaded}

\begin{verbatim}

                              Stepwise Summary                              
--------------------------------------------------------------------------
Step    Variable          AIC        SBC       SBIC       R2       Adj. R2 
--------------------------------------------------------------------------
 0      Base Model      165.209    168.484    55.141    0.00000    0.00000 
 1      flavor (+)      130.021    134.934    21.686    0.62417    0.61373 
 2      oakiness (+)    128.090    134.640    20.124    0.66111    0.64175 
 3      aroma (+)       124.978    133.166    18.070    0.70377    0.67763 
--------------------------------------------------------------------------

Final Model Output 
------------------

                         Model Summary                          
---------------------------------------------------------------
R                       0.839       RMSE                 1.098 
R-Squared               0.704       MSE                  1.207 
Adj. R-Squared          0.678       Coef. Var            9.338 
Pred R-Squared          0.638       AIC                124.978 
MAE                     0.868       SBC                133.166 
---------------------------------------------------------------
 RMSE: Root Mean Square Error 
 MSE: Mean Square Error 
 MAE: Mean Absolute Error 
 AIC: Akaike Information Criteria 
 SBC: Schwarz Bayesian Criteria 

                               ANOVA                                
-------------------------------------------------------------------
               Sum of                                              
              Squares        DF    Mean Square      F         Sig. 
-------------------------------------------------------------------
Regression    108.935         3         36.312    26.925    0.0000 
Residual       45.853        34          1.349                     
Total         154.788        37                                    
-------------------------------------------------------------------

                                  Parameter Estimates                                    
----------------------------------------------------------------------------------------
      model      Beta    Std. Error    Std. Beta      t        Sig      lower     upper 
----------------------------------------------------------------------------------------
(Intercept)     6.467         1.333                  4.852    0.000     3.759     9.176 
     flavor     1.200         0.275        0.603     4.364    0.000     0.641     1.758 
   oakiness    -0.602         0.264       -0.217    -2.278    0.029    -1.140    -0.065 
      aroma     0.580         0.262        0.307     2.213    0.034     0.047     1.113 
----------------------------------------------------------------------------------------
\end{verbatim}

\section{Multicollinearity}\label{multicollinearity}

\subsection{A Simple Example}\label{a-simple-example}

\begin{Shaded}
\begin{Highlighting}[]
\NormalTok{y }\OtherTok{\textless{}{-}} \FunctionTok{c}\NormalTok{(}\DecValTok{19}\NormalTok{, }\DecValTok{20}\NormalTok{, }\DecValTok{37}\NormalTok{, }\DecValTok{39}\NormalTok{, }\DecValTok{36}\NormalTok{, }\DecValTok{38}\NormalTok{)}
\NormalTok{x1 }\OtherTok{\textless{}{-}} \FunctionTok{c}\NormalTok{(}\DecValTok{4}\NormalTok{, }\DecValTok{4}\NormalTok{, }\DecValTok{7}\NormalTok{, }\DecValTok{7}\NormalTok{, }\FloatTok{7.1}\NormalTok{, }\FloatTok{7.1}\NormalTok{)}
\NormalTok{x2 }\OtherTok{\textless{}{-}} \FunctionTok{c}\NormalTok{(}\DecValTok{16}\NormalTok{, }\DecValTok{16}\NormalTok{, }\DecValTok{49}\NormalTok{, }\DecValTok{49}\NormalTok{, }\FloatTok{50.4}\NormalTok{, }\FloatTok{50.4}\NormalTok{)}
\FunctionTok{cor}\NormalTok{(}\FunctionTok{data.frame}\NormalTok{(x1, x2))}
\end{Highlighting}
\end{Shaded}

\begin{verbatim}
          x1        x2
x1 1.0000000 0.9999713
x2 0.9999713 1.0000000
\end{verbatim}

\begin{Shaded}
\begin{Highlighting}[]
\NormalTok{fit\_multi }\OtherTok{\textless{}{-}} \FunctionTok{lm}\NormalTok{(y }\SpecialCharTok{\textasciitilde{}}\NormalTok{ x1 }\SpecialCharTok{+}\NormalTok{ x2)}
\FunctionTok{summary}\NormalTok{(fit\_multi)}
\end{Highlighting}
\end{Shaded}

\begin{verbatim}

Call:
lm(formula = y ~ x1 + x2)

Residuals:
   1    2    3    4    5    6 
-0.5  0.5 -1.0  1.0 -1.0  1.0 

Coefficients:
            Estimate Std. Error t value Pr(>|t|)
(Intercept) -156.056    117.158  -1.332    0.275
x1            65.444     45.890   1.426    0.249
x2            -5.389      4.152  -1.298    0.285

Residual standard error: 1.225 on 3 degrees of freedom
Multiple R-squared:  0.9897,    Adjusted R-squared:  0.9829 
F-statistic: 144.3 on 2 and 3 DF,  p-value: 0.001043
\end{verbatim}

\begin{Shaded}
\begin{Highlighting}[]
\NormalTok{fit1\_multi }\OtherTok{\textless{}{-}} \FunctionTok{lm}\NormalTok{(y }\SpecialCharTok{\textasciitilde{}}\NormalTok{ x1)}
\FunctionTok{summary}\NormalTok{(fit1\_multi)}
\end{Highlighting}
\end{Shaded}

\begin{verbatim}

Call:
lm(formula = y ~ x1)

Residuals:
      1       2       3       4       5       6 
-0.5260  0.4740 -0.1925  1.8075 -1.7814  0.2186 

Coefficients:
            Estimate Std. Error t value Pr(>|t|)    
(Intercept)  -4.0293     2.3332  -1.727    0.159    
x1            5.8888     0.3762  15.654 9.73e-05 ***
---
Signif. codes:  0 '***' 0.001 '**' 0.01 '*' 0.05 '.' 0.1 ' ' 1

Residual standard error: 1.325 on 4 degrees of freedom
Multiple R-squared:  0.9839,    Adjusted R-squared:  0.9799 
F-statistic: 245.1 on 1 and 4 DF,  p-value: 9.725e-05
\end{verbatim}

\begin{Shaded}
\begin{Highlighting}[]
\FunctionTok{ols\_vif\_tol}\NormalTok{(fit\_multi)}
\end{Highlighting}
\end{Shaded}

\begin{verbatim}
  Variables    Tolerance      VIF
1        x1 5.738191e-05 17427.09
2        x2 5.738191e-05 17427.09
\end{verbatim}

\subsection{VIFs in the Wine Quality
Data}\label{vifs-in-the-wine-quality-data}

\begin{Shaded}
\begin{Highlighting}[]
\NormalTok{wine.x }\OtherTok{\textless{}{-}}\NormalTok{ wine[, }\SpecialCharTok{{-}}\FunctionTok{ncol}\NormalTok{(wine)] }\CommentTok{\# Assuming quality is the last column}
\FunctionTok{cor}\NormalTok{(wine.x)}
\end{Highlighting}
\end{Shaded}

\begin{verbatim}
             clarity     aroma       body      flavor  oakiness
clarity   1.00000000 0.0619021 -0.3083783 -0.08515993 0.1832147
aroma     0.06190210 1.0000000  0.5489102  0.73656121 0.2016444
body     -0.30837826 0.5489102  1.0000000  0.64665917 0.1521059
flavor   -0.08515993 0.7365612  0.6466592  1.00000000 0.1797605
oakiness  0.18321471 0.2016444  0.1521059  0.17976051 1.0000000
\end{verbatim}

\begin{Shaded}
\begin{Highlighting}[]
\DocumentationTok{\#\# VIF using olsrr (data frame output)}
\FunctionTok{ols\_vif\_tol}\NormalTok{(model.wine)}
\end{Highlighting}
\end{Shaded}

\begin{verbatim}
  Variables Tolerance      VIF
1   clarity 0.7896462 1.266390
2     aroma 0.4199665 2.381143
3      body 0.4862649 2.056492
4    flavor 0.3728175 2.682277
5  oakiness 0.9118005 1.096731
\end{verbatim}

\subsection{VIFs in the Children Height
Data}\label{vifs-in-the-children-height-data}

\begin{Shaded}
\begin{Highlighting}[]
\DocumentationTok{\#\# Data: Weight, height and age of children}
\NormalTok{wgt }\OtherTok{\textless{}{-}} \FunctionTok{c}\NormalTok{(}\DecValTok{64}\NormalTok{, }\DecValTok{71}\NormalTok{, }\DecValTok{53}\NormalTok{, }\DecValTok{67}\NormalTok{, }\DecValTok{55}\NormalTok{, }\DecValTok{58}\NormalTok{, }\DecValTok{77}\NormalTok{, }\DecValTok{57}\NormalTok{, }\DecValTok{56}\NormalTok{, }\DecValTok{51}\NormalTok{, }\DecValTok{76}\NormalTok{, }\DecValTok{68}\NormalTok{)}
\NormalTok{hgt }\OtherTok{\textless{}{-}} \FunctionTok{c}\NormalTok{(}\DecValTok{57}\NormalTok{, }\DecValTok{59}\NormalTok{, }\DecValTok{49}\NormalTok{, }\DecValTok{62}\NormalTok{, }\DecValTok{51}\NormalTok{, }\DecValTok{50}\NormalTok{, }\DecValTok{55}\NormalTok{, }\DecValTok{48}\NormalTok{, }\DecValTok{42}\NormalTok{, }\DecValTok{42}\NormalTok{, }\DecValTok{61}\NormalTok{, }\DecValTok{57}\NormalTok{)}
\NormalTok{age }\OtherTok{\textless{}{-}} \FunctionTok{c}\NormalTok{(}\DecValTok{8}\NormalTok{, }\DecValTok{10}\NormalTok{, }\DecValTok{6}\NormalTok{, }\DecValTok{11}\NormalTok{, }\DecValTok{8}\NormalTok{, }\DecValTok{7}\NormalTok{, }\DecValTok{10}\NormalTok{, }\DecValTok{9}\NormalTok{, }\DecValTok{10}\NormalTok{, }\DecValTok{6}\NormalTok{, }\DecValTok{12}\NormalTok{, }\DecValTok{9}\NormalTok{)}

\NormalTok{fit\_age\_hgt }\OtherTok{\textless{}{-}} \FunctionTok{lm}\NormalTok{(wgt }\SpecialCharTok{\textasciitilde{}}\NormalTok{ hgt }\SpecialCharTok{+}\NormalTok{ age, }\AttributeTok{data =}\NormalTok{ child.data)}
\FunctionTok{ols\_vif\_tol}\NormalTok{(fit\_age\_hgt)}
\end{Highlighting}
\end{Shaded}

\begin{verbatim}
  Variables Tolerance      VIF
1       hgt 0.6232021 1.604616
2       age 0.6232021 1.604616
\end{verbatim}

\bookmarksetup{startatroot}

\chapter{Understanding the Leverage for Adjusting Residuals of
OLS}\label{understanding-the-leverage-for-adjusting-residuals-of-ols}

A Simulation Illustration with R

\hfill\break

Note: this html contains materials for understanding residuals not the
techniques that students need to master.

\section{Introduction}\label{introduction}

In statistical modeling, identifying data points that don't fit---the
\textbf{outliers}---is a critical step. The most reliable tool for this
job is the \textbf{externally studentized residual}. Its power comes
from a simple, intuitive idea: to judge a point fairly, you shouldn't
use that point when building your model. This is the core principle of
\textbf{Leave-One-Out Cross-Validation (LOOCV)}.

This article provides a complete walkthrough of this essential concept.
We'll start with the basic linear model, introduce the necessary
notation, explore the flaws of simpler residuals, and then formally
define and prove the equivalence of the conceptual and computational
formulas for studentized residuals. Finally, we'll make it all concrete
with a simple example.

\section{The Linear Model}\label{the-linear-model}

Our discussion is based on the standard multiple linear regression
model. In matrix form, the relationship between a response vector
\textbf{Y} and a predictor matrix \textbf{X} is: \[
\mathbf{Y} = \mathbf{X}\beta + \epsilon
\]

where:

\begin{itemize}
\tightlist
\item
  \textbf{Y} is an \emph{n} x 1 vector of the observed outcomes.
\item
  \textbf{X} is the \emph{n} x \emph{p} design matrix of predictor
  variables (where \emph{p} is the number of coefficients, including the
  intercept).
\item
  \(\beta\) is the \emph{p} x 1 vector of unknown true coefficients we
  want to estimate.
\item
  \(\epsilon\) is an \emph{n} x 1 vector of unobservable random errors,
  assumed to be independent and identically distributed with a mean of 0
  and a variance of \(\sigma^2\).
\end{itemize}

\section{Residuals for OLS}\label{residuals-for-ols}

\subsection{Our Notations}\label{our-notations}

To discuss models fit with all data versus those with one point removed,
we need clear notation.

\textbf{Full Data Model (Using all \emph{n} observations)}

\begin{itemize}
\tightlist
\item
  \(\hat{\beta}\): The estimated coefficient vector.
\item
  \(\hat{y}_i\): The predicted value for observation \emph{i} from this
  model.
\item
  \(e_i\): The \textbf{ordinary residual} (\(e_i = y_i - \hat{y}_i\)).
\item
  \(\hat{\sigma}\): The estimated standard deviation of the errors
  (Residual Standard Error).
\item
  \(h_{ii}\): The \textbf{leverage} of observation \emph{i}, a measure
  of how much its x-values influence the model.
\end{itemize}

\textbf{Leave-One-Out (LOOCV) Model}

\begin{itemize}
\tightlist
\item
  \(\hat{\beta}_{-i}\): The coefficient vector estimated after
  \textbf{removing} observation \emph{i}.
\item
  \(\hat{y}_{i,-i}\): The predicted value for observation \emph{i}, from
  the model fit \textbf{without} observation \emph{i}.
\item
  \(e_{i,-i}\): The \textbf{deleted residual}
  (\(e_{i,-i} = y_i - \hat{y}_{i,-i}\)).
\item
  \(\hat{\sigma}_{-i}\): The standard deviation of the errors estimated
  from the model fit \textbf{without} observation \emph{i}.
\end{itemize}

\subsection{Non-studentized Residuals}\label{non-studentized-residuals}

Before getting to the correct solution, it's crucial to understand why
simpler methods of standardizing residuals are flawed.

\subsubsection{\texorpdfstring{\textbf{The Ordinary Residual (}\(e_i\)):
Too Small and \(x_i\)
Dependent}{The Ordinary Residual (e\_i): Too Small and x\_i Dependent}}\label{the-ordinary-residual-e_i-too-small-and-x_i-dependent}

The most basic residual, \(e_i\), is problematic for two key reasons.

An outlier has an undue influence on the model, pulling the regression
line towards itself. This makes its own predicted value, \(\hat{y}_i\),
artificially close to its actual value, \(y_i\). As a result, its
residual, \(e_i\), is \textbf{deceptively small} and doesn't reflect the
true magnitude of the error.

The variance of an ordinary residual is not constant; it depends on the
point's leverage. The variance can be derived from the hat matrix
\(H = X(X^\top X)^{-1}X^\top\). Since \[\hat{y} = HY,\] we have
\begin{equation}
e = (I - H)\epsilon.
\end{equation} Thus, \begin{equation}
\mathrm{Var}(e) = (I-H)\,\sigma^2\,(I-H)^\top = (I-H)\sigma^2.
\end{equation} Therefore, for the \(i\)th residual, \begin{equation}
\mathrm{Var}(e_i) = \sigma^2(1 - h_{ii}).
\end{equation}

Since leverage (\(h_{ii}\)) is always greater than 0, the variance of an
ordinary residual is always \textbf{less than the true error variance,}
\(\sigma^2\). High-leverage points act as ``anchors'' for the line and
have even smaller variance.

\subsubsection{\texorpdfstring{\textbf{The LOOCV Residual
(}\(e_{i,-i}\)): Too large and \(x_i\)-Dependent
Variance}{The LOOCV Residual (e\_\{i,-i\}): Too large and x\_i-Dependent Variance}}\label{the-loocv-residual-e_i-i-too-large-and-x_i-dependent-variance}

The deleted residual, \(e_{i,-i}\), solves the ``too small'' problem.
Because the model isn't influenced by the point it's predicting, the
residual is an honest measure of prediction error. However, its variance
is still not constant. The variance of a deleted residual also depends
on leverage, but in the opposite way. \begin{equation}
    \text{Var}(e_{i,-i}) = \frac{\sigma^2}{1 - h_{ii}}
    \end{equation}

From the key identity Equation~\ref{eq-key}, \begin{equation}
e_{i,-i} = \frac{e_i}{1 - h_{ii}}.
\end{equation} Therefore, \begin{equation}
\mathrm{Var}(e_{i,-i}) = \frac{\mathrm{Var}(e_i)}{(1-h_{ii})^2}
= \frac{\sigma^2(1-h_{ii})}{(1-h_{ii})^2}
= \frac{\sigma^2}{1-h_{ii}}.
\end{equation}

Since \(1-h_{ii}\) is less than 1, the variance of a deleted residual is
always \textbf{greater than the true error variance,} \(\sigma^2\). This
is because it has two sources of randomness: the error in the point
itself (\(y_i\)) and the error in the prediction (\(\hat{y}_{i,-i}\)).

\subsection{Studentized Residuals}\label{studentized-residuals}

\subsubsection{Studentized LOOCV (Deleted)
Residual}\label{studentized-loocv-deleted-residual}

The correct solution is to take the LOOCV residual and divide it by its
true standard error, which properly accounts for its larger, x-dependent
variance. This is the \textbf{externally studentized residual}, \(t_i\),
defined as follows:

\begin{equation}\phantomsection\label{eq-sloo}{
t_i = \frac{e_{i,-i}}{\text{SE}(e_{i,-i})} = \frac{e_{i,-i}}{\frac{\hat{\sigma}_{-i}}{\sqrt{1-h_{ii}}}}
}\end{equation} This final value is a reliable diagnostic. Under the
null hypothesis that the observation is not an outlier, it follows a
\textbf{Student's t-distribution} with \(n - p - 1\) degrees of freedom.

\subsubsection{Studentized Full Data
Residuals}\label{studentized-full-data-residuals}

Calculating the conceptual formula appears to require fitting \emph{n}
different regression models---a computationally expensive task.
Fortunately, a mathematical identity allows us to calculate the exact
same value using only the results from the single, full data model.

\begin{equation}\phantomsection\label{eq-sfull}{
t_i = \frac{e_i}{\hat{\sigma}_{-i}\sqrt{1 - h_{ii}}}
}\end{equation} This is not an approximation; it is an \textbf{exact
algebraic rearrangement} of the conceptual definition.

\subsubsection{\texorpdfstring{Equivalence of Equation~\ref{eq-sfull}
and
Equation~\ref{eq-sloo}}{Equivalence of Equation~ and Equation~}}\label{equivalence-of-eq-sfull-and-eq-sloo}

\paragraph{Proof of Equivalence ⚙️}\label{proof-of-equivalence}

Let's start with the conceptual definition of the studentized LOOCV
residuals Equation~\ref{eq-sloo} and show how it transforms into
Equation~\ref{eq-sfull}.

\begin{itemize}
\item
  \textbf{Start with the conceptual LOOCV definition:} \begin{equation}
    t_i = \frac{e_{i,-i}}{\text{SE}(e_{i,-i})} = \frac{e_{i,-i}}{\frac{\hat{\sigma}_{-i}}{\sqrt{1-h_{ii}}}}
    \end{equation}
\item
  \textbf{Substitute the key identity} into the numerator:
  \begin{equation}
    t_i = \frac{\frac{e_i}{1 - h_{ii}}}{\frac{\hat{\sigma}_{-i}}{\sqrt{1 - h_{ii}}}}
    \end{equation}
\item
  \textbf{Simplify the complex fraction.} We can do this by multiplying
  the numerator by the reciprocal of the denominator: \begin{equation}
    t_i = \frac{e_i}{1-h_{ii}} \cdot \frac{\sqrt{1-h_{ii}}}{\hat{\sigma}_{-i}}
    \end{equation}
\item
  \textbf{Cancel the terms.} Since
  \(1 - h_{ii} = (\sqrt{1 - h_{ii}})^2\), one of the
  \(\sqrt{1 - h_{ii}}\) terms in the denominator cancels with the term
  in the numerator. This leaves us with the computational shortcut
  formula: \begin{equation}
    t_i = \frac{e_i}{\hat{\sigma}_{-i}\sqrt{1-h_{ii}}}
    \end{equation}
\end{itemize}

This proves that the two formulas are mathematically identical. The
computational shortcut is simply a clever algebraic rearrangement of the
more intuitive LOOCV definition, allowing for efficient and accurate
calculation. ✅

Of course. Here is the modified \texttt{.qmd} file with the
\textbf{standardized residual} (which you've labeled \textbf{STD-Full})
added to the table, the descriptions, and the plot.

The main changes include:

\begin{enumerate}
\def\labelenumi{\arabic{enumi}.}
\tightlist
\item
  Adding the \texttt{STD-Full} column to the \texttt{residuals\_df} data
  frame using R's \texttt{rstandard()} function.
\item
  Updating the list of calculated columns to include a description of
  \textbf{STD-Full}.
\item
  Modifying the plotting code to include \textbf{STD-Full} with its own
  distinct color and shape.
\end{enumerate}

\subsection{List of Residuals}\label{list-of-residuals}

In this article, we will compare the four residuals given as:

\begin{longtable}[]{@{}
  >{\raggedright\arraybackslash}p{(\columnwidth - 4\tabcolsep) * \real{0.0909}}
  >{\raggedright\arraybackslash}p{(\columnwidth - 4\tabcolsep) * \real{0.3818}}
  >{\raggedright\arraybackslash}p{(\columnwidth - 4\tabcolsep) * \real{0.5273}}@{}}
\caption{}\label{tbl-residuals-five-plausible-residuals}\tabularnewline
\toprule\noalign{}
\begin{minipage}[b]{\linewidth}\raggedright
Short Name
\end{minipage} & \begin{minipage}[b]{\linewidth}\raggedright
Full Name
\end{minipage} & \begin{minipage}[b]{\linewidth}\raggedright
Formula
\end{minipage} \\
\midrule\noalign{}
\endfirsthead
\toprule\noalign{}
\begin{minipage}[b]{\linewidth}\raggedright
Short Name
\end{minipage} & \begin{minipage}[b]{\linewidth}\raggedright
Full Name
\end{minipage} & \begin{minipage}[b]{\linewidth}\raggedright
Formula
\end{minipage} \\
\midrule\noalign{}
\endhead
\bottomrule\noalign{}
\endlastfoot
\textbf{NS-Full} & Non-studentized Full-Data Residual &
\(\frac{e_i}{\hat{\sigma}}\) \\
\textbf{NS-LOO} & Non-studentized LOOCV Residual &
\(\frac{e_{i,-i}}{\hat{\sigma}_{-i}}\) \\
\textbf{ST-LOO} & Studentized LOOCV Residual &
\(\frac{e_{i,-i}}{\hat{\sigma}_{-i}/\sqrt{1-h_{ii}}}\) \\
\textbf{ST-Full} & Studentized Full-Data Residual &
\(\frac{e_i}{\hat{\sigma}_{-i}\sqrt{1-h_{ii}}}\) \\
\textbf{STD-Full} & Standardized (Internal Studentized) Residual &
\(\frac{e_i}{\hat{\sigma}\sqrt{1-h_{ii}}}\) \\
\end{longtable}

\subsection{Example of Various
Residuals}\label{example-of-various-residuals}

\subsubsection{The Linear Model}\label{the-linear-model-1}

The simulation uses a \textbf{simple linear regression model} to
describe the relationship between a single predictor variable, \(x_i\),
and a response variable, \(y_i\). The underlying ``true'' model from
which the data is generated is: \[y_i = 2 + 3x_i + \epsilon_i\] This
means we have a true intercept of 2, a true slope of 3, and a random
error term, \(\epsilon_i\), drawn from a normal distribution with a mean
of 0 and a standard deviation of 5. One artificial outlier is added to
this data to test the behavior of the different residual types. 5
unrelated predictors are added to the dataset.

\begin{Shaded}
\begin{Highlighting}[]
\DocumentationTok{\#\# Load libraries}
\FunctionTok{library}\NormalTok{(dplyr)}
\FunctionTok{library}\NormalTok{(knitr)}

\DocumentationTok{\#\# {-}{-}{-}{-}{-}{-}{-}{-}{-}{-}{-}{-}{-}{-}{-}{-}{-}{-}{-}{-}{-}{-}{-}{-}{-}{-}{-}{-}{-}{-}{-}}
\DocumentationTok{\#\# 1) Data and full{-}model fit}
\DocumentationTok{\#\# {-}{-}{-}{-}{-}{-}{-}{-}{-}{-}{-}{-}{-}{-}{-}{-}{-}{-}{-}{-}{-}{-}{-}{-}{-}{-}{-}{-}{-}{-}{-}}
\FunctionTok{set.seed}\NormalTok{(}\DecValTok{123}\NormalTok{)}
\NormalTok{n }\OtherTok{\textless{}{-}} \DecValTok{20}
\NormalTok{x }\OtherTok{\textless{}{-}} \DecValTok{1}\SpecialCharTok{:}\NormalTok{n}
\NormalTok{y }\OtherTok{\textless{}{-}} \DecValTok{2} \SpecialCharTok{+} \DecValTok{3} \SpecialCharTok{*}\NormalTok{ x }\SpecialCharTok{+} \FunctionTok{rnorm}\NormalTok{(n, }\AttributeTok{mean =} \DecValTok{0}\NormalTok{, }\AttributeTok{sd =} \DecValTok{5}\NormalTok{)}
\NormalTok{y[}\DecValTok{1}\NormalTok{] }\OtherTok{\textless{}{-}}\NormalTok{ y[}\DecValTok{1}\NormalTok{] }\SpecialCharTok{+} \DecValTok{30}  
\CommentTok{\#y[11] \textless{}{-} y[11] {-} 30 }
\NormalTok{o.index }\OtherTok{\textless{}{-}} \FunctionTok{c}\NormalTok{(}\DecValTok{1}\NormalTok{)}
\NormalTok{flag.outlier }\OtherTok{\textless{}{-}} \FunctionTok{rep}\NormalTok{(}\DecValTok{20}\NormalTok{, n)}
\NormalTok{flag.outlier[o.index] }\OtherTok{\textless{}{-}} \DecValTok{2}
\NormalTok{full\_data  }\OtherTok{\textless{}{-}} \FunctionTok{data.frame}\NormalTok{(}\AttributeTok{x =}\NormalTok{ x, }\FunctionTok{replicate}\NormalTok{(}\DecValTok{5}\NormalTok{, }\FunctionTok{rnorm}\NormalTok{ (n)), }\AttributeTok{y =}\NormalTok{ y)}

\FunctionTok{plot}\NormalTok{(y}\SpecialCharTok{\textasciitilde{}}\NormalTok{x, }\AttributeTok{data=}\NormalTok{full\_data, }\AttributeTok{pch=}\NormalTok{ flag.outlier, }\AttributeTok{col=}\NormalTok{flag.outlier)}
\NormalTok{fit }\OtherTok{\textless{}{-}} \FunctionTok{lm}\NormalTok{(y}\SpecialCharTok{\textasciitilde{}}\NormalTok{., }\AttributeTok{data=}\NormalTok{full\_data)}
\FunctionTok{abline}\NormalTok{ (fit)}
\FunctionTok{abline}\NormalTok{ (}\AttributeTok{a=}\DecValTok{2}\NormalTok{, }\AttributeTok{b=}\DecValTok{3}\NormalTok{, }\AttributeTok{col=}\StringTok{"red"}\NormalTok{, }\AttributeTok{lwd=}\DecValTok{2}\NormalTok{)}
\end{Highlighting}
\end{Shaded}

\includegraphics{unit3-mlr/residuals_files/figure-pdf/unnamed-chunk-1-1.pdf}

\subsubsection{Description of Calculated
Columns}\label{description-of-calculated-columns}

The final table compiles several important quantities calculated during
the simulation. Here's what each column represents:

\begin{itemize}
\tightlist
\item
  \(x_i\): The predictor variable, which is simply the index of the
  observation from 1 to 20.
\item
  \(h_i\): The \textbf{leverage} of the i-th observation. It measures
  how influential a point's x-value is in determining the model's fit. A
  higher value indicates a more influential point.
\item
  \(e_i\): The \textbf{ordinary residual}, calculated as the difference
  between the actual value (\(y_i\)) and the predicted value
  (\(\hat{y}_i\)) from the model fit on all data.
\item
  \(\hat{\sigma}\): The \textbf{residual standard error} (or Root Mean
  Square Error) of the full model, representing the typical size of an
  ordinary residual.
\item
  \(e_{i,-i}\): The \textbf{deleted (or LOOCV) residual}. This is the
  difference between the actual value (\(y_i\)) and the value predicted
  for it by a model that was fit on all other data \emph{except} point
  \emph{i}.
\item
  \(\hat{e}_{i,-i}\): This column shows the deleted residual calculated
  using the efficient algebraic shortcut (\(e_i / (1-h_{ii})\)),
  verifying it's identical to the brute-force \(e_{i,-i}\).
\item
  \(\hat{\sigma}_{-i}\): The \textbf{LOOCV residual standard error},
  calculated from a model that was fit after removing observation
  \emph{i}.
\item
  \(\tilde{\sigma}_{-i}\): The \textbf{LOOCV residual standard error},
  calculated from the shortcut formula \textbf{?@eq-sigma}\_-i.
\item
  \textbf{NS-Full}: The \textbf{Non-studentized Full-Data Residual},
  calculated as the ordinary residual divided by the full model's
  standard error (\(e_i / \hat{\sigma}\)).
\item
  \textbf{NS-LOO}: The \textbf{Non-studentized LOOCV Residual},
  calculated as the deleted residual divided by the corresponding LOOCV
  standard error (\(e_{i,-i} / \hat{\sigma}_{-i}\)).
\item
  \textbf{STD-Full}: The \textbf{Standardized (or Internally
  Studentized) Residual}, calculated as the ordinary residual divided by
  its estimated standard error
  (\(e_i / (\hat{\sigma}\sqrt{1-h_{ii}})\)). This is provided by R's
  \texttt{rstandard()} function.
\item
  \textbf{ST-LOO}: The \textbf{Studentized LOOCV Residual}, calculated
  using the conceptual formula by dividing the deleted residual by its
  true standard error.
\item
  \textbf{ST-Full}: The \textbf{Studentized Full-Data Residual},
  calculated using the efficient shortcut formula, which is provided by
  R's \texttt{rstudent()} function.
\end{itemize}

\begin{Shaded}
\begin{Highlighting}[]
\FunctionTok{library}\NormalTok{(kableExtra)}

\NormalTok{full\_model }\OtherTok{\textless{}{-}} \FunctionTok{lm}\NormalTok{(y }\SpecialCharTok{\textasciitilde{}}\NormalTok{ ., }\AttributeTok{data =}\NormalTok{ full\_data)}
\NormalTok{p }\OtherTok{\textless{}{-}} \FunctionTok{length}\NormalTok{(}\FunctionTok{coef}\NormalTok{(full\_model))}
\NormalTok{leverage    }\OtherTok{\textless{}{-}} \FunctionTok{hatvalues}\NormalTok{(full\_model)}
\NormalTok{e\_full      }\OtherTok{\textless{}{-}} \FunctionTok{resid}\NormalTok{(full\_model)}
\NormalTok{sigma\_hat\_val  }\OtherTok{\textless{}{-}} \FunctionTok{summary}\NormalTok{(full\_model)}\SpecialCharTok{$}\NormalTok{sigma}

\NormalTok{rss\_full }\OtherTok{\textless{}{-}} \FunctionTok{sum}\NormalTok{(e\_full}\SpecialCharTok{\^{}}\DecValTok{2}\NormalTok{)}
\NormalTok{df\_loo }\OtherTok{\textless{}{-}}\NormalTok{ n }\SpecialCharTok{{-}}\NormalTok{ p }\SpecialCharTok{{-}} \DecValTok{1}
\NormalTok{sigma\_minus\_i\_shortcut }\OtherTok{\textless{}{-}} \FunctionTok{sqrt}\NormalTok{((rss\_full }\SpecialCharTok{{-}}\NormalTok{ (e\_full}\SpecialCharTok{\^{}}\DecValTok{2} \SpecialCharTok{/}\NormalTok{ (}\DecValTok{1} \SpecialCharTok{{-}}\NormalTok{ leverage))) }\SpecialCharTok{/}\NormalTok{ df\_loo)}

\DocumentationTok{\#\# {-}{-}{-}{-}{-}{-}{-}{-}{-}{-}{-}{-}{-}{-}{-}{-}{-}{-}{-}{-}{-}{-}{-}{-}{-}{-}{-}{-}{-}{-}{-}}
\DocumentationTok{\#\# 2) LOOCV quantities (refit n times)}
\DocumentationTok{\#\# {-}{-}{-}{-}{-}{-}{-}{-}{-}{-}{-}{-}{-}{-}{-}{-}{-}{-}{-}{-}{-}{-}{-}{-}{-}{-}{-}{-}{-}{-}{-}}
\NormalTok{e\_del\_val }\OtherTok{\textless{}{-}} \FunctionTok{numeric}\NormalTok{(n)}
\NormalTok{sigma\_minus\_i\_val }\OtherTok{\textless{}{-}} \FunctionTok{numeric}\NormalTok{(n)}

\ControlFlowTok{for}\NormalTok{ (i }\ControlFlowTok{in} \DecValTok{1}\SpecialCharTok{:}\NormalTok{n) \{}
\NormalTok{  loocv\_model }\OtherTok{\textless{}{-}} \FunctionTok{lm}\NormalTok{(y }\SpecialCharTok{\textasciitilde{}}\NormalTok{ ., }\AttributeTok{data =}\NormalTok{ full\_data[}\SpecialCharTok{{-}}\NormalTok{i, ])}
\NormalTok{  yhat\_minus  }\OtherTok{\textless{}{-}} \FunctionTok{predict}\NormalTok{(loocv\_model, }\AttributeTok{newdata =}\NormalTok{ full\_data[i, , }\AttributeTok{drop =} \ConstantTok{FALSE}\NormalTok{])}
\NormalTok{  e\_del\_val[i]    }\OtherTok{\textless{}{-}}\NormalTok{ full\_data}\SpecialCharTok{$}\NormalTok{y[i] }\SpecialCharTok{{-}}\NormalTok{ yhat\_minus}
\NormalTok{  sigma\_minus\_i\_val[i] }\OtherTok{\textless{}{-}} \FunctionTok{summary}\NormalTok{(loocv\_model)}\SpecialCharTok{$}\NormalTok{sigma}
\NormalTok{\}}

\DocumentationTok{\#\# {-}{-}{-}{-}{-}{-}{-}{-}{-}{-}{-}{-}{-}{-}{-}{-}{-}{-}{-}{-}{-}{-}{-}{-}{-}{-}{-}{-}{-}{-}{-}}
\DocumentationTok{\#\# 3) Assemble and round results}
\DocumentationTok{\#\# {-}{-}{-}{-}{-}{-}{-}{-}{-}{-}{-}{-}{-}{-}{-}{-}{-}{-}{-}{-}{-}{-}{-}{-}{-}{-}{-}{-}{-}{-}{-}}
\NormalTok{residuals\_df }\OtherTok{\textless{}{-}} \FunctionTok{data.frame}\NormalTok{(}
  \AttributeTok{x =}\NormalTok{ full\_data}\SpecialCharTok{$}\NormalTok{x,}
  \AttributeTok{h =} \FunctionTok{as.numeric}\NormalTok{(leverage),}
  \AttributeTok{e\_i =} \FunctionTok{as.numeric}\NormalTok{(e\_full),}
  \AttributeTok{sigma\_hat =} \FunctionTok{as.numeric}\NormalTok{(sigma\_hat\_val),}
  \AttributeTok{e\_i\_minus\_i =} \FunctionTok{as.numeric}\NormalTok{(e\_del\_val),}
  \AttributeTok{e\_i\_minus\_i\_2 =}\NormalTok{ e\_full}\SpecialCharTok{/}\NormalTok{(}\DecValTok{1}\SpecialCharTok{{-}}\NormalTok{leverage),}
  \AttributeTok{sigma\_minus\_i =} \FunctionTok{as.numeric}\NormalTok{(sigma\_minus\_i\_val),}
  \AttributeTok{sigma\_minus\_i\_shortcut =} \FunctionTok{as.numeric}\NormalTok{(sigma\_minus\_i\_shortcut),}
  \StringTok{\textasciigrave{}}\AttributeTok{NS{-}Full}\StringTok{\textasciigrave{}} \OtherTok{=}\NormalTok{ e\_full }\SpecialCharTok{/}\NormalTok{ sigma\_hat\_val,}
  \StringTok{\textasciigrave{}}\AttributeTok{NS{-}LOO}\StringTok{\textasciigrave{}} \OtherTok{=}\NormalTok{ e\_del\_val }\SpecialCharTok{/}\NormalTok{ sigma\_minus\_i\_val,}
  \StringTok{\textasciigrave{}}\AttributeTok{STD{-}Full}\StringTok{\textasciigrave{}} \OtherTok{=} \FunctionTok{rstandard}\NormalTok{(full\_model), }\CommentTok{\# \textless{}{-}{-} ADDED STANDARDIZED RESIDUAL}
  \StringTok{\textasciigrave{}}\AttributeTok{ST{-}LOO}\StringTok{\textasciigrave{}} \OtherTok{=}\NormalTok{ e\_del\_val }\SpecialCharTok{/}\NormalTok{ (sigma\_minus\_i\_val }\SpecialCharTok{/} \FunctionTok{sqrt}\NormalTok{(}\DecValTok{1} \SpecialCharTok{{-}}\NormalTok{ leverage)),}
  \StringTok{\textasciigrave{}}\AttributeTok{ST{-}Full}\StringTok{\textasciigrave{}} \OtherTok{=} \FunctionTok{rstudent}\NormalTok{(full\_model)}
\NormalTok{) }\SpecialCharTok{\%\textgreater{}\%}
  \FunctionTok{mutate}\NormalTok{(}\FunctionTok{across}\NormalTok{(}\FunctionTok{where}\NormalTok{(is.numeric), }\SpecialCharTok{\textasciitilde{}} \FunctionTok{round}\NormalTok{(.x, }\DecValTok{3}\NormalTok{)))}


\DocumentationTok{\#\# 4) Create simple display names}
\NormalTok{display\_names }\OtherTok{\textless{}{-}} \FunctionTok{c}\NormalTok{(}\StringTok{"$x\_i$"}\NormalTok{, }\StringTok{"$h\_i$"}\NormalTok{, }\StringTok{"$e\_i$"}\NormalTok{, }\StringTok{"$}\SpecialCharTok{\textbackslash{}\textbackslash{}}\StringTok{hat\{}\SpecialCharTok{\textbackslash{}\textbackslash{}}\StringTok{sigma\}$"}\NormalTok{, }
                   \StringTok{"$e\_\{i,{-}i\}$"}\NormalTok{, }\StringTok{"$}\SpecialCharTok{\textbackslash{}\textbackslash{}}\StringTok{hat\{e\}\_\{i,{-}i\}$"}\NormalTok{,}
                   \StringTok{"$}\SpecialCharTok{\textbackslash{}\textbackslash{}}\StringTok{hat\{}\SpecialCharTok{\textbackslash{}\textbackslash{}}\StringTok{sigma\}\_\{{-}i\}$"}\NormalTok{, }\StringTok{"$}\SpecialCharTok{\textbackslash{}\textbackslash{}}\StringTok{tilde\{}\SpecialCharTok{\textbackslash{}\textbackslash{}}\StringTok{sigma\}\_\{{-}i\}$"}\NormalTok{, }
                   \StringTok{"NS{-}Full"}\NormalTok{, }\StringTok{"NS{-}LOO"}\NormalTok{, }\StringTok{"STD{-}Full"}\NormalTok{, }\StringTok{"ST{-}LOO"}\NormalTok{, }\StringTok{"ST{-}Full"}\NormalTok{) }\CommentTok{\# \textless{}{-}{-} ADDED LABEL}

\DocumentationTok{\#\# 5) Display the table}
\DocumentationTok{\#\# Conditional check for output format}
\ControlFlowTok{if}\NormalTok{ (knitr}\SpecialCharTok{::}\FunctionTok{is\_html\_output}\NormalTok{()) \{}
  \DocumentationTok{\#\# {-}{-}{-} Code for HTML Output (using kableExtra) {-}{-}{-}}
\NormalTok{  knitr}\SpecialCharTok{::}\FunctionTok{kable}\NormalTok{(}
\NormalTok{    residuals\_df,}
    \AttributeTok{caption =} \StringTok{"Residual variants"}\NormalTok{,}
    \AttributeTok{col.names =}\NormalTok{ display\_names,}
    \AttributeTok{align =} \StringTok{"r"}\NormalTok{,}
    \CommentTok{\#format="html",}
    \AttributeTok{escape =} \ConstantTok{FALSE} \CommentTok{\# Allows LaTeX and \textless{}br/\textgreater{} to render}
\NormalTok{  ) }
\NormalTok{\} }\ControlFlowTok{else}\NormalTok{ \{}
  \DocumentationTok{\#\# {-}{-}{-} Code for PDF/Other Output (using kableExtra) {-}{-}{-}}
\NormalTok{  knitr}\SpecialCharTok{::}\FunctionTok{kable}\NormalTok{(}
\NormalTok{    residuals\_df,}
    \AttributeTok{caption =} \StringTok{"Residual variants."}\NormalTok{,}
    \AttributeTok{col.names =}\NormalTok{ display\_names,}
    \AttributeTok{align =} \StringTok{"r"}\NormalTok{,}
    \AttributeTok{format =} \StringTok{"latex"}\NormalTok{,}
    \AttributeTok{booktabs =} \ConstantTok{TRUE}\NormalTok{,}
    \AttributeTok{escape =} \ConstantTok{FALSE} \CommentTok{\# Allows LaTeX and \textbackslash{}\textbackslash{} to render}
\NormalTok{  ) }\SpecialCharTok{\%\textgreater{}\%}
    \FunctionTok{kable\_styling}\NormalTok{(}
      \AttributeTok{latex\_options =} \StringTok{"scale\_down"}
\NormalTok{    )}
\NormalTok{\}}
\end{Highlighting}
\end{Shaded}

\begin{table}
\centering
\caption{Residual variants.}
\centering
\resizebox{\ifdim\width>\linewidth\linewidth\else\width\fi}{!}{
\begin{tabular}[t]{rrrrrrrrrrrrr}
\toprule
$x_i$ & $h_i$ & $e_i$ & $\hat{\sigma}$ & $e_{i,-i}$ & $\hat{e}_{i,-i}$ & $\hat{\sigma}_{-i}$ & $\tilde{\sigma}_{-i}$ & NS-Full & NS-LOO & STD-Full & ST-LOO & ST-Full\\
\midrule
1 & 0.247 & 17.638 & 7.57 & 23.434 & 23.434 & 5.257 & 5.257 & 2.330 & 4.457 & 2.686 & 3.867 & 3.867\\
2 & 0.241 & -7.909 & 7.57 & -10.418 & -10.418 & 7.431 & 7.431 & -1.045 & -1.402 & -1.199 & -1.222 & -1.222\\
3 & 0.364 & -3.271 & 7.57 & -5.147 & -5.147 & 7.790 & 7.790 & -0.432 & -0.661 & -0.542 & -0.527 & -0.527\\
4 & 0.555 & 1.553 & 7.57 & 3.490 & 3.490 & 7.851 & 7.851 & 0.205 & 0.445 & 0.308 & 0.297 & 0.297\\
5 & 0.257 & -1.515 & 7.57 & -2.038 & -2.038 & 7.863 & 7.863 & -0.200 & -0.259 & -0.232 & -0.223 & -0.223\\
\addlinespace
6 & 0.400 & -0.400 & 7.57 & -0.666 & -0.666 & 7.878 & 7.878 & -0.053 & -0.085 & -0.068 & -0.066 & -0.066\\
7 & 0.253 & 0.247 & 7.57 & 0.331 & 0.331 & 7.879 & 7.879 & 0.033 & 0.042 & 0.038 & 0.036 & 0.036\\
8 & 0.303 & -8.972 & 7.57 & -12.871 & -12.871 & 7.243 & 7.243 & -1.185 & -1.777 & -1.419 & -1.484 & -1.484\\
9 & 0.347 & -8.394 & 7.57 & -12.852 & -12.852 & 7.287 & 7.287 & -1.109 & -1.764 & -1.372 & -1.425 & -1.425\\
10 & 0.573 & -5.696 & 7.57 & -13.342 & -13.342 & 7.467 & 7.467 & -0.752 & -1.787 & -1.152 & -1.168 & -1.168\\
\addlinespace
11 & 0.117 & 7.175 & 7.57 & 8.127 & 8.127 & 7.565 & 7.565 & 0.948 & 1.074 & 1.009 & 1.009 & 1.009\\
12 & 0.441 & 0.998 & 7.57 & 1.786 & 1.786 & 7.870 & 7.870 & 0.132 & 0.227 & 0.176 & 0.170 & 0.170\\
13 & 0.359 & 1.298 & 7.57 & 2.026 & 2.026 & 7.865 & 7.865 & 0.171 & 0.258 & 0.214 & 0.206 & 0.206\\
14 & 0.395 & 0.698 & 7.57 & 1.153 & 1.153 & 7.875 & 7.875 & 0.092 & 0.146 & 0.119 & 0.114 & 0.114\\
15 & 0.228 & -1.221 & 7.57 & -1.583 & -1.583 & 7.869 & 7.869 & -0.161 & -0.201 & -0.184 & -0.177 & -0.177\\
\addlinespace
16 & 0.402 & 7.998 & 7.57 & 13.365 & 13.365 & 7.292 & 7.292 & 1.057 & 1.833 & 1.366 & 1.418 & 1.418\\
17 & 0.433 & 3.996 & 7.57 & 7.048 & 7.048 & 7.729 & 7.729 & 0.528 & 0.912 & 0.701 & 0.687 & 0.687\\
18 & 0.414 & -3.369 & 7.57 & -5.746 & -5.746 & 7.776 & 7.776 & -0.445 & -0.739 & -0.581 & -0.566 & -0.566\\
19 & 0.258 & 3.073 & 7.57 & 4.141 & 4.141 & 7.812 & 7.812 & 0.406 & 0.530 & 0.471 & 0.457 & 0.457\\
20 & 0.414 & -3.930 & 7.57 & -6.707 & -6.707 & 7.739 & 7.739 & -0.519 & -0.867 & -0.678 & -0.663 & -0.663\\
\bottomrule
\end{tabular}}
\end{table}

\begin{Shaded}
\begin{Highlighting}[]
\DocumentationTok{\#\# Load libraries}
\FunctionTok{library}\NormalTok{(dplyr)}
\FunctionTok{library}\NormalTok{(tidyr)}
\FunctionTok{library}\NormalTok{(ggplot2)}
\FunctionTok{library}\NormalTok{(knitr)}


\DocumentationTok{\#\# {-}{-}{-}{-}{-}{-}{-}{-}{-}{-}{-}{-}{-}{-}{-}{-}{-}{-}{-}{-}{-}{-}{-}{-}{-}{-}{-}{-}{-}{-}{-}}
\DocumentationTok{\#\# 3) Plotting Code with Updated Names}
\DocumentationTok{\#\# {-}{-}{-}{-}{-}{-}{-}{-}{-}{-}{-}{-}{-}{-}{-}{-}{-}{-}{-}{-}{-}{-}{-}{-}{-}{-}{-}{-}{-}{-}{-}}

\DocumentationTok{\#\# Prepare data for plotting}
\NormalTok{plot\_df }\OtherTok{\textless{}{-}}\NormalTok{ residuals\_df }\SpecialCharTok{\%\textgreater{}\%}
  \DocumentationTok{\#\# Use the new, simple column names (R converts \textquotesingle{}{-}\textquotesingle{} to \textquotesingle{}.\textquotesingle{})}
  \FunctionTok{select}\NormalTok{(}
\NormalTok{    x,}
\NormalTok{    NS.Full,}
\NormalTok{    NS.LOO,}
\NormalTok{    STD.Full, }\CommentTok{\# \textless{}{-}{-} ADDED FOR PLOTTING}
\NormalTok{    ST.LOO,}
\NormalTok{    ST.Full}
\NormalTok{  ) }\SpecialCharTok{\%\textgreater{}\%}
  \FunctionTok{pivot\_longer}\NormalTok{(}
    \AttributeTok{cols =} \SpecialCharTok{{-}}\NormalTok{x,}
    \AttributeTok{names\_to =} \StringTok{"residual\_type"}\NormalTok{,}
    \AttributeTok{values\_to =} \StringTok{"residual\_value"}
\NormalTok{  )}

\DocumentationTok{\#\# Update the names in the mapping vectors}
\NormalTok{shape\_map }\OtherTok{\textless{}{-}} \FunctionTok{c}\NormalTok{(}
  \AttributeTok{NS.Full  =} \DecValTok{16}\NormalTok{,  }\CommentTok{\# solid circle}
  \AttributeTok{NS.LOO   =} \DecValTok{1}\NormalTok{,   }\CommentTok{\# hollow circle}
  \AttributeTok{STD.Full =} \DecValTok{2}\NormalTok{,   }\CommentTok{\# hollow triangle \textless{}{-}{-} ADDED}
  \AttributeTok{ST.LOO   =} \DecValTok{6}\NormalTok{,   }\CommentTok{\# asterisk}
  \AttributeTok{ST.Full  =} \DecValTok{10}   \CommentTok{\# asterisk}
\NormalTok{)}

\NormalTok{labels\_map }\OtherTok{\textless{}{-}} \FunctionTok{c}\NormalTok{(}
  \AttributeTok{NS.Full  =} \StringTok{"NS{-}Full"}\NormalTok{,}
  \AttributeTok{NS.LOO   =} \StringTok{"NS{-}LOO"}\NormalTok{,}
  \AttributeTok{STD.Full =} \StringTok{"STD{-}Full"}\NormalTok{, }\CommentTok{\# \textless{}{-}{-} ADDED}
  \AttributeTok{ST.LOO   =} \StringTok{"ST{-}LOO"}\NormalTok{,}
  \AttributeTok{ST.Full  =} \StringTok{"ST{-}Full"}
\NormalTok{)}

\NormalTok{color\_map }\OtherTok{\textless{}{-}} \FunctionTok{c}\NormalTok{(}
  \AttributeTok{NS.Full  =} \StringTok{"\#1f77b4"}\NormalTok{,  }\CommentTok{\# blue}
  \AttributeTok{NS.LOO   =} \StringTok{"\#ff7f0e"}\NormalTok{, }\CommentTok{\# orange}
  \AttributeTok{STD.Full =} \StringTok{"\#9467bd"}\NormalTok{,  }\CommentTok{\# purple \textless{}{-}{-} ADDED}
  \AttributeTok{ST.LOO   =} \StringTok{"\#2ca02c"}\NormalTok{,  }\CommentTok{\# green}
  \AttributeTok{ST.Full  =} \StringTok{"\#d62728"}   \CommentTok{\# red}
\NormalTok{)}

\DocumentationTok{\#\# Generate the plot}
\FunctionTok{ggplot}\NormalTok{(}
\NormalTok{  plot\_df,}
  \FunctionTok{aes}\NormalTok{(}\AttributeTok{x =}\NormalTok{ x, }\AttributeTok{y =}\NormalTok{ residual\_value,}
      \AttributeTok{shape =}\NormalTok{ residual\_type, }\AttributeTok{color =}\NormalTok{ residual\_type, }\AttributeTok{group =}\NormalTok{ residual\_type)}
\NormalTok{) }\SpecialCharTok{+}
  \FunctionTok{geom\_hline}\NormalTok{(}\AttributeTok{yintercept =} \DecValTok{0}\NormalTok{, }\AttributeTok{linetype =} \StringTok{"dashed"}\NormalTok{) }\SpecialCharTok{+}
  \FunctionTok{geom\_point}\NormalTok{(}\AttributeTok{size =} \DecValTok{3}\NormalTok{, }\AttributeTok{stroke =} \FloatTok{1.2}\NormalTok{) }\SpecialCharTok{+} \CommentTok{\# Increased stroke for visibility}
  \FunctionTok{scale\_shape\_manual}\NormalTok{(}
    \AttributeTok{values =}\NormalTok{ shape\_map,}
    \AttributeTok{breaks =} \FunctionTok{names}\NormalTok{(labels\_map),}
    \AttributeTok{labels =} \FunctionTok{unname}\NormalTok{(labels\_map),}
    \AttributeTok{name =} \StringTok{"Residual Type"}
\NormalTok{  ) }\SpecialCharTok{+}
  \FunctionTok{scale\_color\_manual}\NormalTok{(}
    \AttributeTok{values =}\NormalTok{ color\_map,}
    \AttributeTok{breaks =} \FunctionTok{names}\NormalTok{(labels\_map),}
    \AttributeTok{labels =} \FunctionTok{unname}\NormalTok{(labels\_map),}
    \AttributeTok{name =} \StringTok{"Residual Type"}
\NormalTok{  ) }\SpecialCharTok{+}
  \FunctionTok{labs}\NormalTok{(}
    \AttributeTok{title =} \StringTok{"Five Residual Variants vs x"}\NormalTok{,}
    \AttributeTok{x =} \StringTok{"x\_i"}\NormalTok{,}
    \AttributeTok{y =} \StringTok{"Residual value"}
\NormalTok{  ) }\SpecialCharTok{+}
  \FunctionTok{theme\_bw}\NormalTok{() }\SpecialCharTok{+}
  \FunctionTok{theme}\NormalTok{(}
    \AttributeTok{legend.position =} \StringTok{"right"}\NormalTok{,}
    \AttributeTok{legend.title =} \FunctionTok{element\_text}\NormalTok{(}\AttributeTok{face =} \StringTok{"bold"}\NormalTok{)}
\NormalTok{  )}
\end{Highlighting}
\end{Shaded}

\begin{figure}[H]

{\centering \includegraphics{unit3-mlr/residuals_files/figure-pdf/plot-five-residuals-new-names-1.pdf}

}

\caption{Five residual variants plotted against the predictor variable
x.}

\end{figure}%

From the above simulation results, we observe the following important
facts:

\begin{itemize}
\item
  \textbf{Leverage and Influence:} The simulation confirms that leverage
  (\(h_{ii}\)) measures an observation's influence on the model's
  coefficients. It shows that points with higher leverage pull the
  regression line toward them, resulting in smaller, deceptively
  conservative full-data residuals (\(e_i\)).
\item
  \textbf{Conservative Residuals:} The study highlights that the
  ordinary residual (\(e_i\)) is a ``conservative'' measure of error
  because its value for an outlier is systematically reduced by that
  same outlier's influence on the model.
\item
  \textbf{Identity Verification:} The numerical results validated the
  key algebraic identity that connects the full-data residual (\(e_i\))
  to the leave-one-out (deleted) residual (\(e_{i,-i}\)), as well as the
  identity for calculating the LOOCV standard error
  (\(\hat{\sigma}_{-i}\)) from the full model's statistics. This
  demonstrates that all key LOOCV errors can be calculated efficiently
  from a single model fit.
\item
  \textbf{Effective Studentization:} The final step of studentization,
  which uses leverage to properly scale the residuals, is shown to be
  crucial. It successfully transforms the residuals into a reliable
  diagnostic tool with a constant variance across all predictor values
  (\(x_i\)), causing them to behave much more like a standard normal or
  t-distribution.
\end{itemize}

\section{Cook's Distance}\label{cooks-distance}

\subsection{Definition from the change in
coefficients}\label{definition-from-the-change-in-coefficients}

Let \(\hat{\boldsymbol\beta}\) be the OLS estimate on all \(n\) cases
and \(\hat{\boldsymbol\beta}_{(-i)}\) the estimate after deleting case
\(i\).\\
With \(p\) parameters (including intercept) and
\(\hat\sigma^2=\mathrm{MSE}\),

\[
D_i
= \frac{\big(\hat{\boldsymbol\beta}-\hat{\boldsymbol\beta}_{(-i)}\big)^{\!\top}
(X^{\!\top}X)\,\big(\hat{\boldsymbol\beta}-\hat{\boldsymbol\beta}_{(-i)}\big)}{p\,\hat\sigma^{2}} \, .
\]

This measures how far the whole coefficient vector moves (in the
\(X^{\!\top}X\) metric) when case \(i\) is removed, scaled \textbf{per
parameter}.

\subsubsection{\texorpdfstring{Express \(D_i\) via the
\textbf{studentized LOOCV residual}:
\(t_i\)}{Express D\_i via the studentized LOOCV residual: t\_i}}\label{express-d_i-via-the-studentized-loocv-residual-t_i}

Let \(h_{ii}\) be the leverage and define the LOOCV quantities
\(e_{i,-i}\) and \(\hat\sigma_{-i}\) from the model refit
\textbf{without} case \(i\).\\
The externally studentized residual is

\[
t_i
=\frac{e_i}{\hat{\sigma}_{-i}\sqrt{\,1-h_{ii}\,}}
=\frac{e_{i,-i}\,\sqrt{\,1-h_{ii}\,}}{\hat{\sigma}_{-i}},
\qquad\text{since}\quad e_{i,-i}=\frac{e_i}{1-h_{ii}}.
\]

Then Cook's distance can be written as

\[
D_i
= \frac{n-p}{p}\;\frac{h_{ii}}{1-h_{ii}}\;
\frac{t_i^{\,2}}{(n-p-1)+t_i^{\,2}} \, .
\]

\subsubsection{Exact null distribution}\label{exact-null-distribution}

Under the classical linear model,

\[
t_i^{\,2}\ \sim\ F_{1,\ \nu},\qquad \nu=n-p-1 .
\]

Let

\[
c_i=\frac{n-p}{p}\cdot\frac{h_{ii}}{1-h_{ii}},\qquad
W_i=\frac{t_i^{\,2}}{\nu+t_i^{\,2}} .
\]

Because \(t_i^{\,2}\sim F_{1,\nu}\),

\[
W_i\sim \mathrm{Beta}\!\Big(\tfrac12,\tfrac{\nu}{2}\Big),
\qquad
\frac{D_i}{c_i}=W_i\in[0,1] \, .
\]

This yields exact, per--case \(p\)-values and critical values:

\[
p_i=\Pr\!\left(W_i\ge \frac{D_i}{c_i}\right)
= S_\mathrm{Beta}\!\left(\frac{D_i}{c_i};\,\tfrac12,\,\tfrac{n-p-1}{2}\right),
\]

\[
d_{i,\alpha}=c_i\ q_\mathrm{Beta}\!\Big(1-\alpha;\tfrac12,\tfrac{n-p-1}{2}\Big).
\]

where \(S_\mathrm{Beta}\) and \(q_\mathrm{Beta}\) stand for the survival
and quantile functions of Beta distribution.

\subsubsection{\texorpdfstring{The rough \(4/n\) rule (average-leverage
simplification)}{The rough 4/n rule (average-leverage simplification)}}\label{the-rough-4n-rule-average-leverage-simplification}

Approximating a ``typical'' case by \textbf{average leverage}
\(h_{ii}\approx p/n\) gives

\[
c_i=\frac{n-p}{p}\cdot\frac{p/n}{1-p/n}\approx 1 .
\]

The 95th percentile of \(W_i\) is

\[
\mathrm{qbeta}\!\Big(0.95;\tfrac12,\tfrac{n-p-1}{2}\Big)
\ \approx\ \frac{F_{1,\nu,\,0.95}}{\nu+F_{1,\nu,\,0.95}}
\ \approx\ \frac{3.84}{\,n-p-1\,}
\ \approx\ \frac{4}{n}\quad (\text{when }p\ll n).
\]

So \(4/n\) is a \textbf{rule-of-thumb} 95\% cutoff for an
\textbf{average-leverage} point; the exact leverage--aware cutoff is
\(d_{i,\alpha}\) above (larger for high \(h_{ii}\), smaller for low
\(h_{ii}\)).

\subsubsection{\texorpdfstring{Comparing \(4/n\) rules with the actual
critical
values}{Comparing 4/n rules with the actual critical values}}\label{comparing-4n-rules-with-the-actual-critical-values}

\begin{Shaded}
\begin{Highlighting}[]
\FunctionTok{library}\NormalTok{(dplyr)}
\FunctionTok{library}\NormalTok{(tidyr)}
\FunctionTok{library}\NormalTok{(knitr)}

\DocumentationTok{\#\# Exact 95\% Cook\textquotesingle{}s D cutoff under average leverage h\_ii = p/n (so c\_i = 1):}
\DocumentationTok{\#\# d\_\{i,0.95\} = qbeta(0.95; 1/2, (n {-} p {-} 1)/2), valid when nu = n {-} p {-} 1 \textgreater{} 0}
\NormalTok{cook\_crit\_avg }\OtherTok{\textless{}{-}} \ControlFlowTok{function}\NormalTok{(n, p, }\AttributeTok{alpha =} \FloatTok{0.05}\NormalTok{) \{}
\NormalTok{  nu }\OtherTok{\textless{}{-}}\NormalTok{ n }\SpecialCharTok{{-}}\NormalTok{ p }\SpecialCharTok{{-}} \DecValTok{1}
  \ControlFlowTok{if}\NormalTok{ (nu }\SpecialCharTok{\textless{}=} \DecValTok{0}\NormalTok{) }\FunctionTok{return}\NormalTok{(}\ConstantTok{NA\_real\_}\NormalTok{)}
\NormalTok{  stats}\SpecialCharTok{::}\FunctionTok{qbeta}\NormalTok{(}\DecValTok{1} \SpecialCharTok{{-}}\NormalTok{ alpha, }\AttributeTok{shape1 =} \FloatTok{0.5}\NormalTok{, }\AttributeTok{shape2 =}\NormalTok{ nu }\SpecialCharTok{/} \DecValTok{2}\NormalTok{)}
\NormalTok{\}}

\DocumentationTok{\#\# Grids (edit as needed)}
\NormalTok{n\_vals }\OtherTok{\textless{}{-}} \FunctionTok{c}\NormalTok{(}\DecValTok{20}\NormalTok{, }\DecValTok{30}\NormalTok{, }\DecValTok{50}\NormalTok{, }\DecValTok{80}\NormalTok{, }\DecValTok{100}\NormalTok{, }\DecValTok{150}\NormalTok{, }\DecValTok{200}\NormalTok{, }\DecValTok{500}\NormalTok{)}
\NormalTok{p\_vals }\OtherTok{\textless{}{-}} \FunctionTok{c}\NormalTok{(}\DecValTok{2}\NormalTok{, }\DecValTok{3}\NormalTok{, }\DecValTok{5}\NormalTok{, }\DecValTok{10}\NormalTok{, }\DecValTok{15}\NormalTok{, }\DecValTok{20}\NormalTok{, }\DecValTok{30}\NormalTok{, }\DecValTok{50}\NormalTok{)}

\NormalTok{df }\OtherTok{\textless{}{-}}\NormalTok{ tidyr}\SpecialCharTok{::}\FunctionTok{crossing}\NormalTok{(}\AttributeTok{n =}\NormalTok{ n\_vals, }\AttributeTok{p =}\NormalTok{ p\_vals) }\SpecialCharTok{\%\textgreater{}\%}
  \FunctionTok{mutate}\NormalTok{(}\AttributeTok{valid =}\NormalTok{ p }\SpecialCharTok{\textless{}=}\NormalTok{ n }\SpecialCharTok{{-}} \DecValTok{2}\NormalTok{,}
         \AttributeTok{nu    =}\NormalTok{ n }\SpecialCharTok{{-}}\NormalTok{ p }\SpecialCharTok{{-}} \DecValTok{1}\NormalTok{L) }\SpecialCharTok{\%\textgreater{}\%}
  \FunctionTok{rowwise}\NormalTok{() }\SpecialCharTok{\%\textgreater{}\%}
  \FunctionTok{mutate}\NormalTok{(}
    \AttributeTok{cook\_crit\_95 =} \ControlFlowTok{if}\NormalTok{ (valid) }\FunctionTok{cook\_crit\_avg}\NormalTok{(n, p, }\FloatTok{0.05}\NormalTok{) }\ControlFlowTok{else} \ConstantTok{NA\_real\_}\NormalTok{,}
    \StringTok{\textasciigrave{}}\AttributeTok{4/n}\StringTok{\textasciigrave{}}        \OtherTok{=} \DecValTok{4} \SpecialCharTok{/}\NormalTok{ n,}
    \AttributeTok{ratio        =}\NormalTok{ cook\_crit\_95 }\SpecialCharTok{/} \StringTok{\textasciigrave{}}\AttributeTok{4/n}\StringTok{\textasciigrave{}}\NormalTok{,}
    \StringTok{\textasciigrave{}}\AttributeTok{p/n}\StringTok{\textasciigrave{}}        \OtherTok{=}\NormalTok{ p }\SpecialCharTok{/}\NormalTok{ n}
\NormalTok{  ) }\SpecialCharTok{\%\textgreater{}\%}
  \FunctionTok{ungroup}\NormalTok{() }\SpecialCharTok{\%\textgreater{}\%}
  \FunctionTok{filter}\NormalTok{(valid) }\SpecialCharTok{\%\textgreater{}\%}
  \FunctionTok{select}\NormalTok{(n, p, }\StringTok{\textasciigrave{}}\AttributeTok{p/n}\StringTok{\textasciigrave{}}\NormalTok{, nu, cook\_crit\_95, }\StringTok{\textasciigrave{}}\AttributeTok{4/n}\StringTok{\textasciigrave{}}\NormalTok{, ratio) }\SpecialCharTok{\%\textgreater{}\%}
  \FunctionTok{mutate}\NormalTok{(}
    \StringTok{\textasciigrave{}}\AttributeTok{p/n}\StringTok{\textasciigrave{}}         \OtherTok{=} \FunctionTok{round}\NormalTok{(}\StringTok{\textasciigrave{}}\AttributeTok{p/n}\StringTok{\textasciigrave{}}\NormalTok{, }\DecValTok{3}\NormalTok{),}
    \AttributeTok{cook\_crit\_95  =} \FunctionTok{round}\NormalTok{(cook\_crit\_95, }\DecValTok{6}\NormalTok{),}
    \StringTok{\textasciigrave{}}\AttributeTok{4/n}\StringTok{\textasciigrave{}}         \OtherTok{=} \FunctionTok{round}\NormalTok{(}\StringTok{\textasciigrave{}}\AttributeTok{4/n}\StringTok{\textasciigrave{}}\NormalTok{, }\DecValTok{6}\NormalTok{),}
    \AttributeTok{ratio         =} \FunctionTok{round}\NormalTok{(ratio, }\DecValTok{4}\NormalTok{)}
\NormalTok{  )}


\FunctionTok{library}\NormalTok{(knitr)}
\FunctionTok{library}\NormalTok{(kableExtra)}

\NormalTok{cap }\OtherTok{\textless{}{-}} \StringTok{"Exact 95\% Cook\textquotesingle{}s D critical value (average leverage $h\_\{ii\}=p/n }\SpecialCharTok{\textbackslash{}\textbackslash{}}\StringTok{Rightarrow c\_i=1$) vs heuristic $4/n$."}

\ControlFlowTok{if}\NormalTok{ (knitr}\SpecialCharTok{::}\FunctionTok{is\_html\_output}\NormalTok{()) \{}
  \FunctionTok{kable}\NormalTok{(}
\NormalTok{    df,}
    \AttributeTok{format   =} \StringTok{"html"}\NormalTok{,}
    \AttributeTok{row.names =} \ConstantTok{FALSE}\NormalTok{,}
    \AttributeTok{align    =} \StringTok{"r"}\NormalTok{,}
    \AttributeTok{escape   =} \ConstantTok{FALSE}\NormalTok{,}
    \AttributeTok{caption  =}\NormalTok{ cap}
\NormalTok{  ) }\SpecialCharTok{|\textgreater{}}
    \FunctionTok{kable\_styling}\NormalTok{(}\AttributeTok{full\_width =} \ConstantTok{FALSE}\NormalTok{, }\AttributeTok{bootstrap\_options =} \FunctionTok{c}\NormalTok{(}\StringTok{"striped"}\NormalTok{, }\StringTok{"hover"}\NormalTok{, }\StringTok{"condensed"}\NormalTok{, }\StringTok{"responsive"}\NormalTok{)) }\SpecialCharTok{|\textgreater{}}
    \FunctionTok{scroll\_box}\NormalTok{(}\AttributeTok{width =} \StringTok{"100\%"}\NormalTok{)}
\NormalTok{\} }\ControlFlowTok{else} \ControlFlowTok{if}\NormalTok{ (knitr}\SpecialCharTok{::}\FunctionTok{is\_latex\_output}\NormalTok{()) \{}
  \FunctionTok{kable}\NormalTok{(}
\NormalTok{    df,}
    \AttributeTok{format   =} \StringTok{"latex"}\NormalTok{,}
    \AttributeTok{row.names =} \ConstantTok{FALSE}\NormalTok{,}
    \AttributeTok{align    =} \StringTok{"r"}\NormalTok{,}
    \AttributeTok{booktabs =} \ConstantTok{TRUE}\NormalTok{,}
    \AttributeTok{escape   =} \ConstantTok{FALSE}\NormalTok{,}
    \AttributeTok{caption  =}\NormalTok{ cap}
\NormalTok{  ) }
\NormalTok{\} }\ControlFlowTok{else}\NormalTok{ \{}
  \CommentTok{\# e.g., Word/RTF}
  \FunctionTok{kable}\NormalTok{(}
\NormalTok{    df,}
    \AttributeTok{row.names =} \ConstantTok{FALSE}\NormalTok{,}
    \AttributeTok{align    =} \StringTok{"r"}\NormalTok{,}
    \AttributeTok{escape   =} \ConstantTok{FALSE}\NormalTok{,}
    \AttributeTok{caption  =}\NormalTok{ cap}
\NormalTok{  )}
\NormalTok{\}}
\end{Highlighting}
\end{Shaded}

\textbackslash begin\{table\}

\textbackslash caption\{\label{tab:cooksD-crit-table}Exact 95\% Cook's D
critical value (average leverage \(h_{ii}=p/n \Rightarrow c_i=1\)) vs
heuristic \(4/n\).\} \centering

\begin{tabular}[t]{rrrrrrr}
\toprule
n & p & p/n & nu & cook_crit_95 & 4/n & ratio\\
\midrule
20 & 2 & 0.100 & 17 & 0.207508 & 0.200000 & 1.0375\\
20 & 3 & 0.150 & 16 & 0.219284 & 0.200000 & 1.0964\\
20 & 5 & 0.250 & 14 & 0.247316 & 0.200000 & 1.2366\\
20 & 10 & 0.500 & 9 & 0.362487 & 0.200000 & 1.8124\\
20 & 15 & 0.750 & 4 & 0.658372 & 0.200000 & 3.2919\\
\addlinespace
30 & 2 & 0.067 & 27 & 0.134893 & 0.133333 & 1.0117\\
30 & 3 & 0.100 & 26 & 0.139791 & 0.133333 & 1.0484\\
30 & 5 & 0.167 & 24 & 0.150733 & 0.133333 & 1.1305\\
30 & 10 & 0.333 & 19 & 0.187366 & 0.133333 & 1.4052\\
30 & 15 & 0.500 & 14 & 0.247316 & 0.133333 & 1.8549\\
\addlinespace
30 & 20 & 0.667 & 9 & 0.362487 & 0.133333 & 2.7187\\
50 & 2 & 0.040 & 47 & 0.079282 & 0.080000 & 0.9910\\
50 & 3 & 0.060 & 46 & 0.080951 & 0.080000 & 1.0119\\
50 & 5 & 0.100 & 44 & 0.084510 & 0.080000 & 1.0564\\
50 & 10 & 0.200 & 39 & 0.094944 & 0.080000 & 1.1868\\
\addlinespace
50 & 15 & 0.300 & 34 & 0.108314 & 0.080000 & 1.3539\\
50 & 20 & 0.400 & 29 & 0.126058 & 0.080000 & 1.5757\\
50 & 30 & 0.600 & 19 & 0.187366 & 0.080000 & 2.3421\\
80 & 2 & 0.025 & 77 & 0.048973 & 0.050000 & 0.9795\\
80 & 3 & 0.038 & 76 & 0.049605 & 0.050000 & 0.9921\\
\addlinespace
80 & 5 & 0.062 & 74 & 0.050920 & 0.050000 & 1.0184\\
80 & 10 & 0.125 & 69 & 0.054533 & 0.050000 & 1.0907\\
80 & 15 & 0.188 & 64 & 0.058698 & 0.050000 & 1.1740\\
80 & 20 & 0.250 & 59 & 0.063551 & 0.050000 & 1.2710\\
80 & 30 & 0.375 & 49 & 0.076141 & 0.050000 & 1.5228\\
\addlinespace
80 & 50 & 0.625 & 29 & 0.126058 & 0.050000 & 2.5212\\
100 & 2 & 0.020 & 97 & 0.039025 & 0.040000 & 0.9756\\
100 & 3 & 0.030 & 96 & 0.039425 & 0.040000 & 0.9856\\
100 & 5 & 0.050 & 94 & 0.040251 & 0.040000 & 1.0063\\
100 & 10 & 0.100 & 89 & 0.042476 & 0.040000 & 1.0619\\
\addlinespace
100 & 15 & 0.150 & 84 & 0.044961 & 0.040000 & 1.1240\\
100 & 20 & 0.200 & 79 & 0.047756 & 0.040000 & 1.1939\\
100 & 30 & 0.300 & 69 & 0.054533 & 0.040000 & 1.3633\\
100 & 50 & 0.500 & 49 & 0.076141 & 0.040000 & 1.9035\\
150 & 2 & 0.013 & 147 & 0.025880 & 0.026667 & 0.9705\\
\addlinespace
150 & 3 & 0.020 & 146 & 0.026056 & 0.026667 & 0.9771\\
150 & 5 & 0.033 & 144 & 0.026414 & 0.026667 & 0.9905\\
150 & 10 & 0.067 & 139 & 0.027355 & 0.026667 & 1.0258\\
150 & 15 & 0.100 & 134 & 0.028364 & 0.026667 & 1.0637\\
150 & 20 & 0.133 & 129 & 0.029452 & 0.026667 & 1.1044\\
\addlinespace
150 & 30 & 0.200 & 119 & 0.031897 & 0.026667 & 1.1961\\
150 & 50 & 0.333 & 99 & 0.038248 & 0.026667 & 1.4343\\
200 & 2 & 0.010 & 197 & 0.019359 & 0.020000 & 0.9680\\
200 & 3 & 0.015 & 196 & 0.019457 & 0.020000 & 0.9729\\
200 & 5 & 0.025 & 194 & 0.019657 & 0.020000 & 0.9828\\
\addlinespace
200 & 10 & 0.050 & 189 & 0.020173 & 0.020000 & 1.0086\\
200 & 15 & 0.075 & 184 & 0.020717 & 0.020000 & 1.0358\\
200 & 20 & 0.100 & 179 & 0.021291 & 0.020000 & 1.0645\\
200 & 30 & 0.150 & 169 & 0.022540 & 0.020000 & 1.1270\\
200 & 50 & 0.250 & 149 & 0.025536 & 0.020000 & 1.2768\\
\addlinespace
500 & 2 & 0.004 & 497 & 0.007707 & 0.008000 & 0.9634\\
500 & 3 & 0.006 & 496 & 0.007723 & 0.008000 & 0.9653\\
500 & 5 & 0.010 & 494 & 0.007754 & 0.008000 & 0.9692\\
500 & 10 & 0.020 & 489 & 0.007833 & 0.008000 & 0.9791\\
500 & 15 & 0.030 & 484 & 0.007914 & 0.008000 & 0.9892\\
\addlinespace
500 & 20 & 0.040 & 479 & 0.007996 & 0.008000 & 0.9995\\
500 & 30 & 0.060 & 469 & 0.008166 & 0.008000 & 1.0207\\
500 & 50 & 0.100 & 449 & 0.008529 & 0.008000 & 1.0661\\
\bottomrule
\end{tabular}

\textbackslash end\{table\}

\subsection{Example for Cook's
Distance:}\label{example-for-cooks-distance}

\begin{Shaded}
\begin{Highlighting}[]
\FunctionTok{library}\NormalTok{(dplyr)}
\FunctionTok{library}\NormalTok{(ggplot2)}
\FunctionTok{library}\NormalTok{(knitr)}

\DocumentationTok{\#\# {-}{-}{-} 1) Data and full{-}model fit (your setup)}

\NormalTok{n }\OtherTok{\textless{}{-}} \DecValTok{20}
\NormalTok{x }\OtherTok{\textless{}{-}} \DecValTok{1}\SpecialCharTok{:}\NormalTok{n}
\NormalTok{y }\OtherTok{\textless{}{-}} \DecValTok{2} \SpecialCharTok{+} \DecValTok{3} \SpecialCharTok{*}\NormalTok{ x }\SpecialCharTok{+} \FunctionTok{rnorm}\NormalTok{(n, }\AttributeTok{mean =} \DecValTok{0}\NormalTok{, }\AttributeTok{sd =} \DecValTok{5}\NormalTok{)}
\NormalTok{y[}\DecValTok{1}\NormalTok{] }\OtherTok{\textless{}{-}}\NormalTok{ y[}\DecValTok{1}\NormalTok{] }\SpecialCharTok{+} \DecValTok{30}   \CommentTok{\# add an outlier at i = 1}
\NormalTok{y[}\DecValTok{11}\NormalTok{] }\OtherTok{\textless{}{-}}\NormalTok{ y[}\DecValTok{11}\NormalTok{] }\SpecialCharTok{+} \DecValTok{30}
\NormalTok{full\_data  }\OtherTok{\textless{}{-}} \FunctionTok{data.frame}\NormalTok{(}\AttributeTok{x =}\NormalTok{ x, }\FunctionTok{replicate}\NormalTok{(}\DecValTok{5}\NormalTok{, }\FunctionTok{rnorm}\NormalTok{(n)), }\AttributeTok{y =}\NormalTok{ y)}

\NormalTok{fit }\OtherTok{\textless{}{-}} \FunctionTok{lm}\NormalTok{(y }\SpecialCharTok{\textasciitilde{}}\NormalTok{ ., }\AttributeTok{data =}\NormalTok{ full\_data)}

\DocumentationTok{\#\# {-}{-}{-} 2) Ingredients for Cook\textquotesingle{}s D and thresholds}
\NormalTok{p      }\OtherTok{\textless{}{-}} \FunctionTok{length}\NormalTok{(}\FunctionTok{coef}\NormalTok{(fit))        }\CommentTok{\# includes intercept}
\NormalTok{h      }\OtherTok{\textless{}{-}} \FunctionTok{hatvalues}\NormalTok{(fit)           }\CommentTok{\# leverage h\_ii}
\NormalTok{D      }\OtherTok{\textless{}{-}} \FunctionTok{cooks.distance}\NormalTok{(fit)      }\CommentTok{\# Cook\textquotesingle{}s D\_i}
\NormalTok{nu     }\OtherTok{\textless{}{-}}\NormalTok{ n }\SpecialCharTok{{-}}\NormalTok{ p }\SpecialCharTok{{-}} \DecValTok{1}                \CommentTok{\# df for deleted{-}studentized residual}
\NormalTok{alpha  }\OtherTok{\textless{}{-}} \FloatTok{0.05}

\DocumentationTok{\#\# Per{-}case scaling factor: c\_i = ((n{-}p)/p) * (h\_ii/(1{-}h\_ii))}
\NormalTok{c\_i        }\OtherTok{\textless{}{-}}\NormalTok{ ((n }\SpecialCharTok{{-}}\NormalTok{ p) }\SpecialCharTok{/}\NormalTok{ p) }\SpecialCharTok{*}\NormalTok{ (h }\SpecialCharTok{/}\NormalTok{ (}\DecValTok{1} \SpecialCharTok{{-}}\NormalTok{ h))}

\DocumentationTok{\#\# Exact per{-}case Beta 95\% critical values:}
\DocumentationTok{\#\# D\_i / c\_i \textasciitilde{} Beta(1/2, (n{-}p{-}1)/2)  =\textgreater{}  D\_crit\_i = c\_i * qbeta(0.95, 1/2, (n{-}p{-}1)/2)}
\NormalTok{crit\_beta\_i }\OtherTok{\textless{}{-}}\NormalTok{ c\_i }\SpecialCharTok{*} \FunctionTok{qbeta}\NormalTok{(}\DecValTok{1} \SpecialCharTok{{-}}\NormalTok{ alpha, }\AttributeTok{shape1 =} \FloatTok{0.5}\NormalTok{, }\AttributeTok{shape2 =}\NormalTok{ nu }\SpecialCharTok{/} \DecValTok{2}\NormalTok{)}

\DocumentationTok{\#\# Average{-}leverage Beta 95\% critical value:}
\DocumentationTok{\#\# If h\_ii = p/n (average leverage), then c\_i = 1 exactly}
\NormalTok{crit\_beta\_avg }\OtherTok{\textless{}{-}} \FunctionTok{qbeta}\NormalTok{(}\DecValTok{1} \SpecialCharTok{{-}}\NormalTok{ alpha, }\AttributeTok{shape1 =} \FloatTok{0.5}\NormalTok{, }\AttributeTok{shape2 =}\NormalTok{ nu }\SpecialCharTok{/} \DecValTok{2}\NormalTok{)}

\DocumentationTok{\#\# Heuristic 4/n line}
\NormalTok{crit\_heur }\OtherTok{\textless{}{-}} \DecValTok{4} \SpecialCharTok{/}\NormalTok{ n}

\DocumentationTok{\#\# {-}{-}{-} 3) Assemble results table}
\NormalTok{df\_cook }\OtherTok{\textless{}{-}} \FunctionTok{tibble}\NormalTok{(}
  \AttributeTok{i            =} \FunctionTok{seq\_len}\NormalTok{(n),}
  \AttributeTok{h\_ii         =} \FunctionTok{as.numeric}\NormalTok{(h),}
  \AttributeTok{D\_i          =} \FunctionTok{as.numeric}\NormalTok{(D),}
  \AttributeTok{c\_i          =} \FunctionTok{as.numeric}\NormalTok{(c\_i),}
  \AttributeTok{crit\_beta\_i  =} \FunctionTok{as.numeric}\NormalTok{(crit\_beta\_i),}
  \AttributeTok{crit\_beta\_avg =}\NormalTok{ crit\_beta\_avg,}
  \AttributeTok{crit\_heur    =}\NormalTok{ crit\_heur}
\NormalTok{)}

\DocumentationTok{\#\# Print a compact table}
\NormalTok{knitr}\SpecialCharTok{::}\FunctionTok{kable}\NormalTok{(}
\NormalTok{  df\_cook }\SpecialCharTok{\%\textgreater{}\%} \FunctionTok{mutate}\NormalTok{(}\FunctionTok{across}\NormalTok{(}\FunctionTok{where}\NormalTok{(is.numeric), }\SpecialCharTok{\textasciitilde{}} \FunctionTok{round}\NormalTok{(.x, }\DecValTok{5}\NormalTok{))),}
  \AttributeTok{caption =} \StringTok{"Cook\textquotesingle{}s D, leverage, and thresholds: per{-}case Beta 95\%, average{-}leverage Beta 95\%, and 4/n."}
\NormalTok{)}
\end{Highlighting}
\end{Shaded}

\begin{longtable}[]{@{}rrrrrrr@{}}
\caption{Cook's D, leverage, and thresholds: per-case Beta 95\%,
average-leverage Beta 95\%, and 4/n.}\tabularnewline
\toprule\noalign{}
i & h\_ii & D\_i & c\_i & crit\_beta\_i & crit\_beta\_avg &
crit\_heur \\
\midrule\noalign{}
\endfirsthead
\toprule\noalign{}
i & h\_ii & D\_i & c\_i & crit\_beta\_i & crit\_beta\_avg &
crit\_heur \\
\midrule\noalign{}
\endhead
\bottomrule\noalign{}
\endlastfoot
1 & 0.67204 & 1.62738 & 3.80555 & 1.07873 & 0.28346 & 0.2 \\
2 & 0.31296 & 0.07715 & 0.84595 & 0.23979 & 0.28346 & 0.2 \\
3 & 0.49722 & 0.08784 & 1.83660 & 0.52061 & 0.28346 & 0.2 \\
4 & 0.61111 & 0.23875 & 2.91831 & 0.82724 & 0.28346 & 0.2 \\
5 & 0.21633 & 0.02983 & 0.51265 & 0.14532 & 0.28346 & 0.2 \\
6 & 0.14309 & 0.00469 & 0.31011 & 0.08790 & 0.28346 & 0.2 \\
7 & 0.47629 & 0.07632 & 1.68900 & 0.47877 & 0.28346 & 0.2 \\
8 & 0.41204 & 0.00004 & 1.30148 & 0.36892 & 0.28346 & 0.2 \\
9 & 0.40499 & 0.29801 & 1.26407 & 0.35832 & 0.28346 & 0.2 \\
10 & 0.17593 & 0.00026 & 0.39648 & 0.11239 & 0.28346 & 0.2 \\
11 & 0.39917 & 0.73199 & 1.23380 & 0.34974 & 0.28346 & 0.2 \\
12 & 0.10715 & 0.00088 & 0.22287 & 0.06318 & 0.28346 & 0.2 \\
13 & 0.21158 & 0.01277 & 0.49837 & 0.14127 & 0.28346 & 0.2 \\
14 & 0.30757 & 0.00026 & 0.82491 & 0.23383 & 0.28346 & 0.2 \\
15 & 0.26786 & 0.04549 & 0.67944 & 0.19260 & 0.28346 & 0.2 \\
16 & 0.71261 & 0.48354 & 4.60490 & 1.30532 & 0.28346 & 0.2 \\
17 & 0.20630 & 0.01202 & 0.48271 & 0.13683 & 0.28346 & 0.2 \\
18 & 0.27965 & 0.00082 & 0.72098 & 0.20437 & 0.28346 & 0.2 \\
19 & 0.29482 & 0.03388 & 0.77642 & 0.22009 & 0.28346 & 0.2 \\
20 & 0.29132 & 0.00775 & 0.76341 & 0.21640 & 0.28346 & 0.2 \\
\end{longtable}

Cook's D by case with heuristic 4/n (red dotted), per-case Beta 95\%
critical values (blue dashed), and average-leverage Beta 95\% line
(purple dot-dash).

\begin{Shaded}
\begin{Highlighting}[]
\DocumentationTok{\#\# {-}{-}{-} 4) Plot D\_i with thresholds}
\NormalTok{thresh\_df }\OtherTok{\textless{}{-}} \FunctionTok{bind\_rows}\NormalTok{(}
\NormalTok{  df\_cook }\SpecialCharTok{\%\textgreater{}\%} \FunctionTok{transmute}\NormalTok{(i, }\AttributeTok{value =}\NormalTok{ crit\_beta\_i,   }\AttributeTok{Type =} \StringTok{"Beta 95\% (per{-}case)"}\NormalTok{),}
  \FunctionTok{tibble}\NormalTok{(}\AttributeTok{i =} \FunctionTok{seq\_len}\NormalTok{(n), }\AttributeTok{value =}\NormalTok{ crit\_beta\_avg,   }\AttributeTok{Type =} \StringTok{"Beta 95\% (avg leverage)"}\NormalTok{),}
  \FunctionTok{tibble}\NormalTok{(}\AttributeTok{i =} \FunctionTok{seq\_len}\NormalTok{(n), }\AttributeTok{value =}\NormalTok{ crit\_heur,       }\AttributeTok{Type =} \StringTok{"4/n rule"}\NormalTok{)}
\NormalTok{)}

\FunctionTok{ggplot}\NormalTok{(df\_cook, }\FunctionTok{aes}\NormalTok{(}\AttributeTok{x =}\NormalTok{ i, }\AttributeTok{y =}\NormalTok{ D\_i)) }\SpecialCharTok{+}
  \FunctionTok{geom\_point}\NormalTok{(}\AttributeTok{size =} \DecValTok{2}\NormalTok{) }\SpecialCharTok{+}
  \FunctionTok{geom\_line}\NormalTok{(}\AttributeTok{data =}\NormalTok{ thresh\_df,}
            \FunctionTok{aes}\NormalTok{(}\AttributeTok{y =}\NormalTok{ value, }\AttributeTok{color =}\NormalTok{ Type, }\AttributeTok{linetype =}\NormalTok{ Type),}
            \AttributeTok{linewidth =} \FloatTok{0.9}\NormalTok{) }\SpecialCharTok{+}
  \FunctionTok{scale\_color\_manual}\NormalTok{(}\AttributeTok{values =} \FunctionTok{c}\NormalTok{(}
    \StringTok{"Beta 95\% (per{-}case)"}      \OtherTok{=} \StringTok{"\#1f77b4"}\NormalTok{,}
    \StringTok{"Beta 95\% (avg leverage)"}  \OtherTok{=} \StringTok{"\#7f3c8d"}\NormalTok{,}
    \StringTok{"4/n rule"}                 \OtherTok{=} \StringTok{"\#d62728"}
\NormalTok{  )) }\SpecialCharTok{+}
  \FunctionTok{scale\_linetype\_manual}\NormalTok{(}\AttributeTok{values =} \FunctionTok{c}\NormalTok{(}
    \StringTok{"Beta 95\% (per{-}case)"}      \OtherTok{=} \StringTok{"dashed"}\NormalTok{,}
    \StringTok{"Beta 95\% (avg leverage)"}  \OtherTok{=} \StringTok{"dotdash"}\NormalTok{,}
    \StringTok{"4/n rule"}                 \OtherTok{=} \StringTok{"dotted"}
\NormalTok{  )) }\SpecialCharTok{+}
  \FunctionTok{labs}\NormalTok{(}\AttributeTok{x =} \StringTok{"Observation i"}\NormalTok{, }\AttributeTok{y =} \StringTok{"Cook\textquotesingle{}s D\_i"}\NormalTok{) }\SpecialCharTok{+}
  \FunctionTok{theme\_bw}\NormalTok{() }\SpecialCharTok{+}
  \FunctionTok{theme}\NormalTok{(}\AttributeTok{legend.title =} \FunctionTok{element\_blank}\NormalTok{())}
\end{Highlighting}
\end{Shaded}

\begin{figure}[H]

{\centering \includegraphics{unit3-mlr/residuals_files/figure-pdf/cooksD-compute-and-plot-1.pdf}

}

\caption{Cook's D by case with heuristic 4/n (red dotted), per-case Beta
95\% critical values (blue dashed), and average-leverage Beta 95\% line
(purple dot-dash).}

\end{figure}%

\section*{Appendix: Key Identities}\label{appendix-key-identities}
\addcontentsline{toc}{section}{Appendix: Key Identities}

\markright{Appendix: Key Identities}

The power of modern regression diagnostics comes from algebraic
shortcuts that allow us to find the results of a leave-one-out process
without the computational cost of refitting the model \emph{n} times.
The following two identities are fundamental to this efficiency.

\subsection*{\texorpdfstring{Finding the LOOCV Residual (\(e_{i,-i}\))
from the Ordinary Residual
(\(e_i\))}{Finding the LOOCV Residual (e\_\{i,-i\}) from the Ordinary Residual (e\_i)}}\label{finding-the-loocv-residual-e_i-i-from-the-ordinary-residual-e_i}
\addcontentsline{toc}{subsection}{Finding the LOOCV Residual
(\(e_{i,-i}\)) from the Ordinary Residual (\(e_i\))}

This identity shows that we can find the ``pure'' leave-one-out residual
using only the results from the single model fit on all data.

\begin{equation}\phantomsection\label{eq-key}{e_{i,-i} = \frac{e_i}{1 - h_{ii}}}\end{equation}

\subsection*{Finding the LOOCV Standard Error from the Full-Model
Standard
Error}\label{finding-the-loocv-standard-error-from-the-full-model-standard-error}
\addcontentsline{toc}{subsection}{Finding the LOOCV Standard Error from
the Full-Model Standard Error}

Similarly, this formula provides an efficient shortcut to see how the
model's overall error changes when a single point is removed.

\begin{equation}\phantomsection\label{eq-sigma_-i}{\hat{\sigma}_{-i} = \sqrt{\frac{(n-p)\hat{\sigma}^2 - \frac{e_i^2}{1-h_{ii}}}{n-p-1}}}\end{equation}

The derivation of this formula relies on first proving the relationship
between the full model's Residual Sum of Squares (\(RSS\)) and the
leave-one-out version (\(RSS_{-i}\)).

\begin{enumerate}
\def\labelenumi{\arabic{enumi}.}
\item
  \textbf{Start with the definition} of the leave-one-out residual sum
  of squares: \[
  RSS_{-i} = \sum_{k \neq i} (y_k - \mathbf{x}_k^T\hat{\beta}_{-i})^2
  \]
\item
  \textbf{Introduce the key identity} that relates the leave-one-out
  coefficient vector (\(\hat{\beta}_{-i}\)) to the full model's
  coefficient vector (\(\hat{\beta}\)): \[
  \hat{\beta}_{-i} = \hat{\beta} - (X^TX)^{-1}\mathbf{x}_i \frac{e_i}{1 - h_{ii}}
  \]
\item
  \textbf{Substitute this identity} into the expression for a generic
  leave-one-out residual,
  \(e_{k,-i} = y_k - \mathbf{x}_k^T\hat{\beta}_{-i}\). After
  simplification, this yields: \[
  e_{k,-i} = e_k + h_{ki} \frac{e_i}{1 - h_{ii}}
  \] where \(e_k\) is the ordinary residual and \(h_{ki}\) is the
  \((k,i)\)-th element of the hat matrix.
\item
  \textbf{Substitute this back into the definition of} \(RSS_{-i}\).
  After expanding the squared term and performing the summation (which
  involves considerable but standard matrix algebra), the expression
  simplifies to the elegant result: \[
  RSS_{-i} = RSS - \frac{e_i^2}{1 - h_{ii}}
  \]
\item
  \textbf{Finally, derive the formula for} \(\hat{\sigma}_{-i}\). We
  know that \(\hat{\sigma}^2_{-i} = \frac{RSS_{-i}}{n-p-1}\) and that
  \(RSS = (n-p)\hat{\sigma}^2\). By substituting the result from Step 4,
  we arrive at the formula for the variance, and taking the square root
  gives us the standard error. ✅
\end{enumerate}

\bookmarksetup{startatroot}

\chapter{Logistic Regression}\label{logistic-regression}

\section{Odds as a Function of
Probability}\label{odds-as-a-function-of-probability}

For an event with probability \(p\), the odds is
\[\mathrm{odds}(p)=\frac{p}{1-p}\] and the log-odds (logit) is
\[\mathrm{logit}(p)=\log\left(\frac{p}{1-p}\right)\].

\begin{Shaded}
\begin{Highlighting}[]
\DocumentationTok{\#\# Plot odds(p) with a right{-}hand axis for log(odds(p)),}
\DocumentationTok{\#\# using different line colors for the two curves.}
\DocumentationTok{\#\# Defaults: p in [0.01, 0.99].}
\DocumentationTok{\#\# Args:}
\DocumentationTok{\#\#   p\_min, p\_max : endpoints for p{-}grid (0\textless{}p\_min\textless{}p\_max\textless{}1)}
\DocumentationTok{\#\#   n            : number of grid points}
\DocumentationTok{\#\#   annotate     : add reference lines/labels if TRUE}
\DocumentationTok{\#\#   odds\_col     : color for odds(p)}
\DocumentationTok{\#\#   logit\_col    : color for log(odds(p))}
\DocumentationTok{\#\#   lwd1, lwd2   : line widths for the two curves}


\NormalTok{plot\_odds }\OtherTok{\textless{}{-}} \ControlFlowTok{function}\NormalTok{(}\AttributeTok{p\_min =} \FloatTok{0.01}\NormalTok{, }\AttributeTok{p\_max =} \FloatTok{0.99}\NormalTok{, }\AttributeTok{n =} \DecValTok{400}\NormalTok{,}
                      \AttributeTok{annotate =} \ConstantTok{TRUE}\NormalTok{,}
                      \AttributeTok{odds\_col =} \StringTok{"steelblue"}\NormalTok{,}
                      \AttributeTok{logit\_col =} \StringTok{"firebrick"}\NormalTok{,}
                      \AttributeTok{lwd1 =} \DecValTok{2}\NormalTok{, }\AttributeTok{lwd2 =} \DecValTok{2}\NormalTok{) \{}
  \FunctionTok{stopifnot}\NormalTok{(p\_min }\SpecialCharTok{\textgreater{}} \DecValTok{0}\NormalTok{, p\_max }\SpecialCharTok{\textless{}} \DecValTok{1}\NormalTok{, p\_min }\SpecialCharTok{\textless{}}\NormalTok{ p\_max, n }\SpecialCharTok{\textgreater{}=} \DecValTok{10}\NormalTok{)}
\NormalTok{  p }\OtherTok{\textless{}{-}} \FunctionTok{seq}\NormalTok{(p\_min, p\_max, }\AttributeTok{length.out =}\NormalTok{ n)}
\NormalTok{  odds }\OtherTok{\textless{}{-}}\NormalTok{ p }\SpecialCharTok{/}\NormalTok{ (}\DecValTok{1} \SpecialCharTok{{-}}\NormalTok{ p)}
\NormalTok{  logit }\OtherTok{\textless{}{-}} \FunctionTok{log}\NormalTok{(odds)}

  \DocumentationTok{\#\# Left y{-}axis: odds(p)}
  \FunctionTok{plot}\NormalTok{(p, odds, }\AttributeTok{type =} \StringTok{"l"}\NormalTok{, }\AttributeTok{lwd =}\NormalTok{ lwd1, }\AttributeTok{col =}\NormalTok{ odds\_col,}
       \AttributeTok{xlab =} \StringTok{"Probability p"}\NormalTok{,}
       \AttributeTok{ylab =} \StringTok{"odds(p) = p / (1 {-} p)"}\NormalTok{)}
  \ControlFlowTok{if}\NormalTok{ (annotate) \{}
    \FunctionTok{abline}\NormalTok{(}\AttributeTok{h =} \DecValTok{1}\NormalTok{, }\AttributeTok{v =} \FloatTok{0.5}\NormalTok{, }\AttributeTok{lty =} \DecValTok{2}\NormalTok{)}
    \FunctionTok{text}\NormalTok{(}\FloatTok{0.52}\NormalTok{, }\FloatTok{1.05}\NormalTok{, }\StringTok{"p = 0.5 → odds = 1"}\NormalTok{, }\AttributeTok{adj =} \DecValTok{0}\NormalTok{)}
\NormalTok{  \}}

  \DocumentationTok{\#\# Right y{-}axis: logit(p) = log(odds)}
\NormalTok{  op }\OtherTok{\textless{}{-}} \FunctionTok{par}\NormalTok{(}\AttributeTok{new =} \ConstantTok{TRUE}\NormalTok{)}
  \FunctionTok{on.exit}\NormalTok{(}\FunctionTok{par}\NormalTok{(op), }\AttributeTok{add =} \ConstantTok{TRUE}\NormalTok{)}
  \FunctionTok{plot}\NormalTok{(p, logit, }\AttributeTok{type =} \StringTok{"l"}\NormalTok{, }\AttributeTok{lwd =}\NormalTok{ lwd2, }\AttributeTok{col =}\NormalTok{ logit\_col,}
       \AttributeTok{axes =} \ConstantTok{FALSE}\NormalTok{, }\AttributeTok{xlab =} \StringTok{""}\NormalTok{, }\AttributeTok{ylab =} \StringTok{""}\NormalTok{)}
  \FunctionTok{axis}\NormalTok{(}\DecValTok{4}\NormalTok{)}
  \FunctionTok{mtext}\NormalTok{(}\StringTok{"log\{odds(p)\} = log\{p/(1 {-} p)\}"}\NormalTok{, }\AttributeTok{side =} \DecValTok{4}\NormalTok{, }\AttributeTok{line =} \DecValTok{3}\NormalTok{)}

  \ControlFlowTok{if}\NormalTok{ (annotate) \{}
    \FunctionTok{abline}\NormalTok{(}\AttributeTok{v =} \FloatTok{0.5}\NormalTok{, }\AttributeTok{lty =} \DecValTok{2}\NormalTok{)}
    \CommentTok{\# logit(0.5) = 0 reference (horizontal) on the right{-}axis scale}
\NormalTok{    usr }\OtherTok{\textless{}{-}} \FunctionTok{par}\NormalTok{(}\StringTok{"usr"}\NormalTok{)}
    \FunctionTok{segments}\NormalTok{(}\AttributeTok{x0 =}\NormalTok{ usr[}\DecValTok{1}\NormalTok{], }\AttributeTok{y0 =} \DecValTok{0}\NormalTok{, }\AttributeTok{x1 =} \FloatTok{0.5}\NormalTok{, }\AttributeTok{y1 =} \DecValTok{0}\NormalTok{, }\AttributeTok{lty =} \DecValTok{3}\NormalTok{)}
\NormalTok{  \}}

  \FunctionTok{legend}\NormalTok{(}\StringTok{"topleft"}\NormalTok{,}
         \AttributeTok{legend =} \FunctionTok{c}\NormalTok{(}\StringTok{"odds(p)"}\NormalTok{, }\StringTok{"log\{odds(p)\}"}\NormalTok{),}
         \AttributeTok{col =} \FunctionTok{c}\NormalTok{(odds\_col, logit\_col),}
         \AttributeTok{lwd =} \FunctionTok{c}\NormalTok{(lwd1, lwd2), }\AttributeTok{bty =} \StringTok{"n"}\NormalTok{)}

  \FunctionTok{invisible}\NormalTok{(}\FunctionTok{list}\NormalTok{(}\AttributeTok{p =}\NormalTok{ p, }\AttributeTok{odds =}\NormalTok{ odds, }\AttributeTok{logit =}\NormalTok{ logit))}
\NormalTok{\}}

\DocumentationTok{\#\# Example usage:}
\DocumentationTok{\#\# plot\_odds()  \# defaults: steelblue for odds, firebrick for log{-}odds (right axis)}
\FunctionTok{plot\_odds}\NormalTok{(}\AttributeTok{odds\_col =} \StringTok{"\#1f77b4"}\NormalTok{, }\AttributeTok{logit\_col =} \StringTok{"\#d62728"}\NormalTok{, }\AttributeTok{n =} \DecValTok{600}\NormalTok{)}
\end{Highlighting}
\end{Shaded}

\includegraphics{unit4-lr/logistic_files/figure-pdf/plot-odds-colored-1.pdf}

\section{Logistic Regression}\label{logistic-regression-1}

Let \(Y_i \in \{0,1\}\) denote a binary outcome, and let
\(\mathbf{x}_i = (x_{i1}, \ldots, x_{ip})^\top\) be the vector of
predictors \textbf{excluding} the intercept.\\
Let \(\boldsymbol{\beta} = (\beta_1, \ldots, \beta_p)^\top\) denote the
slope coefficients, and \(\beta_0\) the intercept.\\
The logistic regression model specifies \[
p_i \;=\; P(Y_i = 1 \mid \mathbf{x}_i),
\] with \[
\log\!\left(\frac{p_i}{1 - p_i}\right)
\;=\; \beta_0 + \mathbf{x}_i^\top \boldsymbol{\beta}
\;=\; \beta_0 + \sum_{j=1}^p \beta_j x_{ij}.
\]

Equivalently, the fitted probability is \[
p_i(\mathbf{x}_i;\boldsymbol{\beta})
\;=\;
\frac{\exp(\beta_0 + \mathbf{x}_i^\top \boldsymbol{\beta})}
     {1+\exp(\beta_0 + \mathbf{x}_i^\top \boldsymbol{\beta})}
\;=\;
\frac{1}{1+\exp\!\big(-(\beta_0 + \mathbf{x}_i^\top \boldsymbol{\beta})\big)}.
\]

This formulation expresses the \textbf{log-odds} of success as a linear
function of predictors, ensuring that fitted probabilities remain in the
range \((0,1)\).\\
The \textbf{odds} satisfy \[
\frac{p_i}{1-p_i} \;=\; \exp(\beta_0 + \mathbf{x}_i^\top \boldsymbol{\beta}),
\] so a one-unit increase in \(x_{ij}\) (holding other predictors fixed)
multiplies the odds of success by \(\exp(\beta_j)\).

\section{A Dataset Simulated from Logistic
Regression}\label{a-dataset-simulated-from-logistic-regression}

We simulate data from a logistic model where the \textbf{logit} is a
linear function of \(x\):

\[
\operatorname{logit}{p(x)}
=
\log\left(\frac{p(x)}{1-p(x)}\right)
=
\beta_0 + \beta_1 x,
\]

so that

\[
p(x)
=
\operatorname{logit}^{-1}(\beta_0+\beta_1 x)
=
\frac{1}{1+\exp\{-(\beta_0+\beta_1 x)\}}.
\]

We then display the observed \(y_i\) (binary outcomes) and the true
probability curve \(p(x)\) in red.

\begin{Shaded}
\begin{Highlighting}[]
\FunctionTok{set.seed}\NormalTok{(}\DecValTok{123}\NormalTok{)}

\DocumentationTok{\#\# {-}{-} Truth (edit as desired) {-}{-}}
\NormalTok{n     }\OtherTok{\textless{}{-}} \DecValTok{200}
\NormalTok{beta0 }\OtherTok{\textless{}{-}} \DecValTok{0}
\NormalTok{beta1 }\OtherTok{\textless{}{-}}  \DecValTok{4}

\DocumentationTok{\#\# {-}{-} Simulate {-}{-}}
\NormalTok{x   }\OtherTok{\textless{}{-}} \FunctionTok{runif}\NormalTok{(n, }\SpecialCharTok{{-}}\DecValTok{1}\NormalTok{, }\DecValTok{1}\NormalTok{)             }\CommentTok{\# predictor}
\NormalTok{eta }\OtherTok{\textless{}{-}}\NormalTok{ beta0 }\SpecialCharTok{+}\NormalTok{ beta1 }\SpecialCharTok{*}\NormalTok{ x}
\NormalTok{p   }\OtherTok{\textless{}{-}} \FunctionTok{plogis}\NormalTok{(eta)                 }\CommentTok{\# true p(x)}
\NormalTok{y   }\OtherTok{\textless{}{-}} \FunctionTok{rbinom}\NormalTok{(n, }\AttributeTok{size =} \DecValTok{1}\NormalTok{, }\AttributeTok{prob =}\NormalTok{ p) }\CommentTok{\# outcomes}

\NormalTok{sim.data }\OtherTok{\textless{}{-}} \FunctionTok{data.frame}\NormalTok{(}\AttributeTok{x =}\NormalTok{ x, }\AttributeTok{y =}\NormalTok{ y, }\AttributeTok{p =}\NormalTok{ p)}
\end{Highlighting}
\end{Shaded}

\subsection{Fit a logistic model to the simulated
data}\label{fit-a-logistic-model-to-the-simulated-data}

\begin{Shaded}
\begin{Highlighting}[]
\DocumentationTok{\#\# {-}{-} Optional: fit a model to the simulated data {-}{-}}
\NormalTok{sim.fit }\OtherTok{\textless{}{-}} \FunctionTok{glm}\NormalTok{(y }\SpecialCharTok{\textasciitilde{}}\NormalTok{ x, }\AttributeTok{data =}\NormalTok{ sim.data, }\AttributeTok{family =} \FunctionTok{binomial}\NormalTok{())}
\NormalTok{p\_fit }\OtherTok{\textless{}{-}} \FunctionTok{predict}\NormalTok{(sim.fit, }\AttributeTok{newdata =} \FunctionTok{data.frame}\NormalTok{(}\AttributeTok{x =}\NormalTok{ x), }\AttributeTok{type =} \StringTok{"response"}\NormalTok{)}

\DocumentationTok{\#\# {-}{-} Plot: points for y\_i (jittered), red line for true p(x) {-}{-}}
\DocumentationTok{\#\# Define jitter amount}
\NormalTok{jit }\OtherTok{\textless{}{-}} \FloatTok{0.05} 
\DocumentationTok{\#\# jitter to separate 0/1 visually}
\NormalTok{yj }\OtherTok{\textless{}{-}} \FunctionTok{jitter}\NormalTok{(sim.data}\SpecialCharTok{$}\NormalTok{y, }\AttributeTok{amount =}\NormalTok{ jit) }

\FunctionTok{plot}\NormalTok{(sim.data}\SpecialCharTok{$}\NormalTok{x, yj,}
     \AttributeTok{pch =} \DecValTok{16}\NormalTok{, }\AttributeTok{col =} \FunctionTok{rgb}\NormalTok{(}\DecValTok{0}\NormalTok{, }\DecValTok{0}\NormalTok{, }\DecValTok{0}\NormalTok{, }\FloatTok{0.45}\NormalTok{),}
     \AttributeTok{xlab =} \StringTok{"x"}\NormalTok{,}
     \AttributeTok{ylab =} \StringTok{"Observed y (points) \& p(x) (curves)"}\NormalTok{,}
     \AttributeTok{ylim =} \FunctionTok{c}\NormalTok{(}\SpecialCharTok{{-}}\FloatTok{0.1}\NormalTok{, }\FloatTok{1.1}\NormalTok{))}

\DocumentationTok{\#\# True probability curve (red)}
\NormalTok{xg }\OtherTok{\textless{}{-}} \FunctionTok{seq}\NormalTok{(}\FunctionTok{min}\NormalTok{(x), }\FunctionTok{max}\NormalTok{(x), }\AttributeTok{length.out =} \DecValTok{500}\NormalTok{)}
\FunctionTok{lines}\NormalTok{(xg, }\FunctionTok{plogis}\NormalTok{(beta0 }\SpecialCharTok{+}\NormalTok{ beta1 }\SpecialCharTok{*}\NormalTok{ xg), }\AttributeTok{col =} \StringTok{"red"}\NormalTok{, }\AttributeTok{lwd =} \DecValTok{2}\NormalTok{)}

\DocumentationTok{\#\# Optional: add fitted probability curve (dashed dark red)}
\FunctionTok{lines}\NormalTok{(xg, }\FunctionTok{predict}\NormalTok{(sim.fit, }\AttributeTok{newdata =} \FunctionTok{data.frame}\NormalTok{(}\AttributeTok{x =}\NormalTok{ xg), }\AttributeTok{type =} \StringTok{"response"}\NormalTok{),}
      \AttributeTok{col =} \StringTok{"darkred"}\NormalTok{, }\AttributeTok{lwd =} \DecValTok{2}\NormalTok{, }\AttributeTok{lty =} \DecValTok{2}\NormalTok{)}

\FunctionTok{legend}\NormalTok{(}\StringTok{"topleft"}\NormalTok{,}
       \AttributeTok{legend =} \FunctionTok{c}\NormalTok{(}\StringTok{"y (jittered points)"}\NormalTok{, }\StringTok{"true p(x)"}\NormalTok{, }\StringTok{"fitted p(x)"}\NormalTok{),}
       \AttributeTok{pch    =} \FunctionTok{c}\NormalTok{(}\DecValTok{16}\NormalTok{, }\ConstantTok{NA}\NormalTok{, }\ConstantTok{NA}\NormalTok{),}
       \AttributeTok{lty    =} \FunctionTok{c}\NormalTok{(}\ConstantTok{NA}\NormalTok{, }\DecValTok{1}\NormalTok{, }\DecValTok{2}\NormalTok{),}
       \AttributeTok{col    =} \FunctionTok{c}\NormalTok{(}\FunctionTok{rgb}\NormalTok{(}\DecValTok{0}\NormalTok{,}\DecValTok{0}\NormalTok{,}\DecValTok{0}\NormalTok{,}\FloatTok{0.45}\NormalTok{), }\StringTok{"red"}\NormalTok{, }\StringTok{"darkred"}\NormalTok{),}
       \AttributeTok{lwd    =} \FunctionTok{c}\NormalTok{(}\ConstantTok{NA}\NormalTok{, }\DecValTok{2}\NormalTok{, }\DecValTok{2}\NormalTok{),}
       \AttributeTok{bty    =} \StringTok{"n"}\NormalTok{)}
\end{Highlighting}
\end{Shaded}

\includegraphics{unit4-lr/logistic_files/figure-pdf/unnamed-chunk-2-1.pdf}

\section{Example: Coronary Heart Disease
Data}\label{example-coronary-heart-disease-data}

\subsection{Load a dataset}\label{load-a-dataset}

This dataset is about a follow-up study to determine the development of
coronary heart disease (CHD) over 9 years of follow-up of 609 white
males from Evans County, Georgia.

\textbf{Variable meanings (as provided):}

\begin{itemize}
\tightlist
\item
  \texttt{chd}: 1 if a person has the disease, 0 otherwise.
\item
  \texttt{smk}: 1 if smoker, 0 if not.
\item
  \texttt{cat}: 1 if catecholamine level is high, 0 if low.
\item
  \texttt{sbp}: systolic blood pressure (continuous).
\item
  \texttt{age}: age in years (continuous).
\item
  \texttt{chl}: cholesterol level (continuous).
\item
  \texttt{ecg}: 1 if electrocardiogram is abnormal, 0 if normal.
\item
  \texttt{hpt}: 1 if high blood pressure, 0 if normal.
\end{itemize}

\begin{Shaded}
\begin{Highlighting}[]
\DocumentationTok{\#\# Adjust the path if needed. The default is your original V: drive path.}
\NormalTok{data\_path }\OtherTok{\textless{}{-}} \StringTok{"evans.dat"}

\DocumentationTok{\#\# Read data (expects a header row)}
\NormalTok{CHD.data }\OtherTok{\textless{}{-}} \FunctionTok{read.table}\NormalTok{(data\_path, }\AttributeTok{header =} \ConstantTok{TRUE}\NormalTok{)}
\ControlFlowTok{if}\NormalTok{ (knitr}\SpecialCharTok{::}\FunctionTok{is\_html\_output}\NormalTok{())\{}
\NormalTok{  CHD.data}
\NormalTok{\} }\ControlFlowTok{else}\NormalTok{\{}
\NormalTok{  CHD.data[}\DecValTok{1}\SpecialCharTok{:}\DecValTok{20}\NormalTok{,]}
\NormalTok{\}}
\end{Highlighting}
\end{Shaded}

\begin{verbatim}
    id chd age cat chl dbp ecg sbp smk hpt
1   21   0  56   0 270  80   0 138   0   0
2   31   0  43   0 159  74   0 128   1   0
3   51   1  56   1 201 112   1 164   1   1
4   71   0  64   1 179 100   0 200   1   1
5   74   0  49   0 243  82   0 145   1   0
6   91   0  46   0 252  88   0 142   1   0
7  111   1  52   0 179  80   1 128   1   0
8  131   0  63   0 217  92   0 135   0   0
9  141   0  42   0 176  76   0 114   1   0
10 191   0  55   0 250 114   1 182   0   1
11 201   0  74   0 293 100   0 166   0   1
12 241   0  53   0 179  90   0 158   0   0
13 251   0  58   0 201  86   0 142   1   0
14 261   0  56   0 206  85   0 120   1   0
15 271   0  69   0 225  84   0 168   0   1
16 283   1  51   1 259 102   1 135   0   1
17 291   0  43   0 193  78   0 118   1   0
18 311   0  64   1 185 100   1 180   0   1
19 312   0  44   0 150 108   0 160   0   1
20 331   0  42   0 211  86   1 122   0   0
\end{verbatim}

\subsection{Fit Logistic Regression Model for a Single
Variable}\label{fit-logistic-regression-model-for-a-single-variable}

\begin{Shaded}
\begin{Highlighting}[]
\NormalTok{vars }\OtherTok{\textless{}{-}} \FunctionTok{c}\NormalTok{(}\StringTok{"smk"}\NormalTok{, }\StringTok{"sbp"}\NormalTok{, }\StringTok{"age"}\NormalTok{, }\StringTok{"chl"}\NormalTok{)}
\CommentTok{\#jit  \textless{}{-} 0.01  \# global jitter amount for y}

\DocumentationTok{\#\# par(mfrow = c(2, 2), mar = c(4, 4, 2, 4) + 0.1)  \# extra right margin for axis(4)}

\ControlFlowTok{for}\NormalTok{ (v }\ControlFlowTok{in}\NormalTok{ vars) \{}
  \DocumentationTok{\#\# Univariate logistic regression using ORIGINAL variable name in the formula}
\NormalTok{  fit }\OtherTok{\textless{}{-}} \FunctionTok{glm}\NormalTok{(}
    \AttributeTok{formula =} \FunctionTok{reformulate}\NormalTok{(v, }\AttributeTok{response =} \StringTok{"chd"}\NormalTok{),}
    \AttributeTok{data    =}\NormalTok{ CHD.data,}
    \AttributeTok{family  =} \FunctionTok{binomial}\NormalTok{()}
\NormalTok{  )}
  \FunctionTok{print}\NormalTok{(}\FunctionTok{summary}\NormalTok{(fit))}

  \DocumentationTok{\#\# Base scatter of chd with small jitter (left axis: probability scale)}
  \FunctionTok{plot}\NormalTok{(}
\NormalTok{    CHD.data[[v]],}
    \FunctionTok{jitter}\NormalTok{(CHD.data}\SpecialCharTok{$}\NormalTok{chd, }\AttributeTok{amount =}\NormalTok{ jit),}
    \AttributeTok{pch  =} \DecValTok{16}\NormalTok{, }\AttributeTok{col =} \FunctionTok{rgb}\NormalTok{(}\DecValTok{0}\NormalTok{, }\DecValTok{0}\NormalTok{, }\DecValTok{0}\NormalTok{, }\FloatTok{0.45}\NormalTok{),}
    \AttributeTok{xlab =}\NormalTok{ v, }\AttributeTok{ylab =} \StringTok{"chd (jittered)"}\NormalTok{,}
    \AttributeTok{main =} \FunctionTok{paste}\NormalTok{(}\StringTok{"chd vs"}\NormalTok{, v),}
    \AttributeTok{ylim =} \FunctionTok{c}\NormalTok{(}\SpecialCharTok{{-}}\FloatTok{0.1}\NormalTok{, }\FloatTok{1.1}\NormalTok{)}
\NormalTok{  )}

  \DocumentationTok{\#\# Fitted π(x) in red (left axis)}
  \ControlFlowTok{if}\NormalTok{ (}\FunctionTok{length}\NormalTok{(}\FunctionTok{unique}\NormalTok{(CHD.data[[v]])) }\SpecialCharTok{==} \DecValTok{2}\NormalTok{) \{}
    \CommentTok{\# binary predictor}
\NormalTok{    xcat }\OtherTok{\textless{}{-}} \FunctionTok{sort}\NormalTok{(}\FunctionTok{unique}\NormalTok{(CHD.data[[v]]))}
\NormalTok{    nd   }\OtherTok{\textless{}{-}} \FunctionTok{setNames}\NormalTok{(}\FunctionTok{data.frame}\NormalTok{(xcat), v)}
\NormalTok{    pcat }\OtherTok{\textless{}{-}} \FunctionTok{predict}\NormalTok{(fit, }\AttributeTok{newdata =}\NormalTok{ nd, }\AttributeTok{type =} \StringTok{"response"}\NormalTok{)}
    \FunctionTok{points}\NormalTok{(xcat, pcat, }\AttributeTok{pch =} \DecValTok{19}\NormalTok{, }\AttributeTok{col =} \StringTok{"red"}\NormalTok{)}
    \FunctionTok{lines}\NormalTok{(xcat, pcat, }\AttributeTok{col =} \StringTok{"red"}\NormalTok{, }\AttributeTok{lwd =} \DecValTok{2}\NormalTok{)}

    \CommentTok{\# Right{-}axis: logit\{π(x)\} with fixed y{-}limits}
\NormalTok{    logit\_p }\OtherTok{\textless{}{-}} \FunctionTok{log}\NormalTok{(pcat }\SpecialCharTok{/}\NormalTok{ (}\DecValTok{1} \SpecialCharTok{{-}}\NormalTok{ pcat))}
    \FunctionTok{par}\NormalTok{(}\AttributeTok{new =} \ConstantTok{TRUE}\NormalTok{)}
    \FunctionTok{plot}\NormalTok{(}
\NormalTok{      xcat, logit\_p, }\AttributeTok{type =} \StringTok{"l"}\NormalTok{, }\AttributeTok{lwd =} \DecValTok{2}\NormalTok{, }\AttributeTok{col =} \StringTok{"blue"}\NormalTok{,}
      \AttributeTok{axes =} \ConstantTok{FALSE}\NormalTok{, }\AttributeTok{xlab =} \StringTok{""}\NormalTok{, }\AttributeTok{ylab =} \StringTok{""}\NormalTok{,}
      \AttributeTok{xlim =} \FunctionTok{range}\NormalTok{(CHD.data[[v]]), }\AttributeTok{ylim =} \FunctionTok{c}\NormalTok{(}\SpecialCharTok{{-}}\FloatTok{2.5}\NormalTok{, }\DecValTok{0}\NormalTok{)}
\NormalTok{    )}
    \FunctionTok{axis}\NormalTok{(}\DecValTok{4}\NormalTok{)}
    \FunctionTok{mtext}\NormalTok{(}\StringTok{"logit(p(x))"}\NormalTok{, }\AttributeTok{side =} \DecValTok{4}\NormalTok{, }\AttributeTok{line =} \DecValTok{3}\NormalTok{)}
    \FunctionTok{par}\NormalTok{(}\AttributeTok{new =} \ConstantTok{FALSE}\NormalTok{)}

\NormalTok{  \} }\ControlFlowTok{else}\NormalTok{ \{}
    \CommentTok{\# continuous predictor}
\NormalTok{    xg }\OtherTok{\textless{}{-}} \FunctionTok{seq}\NormalTok{(}\FunctionTok{min}\NormalTok{(CHD.data[[v]]), }\FunctionTok{max}\NormalTok{(CHD.data[[v]]), }\AttributeTok{length.out =} \DecValTok{400}\NormalTok{)}
\NormalTok{    nd }\OtherTok{\textless{}{-}} \FunctionTok{setNames}\NormalTok{(}\FunctionTok{data.frame}\NormalTok{(xg), v)}
\NormalTok{    pg }\OtherTok{\textless{}{-}} \FunctionTok{predict}\NormalTok{(fit, }\AttributeTok{newdata =}\NormalTok{ nd, }\AttributeTok{type =} \StringTok{"response"}\NormalTok{)}
    \FunctionTok{lines}\NormalTok{(xg, pg, }\AttributeTok{col =} \StringTok{"red"}\NormalTok{, }\AttributeTok{lwd =} \DecValTok{2}\NormalTok{)}

    \CommentTok{\# Right{-}axis: logit\{π(x)\} with fixed y{-}limits}
\NormalTok{    logit\_pg }\OtherTok{\textless{}{-}} \FunctionTok{log}\NormalTok{(pg }\SpecialCharTok{/}\NormalTok{ (}\DecValTok{1} \SpecialCharTok{{-}}\NormalTok{ pg))}
    \FunctionTok{par}\NormalTok{(}\AttributeTok{new =} \ConstantTok{TRUE}\NormalTok{)}
    \FunctionTok{plot}\NormalTok{(}
\NormalTok{      xg, logit\_pg, }\AttributeTok{type =} \StringTok{"l"}\NormalTok{, }\AttributeTok{lwd =} \DecValTok{2}\NormalTok{, }\AttributeTok{col =} \StringTok{"blue"}\NormalTok{,}
      \AttributeTok{axes =} \ConstantTok{FALSE}\NormalTok{, }\AttributeTok{xlab =} \StringTok{""}\NormalTok{, }\AttributeTok{ylab =} \StringTok{""}\NormalTok{,}
      \AttributeTok{xlim =} \FunctionTok{range}\NormalTok{(xg), }\AttributeTok{ylim =} \FunctionTok{c}\NormalTok{(}\SpecialCharTok{{-}}\FloatTok{2.5}\NormalTok{, }\DecValTok{0}\NormalTok{)}
\NormalTok{    )}
    \FunctionTok{axis}\NormalTok{(}\DecValTok{4}\NormalTok{)}
    \FunctionTok{mtext}\NormalTok{(}\StringTok{"logit(p(x))"}\NormalTok{, }\AttributeTok{side =} \DecValTok{4}\NormalTok{, }\AttributeTok{line =} \DecValTok{3}\NormalTok{)}
    \FunctionTok{par}\NormalTok{(}\AttributeTok{new =} \ConstantTok{FALSE}\NormalTok{)}
\NormalTok{  \}}
\NormalTok{\}}
\end{Highlighting}
\end{Shaded}

\begin{verbatim}

Call:
glm(formula = reformulate(v, response = "chd"), family = binomial(), 
    data = CHD.data)

Coefficients:
            Estimate Std. Error z value Pr(>|z|)    
(Intercept)  -2.4898     0.2524  -9.865   <2e-16 ***
smk           0.6706     0.2919   2.297   0.0216 *  
---
Signif. codes:  0 '***' 0.001 '**' 0.01 '*' 0.05 '.' 0.1 ' ' 1

(Dispersion parameter for binomial family taken to be 1)

    Null deviance: 438.56  on 608  degrees of freedom
Residual deviance: 432.81  on 607  degrees of freedom
AIC: 436.81

Number of Fisher Scoring iterations: 5
\end{verbatim}

\includegraphics{unit4-lr/logistic_files/figure-pdf/chd-univariate-2x2-jitter-fixedlogit-1.pdf}

\begin{verbatim}

Call:
glm(formula = reformulate(v, response = "chd"), family = binomial(), 
    data = CHD.data)

Coefficients:
             Estimate Std. Error z value Pr(>|z|)    
(Intercept) -3.837912   0.629805  -6.094  1.1e-09 ***
sbp          0.012154   0.004036   3.011   0.0026 ** 
---
Signif. codes:  0 '***' 0.001 '**' 0.01 '*' 0.05 '.' 0.1 ' ' 1

(Dispersion parameter for binomial family taken to be 1)

    Null deviance: 438.56  on 608  degrees of freedom
Residual deviance: 430.06  on 607  degrees of freedom
AIC: 434.06

Number of Fisher Scoring iterations: 4
\end{verbatim}

\includegraphics{unit4-lr/logistic_files/figure-pdf/chd-univariate-2x2-jitter-fixedlogit-2.pdf}

\begin{verbatim}

Call:
glm(formula = reformulate(v, response = "chd"), family = binomial(), 
    data = CHD.data)

Coefficients:
            Estimate Std. Error z value Pr(>|z|)    
(Intercept) -4.47833    0.75610  -5.923 3.16e-09 ***
age          0.04445    0.01315   3.381 0.000723 ***
---
Signif. codes:  0 '***' 0.001 '**' 0.01 '*' 0.05 '.' 0.1 ' ' 1

(Dispersion parameter for binomial family taken to be 1)

    Null deviance: 438.56  on 608  degrees of freedom
Residual deviance: 427.22  on 607  degrees of freedom
AIC: 431.22

Number of Fisher Scoring iterations: 5
\end{verbatim}

\includegraphics{unit4-lr/logistic_files/figure-pdf/chd-univariate-2x2-jitter-fixedlogit-3.pdf}

\begin{verbatim}

Call:
glm(formula = reformulate(v, response = "chd"), family = binomial(), 
    data = CHD.data)

Coefficients:
             Estimate Std. Error z value Pr(>|z|)    
(Intercept) -3.538260   0.686879  -5.151 2.59e-07 ***
chl          0.007004   0.003064   2.286   0.0223 *  
---
Signif. codes:  0 '***' 0.001 '**' 0.01 '*' 0.05 '.' 0.1 ' ' 1

(Dispersion parameter for binomial family taken to be 1)

    Null deviance: 438.56  on 608  degrees of freedom
Residual deviance: 433.42  on 607  degrees of freedom
AIC: 437.42

Number of Fisher Scoring iterations: 4
\end{verbatim}

\includegraphics{unit4-lr/logistic_files/figure-pdf/chd-univariate-2x2-jitter-fixedlogit-4.pdf}

\subsection{Fit Logistic Regression Model with all
variables}\label{fit-logistic-regression-model-with-all-variables}

We fit a logistic regression with a logit link:

\begin{Shaded}
\begin{Highlighting}[]
\NormalTok{fit1\_chd }\OtherTok{\textless{}{-}} \FunctionTok{glm}\NormalTok{(}
\NormalTok{  chd }\SpecialCharTok{\textasciitilde{}}\NormalTok{ smk }\SpecialCharTok{+}\NormalTok{ cat }\SpecialCharTok{+}\NormalTok{ sbp }\SpecialCharTok{+}\NormalTok{ age }\SpecialCharTok{+}\NormalTok{ chl }\SpecialCharTok{+}\NormalTok{ ecg }\SpecialCharTok{+}\NormalTok{ hpt,}
  \AttributeTok{data =}\NormalTok{ CHD.data,}
  \AttributeTok{family =} \FunctionTok{binomial}\NormalTok{(}\AttributeTok{link =} \StringTok{"logit"}\NormalTok{)}
\NormalTok{)}
\FunctionTok{summary}\NormalTok{(fit1\_chd)}
\end{Highlighting}
\end{Shaded}

\begin{verbatim}

Call:
glm(formula = chd ~ smk + cat + sbp + age + chl + ecg + hpt, 
    family = binomial(link = "logit"), data = CHD.data)

Coefficients:
             Estimate Std. Error z value Pr(>|z|)    
(Intercept) -6.048892   1.345165  -4.497  6.9e-06 ***
smk          0.855951   0.306505   2.793  0.00523 ** 
cat          0.732763   0.376129   1.948  0.05139 .  
sbp         -0.006995   0.006976  -1.003  0.31600    
age          0.033956   0.015344   2.213  0.02690 *  
chl          0.008970   0.003274   2.740  0.00615 ** 
ecg          0.417776   0.295553   1.414  0.15750    
hpt          0.655498   0.359976   1.821  0.06861 .  
---
Signif. codes:  0 '***' 0.001 '**' 0.01 '*' 0.05 '.' 0.1 ' ' 1

(Dispersion parameter for binomial family taken to be 1)

    Null deviance: 438.56  on 608  degrees of freedom
Residual deviance: 399.35  on 601  degrees of freedom
AIC: 415.35

Number of Fisher Scoring iterations: 5
\end{verbatim}

\textbf{Notes for interpretation:}

\begin{itemize}
\tightlist
\item
  Positive coefficients increase the log-odds of CHD; negative
  coefficients decrease it.
\item
  For indicator variables (e.g., \texttt{smk}), \texttt{exp(beta)} is
  the adjusted odds ratio comparing the group with value 1 versus 0,
  holding others fixed.
\item
  For continuous predictors (e.g., \texttt{sbp}, \texttt{age}),
  \texttt{exp(beta)} is the multiplicative change in the odds for a
  one‑unit increase. For a \emph{d}-unit increase, the OR is
  \texttt{exp(d\ *\ beta)}.
\end{itemize}

\section{Inference for Coefficients: Confidence Intervals and Covariance
Matrix}\label{inference-for-coefficients-confidence-intervals-and-covariance-matrix}

We extract profile‑likelihood CIs and the covariance matrix to confirm
standard errors.

\begin{Shaded}
\begin{Highlighting}[]
\NormalTok{ci\_95 }\OtherTok{\textless{}{-}} \FunctionTok{confint}\NormalTok{(fit1\_chd, }\AttributeTok{level =} \FloatTok{0.95}\NormalTok{)     }\CommentTok{\# profile{-}likelihood CI}
\NormalTok{vcov\_mat }\OtherTok{\textless{}{-}} \FunctionTok{vcov}\NormalTok{(fit1\_chd)                    }\CommentTok{\# covariance matrix of coefficients}
\NormalTok{se\_vec   }\OtherTok{\textless{}{-}} \FunctionTok{sqrt}\NormalTok{(}\FunctionTok{diag}\NormalTok{(vcov\_mat))          }\CommentTok{\# standard errors}

\NormalTok{ci\_95}
\end{Highlighting}
\end{Shaded}

\begin{verbatim}
                   2.5 %       97.5 %
(Intercept) -8.718003347 -3.427904298
smk          0.275699158  1.483333169
cat         -0.006873216  1.471885644
sbp         -0.021166144  0.006266328
age          0.003687290  0.064005215
chl          0.002533226  0.015404292
ecg         -0.171584621  0.990632546
hpt         -0.050184520  1.364993401
\end{verbatim}

\begin{Shaded}
\begin{Highlighting}[]
\NormalTok{vcov\_mat}
\end{Highlighting}
\end{Shaded}

\begin{verbatim}
             (Intercept)           smk           cat           sbp
(Intercept)  1.809468553 -1.014526e-01  0.1391440386 -4.908229e-03
smk         -0.101452600  9.394560e-02 -0.0032961000 -1.653230e-04
cat          0.139144039 -3.296100e-03  0.1414730484 -9.299960e-04
sbp         -0.004908229 -1.653230e-04 -0.0009299960  4.866901e-05
age         -0.011142995  7.738971e-04 -0.0017879998 -1.311191e-05
chl         -0.002111134  3.161443e-05  0.0003146354 -1.821907e-06
ecg          0.003442546  9.255483e-03 -0.0204455233 -2.982539e-04
hpt          0.139817180  6.954592e-03 -0.0044220690 -1.486400e-03
                      age           chl           ecg           hpt
(Intercept) -1.114300e-02 -2.111134e-03  3.442546e-03  1.398172e-01
smk          7.738971e-04  3.161443e-05  9.255483e-03  6.954592e-03
cat         -1.788000e-03  3.146354e-04 -2.044552e-02 -4.422069e-03
sbp         -1.311191e-05 -1.821907e-06 -2.982539e-04 -1.486400e-03
age          2.354442e-04 -1.480501e-06 -4.972374e-05  4.044434e-04
chl         -1.480501e-06  1.071766e-05  5.040548e-05 -6.046197e-05
ecg         -4.972374e-05  5.040548e-05  8.735130e-02  9.506863e-04
hpt          4.044434e-04 -6.046197e-05  9.506863e-04  1.295828e-01
\end{verbatim}

\begin{Shaded}
\begin{Highlighting}[]
\NormalTok{se\_vec  }\CommentTok{\# should match the SE column in summary(fit1\_chd)}
\end{Highlighting}
\end{Shaded}

\begin{verbatim}
(Intercept)         smk         cat         sbp         age         chl 
1.345164879 0.306505459 0.376129032 0.006976318 0.015344190 0.003273784 
        ecg         hpt 
0.295552531 0.359976081 
\end{verbatim}

\section{Inference for Odds Ratios}\label{inference-for-odds-ratios}

\subsection{Interpretation of Odds Ratios in Logistic
Regression}\label{interpretation-of-odds-ratios-in-logistic-regression}

A multiple logistic regression model expresses the log-odds (logit) of
an event as a linear function of predictors:

\[
\log\left(\frac{p}{1-p}\right)
= \beta_0 + \beta_1 x_1 + \beta_2 x_2 + \cdots + \beta_k x_k.
\]

Here,

\begin{itemize}
\tightlist
\item
  \(p = \Pr(Y = 1 \mid x_1, x_2, \ldots, x_k)\) is the probability of
  the event,
\item
  \(\beta_0\) is the intercept, and
\item
  each \(\beta_j\) represents the \textbf{change in the log-odds} of the
  event per one-unit increase in \(x_j\), \emph{holding all other
  variables constant}.
\end{itemize}

Exponentiating both sides gives the model in odds form:

\[
\frac{p}{1-p}
= \exp(\beta_0)
\times \exp(\beta_1 x_1)
\times \exp(\beta_2 x_2)
\times \cdots
\times \exp(\beta_k x_k).
\]

\textbf{An R function for OR at two Profiles}

The or\_from\_predict R function is a utility designed to calculate the
Odds Ratio (OR) and its 95\% confidence interval (CI) between two
specific covariate profiles (new1 and new0) for a given logistic
regression model (fit). The calculation is performed on the link (logit)
scale. For a logistic model
\(\text{logit}(p) = \eta = \mathbf{X}\boldsymbol{\beta}\), the log-Odds
Ratio (logOR) is the difference between the linear predictors
(\(\eta_1, \eta_0\)) for the two profiles:
\begin{equation}\phantomsection\label{eq-logor}{\widehat{\text{logOR}} = \eta_1 - \eta_0 = (\mathbf{x}_1^T - \mathbf{x}_0^T) \boldsymbol{\beta} = \mathbf{c}^T \boldsymbol{\beta}}\end{equation}

Here, \(\mathbf{c} = \mathbf{x}_1 - \mathbf{x}_0\) is the linear
contrast vector derived from the model matrices of the two profiles. The
function estimates the variance of this contrast as
\(\text{Var}(\widehat{\text{logOR}}) = \mathbf{c}^T \mathbf{V} \mathbf{c}\),
where \(\mathbf{V}\) is the model's variance-covariance matrix
(vcov(fit)). The standard error
\(SE = \sqrt{\mathbf{c}^T \mathbf{V} \mathbf{c}}\) is used to compute
the \(100(1-\alpha)\%\) confidence interval for the logOR:
\[\widehat{\text{logOR}} \pm z_{1-\alpha/2} \times SE.\] These values
(estimate and CI bounds) are then exponentiated to produce the final
\(\widehat{\text{OR}} = \exp(\widehat{\text{logOR}})\) and its 95\% CI.
The function also prints two helpful summaries to the console: a data
frame showing only the variables that differ between the new0 and new1
profiles, and a 2x3 table presenting the estimates and CIs for both the
OR and the logOR.

\textbf{The R function to find ORs}

\begin{Shaded}
\begin{Highlighting}[]
\DocumentationTok{\#\# Compute OR and 95\% CI via predict() on the LINK scale}
\DocumentationTok{\#\# OR = exp( eta(new1) {-} eta(new0) ), where eta(.) = logit\{π(.)\}}
\DocumentationTok{\#\# Compute OR via predict() contrast on the LINK scale, also:}
\DocumentationTok{\#\# (ii) print a 2{-}row data.frame of only variables that differ between new0 and new1}
\DocumentationTok{\#\# (iii) print a 2x3 table (rows: OR, logOR; cols: Estimate, CI\_low, CI\_up)}
\NormalTok{or\_from\_predict }\OtherTok{\textless{}{-}} \ControlFlowTok{function}\NormalTok{(fit, new1, new0, }\AttributeTok{level =} \FloatTok{0.95}\NormalTok{, }\AttributeTok{digits =} \DecValTok{4}\NormalTok{, }\AttributeTok{tol =} \FloatTok{1e{-}12}\NormalTok{) \{}
  \FunctionTok{stopifnot}\NormalTok{(}\FunctionTok{is.data.frame}\NormalTok{(new1), }\FunctionTok{is.data.frame}\NormalTok{(new0))}

  \DocumentationTok{\#\# {-}{-}{-} REFACTORED SECTION START {-}{-}{-}}
  \DocumentationTok{\#\# {-}{-}{-}{-} (ii) Two{-}row data.frame with only changed variables {-}{-}{-}{-}}
  
  \DocumentationTok{\#\# Helper function to find differing variables between two profiles}
  \DocumentationTok{\#\# This is defined *inside* or\_from\_predict for encapsulation}
\NormalTok{  get\_changed\_vars }\OtherTok{\textless{}{-}} \ControlFlowTok{function}\NormalTok{(d0, d1, tolerance) \{}
\NormalTok{    common }\OtherTok{\textless{}{-}} \FunctionTok{intersect}\NormalTok{(}\FunctionTok{names}\NormalTok{(d0), }\FunctionTok{names}\NormalTok{(d1))}
\NormalTok{    diffv  }\OtherTok{\textless{}{-}} \FunctionTok{vapply}\NormalTok{(common, }\ControlFlowTok{function}\NormalTok{(nm) \{}
\NormalTok{      x0 }\OtherTok{\textless{}{-}}\NormalTok{ d0[[nm]]; x1 }\OtherTok{\textless{}{-}}\NormalTok{ d1[[nm]]}
      \ControlFlowTok{if}\NormalTok{ (}\FunctionTok{is.numeric}\NormalTok{(x0) }\SpecialCharTok{\&\&} \FunctionTok{is.numeric}\NormalTok{(x1)) \{}
        \SpecialCharTok{!}\FunctionTok{isTRUE}\NormalTok{(}\FunctionTok{all.equal}\NormalTok{(}\FunctionTok{as.numeric}\NormalTok{(x0), }\FunctionTok{as.numeric}\NormalTok{(x1), }\AttributeTok{tolerance =}\NormalTok{ tolerance))}
\NormalTok{      \} }\ControlFlowTok{else}\NormalTok{ \{}
        \SpecialCharTok{!}\FunctionTok{identical}\NormalTok{(x0, x1)}
\NormalTok{      \}}
\NormalTok{    \}, }\FunctionTok{logical}\NormalTok{(}\DecValTok{1}\NormalTok{))}
    
\NormalTok{    keep }\OtherTok{\textless{}{-}}\NormalTok{ common[diffv]}
    \ControlFlowTok{if}\NormalTok{ (}\FunctionTok{length}\NormalTok{(keep) }\SpecialCharTok{==} \DecValTok{0}\NormalTok{L) \{}
\NormalTok{      out }\OtherTok{\textless{}{-}} \FunctionTok{data.frame}\NormalTok{(}\StringTok{\textasciigrave{}}\AttributeTok{\_no\_changes\_}\StringTok{\textasciigrave{}} \OtherTok{=} \StringTok{"no differences"}\NormalTok{)}
      \FunctionTok{rownames}\NormalTok{(out) }\OtherTok{\textless{}{-}} \FunctionTok{c}\NormalTok{(}\StringTok{"new0"}\NormalTok{, }\StringTok{"new1"}\NormalTok{)}
      \FunctionTok{return}\NormalTok{(out)}
\NormalTok{    \}}
\NormalTok{    out }\OtherTok{\textless{}{-}} \FunctionTok{rbind}\NormalTok{(d0[keep], d1[keep])}
    \FunctionTok{rownames}\NormalTok{(out) }\OtherTok{\textless{}{-}} \FunctionTok{c}\NormalTok{(}\StringTok{"new0"}\NormalTok{, }\StringTok{"new1"}\NormalTok{)}
\NormalTok{    out}
\NormalTok{  \}}
  
  \DocumentationTok{\#\# Call the helper function}
\NormalTok{  changes\_df }\OtherTok{\textless{}{-}} \FunctionTok{get\_changed\_vars}\NormalTok{(new0, new1, tol)}
  \DocumentationTok{\#\# {-}{-}{-} REFACTORED SECTION }\RegionMarkerTok{END}\DocumentationTok{ {-}{-}{-}}


  \DocumentationTok{\#\# {-}{-}{-}{-} Linear contrast for log{-}OR and its variance {-}{-}{-}{-}}
  
  \DocumentationTok{\#\# (i) Calculate logOR estimate using predict(type="link")}
  \DocumentationTok{\#\# eta(.) = logit\{p(.)\}}
\NormalTok{  eta1 }\OtherTok{\textless{}{-}} \FunctionTok{predict}\NormalTok{(fit, }\AttributeTok{newdata =}\NormalTok{ new1, }\AttributeTok{type =} \StringTok{"link"}\NormalTok{)}
\NormalTok{  eta0 }\OtherTok{\textless{}{-}} \FunctionTok{predict}\NormalTok{(fit, }\AttributeTok{newdata =}\NormalTok{ new0, }\AttributeTok{type =} \StringTok{"link"}\NormalTok{)}
\NormalTok{  logOR\_hat }\OtherTok{\textless{}{-}} \FunctionTok{as.numeric}\NormalTok{(eta1 }\SpecialCharTok{{-}}\NormalTok{ eta0) }\CommentTok{\# logOR = eta1 {-} eta0}
  
  \DocumentationTok{\#\# (ii) Calculate standard error using the contrast vector \textquotesingle{}cvec\textquotesingle{}}
\NormalTok{  X1 }\OtherTok{\textless{}{-}} \FunctionTok{model.matrix}\NormalTok{(}\FunctionTok{delete.response}\NormalTok{(}\FunctionTok{terms}\NormalTok{(fit)), }\AttributeTok{data =}\NormalTok{ new1)}
\NormalTok{  X0 }\OtherTok{\textless{}{-}} \FunctionTok{model.matrix}\NormalTok{(}\FunctionTok{delete.response}\NormalTok{(}\FunctionTok{terms}\NormalTok{(fit)), }\AttributeTok{data =}\NormalTok{ new0)}
\NormalTok{  cvec      }\OtherTok{\textless{}{-}} \FunctionTok{as.numeric}\NormalTok{(X1 }\SpecialCharTok{{-}}\NormalTok{ X0)}
\NormalTok{  V         }\OtherTok{\textless{}{-}} \FunctionTok{vcov}\NormalTok{(fit)}
\NormalTok{  se\_logOR  }\OtherTok{\textless{}{-}} \FunctionTok{sqrt}\NormalTok{(}\FunctionTok{as.numeric}\NormalTok{(}\FunctionTok{t}\NormalTok{(cvec) }\SpecialCharTok{\%*\%}\NormalTok{ V }\SpecialCharTok{\%*\%}\NormalTok{ cvec))}

\NormalTok{  alpha  }\OtherTok{\textless{}{-}} \DecValTok{1} \SpecialCharTok{{-}}\NormalTok{ level}
\NormalTok{  z      }\OtherTok{\textless{}{-}} \FunctionTok{qnorm}\NormalTok{(}\DecValTok{1} \SpecialCharTok{{-}}\NormalTok{ alpha }\SpecialCharTok{/} \DecValTok{2}\NormalTok{)}
\NormalTok{  ci\_log }\OtherTok{\textless{}{-}} \FunctionTok{c}\NormalTok{(logOR\_hat }\SpecialCharTok{{-}}\NormalTok{ z }\SpecialCharTok{*}\NormalTok{ se\_logOR, logOR\_hat }\SpecialCharTok{+}\NormalTok{ z }\SpecialCharTok{*}\NormalTok{ se\_logOR)}

  \DocumentationTok{\#\# {-}{-}{-}{-} 2x3 table: rows OR and logOR; columns Estimate, CI\_low, CI\_up {-}{-}{-}{-}}
\NormalTok{  res\_tab }\OtherTok{\textless{}{-}} \FunctionTok{data.frame}\NormalTok{(}
    \AttributeTok{Estimate =} \FunctionTok{c}\NormalTok{(}\FunctionTok{exp}\NormalTok{(logOR\_hat),          logOR\_hat),}
    \AttributeTok{CI\_low   =} \FunctionTok{c}\NormalTok{(}\FunctionTok{exp}\NormalTok{(ci\_log[}\DecValTok{1}\NormalTok{L]),         ci\_log[}\DecValTok{1}\NormalTok{L]),}
    \AttributeTok{CI\_up    =} \FunctionTok{c}\NormalTok{(}\FunctionTok{exp}\NormalTok{(ci\_log[}\DecValTok{2}\NormalTok{L]),         ci\_log[}\DecValTok{2}\NormalTok{L]),}
    \AttributeTok{row.names =} \FunctionTok{c}\NormalTok{(}\StringTok{"OR"}\NormalTok{, }\StringTok{"logOR"}\NormalTok{)}
\NormalTok{  )}

  \DocumentationTok{\#\# {-}{-}{-}{-} Print requested items {-}{-}{-}{-}}
  \FunctionTok{cat}\NormalTok{(}\StringTok{"}\SpecialCharTok{\textbackslash{}n}\StringTok{Variables that differ between new0 and new1:}\SpecialCharTok{\textbackslash{}n}\StringTok{"}\NormalTok{)}
  \FunctionTok{print}\NormalTok{(changes\_df)}
  \FunctionTok{cat}\NormalTok{(}\StringTok{"}\SpecialCharTok{\textbackslash{}n}\StringTok{Odds Ratio summary:}\SpecialCharTok{\textbackslash{}n}\StringTok{"}\NormalTok{)}
  \FunctionTok{print}\NormalTok{(}\FunctionTok{round}\NormalTok{(res\_tab, }\AttributeTok{digits =}\NormalTok{ digits)) }\CommentTok{\# Added rounding for neatness}

  \DocumentationTok{\#\# {-}{-}{-}{-} Return (invisibly) {-}{-}{-}{-}}
  \FunctionTok{invisible}\NormalTok{(}\FunctionTok{list}\NormalTok{(}
    \AttributeTok{OR        =} \FunctionTok{exp}\NormalTok{(logOR\_hat),}
    \AttributeTok{CI\_OR     =} \FunctionTok{exp}\NormalTok{(ci\_log),}
    \AttributeTok{logOR     =}\NormalTok{ logOR\_hat,}
    \AttributeTok{CI\_logOR  =}\NormalTok{ ci\_log,}
    \AttributeTok{se\_logOR  =}\NormalTok{ se\_logOR,}
    \AttributeTok{changes   =}\NormalTok{ changes\_df,}
    \AttributeTok{table     =}\NormalTok{ res\_tab}
\NormalTok{  ))}
\NormalTok{\}}

\DocumentationTok{\#\# {-}{-}{-} Example usage {-}{-}{-}}
\DocumentationTok{\#\# Suppose \textquotesingle{}fit1\_chd\textquotesingle{} is your fitted model and \textquotesingle{}CHD.data\textquotesingle{} is your data}
\DocumentationTok{\#\# base\_prof \textless{}{-} as.data.frame(lapply(CHD.data, function(col) if (is.numeric(col)) mean(col) else col[1]))}
\DocumentationTok{\#\# new0 \textless{}{-} base\_prof; new0$smk \textless{}{-} 0}
\DocumentationTok{\#\# new1 \textless{}{-} base\_prof; new1$smk \textless{}{-} 1}
\DocumentationTok{\#\# or\_from\_predict(fit1\_chd, new1 = new1, new0 = new0)}


\NormalTok{mean\_profile }\OtherTok{\textless{}{-}} \ControlFlowTok{function}\NormalTok{(data, }\AttributeTok{vars\_binary\_as =} \FunctionTok{c}\NormalTok{(}\DecValTok{0}\NormalTok{,}\DecValTok{1}\NormalTok{)) \{}
  \DocumentationTok{\#\# Build a single{-}row data.frame of typical values:}
\NormalTok{  out }\OtherTok{\textless{}{-}} \FunctionTok{lapply}\NormalTok{(data, }\ControlFlowTok{function}\NormalTok{(col) \{}
    \ControlFlowTok{if}\NormalTok{ (}\FunctionTok{is.numeric}\NormalTok{(col)) \{}
      \CommentTok{\# If strictly 0/1, keep mean (works fine for GLM prediction),}
      \CommentTok{\# or switch to mode if you prefer.}
      \ControlFlowTok{if}\NormalTok{ (}\FunctionTok{all}\NormalTok{(col }\SpecialCharTok{\%in\%} \FunctionTok{c}\NormalTok{(}\DecValTok{0}\NormalTok{,}\DecValTok{1}\NormalTok{))) }\FunctionTok{mean}\NormalTok{(col) }\ControlFlowTok{else} \FunctionTok{mean}\NormalTok{(col, }\AttributeTok{na.rm =} \ConstantTok{TRUE}\NormalTok{)}
\NormalTok{    \} }\ControlFlowTok{else}\NormalTok{ \{}
      \CommentTok{\# Fallback to first level for factors/characters}
      \ControlFlowTok{if}\NormalTok{ (}\FunctionTok{is.factor}\NormalTok{(col)) }\FunctionTok{levels}\NormalTok{(col)[}\DecValTok{1}\NormalTok{] }\ControlFlowTok{else} \FunctionTok{unique}\NormalTok{(col)[}\DecValTok{1}\NormalTok{]}
\NormalTok{    \}}
\NormalTok{  \})}
  \FunctionTok{as.data.frame}\NormalTok{(out)}
\NormalTok{\}}
\end{Highlighting}
\end{Shaded}

\subsection{Examples of Finding ORs and Their CIs for the CHD
Dataset}\label{examples-of-finding-ors-and-their-cis-for-the-chd-dataset}

\subsubsection{\texorpdfstring{OR Smoking (\texttt{smk}) (1 vs
0)}{OR Smoking (smk) (1 vs 0)}}\label{or-smoking-smk-1-vs-0}

\begin{Shaded}
\begin{Highlighting}[]
\DocumentationTok{\#\#\# 1) Smoking OR: smk = 1 vs 0 (other vars at their means)}
\DocumentationTok{\#\# Example profiles at sample means (adjust as you like)}
\NormalTok{base\_prof }\OtherTok{\textless{}{-}} \FunctionTok{mean\_profile}\NormalTok{(CHD.data)}
\NormalTok{new0 }\OtherTok{\textless{}{-}}\NormalTok{ base\_prof; new0}\SpecialCharTok{$}\NormalTok{smk }\OtherTok{\textless{}{-}} \DecValTok{0}
\NormalTok{new1 }\OtherTok{\textless{}{-}}\NormalTok{ base\_prof; new1}\SpecialCharTok{$}\NormalTok{smk }\OtherTok{\textless{}{-}} \DecValTok{1}

\NormalTok{res\_smk }\OtherTok{\textless{}{-}} \FunctionTok{or\_from\_predict}\NormalTok{(fit1\_chd, }\AttributeTok{new1 =}\NormalTok{ new1, }\AttributeTok{new0 =}\NormalTok{ new0)}
\end{Highlighting}
\end{Shaded}

\begin{verbatim}

Variables that differ between new0 and new1:
     smk
new0   0
new1   1

Odds Ratio summary:
      Estimate CI_low  CI_up
OR      2.3536 1.2907 4.2917
logOR   0.8560 0.2552 1.4567
\end{verbatim}

\textbf{How to read this:}

\begin{itemize}
\tightlist
\item
  \texttt{OR\_smk\ \textgreater{}\ 1} suggests higher odds of CHD among
  smokers (adjusted for other variables). If the 95\% CI excludes 1, the
  association is statistically significant at the 5\% level.
\end{itemize}

\subsubsection{\texorpdfstring{OR for Systolic Blood Pressure
(\texttt{sbp}): from 120 to
160}{OR for Systolic Blood Pressure (sbp): from 120 to 160}}\label{or-for-systolic-blood-pressure-sbp-from-120-to-160}

We compute the adjusted OR for a 40‑unit increase in \texttt{sbp} (from
120 to 160):

\begin{Shaded}
\begin{Highlighting}[]
\DocumentationTok{\#\#\# 2) SBP OR: 160 vs 120 (other vars at their means)}
\NormalTok{new0 }\OtherTok{\textless{}{-}}\NormalTok{ base\_prof; new0}\SpecialCharTok{$}\NormalTok{sbp }\OtherTok{\textless{}{-}} \DecValTok{120}
\NormalTok{new1 }\OtherTok{\textless{}{-}}\NormalTok{ base\_prof; new1}\SpecialCharTok{$}\NormalTok{sbp }\OtherTok{\textless{}{-}} \DecValTok{160}

\NormalTok{res\_sbp }\OtherTok{\textless{}{-}} \FunctionTok{or\_from\_predict}\NormalTok{(fit1\_chd, }\AttributeTok{new1 =}\NormalTok{ new1, }\AttributeTok{new0 =}\NormalTok{ new0)}
\end{Highlighting}
\end{Shaded}

\begin{verbatim}

Variables that differ between new0 and new1:
     sbp
new0 120
new1 160

Odds Ratio summary:
      Estimate  CI_low  CI_up
OR      0.7559  0.4375 1.3062
logOR  -0.2798 -0.8267 0.2671
\end{verbatim}

\subsubsection{OR for Combined Effects of Two Variables: Smoking with an
Age
Difference}\label{or-for-combined-effects-of-two-variables-smoking-with-an-age-difference}

Suppose we compare two groups that differ in \textbf{smoking status} and
\textbf{age}:

\begin{itemize}
\tightlist
\item
  \textbf{Group A:} \texttt{smk\ =\ 1}, \texttt{age\ =\ 50} (all other
  covariates equal)
\item
  \textbf{Group B:} \texttt{smk\ =\ 0}, \texttt{age\ =\ 30}
\end{itemize}

The log‑odds contrast is (\(A = \beta_{smk} + (50-20)\beta_{age}\)), so
the OR is (\(\exp(A)\)).

\begin{Shaded}
\begin{Highlighting}[]
\NormalTok{new0 }\OtherTok{\textless{}{-}}\NormalTok{ base\_prof; new0}\SpecialCharTok{$}\NormalTok{age }\OtherTok{\textless{}{-}} \DecValTok{30}\NormalTok{; new0}\SpecialCharTok{$}\NormalTok{smk }\OtherTok{\textless{}{-}} \DecValTok{0}
\NormalTok{new1 }\OtherTok{\textless{}{-}}\NormalTok{ base\_prof; new1}\SpecialCharTok{$}\NormalTok{age }\OtherTok{\textless{}{-}} \DecValTok{50}\NormalTok{; new1}\SpecialCharTok{$}\NormalTok{smk }\OtherTok{\textless{}{-}} \DecValTok{1}

\NormalTok{res\_ageAsmk }\OtherTok{\textless{}{-}} \FunctionTok{or\_from\_predict}\NormalTok{(fit1\_chd, }\AttributeTok{new1 =}\NormalTok{ new1, }\AttributeTok{new0 =}\NormalTok{ new0)}
\end{Highlighting}
\end{Shaded}

\begin{verbatim}

Variables that differ between new0 and new1:
     age smk
new0  30   0
new1  50   1

Odds Ratio summary:
      Estimate CI_low   CI_up
OR      4.6417 1.8546 11.6168
logOR   1.5351 0.6177  2.4525
\end{verbatim}

\section{Assessing Statistical Significance with Wilks' Theorem
(Analogue of F-test for
OLS)}\label{assessing-statistical-significance-with-wilks-theorem-analogue-of-f-test-for-ols}

In the context of logistic regression, Wilks' theorem provides the basis
for the Likelihood Ratio Test (LRT) used to assess the significance of
predictor variables. The theorem states that when comparing a full model
(\(M_1\)) to a nested null model (\(M_0\)), the test statistic,
\(\Lambda\), asymptotically follows a chi-squared (\(\chi^2\))
distribution under the null hypothesis (i.e., that the simpler model
\(M_0\) is correct).

The statistic \(\Lambda\) is calculated as the difference in the
maximized log-likelihoods:
\begin{equation}\phantomsection\label{eq-LRT}{\Lambda = -2(\log L_0 - \log L_1)}\end{equation}
where \(\log L_0\) and \(\log L_1\) are the log-likelihoods of the null
and full models, respectively. In logistic regression, this is
equivalent to the difference in the deviances:
\(\Lambda = \text{Deviance}_0 - \text{Deviance}_1\). This test statistic
\(\Lambda\) represents the reduction in deviance (a measure of
badness-of-fit) achieved by adding the extra predictors to the model.

The following R code chunk generates a conceptual plot of this
relationship:

\begin{Shaded}
\begin{Highlighting}[]
\CommentTok{\#| label: plot{-}lrt{-}concept}
\CommentTok{\#| echo: false}
\CommentTok{\#| fig{-}cap: "Conceptual plot of Deviance versus Number of Parameters, illustrating the Likelihood Ratio Test statistic (Λ) and Residual Deviance. Λ is the drop from the Null Model to the Full Model. The Saturated Model represents perfect fit (Deviance = 0)."}

\FunctionTok{library}\NormalTok{(ggplot2)}

\DocumentationTok{\#\# 1. Create conceptual data for the plot}
\DocumentationTok{\#\# These are just for illustration}
\NormalTok{n\_obs }\OtherTok{\textless{}{-}} \DecValTok{60} \CommentTok{\# Number of observations}
\NormalTok{p0 }\OtherTok{\textless{}{-}} \DecValTok{1}     \CommentTok{\# Parameters in null model (intercept)}
\NormalTok{p1 }\OtherTok{\textless{}{-}} \DecValTok{21}    \CommentTok{\# Parameters in full model (e.g., intercept + 7 predictors)}
\NormalTok{psat }\OtherTok{\textless{}{-}}\NormalTok{ n\_obs }\CommentTok{\# Parameters in saturated model (1 per observation)}

\NormalTok{D0 }\OtherTok{\textless{}{-}} \DecValTok{41} \CommentTok{\# Null deviance}
\NormalTok{D1 }\OtherTok{\textless{}{-}} \DecValTok{20} \CommentTok{\# Full model deviance (residual deviance of M1)}
\NormalTok{D\_sat }\OtherTok{\textless{}{-}} \DecValTok{0} \CommentTok{\# Saturated model deviance}

\DocumentationTok{\#\# Data frame for the three points}
\NormalTok{plot\_data }\OtherTok{\textless{}{-}} \FunctionTok{data.frame}\NormalTok{(}
  \AttributeTok{model =} \FunctionTok{c}\NormalTok{(}\StringTok{"M\_0 (Null)"}\NormalTok{, }\StringTok{"M\_1 (Full)"}\NormalTok{, }\StringTok{"M\_Sat (Saturated)"}\NormalTok{),}
  \AttributeTok{params =} \FunctionTok{c}\NormalTok{(p0, p1, psat),}
  \AttributeTok{deviance =} \FunctionTok{c}\NormalTok{(D0, D1, D\_sat),}
  \DocumentationTok{\#\# Add custom justification and nudges for labels}
  \AttributeTok{hjust\_val =} \FunctionTok{c}\NormalTok{(}\FloatTok{0.5}\NormalTok{, }\FloatTok{0.5}\NormalTok{, }\FloatTok{1.1}\NormalTok{), }\CommentTok{\# Right{-}align the last label}
  \AttributeTok{nudge\_x\_val =} \FunctionTok{c}\NormalTok{(}\DecValTok{0}\NormalTok{, }\DecValTok{0}\NormalTok{, }\DecValTok{0}\NormalTok{) }
\NormalTok{)}

\DocumentationTok{\#\# 2. Create the ggplot}
\FunctionTok{ggplot}\NormalTok{(plot\_data, }\FunctionTok{aes}\NormalTok{(}\AttributeTok{x =}\NormalTok{ params, }\AttributeTok{y =}\NormalTok{ deviance)) }\SpecialCharTok{+}
  \DocumentationTok{\#\# Draw dashed guide lines for D0 and D1}
  \FunctionTok{geom\_segment}\NormalTok{(}\FunctionTok{aes}\NormalTok{(}\AttributeTok{x =}\NormalTok{ p0, }\AttributeTok{y =}\NormalTok{ D0, }\AttributeTok{xend =}\NormalTok{ p1, }\AttributeTok{yend =}\NormalTok{ D0), }\AttributeTok{linetype =} \StringTok{"dashed"}\NormalTok{, }\AttributeTok{color =} \StringTok{"grey70"}\NormalTok{) }\SpecialCharTok{+}
  \FunctionTok{geom\_segment}\NormalTok{(}\FunctionTok{aes}\NormalTok{(}\AttributeTok{x =}\NormalTok{ p1, }\AttributeTok{y =}\NormalTok{ D1, }\AttributeTok{xend =}\NormalTok{ psat, }\AttributeTok{yend =}\NormalTok{ D1), }\AttributeTok{linetype =} \StringTok{"dashed"}\NormalTok{, }\AttributeTok{color =} \StringTok{"grey70"}\NormalTok{) }\SpecialCharTok{+}
  
  \DocumentationTok{\#\# Connect the points with lines}
  \FunctionTok{geom\_line}\NormalTok{(}\AttributeTok{color =} \StringTok{"black"}\NormalTok{, }\AttributeTok{linetype =} \StringTok{"solid"}\NormalTok{, }\AttributeTok{linewidth =} \FloatTok{0.5}\NormalTok{) }\SpecialCharTok{+}
  
  \DocumentationTok{\#\# Draw the main points}
  \FunctionTok{geom\_point}\NormalTok{(}\AttributeTok{size =} \DecValTok{4}\NormalTok{, }\FunctionTok{aes}\NormalTok{(}\AttributeTok{color =}\NormalTok{ model)) }\SpecialCharTok{+}
  
  \DocumentationTok{\#\# {-}{-}{-} MODIFIED LABEL PLACEMENT {-}{-}{-}}
  \DocumentationTok{\#\# Label the points using custom nudge/justification}
  \FunctionTok{geom\_text}\NormalTok{(}
    \FunctionTok{aes}\NormalTok{(}\AttributeTok{label =}\NormalTok{ model, }\AttributeTok{hjust =}\NormalTok{ hjust\_val, }\AttributeTok{nudge\_x =}\NormalTok{ nudge\_x\_val), }
    \AttributeTok{nudge\_y =} \FloatTok{2.5}\NormalTok{,  }\CommentTok{\# Use a much smaller vertical nudge}
    \AttributeTok{size =} \DecValTok{4}
\NormalTok{  ) }\SpecialCharTok{+}
  
  \DocumentationTok{\#\# {-}{-}{-} ADDED BACK D0 and D1 ANNOTATIONS {-}{-}{-}}
  \DocumentationTok{\#\# D0 (Null Deviance)}
  \FunctionTok{geom\_segment}\NormalTok{(}
    \FunctionTok{aes}\NormalTok{(}\AttributeTok{x =}\NormalTok{ p0 }\SpecialCharTok{{-}} \DecValTok{2}\NormalTok{, }\AttributeTok{y =}\NormalTok{ D0, }\AttributeTok{xend =}\NormalTok{ p0 }\SpecialCharTok{{-}} \DecValTok{2}\NormalTok{, }\AttributeTok{yend =}\NormalTok{ D\_sat), }\CommentTok{\# Nudged left}
    \AttributeTok{arrow =} \FunctionTok{arrow}\NormalTok{(}\AttributeTok{ends =} \StringTok{"both"}\NormalTok{, }\AttributeTok{length =} \FunctionTok{unit}\NormalTok{(}\FloatTok{0.1}\NormalTok{, }\StringTok{"inches"}\NormalTok{)),}
    \AttributeTok{color =} \StringTok{"darkgreen"}\NormalTok{,}
    \AttributeTok{linewidth =} \DecValTok{1}
\NormalTok{  ) }\SpecialCharTok{+}
  \FunctionTok{annotate}\NormalTok{(}
    \StringTok{"text"}\NormalTok{,}
    \AttributeTok{x =}\NormalTok{ p0 }\SpecialCharTok{{-}} \DecValTok{3}\NormalTok{, }\AttributeTok{y =}\NormalTok{ D0 }\SpecialCharTok{/} \DecValTok{2}\NormalTok{, }\CommentTok{\# Nudged left}
    \AttributeTok{label =} \StringTok{"D[0]"}\NormalTok{, }\AttributeTok{parse =} \ConstantTok{TRUE}\NormalTok{,}
    \AttributeTok{color =} \StringTok{"darkgreen"}\NormalTok{, }\AttributeTok{hjust =} \FloatTok{0.5}\NormalTok{, }\AttributeTok{size =} \DecValTok{5}
\NormalTok{  ) }\SpecialCharTok{+}
  
  \DocumentationTok{\#\# D1 (Residual Deviance of Full Model)}
  \FunctionTok{geom\_segment}\NormalTok{(}
    \FunctionTok{aes}\NormalTok{(}\AttributeTok{x =}\NormalTok{ psat }\SpecialCharTok{+} \DecValTok{2}\NormalTok{, }\AttributeTok{y =}\NormalTok{ D1, }\AttributeTok{xend =}\NormalTok{ psat }\SpecialCharTok{+} \DecValTok{2}\NormalTok{, }\AttributeTok{yend =}\NormalTok{ D\_sat), }\CommentTok{\# Nudged right}
    \AttributeTok{arrow =} \FunctionTok{arrow}\NormalTok{(}\AttributeTok{ends =} \StringTok{"both"}\NormalTok{, }\AttributeTok{length =} \FunctionTok{unit}\NormalTok{(}\FloatTok{0.1}\NormalTok{, }\StringTok{"inches"}\NormalTok{)),}
    \AttributeTok{color =} \StringTok{"darkblue"}\NormalTok{,}
    \AttributeTok{linewidth =} \DecValTok{1}
\NormalTok{  ) }\SpecialCharTok{+}
  \FunctionTok{annotate}\NormalTok{(}
    \StringTok{"text"}\NormalTok{,}
    \AttributeTok{x =}\NormalTok{ psat }\SpecialCharTok{+} \DecValTok{3}\NormalTok{, }\AttributeTok{y =}\NormalTok{ D1 }\SpecialCharTok{/} \DecValTok{2}\NormalTok{, }\CommentTok{\# Nudged right}
    \AttributeTok{label =} \StringTok{"D[1]"}\NormalTok{, }\AttributeTok{parse =} \ConstantTok{TRUE}\NormalTok{,}
    \AttributeTok{color =} \StringTok{"darkblue"}\NormalTok{, }\AttributeTok{hjust =} \FloatTok{0.5}\NormalTok{, }\AttributeTok{size =} \DecValTok{5}
\NormalTok{  ) }\SpecialCharTok{+}
  
  \DocumentationTok{\#\# LRT statistic Λ = D0 {-} D1}
  \FunctionTok{geom\_segment}\NormalTok{(}
    \FunctionTok{aes}\NormalTok{(}\AttributeTok{x =}\NormalTok{ p1 }\SpecialCharTok{+} \DecValTok{2}\NormalTok{, }\AttributeTok{y =}\NormalTok{ D0, }\AttributeTok{xend =}\NormalTok{ p1 }\SpecialCharTok{+} \DecValTok{2}\NormalTok{, }\AttributeTok{yend =}\NormalTok{ D1), }\CommentTok{\# Nudged right}
    \AttributeTok{arrow =} \FunctionTok{arrow}\NormalTok{(}\AttributeTok{ends =} \StringTok{"both"}\NormalTok{, }\AttributeTok{length =} \FunctionTok{unit}\NormalTok{(}\FloatTok{0.1}\NormalTok{, }\StringTok{"inches"}\NormalTok{)),}
    \AttributeTok{color =} \StringTok{"red"}\NormalTok{,}
    \AttributeTok{linewidth =} \DecValTok{1}
\NormalTok{  ) }\SpecialCharTok{+}
  \FunctionTok{annotate}\NormalTok{(}
    \StringTok{"text"}\NormalTok{,}
    \AttributeTok{x =}\NormalTok{ p1 }\SpecialCharTok{+} \DecValTok{3}\NormalTok{, }\AttributeTok{y =}\NormalTok{ D1 }\SpecialCharTok{+}\NormalTok{ (D0 }\SpecialCharTok{{-}}\NormalTok{ D1) }\SpecialCharTok{/} \DecValTok{2}\NormalTok{, }\CommentTok{\# Nudged right}
    \AttributeTok{label =} \StringTok{"Lambda == D[0] {-} D[1]"}\NormalTok{,}
    \AttributeTok{parse =} \ConstantTok{TRUE}\NormalTok{,}
    \AttributeTok{color =} \StringTok{"red"}\NormalTok{, }\AttributeTok{hjust =} \DecValTok{0}\NormalTok{, }\AttributeTok{size =} \DecValTok{5}
\NormalTok{  ) }\SpecialCharTok{+}
  
  \DocumentationTok{\#\# Customize axes to show the symbolic labels}
  \FunctionTok{scale\_x\_continuous}\NormalTok{(}
    \AttributeTok{breaks =} \FunctionTok{c}\NormalTok{(p0, p1, psat),}
    \AttributeTok{labels =} \FunctionTok{c}\NormalTok{(}\FunctionTok{expression}\NormalTok{(p[}\DecValTok{0}\NormalTok{]), }\FunctionTok{expression}\NormalTok{(p[}\DecValTok{1}\NormalTok{]), }\FunctionTok{expression}\NormalTok{(n)),}
    \AttributeTok{expand =} \FunctionTok{expansion}\NormalTok{(}\AttributeTok{mult =} \FloatTok{0.1}\NormalTok{) }\CommentTok{\# Add some padding}
\NormalTok{  ) }\SpecialCharTok{+}
  \FunctionTok{scale\_y\_continuous}\NormalTok{(}
    \AttributeTok{breaks =} \FunctionTok{c}\NormalTok{(D\_sat, D1, D0),}
    \AttributeTok{labels =} \FunctionTok{c}\NormalTok{(}\FunctionTok{expression}\NormalTok{(}\DecValTok{0}\NormalTok{), }\FunctionTok{expression}\NormalTok{(D[}\DecValTok{1}\NormalTok{]), }\FunctionTok{expression}\NormalTok{(D[}\DecValTok{0}\NormalTok{]))}
\NormalTok{  ) }\SpecialCharTok{+}
  
  \DocumentationTok{\#\# Labels and Title}
  \FunctionTok{labs}\NormalTok{(}
    \AttributeTok{title =} \StringTok{"Relationship between Deviance and Model Complexity"}\NormalTok{,}
    \AttributeTok{x =} \StringTok{"Number of Parameters"}\NormalTok{,}
    \AttributeTok{y =} \StringTok{"Model Deviance"}
\NormalTok{  ) }\SpecialCharTok{+}
  
  \DocumentationTok{\#\# Clean theme}
  \FunctionTok{theme\_minimal}\NormalTok{(}\AttributeTok{base\_size =} \DecValTok{14}\NormalTok{) }\SpecialCharTok{+}
  \FunctionTok{theme}\NormalTok{(}
    \AttributeTok{plot.title =} \FunctionTok{element\_text}\NormalTok{(}\AttributeTok{hjust =} \FloatTok{0.5}\NormalTok{),}
    \AttributeTok{legend.position =} \StringTok{"none"} \CommentTok{\# Remove legend, as points are labeled}
\NormalTok{  ) }\SpecialCharTok{+}
  \FunctionTok{scale\_color\_manual}\NormalTok{(}\AttributeTok{values =} \FunctionTok{c}\NormalTok{(}\StringTok{"M\_0 (Null)"} \OtherTok{=} \StringTok{"blue"}\NormalTok{, }\StringTok{"M\_1 (Full)"} \OtherTok{=} \StringTok{"blue"}\NormalTok{, }\StringTok{"M\_Sat (Saturated)"} \OtherTok{=} \StringTok{"red"}\NormalTok{))}
\end{Highlighting}
\end{Shaded}

\includegraphics{unit4-lr/logistic_files/figure-pdf/unnamed-chunk-6-1.pdf}

As the diagram illustrates, the null model (\(M_0\)) has fewer
parameters (\(p_0\)) and a higher deviance (\(D_0\), or worse fit),
while the full model (\(M_1\)) has more parameters (\(p_1\)) and a lower
deviance (\(D_1\)). The Likelihood Ratio Test statistic \(D\) is the
magnitude of this drop in deviance.

For assessing the overall significance of a regression model
(\texttt{fit1\_chd}), this involves comparing it to its corresponding
intercept-only (null) model. The degrees of freedom for the \(\chi^2\)
test is the difference in the number of parameters, \(df = p_1 - p_0\),
which equals the number of predictors in the full model.

Here is an R code chunk demonstrating how to compute this p-value
directly from a \texttt{glm} fit object, assuming it is named
\texttt{fit1\_chd}.

\begin{Shaded}
\begin{Highlighting}[]
\DocumentationTok{\#\# Calculate the Likelihood Ratio Test statistic (D) and degrees of freedom (df)}
\DocumentationTok{\#\# by comparing the model\textquotesingle{}s deviance to the null (intercept{-}only) deviance,}
\DocumentationTok{\#\# both of which are stored in the \textquotesingle{}fit1\_chd\textquotesingle{} object.}
\FunctionTok{summary}\NormalTok{(fit1\_chd)}
\end{Highlighting}
\end{Shaded}

\begin{verbatim}

Call:
glm(formula = chd ~ smk + cat + sbp + age + chl + ecg + hpt, 
    family = binomial(link = "logit"), data = CHD.data)

Coefficients:
             Estimate Std. Error z value Pr(>|z|)    
(Intercept) -6.048892   1.345165  -4.497  6.9e-06 ***
smk          0.855951   0.306505   2.793  0.00523 ** 
cat          0.732763   0.376129   1.948  0.05139 .  
sbp         -0.006995   0.006976  -1.003  0.31600    
age          0.033956   0.015344   2.213  0.02690 *  
chl          0.008970   0.003274   2.740  0.00615 ** 
ecg          0.417776   0.295553   1.414  0.15750    
hpt          0.655498   0.359976   1.821  0.06861 .  
---
Signif. codes:  0 '***' 0.001 '**' 0.01 '*' 0.05 '.' 0.1 ' ' 1

(Dispersion parameter for binomial family taken to be 1)

    Null deviance: 438.56  on 608  degrees of freedom
Residual deviance: 399.35  on 601  degrees of freedom
AIC: 415.35

Number of Fisher Scoring iterations: 5
\end{verbatim}

\begin{Shaded}
\begin{Highlighting}[]
\NormalTok{lrt\_statistic }\OtherTok{\textless{}{-}}\NormalTok{ fit1\_chd}\SpecialCharTok{$}\NormalTok{null.deviance }\SpecialCharTok{{-}}\NormalTok{ fit1\_chd}\SpecialCharTok{$}\NormalTok{deviance}
\NormalTok{lrt\_df }\OtherTok{\textless{}{-}}\NormalTok{ fit1\_chd}\SpecialCharTok{$}\NormalTok{df.null }\SpecialCharTok{{-}}\NormalTok{ fit1\_chd}\SpecialCharTok{$}\NormalTok{df.residual}

\DocumentationTok{\#\# Compute the p{-}value from the chi{-}squared distribution}
\DocumentationTok{\#\# We use lower.tail = FALSE to get P(ChiSq \textgreater{} D)}
\NormalTok{p\_value }\OtherTok{\textless{}{-}} \FunctionTok{pchisq}\NormalTok{(lrt\_statistic, lrt\_df, }\AttributeTok{lower.tail =} \ConstantTok{FALSE}\NormalTok{)}

\DocumentationTok{\#\# Create and print the result in an ANOVA{-}like table}
\DocumentationTok{\#\# Row 1: Null model}
\DocumentationTok{\#\# Row 2: Full model (fit1\_chd), showing the test against the null}
\NormalTok{lrt\_table }\OtherTok{\textless{}{-}} \FunctionTok{data.frame}\NormalTok{(}
  \StringTok{"Resid. Df"} \OtherTok{=} \FunctionTok{c}\NormalTok{(fit1\_chd}\SpecialCharTok{$}\NormalTok{df.null, fit1\_chd}\SpecialCharTok{$}\NormalTok{df.residual),}
  \StringTok{"Resid. Dev"} \OtherTok{=} \FunctionTok{c}\NormalTok{(}\FunctionTok{round}\NormalTok{(fit1\_chd}\SpecialCharTok{$}\NormalTok{null.deviance, }\DecValTok{4}\NormalTok{), }\FunctionTok{round}\NormalTok{(fit1\_chd}\SpecialCharTok{$}\NormalTok{deviance, }\DecValTok{4}\NormalTok{)),}
  \StringTok{"Test Df"} \OtherTok{=} \FunctionTok{c}\NormalTok{(}\ConstantTok{NA}\NormalTok{, lrt\_df),}
  \StringTok{"Test Statistic (D)"} \OtherTok{=} \FunctionTok{c}\NormalTok{(}\ConstantTok{NA}\NormalTok{, }\FunctionTok{round}\NormalTok{(lrt\_statistic, }\DecValTok{4}\NormalTok{)),}
  \StringTok{"p{-}value"} \OtherTok{=} \FunctionTok{c}\NormalTok{(}\ConstantTok{NA}\NormalTok{, }\FunctionTok{format.pval}\NormalTok{(p\_value, }\AttributeTok{digits =} \DecValTok{4}\NormalTok{)),}
  \AttributeTok{row.names =} \FunctionTok{c}\NormalTok{(}\StringTok{"Null Model"}\NormalTok{, }\StringTok{"Full Model (fit1\_chd)"}\NormalTok{),}
  \AttributeTok{check.names =} \ConstantTok{FALSE} \CommentTok{\# Prevent R from changing \textquotesingle{}p{-}value\textquotesingle{} to \textquotesingle{}p.value\textquotesingle{}}
\NormalTok{)}

\FunctionTok{cat}\NormalTok{(}\StringTok{"Likelihood Ratio Test for Model Significance:}\SpecialCharTok{\textbackslash{}n}\StringTok{"}\NormalTok{)}
\end{Highlighting}
\end{Shaded}

\begin{verbatim}
Likelihood Ratio Test for Model Significance:
\end{verbatim}

\begin{Shaded}
\begin{Highlighting}[]
\NormalTok{lrt\_table}
\end{Highlighting}
\end{Shaded}

\begin{verbatim}
                      Resid. Df Resid. Dev Test Df Test Statistic (D)   p-value
Null Model                  608   438.5583      NA                 NA      <NA>
Full Model (fit1_chd)       601   399.3539       7            39.2044 1.787e-06
\end{verbatim}

\textbf{Using built-in anova() function}

\begin{Shaded}
\begin{Highlighting}[]
\NormalTok{fit0\_chd }\OtherTok{\textless{}{-}} \FunctionTok{glm}\NormalTok{ (chd}\SpecialCharTok{\textasciitilde{}}\DecValTok{1}\NormalTok{, }\AttributeTok{data =}\NormalTok{ CHD.data, }\AttributeTok{family =} \FunctionTok{binomial}\NormalTok{())}
\FunctionTok{anova}\NormalTok{(fit0\_chd, fit1\_chd)}
\end{Highlighting}
\end{Shaded}

\begin{verbatim}
Analysis of Deviance Table

Model 1: chd ~ 1
Model 2: chd ~ smk + cat + sbp + age + chl + ecg + hpt
  Resid. Df Resid. Dev Df Deviance  Pr(>Chi)    
1       608     438.56                          
2       601     399.35  7   39.204 1.787e-06 ***
---
Signif. codes:  0 '***' 0.001 '**' 0.01 '*' 0.05 '.' 0.1 ' ' 1
\end{verbatim}

\begin{Shaded}
\begin{Highlighting}[]
\FunctionTok{anova}\NormalTok{(fit1\_chd, }\AttributeTok{test=}\StringTok{"LRT"}\NormalTok{)}
\end{Highlighting}
\end{Shaded}

\begin{verbatim}
Analysis of Deviance Table

Model: binomial, link: logit

Response: chd

Terms added sequentially (first to last)

     Df Deviance Resid. Df Resid. Dev  Pr(>Chi)    
NULL                   608     438.56              
smk   1   5.7453       607     432.81 0.0165324 *  
cat   1  14.3716       606     418.44 0.0001501 ***
sbp   1   0.7574       605     417.68 0.3841353    
age   1   5.2821       604     412.40 0.0215455 *  
chl   1   7.8619       603     404.54 0.0050489 ** 
ecg   1   1.8701       602     402.67 0.1714609    
hpt   1   3.3159       601     399.35 0.0686113 .  
---
Signif. codes:  0 '***' 0.001 '**' 0.01 '*' 0.05 '.' 0.1 ' ' 1
\end{verbatim}

\section{\texorpdfstring{Assessing Predictive Effect-Size (Anologue to
\(R^2_\mathrm{adj}\))}{Assessing Predictive Effect-Size (Anologue to R\^{}2\_\textbackslash mathrm\{adj\})}}\label{assessing-predictive-effect-size-anologue-to-r2_mathrmadj}

While the LRT assesses overall model significance (in-sample fit), it's
also crucial to evaluate how well the model predicts new, unseen data
(out-of-sample performance). A common method is to split the data into a
training set (e.g., 2/3 of the data) and a test set (e.g., 1/3). The
model is fit using only the training data and then used to make
predictions for the test data. We can then compare these predictions to
the actual outcomes in the test set.

\subsection{Understanding the Confusion Matrix and
Metrics}\label{understanding-the-confusion-matrix-and-metrics}

To evaluate a model's predictive performance, we classify its
probabilistic predictions using a threshold (typically 0.5) and compare
them to the true outcomes in a \textbf{Confusion Matrix}:

\begin{longtable}[]{@{}lll@{}}
\toprule\noalign{}
& \textbf{Predicted: 0} & \textbf{Predicted: 1} \\
\midrule\noalign{}
\endhead
\bottomrule\noalign{}
\endlastfoot
\textbf{Actual: 0} & True Negative (TN) & False Positive (FP) \\
\textbf{Actual: 1} & False Negative (FN) & True Positive (TP) \\
\end{longtable}

From this matrix, we derive several key performance metrics:

\begin{itemize}
\item
  \textbf{Misclassification Error Rate (ER):} The proportion of all
  predictions that were incorrect. \[
    \text{Error Rate} = \frac{FP + FN}{TP + TN + FP + FN}
    \]
\item
  \textbf{Precision (Positive Predictive Value):} Answers: ``Of all the
  times the model predicted positive, how often was it correct?'' This
  is crucial when the cost of a \textbf{False Positive} is high. \[
    \text{Precision} = \frac{TP}{TP + FP}
    \]
\item
  \textbf{Recall (Sensitivity or True Positive Rate):} Answers: ``Of all
  the actual positive cases, how many did the model find?'' This is
  crucial when the cost of a \textbf{False Negative} is high. \[
    \text{Recall (TPR)} = \frac{TP}{TP + FN}
    \]
\item
  \textbf{ROC Curve and AUC:} An \textbf{ROC (Receiver Operating
  Characteristic) Curve} is a graph that shows a model's diagnostic
  ability across \emph{all possible classification thresholds}. It plots
  the \textbf{True Positive Rate (Recall)} on the y-axis against the
  \textbf{False Positive Rate} (FPR = \(\frac{FP}{FP + TN}\)) on the
  x-axis.

  \begin{itemize}
  \tightlist
  \item
    \textbf{Interpretation:} The curve shows the trade-off between
    sensitivity (finding all the positives) and specificity (not
    mislabeling negatives). A random ``no-skill'' classifier is
    represented by a diagonal line from (0,0) to (1,1). A perfect
    classifier would hug the \textbf{top-left corner} (TPR = 1, FPR =
    0).
  \item
    \textbf{AUC (Area Under the Curve):} The AUC summarizes the entire
    curve into a single number from 0 to 1. An AUC of 0.5 corresponds to
    a random guess, while an AUC of 1.0 represents a perfect model.
  \end{itemize}
\item
  \textbf{Precision-Recall (PR) Curve:} A \textbf{PR Curve} plots
  \textbf{Precision} (y-axis) against \textbf{Recall} (x-axis) at all
  possible thresholds.

  \begin{itemize}
  \tightlist
  \item
    \textbf{Interpretation:} This curve shows the trade-off between how
    \emph{reliable} a positive prediction is (Precision) and how
    \emph{complete} the model is at finding all positives (Recall).
  \item
    \textbf{When to Use:} The PR curve is particularly informative when
    the dataset is \textbf{imbalanced} (i.e., one class, like ``fraud''
    or ``disease,'' is much rarer than the other). Unlike the ROC curve,
    the PR curve's baseline (the ``no-skill'' line) is a horizontal line
    at the proportion of positive cases, which makes it easier to see if
    the model is performing significantly better than chance in a
    low-positive-rate scenario. A perfect classifier would hug the
    \textbf{top-right corner} (Precision = 1, Recall = 1).
  \end{itemize}
\end{itemize}

\subsection{Illustration with the Simulated
Dataset}\label{illustration-with-the-simulated-dataset}

This section applies the train/test split and model evaluation workflow
to the \texttt{sim.data} created in the previous step.

\begin{Shaded}
\begin{Highlighting}[]
\DocumentationTok{\#\# Load the pROC library for AUC calculation}
\DocumentationTok{\#\# install.packages("pROC") \# Uncomment to install if needed}
\FunctionTok{library}\NormalTok{(pROC)}

\DocumentationTok{\#\# {-}{-}{-} 1. Split the data {-}{-}{-}}
\DocumentationTok{\#\# We use \textquotesingle{}sim.data\textquotesingle{} which has 200 rows}
\FunctionTok{set.seed}\NormalTok{(}\DecValTok{123}\NormalTok{) }\CommentTok{\# for reproducibility}
\NormalTok{n\_sim }\OtherTok{\textless{}{-}} \FunctionTok{nrow}\NormalTok{(sim.data)}
\NormalTok{train\_size\_sim }\OtherTok{\textless{}{-}} \FunctionTok{floor}\NormalTok{(}\DecValTok{2}\SpecialCharTok{/}\DecValTok{3} \SpecialCharTok{*}\NormalTok{ n\_sim)}
\NormalTok{train\_indices\_sim }\OtherTok{\textless{}{-}} \FunctionTok{sample}\NormalTok{(}\DecValTok{1}\SpecialCharTok{:}\NormalTok{n\_sim, }\AttributeTok{size =}\NormalTok{ train\_size\_sim)}
\NormalTok{train\_data\_sim }\OtherTok{\textless{}{-}}\NormalTok{ sim.data[train\_indices\_sim, ]}
\NormalTok{test\_data\_sim  }\OtherTok{\textless{}{-}}\NormalTok{ sim.data[}\SpecialCharTok{{-}}\NormalTok{train\_indices\_sim, ]}

\DocumentationTok{\#\# {-}{-}{-} 2. Refit the model on the training data {-}{-}{-}}
\DocumentationTok{\#\# We fit the model y \textasciitilde{} x on the training data}
\NormalTok{fit\_train\_sim }\OtherTok{\textless{}{-}} \FunctionTok{glm}\NormalTok{(}
\NormalTok{  y }\SpecialCharTok{\textasciitilde{}}\NormalTok{ x,}
  \AttributeTok{data =}\NormalTok{ train\_data\_sim,}
  \AttributeTok{family =} \FunctionTok{binomial}\NormalTok{(}\AttributeTok{link =} \StringTok{"logit"}\NormalTok{)}
\NormalTok{)}

\DocumentationTok{\#\# {-}{-}{-} 3. Make predictions on the test data {-}{-}{-}}
\DocumentationTok{\#\# Note: The true probabilities \textquotesingle{}p\textquotesingle{} are also in test\_data\_sim}
\DocumentationTok{\#\# We predict from the *fitted* model}
\NormalTok{pred\_probs\_sim }\OtherTok{\textless{}{-}} \FunctionTok{predict}\NormalTok{(fit\_train\_sim, }\AttributeTok{newdata =}\NormalTok{ test\_data\_sim, }\AttributeTok{type =} \StringTok{"response"}\NormalTok{)}
\end{Highlighting}
\end{Shaded}

\textbf{Plotting the Predictive Probabilities with True Labels}

\begin{Shaded}
\begin{Highlighting}[]
\DocumentationTok{\#\# {-}{-}{-} 5. Plot sorted predicted probabilities {-}{-}{-}}

\DocumentationTok{\#\# Create a data frame for plotting}
\NormalTok{plot\_data\_sim }\OtherTok{\textless{}{-}} \FunctionTok{data.frame}\NormalTok{(}
  \AttributeTok{Prob =}\NormalTok{ pred\_probs\_sim,}
  \AttributeTok{Actual =} \FunctionTok{as.factor}\NormalTok{(test\_data\_sim}\SpecialCharTok{$}\NormalTok{y),}
  \AttributeTok{TrueProb =}\NormalTok{ test\_data\_sim}\SpecialCharTok{$}\NormalTok{p }\CommentTok{\# Include true probs for comparison}
\NormalTok{)}

\DocumentationTok{\#\# Sort by predicted probability}
\NormalTok{plot\_data\_sim }\OtherTok{\textless{}{-}}\NormalTok{ plot\_data\_sim[}\FunctionTok{order}\NormalTok{(plot\_data\_sim}\SpecialCharTok{$}\NormalTok{Prob), ]}
\NormalTok{plot\_data\_sim}\SpecialCharTok{$}\NormalTok{Rank }\OtherTok{\textless{}{-}} \DecValTok{1}\SpecialCharTok{:}\FunctionTok{nrow}\NormalTok{(plot\_data\_sim)}

\DocumentationTok{\#\# Create the plot}
\FunctionTok{plot}\NormalTok{(}
\NormalTok{  plot\_data\_sim}\SpecialCharTok{$}\NormalTok{Rank,}
\NormalTok{  plot\_data\_sim}\SpecialCharTok{$}\NormalTok{Prob,}
  \AttributeTok{pch =} \FunctionTok{ifelse}\NormalTok{(plot\_data\_sim}\SpecialCharTok{$}\NormalTok{Actual }\SpecialCharTok{==} \DecValTok{0}\NormalTok{, }\DecValTok{1}\NormalTok{, }\DecValTok{4}\NormalTok{),}
  \AttributeTok{col =} \FunctionTok{ifelse}\NormalTok{(plot\_data\_sim}\SpecialCharTok{$}\NormalTok{Actual }\SpecialCharTok{==} \DecValTok{0}\NormalTok{, }\StringTok{"blue"}\NormalTok{, }\StringTok{"red"}\NormalTok{),}
  \AttributeTok{xlab =} \StringTok{"Index (Sorted by Predicted Probability)"}\NormalTok{,}
  \AttributeTok{ylab =} \StringTok{"Predicted Probability"}\NormalTok{,}
  \AttributeTok{main =} \StringTok{"Predicted Probabilities vs. Actual Class (Simulated Data)"}\NormalTok{,}
  \AttributeTok{ylim =} \FunctionTok{c}\NormalTok{(}\DecValTok{0}\NormalTok{, }\DecValTok{1}\NormalTok{)}
\NormalTok{)}
\FunctionTok{abline}\NormalTok{(}\AttributeTok{h =} \FloatTok{0.5}\NormalTok{, }\AttributeTok{lty =} \DecValTok{2}\NormalTok{, }\AttributeTok{col =} \StringTok{"black"}\NormalTok{)}
\FunctionTok{abline}\NormalTok{(}\AttributeTok{h =} \FloatTok{0.1}\NormalTok{, }\AttributeTok{lty =} \DecValTok{3}\NormalTok{, }\AttributeTok{col =} \StringTok{"grey"}\NormalTok{)}

\DocumentationTok{\#\# Add the true probability curve (sorted by predicted prob)}
\DocumentationTok{\#\# This shows how well the fitted model\textquotesingle{}s predictions align with the true probs}
\CommentTok{\#lines(plot\_data\_sim$Rank, plot\_data\_sim$TrueProb[order(plot\_data\_sim$Prob)], col = "darkgreen", lwd = 2)}


\DocumentationTok{\#\# Add a legend}
\FunctionTok{legend}\NormalTok{(}
  \StringTok{"topleft"}\NormalTok{,}
  \AttributeTok{legend =} \FunctionTok{c}\NormalTok{(}\StringTok{"Actual 0 (o)"}\NormalTok{, }\StringTok{"Actual 1 (x)"}\NormalTok{),}
  \AttributeTok{pch =} \FunctionTok{c}\NormalTok{(}\DecValTok{1}\NormalTok{, }\DecValTok{4}\NormalTok{),}
  \AttributeTok{lty =} \FunctionTok{c}\NormalTok{(}\ConstantTok{NA}\NormalTok{, }\ConstantTok{NA}\NormalTok{),}
  \AttributeTok{lwd =} \FunctionTok{c}\NormalTok{(}\ConstantTok{NA}\NormalTok{, }\ConstantTok{NA}\NormalTok{),}
  \AttributeTok{col =} \FunctionTok{c}\NormalTok{(}\StringTok{"blue"}\NormalTok{, }\StringTok{"red"}\NormalTok{)}
\NormalTok{)}
\end{Highlighting}
\end{Shaded}

\includegraphics{unit4-lr/logistic_files/figure-pdf/plot-sorted-probabilities-sim-1.pdf}

\textbf{Confusion Matrix with threshold=0.5}

\begin{Shaded}
\begin{Highlighting}[]
\DocumentationTok{\#\# {-}{-}{-} 4. Assess accuracy {-}{-}{-}}

\DocumentationTok{\#\# 4a. Misclassification Error Rate (using 0.5 threshold)}
\NormalTok{threshold }\OtherTok{\textless{}{-}} \FloatTok{0.5}
\NormalTok{pred\_class\_sim }\OtherTok{\textless{}{-}} \FunctionTok{ifelse}\NormalTok{(pred\_probs\_sim }\SpecialCharTok{\textgreater{}}\NormalTok{ threshold, }\DecValTok{1}\NormalTok{, }\DecValTok{0}\NormalTok{)}
\NormalTok{conf\_matrix\_sim }\OtherTok{\textless{}{-}} \FunctionTok{table}\NormalTok{(}\AttributeTok{Actual =}\NormalTok{ test\_data\_sim}\SpecialCharTok{$}\NormalTok{y, }\AttributeTok{Predicted =}\NormalTok{ pred\_class\_sim)}

\DocumentationTok{\#\# {-}{-}{-} MODIFIED LINES START {-}{-}{-}}
\FunctionTok{cat}\NormalTok{(}\StringTok{"Confusion Matrix (Counts, threshold = 0.5):}\SpecialCharTok{\textbackslash{}n}\StringTok{"}\NormalTok{)}
\end{Highlighting}
\end{Shaded}

\begin{verbatim}
Confusion Matrix (Counts, threshold = 0.5):
\end{verbatim}

\begin{Shaded}
\begin{Highlighting}[]
\FunctionTok{print}\NormalTok{(conf\_matrix\_sim)}
\end{Highlighting}
\end{Shaded}

\begin{verbatim}
      Predicted
Actual  0  1
     0 26  5
     1  9 27
\end{verbatim}

\begin{Shaded}
\begin{Highlighting}[]
\FunctionTok{cat}\NormalTok{(}\StringTok{"}\SpecialCharTok{\textbackslash{}n}\StringTok{Row Proportions (Given Actual, \% Predicted {-}{-} Relates to TPR/FPR):}\SpecialCharTok{\textbackslash{}n}\StringTok{"}\NormalTok{)}
\end{Highlighting}
\end{Shaded}

\begin{verbatim}

Row Proportions (Given Actual, % Predicted -- Relates to TPR/FPR):
\end{verbatim}

\begin{Shaded}
\begin{Highlighting}[]
\DocumentationTok{\#\# margin = 1 calculates proportions across rows}
\FunctionTok{print}\NormalTok{(}\FunctionTok{round}\NormalTok{(}\FunctionTok{prop.table}\NormalTok{(conf\_matrix\_sim, }\AttributeTok{margin =} \DecValTok{1}\NormalTok{), }\DecValTok{3}\NormalTok{))}
\end{Highlighting}
\end{Shaded}

\begin{verbatim}
      Predicted
Actual     0     1
     0 0.839 0.161
     1 0.250 0.750
\end{verbatim}

\begin{Shaded}
\begin{Highlighting}[]
\FunctionTok{cat}\NormalTok{(}\StringTok{"}\SpecialCharTok{\textbackslash{}n}\StringTok{Column Proportions (Given Predicted, \% Actual {-}{-} Relates to Precision):}\SpecialCharTok{\textbackslash{}n}\StringTok{"}\NormalTok{)}
\end{Highlighting}
\end{Shaded}

\begin{verbatim}

Column Proportions (Given Predicted, % Actual -- Relates to Precision):
\end{verbatim}

\begin{Shaded}
\begin{Highlighting}[]
\DocumentationTok{\#\# margin = 2 calculates proportions across columns}
\FunctionTok{print}\NormalTok{(}\FunctionTok{round}\NormalTok{(}\FunctionTok{prop.table}\NormalTok{(conf\_matrix\_sim, }\AttributeTok{margin =} \DecValTok{2}\NormalTok{), }\DecValTok{3}\NormalTok{))}
\end{Highlighting}
\end{Shaded}

\begin{verbatim}
      Predicted
Actual     0     1
     0 0.743 0.156
     1 0.257 0.844
\end{verbatim}

\begin{Shaded}
\begin{Highlighting}[]
\DocumentationTok{\#\# {-}{-}{-} MODIFIED LINES }\RegionMarkerTok{END}\DocumentationTok{ {-}{-}{-}}


\DocumentationTok{\#\# Check if matrix has 2x2 dimensions, otherwise metrics will fail}
\ControlFlowTok{if}\NormalTok{ (}\FunctionTok{all}\NormalTok{(}\FunctionTok{dim}\NormalTok{(conf\_matrix\_sim) }\SpecialCharTok{==} \FunctionTok{c}\NormalTok{(}\DecValTok{2}\NormalTok{, }\DecValTok{2}\NormalTok{))) \{}
\NormalTok{  TN }\OtherTok{\textless{}{-}}\NormalTok{ conf\_matrix\_sim[}\DecValTok{1}\NormalTok{, }\DecValTok{1}\NormalTok{]}
\NormalTok{  FP }\OtherTok{\textless{}{-}}\NormalTok{ conf\_matrix\_sim[}\DecValTok{1}\NormalTok{, }\DecValTok{2}\NormalTok{]}
\NormalTok{  FN }\OtherTok{\textless{}{-}}\NormalTok{ conf\_matrix\_sim[}\DecValTok{2}\NormalTok{, }\DecValTok{1}\NormalTok{]}
\NormalTok{  TP }\OtherTok{\textless{}{-}}\NormalTok{ conf\_matrix\_sim[}\DecValTok{2}\NormalTok{, }\DecValTok{2}\NormalTok{]}

  \DocumentationTok{\#\# Calculate metrics}
\NormalTok{  error\_rate }\OtherTok{\textless{}{-}}\NormalTok{ (FP }\SpecialCharTok{+}\NormalTok{ FN) }\SpecialCharTok{/}\NormalTok{ (TP }\SpecialCharTok{+}\NormalTok{ TN }\SpecialCharTok{+}\NormalTok{ FP }\SpecialCharTok{+}\NormalTok{ FN)}
\NormalTok{  TPR\_Recall }\OtherTok{\textless{}{-}}\NormalTok{ TP }\SpecialCharTok{/}\NormalTok{ (TP }\SpecialCharTok{+}\NormalTok{ FN) }\CommentTok{\# True Positive Rate (Recall / Sensitivity)}
\NormalTok{  FPR }\OtherTok{\textless{}{-}}\NormalTok{ FP }\SpecialCharTok{/}\NormalTok{ (FP }\SpecialCharTok{+}\NormalTok{ TN)      }\CommentTok{\# False Positive Rate (1 {-} Specificity)}
\NormalTok{  Precision }\OtherTok{\textless{}{-}}\NormalTok{ TP }\SpecialCharTok{/}\NormalTok{ (TP }\SpecialCharTok{+}\NormalTok{ FP)  }\CommentTok{\# Positive Predictive Value}

  \FunctionTok{cat}\NormalTok{(}\FunctionTok{paste}\NormalTok{(}\StringTok{"}\SpecialCharTok{\textbackslash{}n}\StringTok{Misclassification Error Rate:"}\NormalTok{, }\FunctionTok{round}\NormalTok{(error\_rate, }\DecValTok{4}\NormalTok{), }\StringTok{"}\SpecialCharTok{\textbackslash{}n}\StringTok{"}\NormalTok{))}
  \FunctionTok{cat}\NormalTok{(}\FunctionTok{paste}\NormalTok{(}\StringTok{"True Positive Rate (Recall):"}\NormalTok{, }\FunctionTok{round}\NormalTok{(TPR\_Recall, }\DecValTok{4}\NormalTok{), }\StringTok{"}\SpecialCharTok{\textbackslash{}n}\StringTok{"}\NormalTok{))}
  \FunctionTok{cat}\NormalTok{(}\FunctionTok{paste}\NormalTok{(}\StringTok{"False Positive Rate:"}\NormalTok{, }\FunctionTok{round}\NormalTok{(FPR, }\DecValTok{4}\NormalTok{), }\StringTok{"}\SpecialCharTok{\textbackslash{}n}\StringTok{"}\NormalTok{))}
  \FunctionTok{cat}\NormalTok{(}\FunctionTok{paste}\NormalTok{(}\StringTok{"Precision:"}\NormalTok{, }\FunctionTok{round}\NormalTok{(Precision, }\DecValTok{4}\NormalTok{), }\StringTok{"}\SpecialCharTok{\textbackslash{}n}\StringTok{"}\NormalTok{))}
\NormalTok{\} }\ControlFlowTok{else}\NormalTok{ \{}
  \FunctionTok{cat}\NormalTok{(}\StringTok{"}\SpecialCharTok{\textbackslash{}n}\StringTok{Cannot calculate full metrics: model predicted only one class.}\SpecialCharTok{\textbackslash{}n}\StringTok{"}\NormalTok{)}
\NormalTok{\}}
\end{Highlighting}
\end{Shaded}

\begin{verbatim}

Misclassification Error Rate: 0.209 
True Positive Rate (Recall): 0.75 
False Positive Rate: 0.1613 
Precision: 0.8438 
\end{verbatim}

\textbf{Confusion Matrix with threshold=0.1}

\begin{Shaded}
\begin{Highlighting}[]
\NormalTok{threshold }\OtherTok{\textless{}{-}} \FloatTok{0.1}
\NormalTok{pred\_class\_sim }\OtherTok{\textless{}{-}} \FunctionTok{ifelse}\NormalTok{(pred\_probs\_sim }\SpecialCharTok{\textgreater{}}\NormalTok{ threshold, }\DecValTok{1}\NormalTok{, }\DecValTok{0}\NormalTok{)}
\NormalTok{conf\_matrix\_sim }\OtherTok{\textless{}{-}} \FunctionTok{table}\NormalTok{(}\AttributeTok{Actual =}\NormalTok{ test\_data\_sim}\SpecialCharTok{$}\NormalTok{y, }\AttributeTok{Predicted =}\NormalTok{ pred\_class\_sim)}

\DocumentationTok{\#\# {-}{-}{-} MODIFIED LINES START {-}{-}{-}}
\FunctionTok{cat}\NormalTok{(}\StringTok{"Confusion Matrix (Counts, threshold = 0.1):}\SpecialCharTok{\textbackslash{}n}\StringTok{"}\NormalTok{)}
\end{Highlighting}
\end{Shaded}

\begin{verbatim}
Confusion Matrix (Counts, threshold = 0.1):
\end{verbatim}

\begin{Shaded}
\begin{Highlighting}[]
\FunctionTok{print}\NormalTok{(conf\_matrix\_sim)}
\end{Highlighting}
\end{Shaded}

\begin{verbatim}
      Predicted
Actual  0  1
     0 15 16
     1  1 35
\end{verbatim}

\begin{Shaded}
\begin{Highlighting}[]
\FunctionTok{cat}\NormalTok{(}\StringTok{"}\SpecialCharTok{\textbackslash{}n}\StringTok{Row Proportions (Given Actual, \% Predicted {-}{-} Relates to TPR/FPR):}\SpecialCharTok{\textbackslash{}n}\StringTok{"}\NormalTok{)}
\end{Highlighting}
\end{Shaded}

\begin{verbatim}

Row Proportions (Given Actual, % Predicted -- Relates to TPR/FPR):
\end{verbatim}

\begin{Shaded}
\begin{Highlighting}[]
\DocumentationTok{\#\# margin = 1 calculates proportions across rows}
\FunctionTok{print}\NormalTok{(}\FunctionTok{round}\NormalTok{(}\FunctionTok{prop.table}\NormalTok{(conf\_matrix\_sim, }\AttributeTok{margin =} \DecValTok{1}\NormalTok{), }\DecValTok{3}\NormalTok{))}
\end{Highlighting}
\end{Shaded}

\begin{verbatim}
      Predicted
Actual     0     1
     0 0.484 0.516
     1 0.028 0.972
\end{verbatim}

\begin{Shaded}
\begin{Highlighting}[]
\FunctionTok{cat}\NormalTok{(}\StringTok{"}\SpecialCharTok{\textbackslash{}n}\StringTok{Column Proportions (Given Predicted, \% Actual {-}{-} Relates to Precision):}\SpecialCharTok{\textbackslash{}n}\StringTok{"}\NormalTok{)}
\end{Highlighting}
\end{Shaded}

\begin{verbatim}

Column Proportions (Given Predicted, % Actual -- Relates to Precision):
\end{verbatim}

\begin{Shaded}
\begin{Highlighting}[]
\DocumentationTok{\#\# margin = 2 calculates proportions across columns}
\FunctionTok{print}\NormalTok{(}\FunctionTok{round}\NormalTok{(}\FunctionTok{prop.table}\NormalTok{(conf\_matrix\_sim, }\AttributeTok{margin =} \DecValTok{2}\NormalTok{), }\DecValTok{3}\NormalTok{))}
\end{Highlighting}
\end{Shaded}

\begin{verbatim}
      Predicted
Actual     0     1
     0 0.938 0.314
     1 0.062 0.686
\end{verbatim}

\begin{Shaded}
\begin{Highlighting}[]
\DocumentationTok{\#\# {-}{-}{-} MODIFIED LINES }\RegionMarkerTok{END}\DocumentationTok{ {-}{-}{-}}


\DocumentationTok{\#\# Check if matrix has 2x2 dimensions}
\ControlFlowTok{if}\NormalTok{ (}\FunctionTok{all}\NormalTok{(}\FunctionTok{dim}\NormalTok{(conf\_matrix\_sim) }\SpecialCharTok{==} \FunctionTok{c}\NormalTok{(}\DecValTok{2}\NormalTok{, }\DecValTok{2}\NormalTok{))) \{}
\NormalTok{  TN }\OtherTok{\textless{}{-}}\NormalTok{ conf\_matrix\_sim[}\DecValTok{1}\NormalTok{, }\DecValTok{1}\NormalTok{]}
\NormalTok{  FP }\OtherTok{\textless{}{-}}\NormalTok{ conf\_matrix\_sim[}\DecValTok{1}\NormalTok{, }\DecValTok{2}\NormalTok{]}
\NormalTok{  FN }\OtherTok{\textless{}{-}}\NormalTok{ conf\_matrix\_sim[}\DecValTok{2}\NormalTok{, }\DecValTok{1}\NormalTok{]}
\NormalTok{  TP }\OtherTok{\textless{}{-}}\NormalTok{ conf\_matrix\_sim[}\DecValTok{2}\NormalTok{, }\DecValTok{2}\NormalTok{]}

  \DocumentationTok{\#\# Calculate metrics}
\NormalTok{  error\_rate }\OtherTok{\textless{}{-}}\NormalTok{ (FP }\SpecialCharTok{+}\NormalTok{ FN) }\SpecialCharTok{/}\NormalTok{ (TP }\SpecialCharTok{+}\NormalTok{ TN }\SpecialCharTok{+}\NormalTok{ FP }\SpecialCharTok{+}\NormalTok{ FN)}
\NormalTok{  TPR\_Recall }\OtherTok{\textless{}{-}}\NormalTok{ TP }\SpecialCharTok{/}\NormalTok{ (TP }\SpecialCharTok{+}\NormalTok{ FN) }\CommentTok{\# True Positive Rate (Recall / Sensitivity)}
\NormalTok{  FPR }\OtherTok{\textless{}{-}}\NormalTok{ FP }\SpecialCharTok{/}\NormalTok{ (FP }\SpecialCharTok{+}\NormalTok{ TN)      }\CommentTok{\# False Positive Rate (1 {-} Specificity)}
\NormalTok{  Precision }\OtherTok{\textless{}{-}}\NormalTok{ TP }\SpecialCharTok{/}\NormalTok{ (TP }\SpecialCharTok{+}\NormalTok{ FP)  }\CommentTok{\# Positive Predictive Value}

  \FunctionTok{cat}\NormalTok{(}\FunctionTok{paste}\NormalTok{(}\StringTok{"}\SpecialCharTok{\textbackslash{}n}\StringTok{Misclassification Error Rate:"}\NormalTok{, }\FunctionTok{round}\NormalTok{(error\_rate, }\DecValTok{4}\NormalTok{), }\StringTok{"}\SpecialCharTok{\textbackslash{}n}\StringTok{"}\NormalTok{))}
  \FunctionTok{cat}\NormalTok{(}\FunctionTok{paste}\NormalTok{(}\StringTok{"True Positive Rate (Recall):"}\NormalTok{, }\FunctionTok{round}\NormalTok{(TPR\_Recall, }\DecValTok{4}\NormalTok{), }\StringTok{"}\SpecialCharTok{\textbackslash{}n}\StringTok{"}\NormalTok{))}
  \FunctionTok{cat}\NormalTok{(}\FunctionTok{paste}\NormalTok{(}\StringTok{"False Positive Rate:"}\NormalTok{, }\FunctionTok{round}\NormalTok{(FPR, }\DecValTok{4}\NormalTok{), }\StringTok{"}\SpecialCharTok{\textbackslash{}n}\StringTok{"}\NormalTok{))}
  \FunctionTok{cat}\NormalTok{(}\FunctionTok{paste}\NormalTok{(}\StringTok{"Precision:"}\NormalTok{, }\FunctionTok{round}\NormalTok{(Precision, }\DecValTok{4}\NormalTok{), }\StringTok{"}\SpecialCharTok{\textbackslash{}n}\StringTok{"}\NormalTok{))}
\NormalTok{\} }\ControlFlowTok{else}\NormalTok{ \{}
  \FunctionTok{cat}\NormalTok{(}\StringTok{"}\SpecialCharTok{\textbackslash{}n}\StringTok{Cannot calculate full metrics: model predicted only one class.}\SpecialCharTok{\textbackslash{}n}\StringTok{"}\NormalTok{)}
\NormalTok{\}}
\end{Highlighting}
\end{Shaded}

\begin{verbatim}

Misclassification Error Rate: 0.2537 
True Positive Rate (Recall): 0.9722 
False Positive Rate: 0.5161 
Precision: 0.6863 
\end{verbatim}

\textbf{ROC curve and Area Under the ROC (AUC)}

\begin{Shaded}
\begin{Highlighting}[]
\DocumentationTok{\#\# 4b. Area Under the Curve (AUC)}
\NormalTok{roc\_curve\_sim }\OtherTok{\textless{}{-}} \FunctionTok{roc}\NormalTok{(test\_data\_sim}\SpecialCharTok{$}\NormalTok{y, pred\_probs\_sim, }\AttributeTok{quiet =} \ConstantTok{TRUE}\NormalTok{)}

\DocumentationTok{\#\# Plot the ROC curve}
\FunctionTok{plot}\NormalTok{(roc\_curve\_sim, }\AttributeTok{main =} \StringTok{"ROC Curve (Simulated Test Data)"}\NormalTok{, }\AttributeTok{print.auc =} \ConstantTok{TRUE}\NormalTok{)}
\end{Highlighting}
\end{Shaded}

\includegraphics{unit4-lr/logistic_files/figure-pdf/unnamed-chunk-8-1.pdf}

\begin{Shaded}
\begin{Highlighting}[]
\NormalTok{auc\_value\_sim }\OtherTok{\textless{}{-}} \FunctionTok{auc}\NormalTok{(roc\_curve\_sim)}
\FunctionTok{cat}\NormalTok{(}\FunctionTok{paste}\NormalTok{(}\StringTok{"Area Under the Curve (AUC):"}\NormalTok{, }\FunctionTok{round}\NormalTok{(auc\_value\_sim, }\DecValTok{4}\NormalTok{), }\StringTok{"}\SpecialCharTok{\textbackslash{}n\textbackslash{}n}\StringTok{"}\NormalTok{))}
\end{Highlighting}
\end{Shaded}

\begin{verbatim}
Area Under the Curve (AUC): 0.8996 
\end{verbatim}

\textbf{PR curve and Area Under PR Curve (AUPR)}

\begin{Shaded}
\begin{Highlighting}[]
\DocumentationTok{\#\# Load the ROCR library}
\DocumentationTok{\#\# install.packages("ROCR") \# Uncomment to install if needed}
\FunctionTok{library}\NormalTok{(ROCR)}

\DocumentationTok{\#\# {-}{-}{-} 1. Create a \textquotesingle{}prediction\textquotesingle{} object {-}{-}{-}}
\DocumentationTok{\#\# \textquotesingle{}prediction\textquotesingle{} takes all predictions and all true labels}
\NormalTok{pred\_obj }\OtherTok{\textless{}{-}} \FunctionTok{prediction}\NormalTok{(pred\_probs\_sim, test\_data\_sim}\SpecialCharTok{$}\NormalTok{y)}

\DocumentationTok{\#\# {-}{-}{-} 2. Create a \textquotesingle{}performance\textquotesingle{} object for PR {-}{-}{-}}
\DocumentationTok{\#\# "prec" is for precision, "rec" is for recall}
\NormalTok{perf\_pr }\OtherTok{\textless{}{-}} \FunctionTok{performance}\NormalTok{(pred\_obj, }\AttributeTok{measure =} \StringTok{"prec"}\NormalTok{, }\AttributeTok{x.measure =} \StringTok{"rec"}\NormalTok{)}

\DocumentationTok{\#\# {-}{-}{-} 3. Calculate Area Under the PR Curve (AUPR) {-}{-}{-}}
\NormalTok{perf\_auc }\OtherTok{\textless{}{-}} \FunctionTok{performance}\NormalTok{(pred\_obj, }\AttributeTok{measure =} \StringTok{"aucpr"}\NormalTok{) }\CommentTok{\# "aucpr" = Area Under PR Curve}
\NormalTok{aupr\_value }\OtherTok{\textless{}{-}}\NormalTok{ perf\_auc}\SpecialCharTok{@}\NormalTok{y.values[[}\DecValTok{1}\NormalTok{]]}
\FunctionTok{cat}\NormalTok{(}\FunctionTok{paste}\NormalTok{(}\StringTok{"Area Under PR Curve (AUPR):"}\NormalTok{, }\FunctionTok{round}\NormalTok{(aupr\_value, }\DecValTok{4}\NormalTok{), }\StringTok{"}\SpecialCharTok{\textbackslash{}n}\StringTok{"}\NormalTok{))}
\end{Highlighting}
\end{Shaded}

\begin{verbatim}
Area Under PR Curve (AUPR): 0.9157 
\end{verbatim}

\begin{Shaded}
\begin{Highlighting}[]
\DocumentationTok{\#\# {-}{-}{-} 4. Plot the performance object {-}{-}{-}}
\FunctionTok{plot}\NormalTok{(perf\_pr, }
     \AttributeTok{main =} \StringTok{"Precision{-}Recall Curve (Simulated Test Data)"}\NormalTok{, }
     \AttributeTok{xlim =} \FunctionTok{c}\NormalTok{(}\DecValTok{0}\NormalTok{, }\DecValTok{1}\NormalTok{), }
     \AttributeTok{ylim =} \FunctionTok{c}\NormalTok{(}\DecValTok{0}\NormalTok{, }\DecValTok{1}\NormalTok{),}
     \AttributeTok{col =} \StringTok{"black"}\NormalTok{)}

\DocumentationTok{\#\# {-}{-}{-} 5. Calculate and add the \textquotesingle{}no{-}skill\textquotesingle{} baseline {-}{-}{-}}
\NormalTok{baseline\_precision\_sim }\OtherTok{\textless{}{-}} \FunctionTok{sum}\NormalTok{(test\_data\_sim}\SpecialCharTok{$}\NormalTok{y }\SpecialCharTok{==} \DecValTok{1}\NormalTok{) }\SpecialCharTok{/} \FunctionTok{length}\NormalTok{(test\_data\_sim}\SpecialCharTok{$}\NormalTok{y)}
\FunctionTok{abline}\NormalTok{(}\AttributeTok{h =}\NormalTok{ baseline\_precision\_sim, }\AttributeTok{col =} \StringTok{"blue"}\NormalTok{, }\AttributeTok{lty =} \DecValTok{2}\NormalTok{)}

\DocumentationTok{\#\# {-}{-}{-} 6. Add a legend with AUPR {-}{-}{-}}
\FunctionTok{legend}\NormalTok{(}\StringTok{"bottomleft"}\NormalTok{, }
       \AttributeTok{legend =} \FunctionTok{c}\NormalTok{(}
           \FunctionTok{paste}\NormalTok{(}\StringTok{"Model (AUPR ="}\NormalTok{, }\FunctionTok{round}\NormalTok{(aupr\_value, }\DecValTok{4}\NormalTok{), }\StringTok{")"}\NormalTok{),  }\CommentTok{\# \textless{}{-}{-} MODIFIED LINE}
           \FunctionTok{paste}\NormalTok{(}\StringTok{"Baseline ("}\NormalTok{, }\FunctionTok{round}\NormalTok{(baseline\_precision\_sim, }\DecValTok{3}\NormalTok{), }\StringTok{")"}\NormalTok{)}
\NormalTok{       ), }
       \AttributeTok{col =} \FunctionTok{c}\NormalTok{(}\StringTok{"black"}\NormalTok{, }\StringTok{"blue"}\NormalTok{), }
       \AttributeTok{lty =} \FunctionTok{c}\NormalTok{(}\DecValTok{1}\NormalTok{, }\DecValTok{2}\NormalTok{), }
       \AttributeTok{bty =} \StringTok{"n"}\NormalTok{) }\CommentTok{\# bty="n" removes the box}
\end{Highlighting}
\end{Shaded}

\includegraphics{unit4-lr/logistic_files/figure-pdf/pr-curve-rocr-1.pdf}

\subsection{Application to the CHD
Dataset}\label{application-to-the-chd-dataset}

\begin{Shaded}
\begin{Highlighting}[]
\DocumentationTok{\#\# Load the pROC library for AUC calculation}
\DocumentationTok{\#\# install.packages("pROC") \# Uncomment to install if needed}
\FunctionTok{library}\NormalTok{(pROC)}

\DocumentationTok{\#\# {-}{-}{-} 1. Split the data {-}{-}{-}}
\FunctionTok{set.seed}\NormalTok{(}\DecValTok{123}\NormalTok{) }\CommentTok{\# for reproducibility}
\NormalTok{n }\OtherTok{\textless{}{-}} \FunctionTok{nrow}\NormalTok{(CHD.data)}
\NormalTok{train\_size }\OtherTok{\textless{}{-}} \FunctionTok{floor}\NormalTok{(}\DecValTok{2}\SpecialCharTok{/}\DecValTok{3} \SpecialCharTok{*}\NormalTok{ n)}
\NormalTok{train\_indices }\OtherTok{\textless{}{-}} \FunctionTok{sample}\NormalTok{(}\DecValTok{1}\SpecialCharTok{:}\NormalTok{n, }\AttributeTok{size =}\NormalTok{ train\_size)}
\NormalTok{train\_data }\OtherTok{\textless{}{-}}\NormalTok{ CHD.data[train\_indices, ]}
\NormalTok{test\_data  }\OtherTok{\textless{}{-}}\NormalTok{ CHD.data[}\SpecialCharTok{{-}}\NormalTok{train\_indices, ]}

\DocumentationTok{\#\# {-}{-}{-} 2. Refit the model on the training data {-}{-}{-}}
\NormalTok{fit\_train }\OtherTok{\textless{}{-}} \FunctionTok{glm}\NormalTok{(}
\NormalTok{  chd }\SpecialCharTok{\textasciitilde{}}\NormalTok{ smk }\SpecialCharTok{+}\NormalTok{ cat }\SpecialCharTok{+}\NormalTok{ sbp }\SpecialCharTok{+}\NormalTok{ age }\SpecialCharTok{+}\NormalTok{ chl }\SpecialCharTok{+}\NormalTok{ ecg }\SpecialCharTok{+}\NormalTok{ hpt,}
  \AttributeTok{data =}\NormalTok{ train\_data,}
  \AttributeTok{family =} \FunctionTok{binomial}\NormalTok{(}\AttributeTok{link =} \StringTok{"logit"}\NormalTok{)}
\NormalTok{)}

\DocumentationTok{\#\# {-}{-}{-} 3. Make predictions on the test data {-}{-}{-}}
\NormalTok{pred\_probs }\OtherTok{\textless{}{-}} \FunctionTok{predict}\NormalTok{(fit\_train, }\AttributeTok{newdata =}\NormalTok{ test\_data, }\AttributeTok{type =} \StringTok{"response"}\NormalTok{)}
\end{Highlighting}
\end{Shaded}

\textbf{Plot Predictive Probabilities}

\begin{Shaded}
\begin{Highlighting}[]
\DocumentationTok{\#\# {-}{-}{-} 5. Plot sorted predicted probabilities {-}{-}{-}}

\DocumentationTok{\#\# Create a data frame for plotting}
\NormalTok{plot\_data }\OtherTok{\textless{}{-}} \FunctionTok{data.frame}\NormalTok{(}
  \AttributeTok{Prob =}\NormalTok{ pred\_probs,}
  \AttributeTok{Actual =} \FunctionTok{as.factor}\NormalTok{(test\_data}\SpecialCharTok{$}\NormalTok{chd)}
\NormalTok{)}

\DocumentationTok{\#\# Sort by predicted probability}
\NormalTok{plot\_data }\OtherTok{\textless{}{-}}\NormalTok{ plot\_data[}\FunctionTok{order}\NormalTok{(plot\_data}\SpecialCharTok{$}\NormalTok{Prob), ]}
\NormalTok{plot\_data}\SpecialCharTok{$}\NormalTok{Rank }\OtherTok{\textless{}{-}} \DecValTok{1}\SpecialCharTok{:}\FunctionTok{nrow}\NormalTok{(plot\_data)}

\DocumentationTok{\#\# Create the plot}
\DocumentationTok{\#\# We use \textquotesingle{}pch\textquotesingle{} (plot character) to set different symbols}
\DocumentationTok{\#\# \textquotesingle{}pch = 1\textquotesingle{} is \textquotesingle{}o\textquotesingle{} (default)}
\DocumentationTok{\#\# \textquotesingle{}pch = 4\textquotesingle{} is \textquotesingle{}x\textquotesingle{}}
\FunctionTok{plot}\NormalTok{(}
\NormalTok{  plot\_data}\SpecialCharTok{$}\NormalTok{Rank,}
\NormalTok{  plot\_data}\SpecialCharTok{$}\NormalTok{Prob,}
  \AttributeTok{pch =} \FunctionTok{ifelse}\NormalTok{(plot\_data}\SpecialCharTok{$}\NormalTok{Actual }\SpecialCharTok{==} \DecValTok{0}\NormalTok{, }\DecValTok{1}\NormalTok{, }\DecValTok{4}\NormalTok{),}
  \AttributeTok{col =} \FunctionTok{ifelse}\NormalTok{(plot\_data}\SpecialCharTok{$}\NormalTok{Actual }\SpecialCharTok{==} \DecValTok{0}\NormalTok{, }\StringTok{"blue"}\NormalTok{, }\StringTok{"red"}\NormalTok{),}
  \AttributeTok{xlab =} \StringTok{"Index (Sorted by Predicted Probability)"}\NormalTok{,}
  \AttributeTok{ylab =} \StringTok{"Log{-}odds of Predicted Probability"}\NormalTok{,}
  \AttributeTok{main =} \StringTok{"Predicted Probabilities vs. Actual Class"}\NormalTok{,}
  \AttributeTok{ylim =} \FunctionTok{c}\NormalTok{(}\DecValTok{0}\NormalTok{,}\DecValTok{1}\NormalTok{)}
\NormalTok{)}
\FunctionTok{abline}\NormalTok{(}\AttributeTok{h=}\FloatTok{0.5}\NormalTok{)}
\FunctionTok{abline}\NormalTok{(}\AttributeTok{h=}\FloatTok{0.1}\NormalTok{, }\AttributeTok{col=}\StringTok{"grey"}\NormalTok{)}

\DocumentationTok{\#\# Add a legend}
\FunctionTok{legend}\NormalTok{(}
  \StringTok{"topleft"}\NormalTok{,}
  \AttributeTok{legend =} \FunctionTok{c}\NormalTok{(}\StringTok{"Actual 0 (o)"}\NormalTok{, }\StringTok{"Actual 1 (x)"}\NormalTok{),}
  \AttributeTok{pch =} \FunctionTok{c}\NormalTok{(}\DecValTok{1}\NormalTok{, }\DecValTok{4}\NormalTok{),}
  \AttributeTok{col =} \FunctionTok{c}\NormalTok{(}\StringTok{"blue"}\NormalTok{, }\StringTok{"red"}\NormalTok{)}
\NormalTok{)}
\end{Highlighting}
\end{Shaded}

\includegraphics{unit4-lr/logistic_files/figure-pdf/plot-sorted-probabilities-1.pdf}

\textbf{ROC curve and Area Under the ROC (AUC)}

\begin{Shaded}
\begin{Highlighting}[]
\DocumentationTok{\#\# 4b. Area Under the Curve (AUC)}
\NormalTok{roc\_curve }\OtherTok{\textless{}{-}} \FunctionTok{roc}\NormalTok{(test\_data}\SpecialCharTok{$}\NormalTok{chd, pred\_probs, }\AttributeTok{quiet =} \ConstantTok{TRUE}\NormalTok{)}

\DocumentationTok{\#\# Plot the ROC curve}
\FunctionTok{plot}\NormalTok{(roc\_curve, }\AttributeTok{main =} \StringTok{"ROC Curve (Test Data)"}\NormalTok{, }\AttributeTok{print.auc =} \ConstantTok{TRUE}\NormalTok{)}
\end{Highlighting}
\end{Shaded}

\includegraphics{unit4-lr/logistic_files/figure-pdf/unnamed-chunk-9-1.pdf}

\begin{Shaded}
\begin{Highlighting}[]
\NormalTok{auc\_value }\OtherTok{\textless{}{-}} \FunctionTok{auc}\NormalTok{(roc\_curve)}
\FunctionTok{cat}\NormalTok{(}\FunctionTok{paste}\NormalTok{(}\StringTok{"Area Under the Curve (AUC):"}\NormalTok{, }\FunctionTok{round}\NormalTok{(auc\_value, }\DecValTok{4}\NormalTok{), }\StringTok{"}\SpecialCharTok{\textbackslash{}n\textbackslash{}n}\StringTok{"}\NormalTok{))}
\end{Highlighting}
\end{Shaded}

\begin{verbatim}
Area Under the Curve (AUC): 0.6872 
\end{verbatim}

\textbf{PR curve and Area Under PR Curve (AUPR)}

\begin{Shaded}
\begin{Highlighting}[]
\DocumentationTok{\#\# Load the ROCR library}
\DocumentationTok{\#\# install.packages("ROCR") \# Uncomment to install if needed}
\FunctionTok{library}\NormalTok{(ROCR)}

\DocumentationTok{\#\# {-}{-}{-} 1. Create a \textquotesingle{}prediction\textquotesingle{} object {-}{-}{-}}
\DocumentationTok{\#\# \textquotesingle{}prediction\textquotesingle{} takes all predictions and all true labels}
\DocumentationTok{\#\# We use \textquotesingle{}pred\_probs\textquotesingle{} and \textquotesingle{}test\_data$chd\textquotesingle{} from the CHD data split}
\NormalTok{pred\_obj }\OtherTok{\textless{}{-}} \FunctionTok{prediction}\NormalTok{(pred\_probs, test\_data}\SpecialCharTok{$}\NormalTok{chd)}

\DocumentationTok{\#\# {-}{-}{-} 2. Create a \textquotesingle{}performance\textquotesingle{} object for PR {-}{-}{-}}
\DocumentationTok{\#\# "prec" is for precision, "rec" is for recall}
\NormalTok{perf\_pr }\OtherTok{\textless{}{-}} \FunctionTok{performance}\NormalTok{(pred\_obj, }\AttributeTok{measure =} \StringTok{"prec"}\NormalTok{, }\AttributeTok{x.measure =} \StringTok{"rec"}\NormalTok{)}

\DocumentationTok{\#\# {-}{-}{-} 3. Calculate Area Under the PR Curve (AUPR) {-}{-}{-}}
\NormalTok{perf\_auc }\OtherTok{\textless{}{-}} \FunctionTok{performance}\NormalTok{(pred\_obj, }\AttributeTok{measure =} \StringTok{"aucpr"}\NormalTok{) }\CommentTok{\# "aucpr" = Area Under PR Curve}
\NormalTok{aupr\_value }\OtherTok{\textless{}{-}}\NormalTok{ perf\_auc}\SpecialCharTok{@}\NormalTok{y.values[[}\DecValTok{1}\NormalTok{]]}
\FunctionTok{cat}\NormalTok{(}\FunctionTok{paste}\NormalTok{(}\StringTok{"Area Under PR Curve (AUPR):"}\NormalTok{, }\FunctionTok{round}\NormalTok{(aupr\_value, }\DecValTok{4}\NormalTok{), }\StringTok{"}\SpecialCharTok{\textbackslash{}n}\StringTok{"}\NormalTok{))}
\end{Highlighting}
\end{Shaded}

\begin{verbatim}
Area Under PR Curve (AUPR): 0.2826 
\end{verbatim}

\begin{Shaded}
\begin{Highlighting}[]
\DocumentationTok{\#\# {-}{-}{-} 4. Plot the performance object {-}{-}{-}}
\FunctionTok{plot}\NormalTok{(perf\_pr, }
     \AttributeTok{main =} \StringTok{"Precision{-}Recall Curve (Test Data)"}\NormalTok{, }
     \AttributeTok{xlim =} \FunctionTok{c}\NormalTok{(}\DecValTok{0}\NormalTok{, }\DecValTok{1}\NormalTok{), }
     \AttributeTok{ylim =} \FunctionTok{c}\NormalTok{(}\DecValTok{0}\NormalTok{, }\DecValTok{1}\NormalTok{),}
     \AttributeTok{col =} \StringTok{"black"}\NormalTok{)}

\DocumentationTok{\#\# {-}{-}{-} 5. Calculate and add the \textquotesingle{}no{-}skill\textquotesingle{} baseline {-}{-}{-}}
\NormalTok{baseline\_precision }\OtherTok{\textless{}{-}} \FunctionTok{sum}\NormalTok{(test\_data}\SpecialCharTok{$}\NormalTok{chd }\SpecialCharTok{==} \DecValTok{1}\NormalTok{) }\SpecialCharTok{/} \FunctionTok{length}\NormalTok{(test\_data}\SpecialCharTok{$}\NormalTok{chd)}
\FunctionTok{abline}\NormalTok{(}\AttributeTok{h =}\NormalTok{ baseline\_precision, }\AttributeTok{col =} \StringTok{"blue"}\NormalTok{, }\AttributeTok{lty =} \DecValTok{2}\NormalTok{)}

\DocumentationTok{\#\# {-}{-}{-} 6. Add a legend with AUPR {-}{-}{-}}
\FunctionTok{legend}\NormalTok{(}\StringTok{"bottomleft"}\NormalTok{, }
       \AttributeTok{legend =} \FunctionTok{c}\NormalTok{(}
           \FunctionTok{paste}\NormalTok{(}\StringTok{"Model (AUPR ="}\NormalTok{, }\FunctionTok{round}\NormalTok{(aupr\_value, }\DecValTok{4}\NormalTok{), }\StringTok{")"}\NormalTok{),  }\CommentTok{\# \textless{}{-}{-} MODIFIED LINE}
           \FunctionTok{paste}\NormalTok{(}\StringTok{"Baseline ("}\NormalTok{, }\FunctionTok{round}\NormalTok{(baseline\_precision, }\DecValTok{3}\NormalTok{), }\StringTok{")"}\NormalTok{)}
\NormalTok{       ), }
       \AttributeTok{col =} \FunctionTok{c}\NormalTok{(}\StringTok{"black"}\NormalTok{, }\StringTok{"blue"}\NormalTok{), }
       \AttributeTok{lty =} \FunctionTok{c}\NormalTok{(}\DecValTok{1}\NormalTok{, }\DecValTok{2}\NormalTok{), }
       \AttributeTok{bty =} \StringTok{"n"}\NormalTok{) }\CommentTok{\# bty="n" removes the box}
\end{Highlighting}
\end{Shaded}

\includegraphics{unit4-lr/logistic_files/figure-pdf/pr-curve-rocr-chd-1.pdf}

\bookmarksetup{startatroot}

\chapter{Randomized Complete Block
Design}\label{randomized-complete-block-design}

\section{Completely Randomized
Design}\label{completely-randomized-design}

\subsection{Design on R}\label{design-on-r}

\begin{Shaded}
\begin{Highlighting}[]
\NormalTok{treatments }\OtherTok{\textless{}{-}}\NormalTok{ LETTERS[}\DecValTok{1}\SpecialCharTok{:}\DecValTok{4}\NormalTok{]}
\NormalTok{rep.treatments }\OtherTok{\textless{}{-}} \FunctionTok{rep}\NormalTok{(treatments, }\AttributeTok{each =} \DecValTok{5}\NormalTok{)}
\NormalTok{rep.treatments}
\end{Highlighting}
\end{Shaded}

\begin{verbatim}
 [1] "A" "A" "A" "A" "A" "B" "B" "B" "B" "B" "C" "C" "C" "C" "C" "D" "D" "D" "D"
[20] "D"
\end{verbatim}

\begin{Shaded}
\begin{Highlighting}[]
\FunctionTok{sample}\NormalTok{ (rep.treatments)}
\end{Highlighting}
\end{Shaded}

\begin{verbatim}
 [1] "C" "C" "D" "A" "B" "B" "A" "D" "D" "A" "C" "C" "D" "D" "B" "B" "A" "A" "C"
[20] "B"
\end{verbatim}

\begin{Shaded}
\begin{Highlighting}[]
\FunctionTok{data.frame}\NormalTok{(}\AttributeTok{run.ID =} \DecValTok{1}\SpecialCharTok{:}\DecValTok{20}\NormalTok{, }\AttributeTok{treatment =} \FunctionTok{sample}\NormalTok{ (rep.treatments))}
\end{Highlighting}
\end{Shaded}

\begin{verbatim}
   run.ID treatment
1       1         D
2       2         A
3       3         A
4       4         B
5       5         D
6       6         C
7       7         A
8       8         C
9       9         C
10     10         B
11     11         B
12     12         C
13     13         C
14     14         B
15     15         A
16     16         D
17     17         A
18     18         D
19     19         D
20     20         B
\end{verbatim}

\subsection{Plasma Etching Experiment}\label{plasma-etching-experiment}

This section analyzes data from a \textbf{Completely Randomized Design
(CRD)}. In a CRD, experimental units (in this case, the silicon wafers
being etched) are assigned to treatments (the RF Power levels)
completely at random. The primary goal is to determine if changing the
RF Power level has a statistically significant effect on the mean etch
rate.

\subsubsection{Data and Visualization}\label{data-and-visualization}

We begin by loading the data into a single, tidy \texttt{data.frame}.
The response variable, \texttt{rate}, contains all the etch rate
observations. The predictor variable, \texttt{power}, is a
\textbf{factor}, which is R's way of representing a categorical
variable. This tells R to treat the different power levels as distinct
groups.

\begin{Shaded}
\begin{Highlighting}[]
\DocumentationTok{\#\# Define the data vectors}
\NormalTok{rate }\OtherTok{\textless{}{-}} \FunctionTok{c}\NormalTok{(}\DecValTok{575}\NormalTok{, }\DecValTok{542}\NormalTok{, }\DecValTok{530}\NormalTok{, }\DecValTok{539}\NormalTok{, }\DecValTok{570}\NormalTok{, }\DecValTok{565}\NormalTok{, }\DecValTok{593}\NormalTok{, }\DecValTok{590}\NormalTok{, }\DecValTok{579}\NormalTok{, }\DecValTok{610}\NormalTok{,}
          \DecValTok{600}\NormalTok{, }\DecValTok{651}\NormalTok{, }\DecValTok{610}\NormalTok{, }\DecValTok{637}\NormalTok{, }\DecValTok{629}\NormalTok{, }\DecValTok{725}\NormalTok{, }\DecValTok{700}\NormalTok{, }\DecValTok{715}\NormalTok{, }\DecValTok{685}\NormalTok{, }\DecValTok{710}\NormalTok{)}
\NormalTok{power\_levels }\OtherTok{\textless{}{-}} \FunctionTok{c}\NormalTok{(}\DecValTok{160}\NormalTok{, }\DecValTok{180}\NormalTok{, }\DecValTok{200}\NormalTok{, }\DecValTok{220}\NormalTok{)}

\DocumentationTok{\#\# Create the data frame}
\NormalTok{etching\_df }\OtherTok{\textless{}{-}} \FunctionTok{data.frame}\NormalTok{(}
  \AttributeTok{rate =}\NormalTok{ rate,}
  \AttributeTok{power =} \FunctionTok{factor}\NormalTok{(}\FunctionTok{rep}\NormalTok{(power\_levels, }\AttributeTok{each =} \DecValTok{5}\NormalTok{))}
\NormalTok{)}

\DocumentationTok{\#\# Display the first few rows}
\NormalTok{etching\_df}
\end{Highlighting}
\end{Shaded}

\begin{verbatim}
   rate power
1   575   160
2   542   160
3   530   160
4   539   160
5   570   160
6   565   180
7   593   180
8   590   180
9   579   180
10  610   180
11  600   200
12  651   200
13  610   200
14  637   200
15  629   200
16  725   220
17  700   220
18  715   220
19  685   220
20  710   220
\end{verbatim}

\textbf{Grouped Boxplots}

\begin{Shaded}
\begin{Highlighting}[]
\FunctionTok{boxplot}\NormalTok{(rate}\SpecialCharTok{\textasciitilde{}}\NormalTok{power, }\AttributeTok{data=}\NormalTok{etching\_df)}
\end{Highlighting}
\end{Shaded}

\includegraphics{unit5-factor/crbd_files/figure-pdf/unnamed-chunk-2-1.pdf}

\textbf{Using ggplot to visualize grouped data}

\begin{Shaded}
\begin{Highlighting}[]
\FunctionTok{library}\NormalTok{(ggplot2)}
\FunctionTok{library}\NormalTok{(dplyr) }\CommentTok{\# Using dplyr for easier data manipulation}

\DocumentationTok{\#\# Calculate group means and their start/end indices}
\NormalTok{mean\_rates }\OtherTok{\textless{}{-}}\NormalTok{ etching\_df }\SpecialCharTok{\%\textgreater{}\%}
  \FunctionTok{mutate}\NormalTok{(}\AttributeTok{obs\_index =} \FunctionTok{row\_number}\NormalTok{()) }\SpecialCharTok{\%\textgreater{}\%}
  \FunctionTok{group\_by}\NormalTok{(power) }\SpecialCharTok{\%\textgreater{}\%}
  \FunctionTok{summarise}\NormalTok{(}
    \AttributeTok{mean\_rate =} \FunctionTok{mean}\NormalTok{(rate),}
    \AttributeTok{x\_start =} \FunctionTok{min}\NormalTok{(obs\_index) }\SpecialCharTok{{-}} \FloatTok{0.5}\NormalTok{,}
    \AttributeTok{x\_end =} \FunctionTok{max}\NormalTok{(obs\_index) }\SpecialCharTok{+} \FloatTok{0.5}
\NormalTok{  )}

\FunctionTok{ggplot}\NormalTok{(etching\_df, }\FunctionTok{aes}\NormalTok{(}\AttributeTok{x =} \DecValTok{1}\SpecialCharTok{:}\FunctionTok{nrow}\NormalTok{(etching\_df), }\AttributeTok{y =}\NormalTok{ rate, }\AttributeTok{color =}\NormalTok{ power)) }\SpecialCharTok{+}
  \FunctionTok{geom\_hline}\NormalTok{(}
    \AttributeTok{yintercept =} \FunctionTok{mean}\NormalTok{(etching\_df}\SpecialCharTok{$}\NormalTok{rate), }
    \AttributeTok{linetype =} \StringTok{"solid"}\NormalTok{, }
    \AttributeTok{color =} \StringTok{"black"}\NormalTok{, }
    \AttributeTok{size =} \DecValTok{1}
\NormalTok{  ) }\SpecialCharTok{+}
  \CommentTok{\# {-}{-}{-}{-}{-}{-}{-}{-}{-}{-}{-}{-}{-}{-}{-}{-}{-}{-}{-}{-}{-}{-}{-}{-}{-}{-}{-}{-}{-}{-}{-}{-}{-}{-}{-}{-}{-}{-}{-}{-}}
  \FunctionTok{geom\_point}\NormalTok{(}\AttributeTok{size =} \DecValTok{3}\NormalTok{, }\AttributeTok{alpha =} \FloatTok{0.7}\NormalTok{) }\SpecialCharTok{+} \CommentTok{\# Plot individual data points}
  \FunctionTok{geom\_segment}\NormalTok{(}
    \AttributeTok{data =}\NormalTok{ mean\_rates, }
    \FunctionTok{aes}\NormalTok{(}\AttributeTok{x =}\NormalTok{ x\_start, }\AttributeTok{xend =}\NormalTok{ x\_end, }\AttributeTok{y =}\NormalTok{ mean\_rate, }\AttributeTok{yend =}\NormalTok{ mean\_rate),}
    \AttributeTok{linetype =} \StringTok{"dashed"}\NormalTok{, }
    \AttributeTok{size =} \FloatTok{1.2}
\NormalTok{  ) }\SpecialCharTok{+} \CommentTok{\# Add line segments for group means}
  \FunctionTok{labs}\NormalTok{(}
    \AttributeTok{title =} \StringTok{"Etch Rate Observations by RF Power Level"}\NormalTok{,}
    \AttributeTok{x =} \StringTok{"Observation Index"}\NormalTok{,}
    \AttributeTok{y =} \StringTok{"Etch Rate"}\NormalTok{,}
    \AttributeTok{color =} \StringTok{"RF Power (W)"}
\NormalTok{  ) }\SpecialCharTok{+}
  \FunctionTok{scale\_color\_brewer}\NormalTok{(}\AttributeTok{palette =} \StringTok{"Set1"}\NormalTok{) }\SpecialCharTok{+}
  \FunctionTok{theme\_minimal}\NormalTok{() }\SpecialCharTok{+}
  \FunctionTok{theme}\NormalTok{(}\AttributeTok{plot.title =} \FunctionTok{element\_text}\NormalTok{(}\AttributeTok{hjust =} \FloatTok{0.5}\NormalTok{))}
\end{Highlighting}
\end{Shaded}

\begin{figure}[H]

{\centering \includegraphics{unit5-factor/crbd_files/figure-pdf/index-plot-rate-power-segments-1.pdf}

}

\caption{Index Plot of Etch Rate with Group-Specific Mean Lines}

\end{figure}%

\subsubsection{Cell Mean Models
(zero-intercept)}\label{cell-mean-models-zero-intercept}

\begin{Shaded}
\begin{Highlighting}[]
\NormalTok{fit\_nointercpt }\OtherTok{\textless{}{-}} \FunctionTok{lm}\NormalTok{(rate }\SpecialCharTok{\textasciitilde{}} \DecValTok{0}\SpecialCharTok{+}\NormalTok{power, }\AttributeTok{data =}\NormalTok{ etching\_df)}
\FunctionTok{model.matrix}\NormalTok{(fit\_nointercpt)}
\end{Highlighting}
\end{Shaded}

\begin{verbatim}
   power160 power180 power200 power220
1         1        0        0        0
2         1        0        0        0
3         1        0        0        0
4         1        0        0        0
5         1        0        0        0
6         0        1        0        0
7         0        1        0        0
8         0        1        0        0
9         0        1        0        0
10        0        1        0        0
11        0        0        1        0
12        0        0        1        0
13        0        0        1        0
14        0        0        1        0
15        0        0        1        0
16        0        0        0        1
17        0        0        0        1
18        0        0        0        1
19        0        0        0        1
20        0        0        0        1
attr(,"assign")
[1] 1 1 1 1
attr(,"contrasts")
attr(,"contrasts")$power
[1] "contr.treatment"
\end{verbatim}

\begin{Shaded}
\begin{Highlighting}[]
\FunctionTok{summary}\NormalTok{(fit\_nointercpt)}
\end{Highlighting}
\end{Shaded}

\begin{verbatim}

Call:
lm(formula = rate ~ 0 + power, data = etching_df)

Residuals:
   Min     1Q Median     3Q    Max 
 -25.4  -13.0    2.8   13.2   25.6 

Coefficients:
         Estimate Std. Error t value Pr(>|t|)    
power160  551.200      8.169   67.47   <2e-16 ***
power180  587.400      8.169   71.90   <2e-16 ***
power200  625.400      8.169   76.55   <2e-16 ***
power220  707.000      8.169   86.54   <2e-16 ***
---
Signif. codes:  0 '***' 0.001 '**' 0.01 '*' 0.05 '.' 0.1 ' ' 1

Residual standard error: 18.27 on 16 degrees of freedom
Multiple R-squared:  0.9993,    Adjusted R-squared:  0.9991 
F-statistic:  5768 on 4 and 16 DF,  p-value: < 2.2e-16
\end{verbatim}

\begin{Shaded}
\begin{Highlighting}[]
\FunctionTok{confint}\NormalTok{(fit\_nointercpt)}
\end{Highlighting}
\end{Shaded}

\begin{verbatim}
            2.5 %   97.5 %
power160 533.8815 568.5185
power180 570.0815 604.7185
power200 608.0815 642.7185
power220 689.6815 724.3185
\end{verbatim}

\begin{Shaded}
\begin{Highlighting}[]
\FunctionTok{anova}\NormalTok{(fit\_nointercpt)}
\end{Highlighting}
\end{Shaded}

\begin{verbatim}
Analysis of Variance Table

Response: rate
          Df  Sum Sq Mean Sq F value    Pr(>F)    
power      4 7699172 1924793    5768 < 2.2e-16 ***
Residuals 16    5339     334                      
---
Signif. codes:  0 '***' 0.001 '**' 0.01 '*' 0.05 '.' 0.1 ' ' 1
\end{verbatim}

This ANOVA result is typically unwanted. If we want to see the causal
effect of power, we need to compare to intercept-only model:

\begin{Shaded}
\begin{Highlighting}[]
\NormalTok{fit\_intercept }\OtherTok{\textless{}{-}} \FunctionTok{lm}\NormalTok{(rate }\SpecialCharTok{\textasciitilde{}} \DecValTok{1}\NormalTok{, }\AttributeTok{data =}\NormalTok{ etching\_df)}
\FunctionTok{anova}\NormalTok{(fit\_intercept, fit\_nointercpt)}
\end{Highlighting}
\end{Shaded}

\begin{verbatim}
Analysis of Variance Table

Model 1: rate ~ 1
Model 2: rate ~ 0 + power
  Res.Df   RSS Df Sum of Sq      F    Pr(>F)    
1     19 72210                                  
2     16  5339  3     66871 66.797 2.883e-09 ***
---
Signif. codes:  0 '***' 0.001 '**' 0.01 '*' 0.05 '.' 0.1 ' ' 1
\end{verbatim}

\subsubsection{Centralized Effect Model (Sum-to-Zero
Constraint)}\label{centralized-effect-model-sum-to-zero-constraint}

We fit a linear model using the \texttt{lm()} function to perform an
\textbf{Analysis of Variance (ANOVA)}. The model is specified as
\texttt{rate\ \textasciitilde{}\ power}, and we now include the
\texttt{data\ =\ etching\_df} argument.

To get interpretable estimates for the treatment effects (\(\tau_i\)),
we use a \textbf{sum-to-zero constraint} (\texttt{contr.sum}), which
forces the sum of the treatment effects to be zero
(\(\sum \tau_i = 0\)).

\begin{Shaded}
\begin{Highlighting}[]
\NormalTok{fit }\OtherTok{\textless{}{-}} \FunctionTok{lm}\NormalTok{(rate }\SpecialCharTok{\textasciitilde{}}\NormalTok{ power, }\AttributeTok{data =}\NormalTok{ etching\_df, }\AttributeTok{contrasts =} \FunctionTok{list}\NormalTok{(}\AttributeTok{power =}\NormalTok{ contr.sum))}
\end{Highlighting}
\end{Shaded}

Model Matrix:

\begin{Shaded}
\begin{Highlighting}[]
\FunctionTok{model.matrix}\NormalTok{(fit)}
\end{Highlighting}
\end{Shaded}

\begin{verbatim}
   (Intercept) power1 power2 power3
1            1      1      0      0
2            1      1      0      0
3            1      1      0      0
4            1      1      0      0
5            1      1      0      0
6            1      0      1      0
7            1      0      1      0
8            1      0      1      0
9            1      0      1      0
10           1      0      1      0
11           1      0      0      1
12           1      0      0      1
13           1      0      0      1
14           1      0      0      1
15           1      0      0      1
16           1     -1     -1     -1
17           1     -1     -1     -1
18           1     -1     -1     -1
19           1     -1     -1     -1
20           1     -1     -1     -1
attr(,"assign")
[1] 0 1 1 1
attr(,"contrasts")
attr(,"contrasts")$power
    [,1] [,2] [,3]
160    1    0    0
180    0    1    0
200    0    0    1
220   -1   -1   -1
\end{verbatim}

Summary of lm fitting results:

\begin{Shaded}
\begin{Highlighting}[]
\NormalTok{summary.fit }\OtherTok{\textless{}{-}} \FunctionTok{summary}\NormalTok{(fit)}
\NormalTok{summary.fit}
\end{Highlighting}
\end{Shaded}

\begin{verbatim}

Call:
lm(formula = rate ~ power, data = etching_df, contrasts = list(power = contr.sum))

Residuals:
   Min     1Q Median     3Q    Max 
 -25.4  -13.0    2.8   13.2   25.6 

Coefficients:
            Estimate Std. Error t value Pr(>|t|)    
(Intercept)  617.750      4.085 151.234  < 2e-16 ***
power1       -66.550      7.075  -9.406 6.39e-08 ***
power2       -30.350      7.075  -4.290 0.000563 ***
power3         7.650      7.075   1.081 0.295602    
---
Signif. codes:  0 '***' 0.001 '**' 0.01 '*' 0.05 '.' 0.1 ' ' 1

Residual standard error: 18.27 on 16 degrees of freedom
Multiple R-squared:  0.9261,    Adjusted R-squared:  0.9122 
F-statistic:  66.8 on 3 and 16 DF,  p-value: 2.883e-09
\end{verbatim}

Confidence Interval for Centralized Effects:

The output of the model provides estimates for the overall mean
(\(\hat{\mu}\)) and the treatment effects for the first k-1 levels
(\(\hat{\tau}_1, \hat{\tau}_2, \hat{\tau}_3\)).

\begin{itemize}
\tightlist
\item
  \(\hat{\mu}\) (the Intercept) is the estimate of the grand mean etch
  rate across all power levels.
\item
  \(\hat{\tau}_i\) is the estimated effect of the i-th power level,
  representing how much that level's mean deviates from the grand mean.
\end{itemize}

Using the sum-to-zero constraint, we can manually calculate the effect
for the final level, \(\hat{\tau}_4\).

\begin{Shaded}
\begin{Highlighting}[]
\CommentTok{\# Define a named vector of all linear combinations}
\CommentTok{\#}
\CommentTok{\# }\AlertTok{NOTE}\CommentTok{: Replace \textquotesingle{}power1\textquotesingle{}, \textquotesingle{}power2\textquotesingle{}, \textquotesingle{}power3\textquotesingle{} with the}
\CommentTok{\# actual names from your coef(fit) output.}
\CommentTok{\#}
\CommentTok{\# The names on the left (e.g., "Level\_1\_Effect") are}
\CommentTok{\# for you; they will label the output rows.}
\FunctionTok{library}\NormalTok{(biostat3)}
\NormalTok{k\_all\_levels }\OtherTok{\textless{}{-}} \FunctionTok{c}\NormalTok{(}
  \StringTok{"Effect for Level 1"} \OtherTok{=} \StringTok{"power1"}\NormalTok{,}
  \StringTok{"Effect for Level 2"} \OtherTok{=} \StringTok{"power2"}\NormalTok{,}
  \StringTok{"Effect for Level 3"} \OtherTok{=} \StringTok{"power3"}\NormalTok{,}
  \StringTok{"Effect for Level 4"} \OtherTok{=} \StringTok{"{-}power1 {-} power2 {-} power3"}
\NormalTok{)}
\CommentTok{\# Run lincom on the whole vector}
\FunctionTok{lincom}\NormalTok{(fit, k\_all\_levels)}
\end{Highlighting}
\end{Shaded}

\begin{verbatim}
                   Estimate 2.5 %     97.5 %    Df Sum of Sq
Effect for Level 1 -66.55   -80.41666 -52.68334 1  29526.02 
Effect for Level 2 -30.35   -44.21666 -16.48334 1  6140.817 
Effect for Level 3 7.65     -6.216659 21.51666  1  390.15   
Effect for Level 4 89.25    75.38334  103.1167  1  53103.75 
\end{verbatim}

\begin{Shaded}
\begin{Highlighting}[]
\DocumentationTok{\#\#\#\#\# Estimate of Treatment Means}
\end{Highlighting}
\end{Shaded}

We can verify the above results by looking at the point estimate of
centralized effect by direct calculation. The construction of CI is more
complicated, hence, omitted.

\begin{Shaded}
\begin{Highlighting}[]
\DocumentationTok{\#\# Extract coefficients}
\NormalTok{est }\OtherTok{\textless{}{-}} \FunctionTok{coef}\NormalTok{(fit)}
\NormalTok{tau4.hat }\OtherTok{\textless{}{-}} \SpecialCharTok{{-}}\FunctionTok{sum}\NormalTok{(est[}\SpecialCharTok{{-}}\DecValTok{1}\NormalTok{])}
\NormalTok{taui.hat }\OtherTok{\textless{}{-}} \FunctionTok{c}\NormalTok{(est[}\SpecialCharTok{{-}}\DecValTok{1}\NormalTok{], tau4.hat)}
\FunctionTok{print}\NormalTok{(taui.hat)}
\end{Highlighting}
\end{Shaded}

\begin{verbatim}
power1 power2 power3        
-66.55 -30.35   7.65  89.25 
\end{verbatim}

\paragraph{ANOVA}\label{anova}

The ANOVA table partitions the total variation into variation
\textbf{between} treatment groups (\texttt{power}) and variation
\textbf{within} treatment groups (random error). The \textbf{p-value}
(Pr(\textgreater F)) indicates if the treatment has a significant
effect.

\begin{Shaded}
\begin{Highlighting}[]
\FunctionTok{anova}\NormalTok{(fit)}
\end{Highlighting}
\end{Shaded}

\begin{verbatim}
Analysis of Variance Table

Response: rate
          Df Sum Sq Mean Sq F value    Pr(>F)    
power      3  66871 22290.2  66.797 2.883e-09 ***
Residuals 16   5339   333.7                      
---
Signif. codes:  0 '***' 0.001 '**' 0.01 '*' 0.05 '.' 0.1 ' ' 1
\end{verbatim}

\subsubsection{Baseline Model (Default for
R)}\label{baseline-model-default-for-r}

Fitting the model without specifying contrasts uses R's default
(``treatment'' contrast), which sets \(\tau_1 = 0\). The fundamental
results (ANOVA, treatment means) remain unchanged.

\begin{Shaded}
\begin{Highlighting}[]
\NormalTok{fit1 }\OtherTok{\textless{}{-}} \FunctionTok{lm}\NormalTok{(rate }\SpecialCharTok{\textasciitilde{}}\NormalTok{ power, }\AttributeTok{data =}\NormalTok{ etching\_df)}
\FunctionTok{model.matrix}\NormalTok{(fit1)}
\end{Highlighting}
\end{Shaded}

\begin{verbatim}
   (Intercept) power180 power200 power220
1            1        0        0        0
2            1        0        0        0
3            1        0        0        0
4            1        0        0        0
5            1        0        0        0
6            1        1        0        0
7            1        1        0        0
8            1        1        0        0
9            1        1        0        0
10           1        1        0        0
11           1        0        1        0
12           1        0        1        0
13           1        0        1        0
14           1        0        1        0
15           1        0        1        0
16           1        0        0        1
17           1        0        0        1
18           1        0        0        1
19           1        0        0        1
20           1        0        0        1
attr(,"assign")
[1] 0 1 1 1
attr(,"contrasts")
attr(,"contrasts")$power
[1] "contr.treatment"
\end{verbatim}

\begin{Shaded}
\begin{Highlighting}[]
\FunctionTok{summary}\NormalTok{(fit1)}
\end{Highlighting}
\end{Shaded}

\begin{verbatim}

Call:
lm(formula = rate ~ power, data = etching_df)

Residuals:
   Min     1Q Median     3Q    Max 
 -25.4  -13.0    2.8   13.2   25.6 

Coefficients:
            Estimate Std. Error t value Pr(>|t|)    
(Intercept)  551.200      8.169  67.471  < 2e-16 ***
power180      36.200     11.553   3.133  0.00642 ** 
power200      74.200     11.553   6.422 8.44e-06 ***
power220     155.800     11.553  13.485 3.73e-10 ***
---
Signif. codes:  0 '***' 0.001 '**' 0.01 '*' 0.05 '.' 0.1 ' ' 1

Residual standard error: 18.27 on 16 degrees of freedom
Multiple R-squared:  0.9261,    Adjusted R-squared:  0.9122 
F-statistic:  66.8 on 3 and 16 DF,  p-value: 2.883e-09
\end{verbatim}

\textbf{Using 200 as the baseline}

If we want to treat power200 as the baseline, we do this:

\begin{Shaded}
\begin{Highlighting}[]
\CommentTok{\# modifying the levels of power}
\NormalTok{etching\_df}\SpecialCharTok{$}\NormalTok{power }\OtherTok{\textless{}{-}} \FunctionTok{factor}\NormalTok{(etching\_df}\SpecialCharTok{$}\NormalTok{power, }\AttributeTok{levels=}\FunctionTok{c}\NormalTok{(}\StringTok{"200"}\NormalTok{,}\StringTok{"160"}\NormalTok{, }\StringTok{"180"}\NormalTok{, }\StringTok{"220"}\NormalTok{))}
\end{Highlighting}
\end{Shaded}

\begin{Shaded}
\begin{Highlighting}[]
\NormalTok{fit1\_200 }\OtherTok{\textless{}{-}} \FunctionTok{lm}\NormalTok{(rate }\SpecialCharTok{\textasciitilde{}}\NormalTok{ power, }\AttributeTok{data =}\NormalTok{ etching\_df)}
\FunctionTok{model.matrix}\NormalTok{(fit1\_200)}
\end{Highlighting}
\end{Shaded}

\begin{verbatim}
   (Intercept) power160 power180 power220
1            1        1        0        0
2            1        1        0        0
3            1        1        0        0
4            1        1        0        0
5            1        1        0        0
6            1        0        1        0
7            1        0        1        0
8            1        0        1        0
9            1        0        1        0
10           1        0        1        0
11           1        0        0        0
12           1        0        0        0
13           1        0        0        0
14           1        0        0        0
15           1        0        0        0
16           1        0        0        1
17           1        0        0        1
18           1        0        0        1
19           1        0        0        1
20           1        0        0        1
attr(,"assign")
[1] 0 1 1 1
attr(,"contrasts")
attr(,"contrasts")$power
[1] "contr.treatment"
\end{verbatim}

\begin{Shaded}
\begin{Highlighting}[]
\FunctionTok{summary}\NormalTok{(fit1\_200)}
\end{Highlighting}
\end{Shaded}

\begin{verbatim}

Call:
lm(formula = rate ~ power, data = etching_df)

Residuals:
   Min     1Q Median     3Q    Max 
 -25.4  -13.0    2.8   13.2   25.6 

Coefficients:
            Estimate Std. Error t value Pr(>|t|)    
(Intercept)  625.400      8.169  76.553  < 2e-16 ***
power160     -74.200     11.553  -6.422 8.44e-06 ***
power180     -38.000     11.553  -3.289  0.00462 ** 
power220      81.600     11.553   7.063 2.68e-06 ***
---
Signif. codes:  0 '***' 0.001 '**' 0.01 '*' 0.05 '.' 0.1 ' ' 1

Residual standard error: 18.27 on 16 degrees of freedom
Multiple R-squared:  0.9261,    Adjusted R-squared:  0.9122 
F-statistic:  66.8 on 3 and 16 DF,  p-value: 2.883e-09
\end{verbatim}

\textbf{ANOVA}

\begin{Shaded}
\begin{Highlighting}[]
\FunctionTok{anova}\NormalTok{(fit1\_200)}
\end{Highlighting}
\end{Shaded}

\begin{verbatim}
Analysis of Variance Table

Response: rate
          Df Sum Sq Mean Sq F value    Pr(>F)    
power      3  66871 22290.2  66.797 2.883e-09 ***
Residuals 16   5339   333.7                      
---
Signif. codes:  0 '***' 0.001 '**' 0.01 '*' 0.05 '.' 0.1 ' ' 1
\end{verbatim}

\textbf{Using 220 as the baseline}

If we want to treat power220 as the baseline, we do this:

\begin{Shaded}
\begin{Highlighting}[]
\NormalTok{etching\_df}\SpecialCharTok{$}\NormalTok{power }\OtherTok{\textless{}{-}} \FunctionTok{factor}\NormalTok{(etching\_df}\SpecialCharTok{$}\NormalTok{power, }\AttributeTok{levels=}\FunctionTok{c}\NormalTok{(}\StringTok{"220"}\NormalTok{,}\StringTok{"160"}\NormalTok{, }\StringTok{"180"}\NormalTok{,}\StringTok{"200"}\NormalTok{ ))}
\end{Highlighting}
\end{Shaded}

\begin{Shaded}
\begin{Highlighting}[]
\NormalTok{fit1\_220 }\OtherTok{\textless{}{-}} \FunctionTok{lm}\NormalTok{(rate }\SpecialCharTok{\textasciitilde{}}\NormalTok{ power, }\AttributeTok{data =}\NormalTok{ etching\_df)}
\FunctionTok{model.matrix}\NormalTok{(fit1\_220)}
\end{Highlighting}
\end{Shaded}

\begin{verbatim}
   (Intercept) power160 power180 power200
1            1        1        0        0
2            1        1        0        0
3            1        1        0        0
4            1        1        0        0
5            1        1        0        0
6            1        0        1        0
7            1        0        1        0
8            1        0        1        0
9            1        0        1        0
10           1        0        1        0
11           1        0        0        1
12           1        0        0        1
13           1        0        0        1
14           1        0        0        1
15           1        0        0        1
16           1        0        0        0
17           1        0        0        0
18           1        0        0        0
19           1        0        0        0
20           1        0        0        0
attr(,"assign")
[1] 0 1 1 1
attr(,"contrasts")
attr(,"contrasts")$power
[1] "contr.treatment"
\end{verbatim}

\begin{Shaded}
\begin{Highlighting}[]
\FunctionTok{summary}\NormalTok{(fit1\_220)}
\end{Highlighting}
\end{Shaded}

\begin{verbatim}

Call:
lm(formula = rate ~ power, data = etching_df)

Residuals:
   Min     1Q Median     3Q    Max 
 -25.4  -13.0    2.8   13.2   25.6 

Coefficients:
            Estimate Std. Error t value Pr(>|t|)    
(Intercept)  707.000      8.169  86.542  < 2e-16 ***
power160    -155.800     11.553 -13.485 3.73e-10 ***
power180    -119.600     11.553 -10.352 1.69e-08 ***
power200     -81.600     11.553  -7.063 2.68e-06 ***
---
Signif. codes:  0 '***' 0.001 '**' 0.01 '*' 0.05 '.' 0.1 ' ' 1

Residual standard error: 18.27 on 16 degrees of freedom
Multiple R-squared:  0.9261,    Adjusted R-squared:  0.9122 
F-statistic:  66.8 on 3 and 16 DF,  p-value: 2.883e-09
\end{verbatim}

\textbf{ANOVA}

\begin{Shaded}
\begin{Highlighting}[]
\FunctionTok{anova}\NormalTok{(fit1\_220)}
\end{Highlighting}
\end{Shaded}

\begin{verbatim}
Analysis of Variance Table

Response: rate
          Df Sum Sq Mean Sq F value    Pr(>F)    
power      3  66871 22290.2  66.797 2.883e-09 ***
Residuals 16   5339   333.7                      
---
Signif. codes:  0 '***' 0.001 '**' 0.01 '*' 0.05 '.' 0.1 ' ' 1
\end{verbatim}

\subsubsection{Pairwise Comparisons}\label{pairwise-comparisons}

Since our ANOVA result was significant, we perform \textbf{post-hoc
tests} to determine exactly which pairs of power levels have different
means.

\paragraph{Fisher's LSD Test}\label{fishers-lsd-test}

The \textbf{Fisher's Least Significant Difference (LSD)} test does not
control the family-wise error rate but is more powerful.

\begin{Shaded}
\begin{Highlighting}[]
\FunctionTok{with}\NormalTok{(etching\_df, }\FunctionTok{pairwise.t.test}\NormalTok{(rate, power, }\AttributeTok{p.adj =} \StringTok{"none"}\NormalTok{))}
\end{Highlighting}
\end{Shaded}

\begin{verbatim}

    Pairwise comparisons using t tests with pooled SD 

data:  rate and power 

    220     160     180   
160 3.7e-10 -       -     
180 1.7e-08 0.0064  -     
200 2.7e-06 8.4e-06 0.0046

P value adjustment method: none 
\end{verbatim}

\paragraph{Tukey's HSD Test}\label{tukeys-hsd-test}

\textbf{Tukey's Honest Significant Difference (HSD)} controls the
\textbf{family-wise error rate}, adjusting p-values to account for
multiple comparisons.

\begin{Shaded}
\begin{Highlighting}[]
\NormalTok{fit.aov }\OtherTok{\textless{}{-}} \FunctionTok{aov}\NormalTok{(rate }\SpecialCharTok{\textasciitilde{}}\NormalTok{ power, }\AttributeTok{data =}\NormalTok{ etching\_df)}
\FunctionTok{TukeyHSD}\NormalTok{(fit.aov)}
\end{Highlighting}
\end{Shaded}

\begin{verbatim}
  Tukey multiple comparisons of means
    95% family-wise confidence level

Fit: aov(formula = rate ~ power, data = etching_df)

$power
          diff         lwr        upr     p adj
160-220 -155.8 -188.854376 -122.74562 0.0000000
180-220 -119.6 -152.654376  -86.54562 0.0000001
200-220  -81.6 -114.654376  -48.54562 0.0000146
180-160   36.2    3.145624   69.25438 0.0294279
200-160   74.2   41.145624  107.25438 0.0000455
200-180   38.0    4.945624   71.05438 0.0215995
\end{verbatim}

\subsubsection{Checking Model
Assumptions}\label{checking-model-assumptions}

The validity of our ANOVA results depends on three key assumptions about
the model's residuals. We use diagnostic plots to check them.

\begin{Shaded}
\begin{Highlighting}[]
\NormalTok{r }\OtherTok{\textless{}{-}} \FunctionTok{rstudent}\NormalTok{(fit)}
\NormalTok{fitted }\OtherTok{\textless{}{-}} \FunctionTok{fitted.values}\NormalTok{(fit)}
\end{Highlighting}
\end{Shaded}

\paragraph{Normality of Residuals}\label{normality-of-residuals}

A \textbf{Normal Q-Q plot} is used to check if the residuals are
normally distributed. The points should fall closely along the straight
diagonal line.

\begin{Shaded}
\begin{Highlighting}[]
\FunctionTok{qqnorm}\NormalTok{(r)}
\FunctionTok{qqline}\NormalTok{(r)}
\end{Highlighting}
\end{Shaded}

\begin{figure}[H]

{\centering \includegraphics{unit5-factor/crbd_files/figure-pdf/qq-plot-1.pdf}

}

\caption{Normal Q-Q plot of standardized residuals.}

\end{figure}%

\paragraph{Independence of Residuals}\label{independence-of-residuals}

A plot of \textbf{residuals versus run order} helps check for
independence. We look for random scatter around the zero line.

\begin{Shaded}
\begin{Highlighting}[]
\FunctionTok{plot}\NormalTok{(r, }\AttributeTok{ylab =} \StringTok{"Standardized residuals"}\NormalTok{, }\AttributeTok{xlab =} \StringTok{"Run order"}\NormalTok{,}
     \AttributeTok{main =} \StringTok{"Plot of residuals vs. run order"}\NormalTok{)}
\FunctionTok{abline}\NormalTok{(}\AttributeTok{h =} \DecValTok{0}\NormalTok{)}
\end{Highlighting}
\end{Shaded}

\begin{figure}[H]

{\centering \includegraphics{unit5-factor/crbd_files/figure-pdf/residuals-vs-order-1.pdf}

}

\caption{Standardized residuals vs.~run order.}

\end{figure}%

\paragraph{Constant Variance
(Homoscedasticity)}\label{constant-variance-homoscedasticity}

A plot of \textbf{residuals versus fitted values} helps check for
constant variance. The spread of residuals should be roughly constant
across all fitted values.

\begin{Shaded}
\begin{Highlighting}[]
\FunctionTok{plot}\NormalTok{(fitted, r, }\AttributeTok{ylab =} \StringTok{"Standardized residuals"}\NormalTok{, }
     \AttributeTok{xlab =} \StringTok{"Fitted values"}\NormalTok{, }\AttributeTok{main =} \StringTok{"Plot of residuals vs. fitted values"}\NormalTok{)}
\FunctionTok{abline}\NormalTok{(}\AttributeTok{h =} \DecValTok{0}\NormalTok{)}
\end{Highlighting}
\end{Shaded}

\begin{figure}[H]

{\centering \includegraphics{unit5-factor/crbd_files/figure-pdf/residuals-vs-fitted-1.pdf}

}

\caption{Standardized residuals vs.~fitted values.}

\end{figure}%

\section{Unbalanced Designs with Unequal Sample
Sizes}\label{unbalanced-designs-with-unequal-sample-sizes}

The ANOVA framework also handles \textbf{unbalanced designs}. We again
start by creating a data frame.

\begin{Shaded}
\begin{Highlighting}[]
\DocumentationTok{\#\# Create the data frame}
\NormalTok{bricks\_df }\OtherTok{\textless{}{-}} \FunctionTok{data.frame}\NormalTok{(}
  \AttributeTok{density =} \FunctionTok{c}\NormalTok{(}\FloatTok{21.8}\NormalTok{, }\FloatTok{21.9}\NormalTok{, }\FloatTok{21.7}\NormalTok{, }\FloatTok{21.6}\NormalTok{, }\FloatTok{21.7}\NormalTok{,}
              \FloatTok{21.7}\NormalTok{, }\FloatTok{21.4}\NormalTok{, }\FloatTok{21.5}\NormalTok{, }\FloatTok{21.4}\NormalTok{,}
              \FloatTok{21.9}\NormalTok{, }\FloatTok{21.8}\NormalTok{, }\FloatTok{21.8}\NormalTok{, }\FloatTok{21.6}\NormalTok{, }\FloatTok{21.5}\NormalTok{,}
              \FloatTok{21.9}\NormalTok{, }\FloatTok{21.7}\NormalTok{, }\FloatTok{21.8}\NormalTok{, }\FloatTok{21.4}\NormalTok{),}
  \AttributeTok{temperature =} \FunctionTok{factor}\NormalTok{(}\FunctionTok{c}\NormalTok{(}\FunctionTok{rep}\NormalTok{(}\DecValTok{100}\NormalTok{, }\DecValTok{5}\NormalTok{), }\FunctionTok{rep}\NormalTok{(}\DecValTok{125}\NormalTok{, }\DecValTok{4}\NormalTok{), }\FunctionTok{rep}\NormalTok{(}\DecValTok{150}\NormalTok{, }\DecValTok{5}\NormalTok{), }\FunctionTok{rep}\NormalTok{(}\DecValTok{175}\NormalTok{, }\DecValTok{4}\NormalTok{)))}
\NormalTok{)}

\NormalTok{bricks\_df}
\end{Highlighting}
\end{Shaded}

\begin{verbatim}
   density temperature
1     21.8         100
2     21.9         100
3     21.7         100
4     21.6         100
5     21.7         100
6     21.7         125
7     21.4         125
8     21.5         125
9     21.4         125
10    21.9         150
11    21.8         150
12    21.8         150
13    21.6         150
14    21.5         150
15    21.9         175
16    21.7         175
17    21.8         175
18    21.4         175
\end{verbatim}

\begin{Shaded}
\begin{Highlighting}[]
\DocumentationTok{\#\# Fit the model and get the ANOVA table}
\NormalTok{fit2 }\OtherTok{\textless{}{-}} \FunctionTok{lm}\NormalTok{(density }\SpecialCharTok{\textasciitilde{}}\NormalTok{ temperature, }\AttributeTok{data =}\NormalTok{ bricks\_df)}
\FunctionTok{summary}\NormalTok{(fit2)}
\end{Highlighting}
\end{Shaded}

\begin{verbatim}

Call:
lm(formula = density ~ temperature, data = bricks_df)

Residuals:
   Min     1Q Median     3Q    Max 
-0.300 -0.100  0.000  0.095  0.200 

Coefficients:
               Estimate Std. Error t value Pr(>|t|)    
(Intercept)    21.74000    0.07171 303.150   <2e-16 ***
temperature125 -0.24000    0.10757  -2.231   0.0425 *  
temperature150 -0.02000    0.10142  -0.197   0.8465    
temperature175 -0.04000    0.10757  -0.372   0.7156    
---
Signif. codes:  0 '***' 0.001 '**' 0.01 '*' 0.05 '.' 0.1 ' ' 1

Residual standard error: 0.1604 on 14 degrees of freedom
Multiple R-squared:  0.3025,    Adjusted R-squared:  0.153 
F-statistic: 2.024 on 3 and 14 DF,  p-value: 0.1569
\end{verbatim}

\begin{Shaded}
\begin{Highlighting}[]
\FunctionTok{anova}\NormalTok{(fit2)}
\end{Highlighting}
\end{Shaded}

\begin{verbatim}
Analysis of Variance Table

Response: density
            Df  Sum Sq  Mean Sq F value Pr(>F)
temperature  3 0.15611 0.052037  2.0237 0.1569
Residuals   14 0.36000 0.025714               
\end{verbatim}

In this case, the large p-value (0.133) indicates that there is no
statistically significant evidence that firing temperature affects brick
density.

\section{Randomized Complete Block
Design}\label{randomized-complete-block-design-1}

\subsection{Vascular Graft Experiment}\label{vascular-graft-experiment}

This section analyzes a \textbf{Randomized Complete Block Design
(RCBD)}, used to control for a known source of variability (here,
``batches of resin,'' treated as \textbf{blocks}).

\subsubsection{Data and Visulization}\label{data-and-visulization}

We structure the data in a \texttt{data.frame} to identify the response,
treatment (\texttt{pressure}), and block (\texttt{batch}) for each
observation.

\begin{Shaded}
\begin{Highlighting}[]
\DocumentationTok{\#\# Define data vectors}
\NormalTok{strength }\OtherTok{\textless{}{-}} \FunctionTok{c}\NormalTok{(}\FloatTok{90.3}\NormalTok{, }\FloatTok{89.2}\NormalTok{, }\FloatTok{98.2}\NormalTok{, }\FloatTok{93.9}\NormalTok{, }\FloatTok{87.4}\NormalTok{, }\FloatTok{97.9}\NormalTok{,}
              \FloatTok{92.5}\NormalTok{, }\FloatTok{89.5}\NormalTok{, }\FloatTok{90.6}\NormalTok{, }\FloatTok{94.7}\NormalTok{, }\FloatTok{87.0}\NormalTok{, }\FloatTok{95.8}\NormalTok{,}
              \FloatTok{85.5}\NormalTok{, }\FloatTok{90.8}\NormalTok{, }\FloatTok{89.6}\NormalTok{, }\FloatTok{86.2}\NormalTok{, }\FloatTok{88.0}\NormalTok{, }\FloatTok{93.4}\NormalTok{,}
              \FloatTok{82.5}\NormalTok{, }\FloatTok{89.5}\NormalTok{, }\FloatTok{85.6}\NormalTok{, }\FloatTok{87.4}\NormalTok{, }\FloatTok{78.9}\NormalTok{, }\FloatTok{90.7}\NormalTok{)}
\NormalTok{pressure\_levels }\OtherTok{\textless{}{-}} \FunctionTok{rep}\NormalTok{(}\FunctionTok{c}\NormalTok{(}\DecValTok{8500}\NormalTok{, }\DecValTok{8700}\NormalTok{, }\DecValTok{8900}\NormalTok{, }\DecValTok{9100}\NormalTok{), }\AttributeTok{each =} \DecValTok{6}\NormalTok{)}
\NormalTok{batch\_levels }\OtherTok{\textless{}{-}} \FunctionTok{rep}\NormalTok{(}\DecValTok{1}\SpecialCharTok{:}\DecValTok{6}\NormalTok{, }\DecValTok{4}\NormalTok{)}

\DocumentationTok{\#\# Create the data frame}
\NormalTok{graft\_df }\OtherTok{\textless{}{-}} \FunctionTok{data.frame}\NormalTok{(}
  \AttributeTok{strength =}\NormalTok{ strength,}
  \AttributeTok{pressure =} \FunctionTok{factor}\NormalTok{(pressure\_levels),}
  \AttributeTok{batch =} \FunctionTok{factor}\NormalTok{(batch\_levels)}
\NormalTok{)}


\NormalTok{graft\_df}
\end{Highlighting}
\end{Shaded}

\begin{verbatim}
   strength pressure batch
1      90.3     8500     1
2      89.2     8500     2
3      98.2     8500     3
4      93.9     8500     4
5      87.4     8500     5
6      97.9     8500     6
7      92.5     8700     1
8      89.5     8700     2
9      90.6     8700     3
10     94.7     8700     4
11     87.0     8700     5
12     95.8     8700     6
13     85.5     8900     1
14     90.8     8900     2
15     89.6     8900     3
16     86.2     8900     4
17     88.0     8900     5
18     93.4     8900     6
19     82.5     9100     1
20     89.5     9100     2
21     85.6     9100     3
22     87.4     9100     4
23     78.9     9100     5
24     90.7     9100     6
\end{verbatim}

\textbf{Visualize the Block and Treatment Effects}

\begin{Shaded}
\begin{Highlighting}[]
\FunctionTok{par}\NormalTok{ (}\AttributeTok{mfrow =} \FunctionTok{c}\NormalTok{(}\DecValTok{1}\NormalTok{,}\DecValTok{2}\NormalTok{))}
\CommentTok{\#boxplot}
\FunctionTok{plot}\NormalTok{(strength }\SpecialCharTok{\textasciitilde{}}\NormalTok{ batch, }\AttributeTok{data=}\NormalTok{graft\_df , }\AttributeTok{main =} \StringTok{"Block"}\NormalTok{)}
\FunctionTok{plot}\NormalTok{(strength }\SpecialCharTok{\textasciitilde{}}\NormalTok{ pressure, }\AttributeTok{data=}\NormalTok{graft\_df , }\AttributeTok{main =} \StringTok{"Pressure"}\NormalTok{)}
\end{Highlighting}
\end{Shaded}

\includegraphics{unit5-factor/crbd_files/figure-pdf/unnamed-chunk-10-1.pdf}

\textbf{Interaction Plots}

\begin{Shaded}
\begin{Highlighting}[]
\FunctionTok{ggplot}\NormalTok{(graft\_df, }\FunctionTok{aes}\NormalTok{(}\AttributeTok{x =}\NormalTok{ pressure, }\AttributeTok{y =}\NormalTok{ strength, }\AttributeTok{group =}\NormalTok{ batch, }\AttributeTok{color =}\NormalTok{ batch)) }\SpecialCharTok{+}
  \FunctionTok{stat\_summary}\NormalTok{(}\AttributeTok{fun =}\NormalTok{ mean, }\AttributeTok{geom =} \StringTok{"line"}\NormalTok{, }\AttributeTok{size =} \DecValTok{1}\NormalTok{) }\SpecialCharTok{+}
  \FunctionTok{stat\_summary}\NormalTok{(}\AttributeTok{fun =}\NormalTok{ mean, }\AttributeTok{geom =} \StringTok{"point"}\NormalTok{, }\AttributeTok{size =} \DecValTok{3}\NormalTok{) }\SpecialCharTok{+}
  \FunctionTok{labs}\NormalTok{(}
    \AttributeTok{title =} \StringTok{"Interaction Plot: Batch and Pressure"}\NormalTok{,}
    \AttributeTok{x =} \StringTok{"Pressue"}\NormalTok{,}
    \AttributeTok{y =} \StringTok{"Strength"}\NormalTok{,}
    \AttributeTok{color =} \StringTok{"Batch"}
\NormalTok{  ) }\SpecialCharTok{+}
  \FunctionTok{theme\_minimal}\NormalTok{() }\SpecialCharTok{+}
  \FunctionTok{theme}\NormalTok{(}\AttributeTok{plot.title =} \FunctionTok{element\_text}\NormalTok{(}\AttributeTok{hjust =} \FloatTok{0.5}\NormalTok{))}
\end{Highlighting}
\end{Shaded}

\begin{figure}[H]

{\centering \includegraphics{unit5-factor/crbd_files/figure-pdf/plot-interaction-1.pdf}

}

\caption{Interaction between Material Type and Temperature.}

\end{figure}%

\subsubsection{Model Fitting and ANOVA}\label{model-fitting-and-anova}

The model \texttt{strength\ \textasciitilde{}\ pressure\ +\ batch}
partitions the total variance into treatment, block, and error
components. Our primary interest is in the significance of the
\texttt{pressure} factor.

\begin{Shaded}
\begin{Highlighting}[]
\NormalTok{rcbd.fit1 }\OtherTok{\textless{}{-}} \FunctionTok{aov}\NormalTok{(strength }\SpecialCharTok{\textasciitilde{}}\NormalTok{ pressure }\SpecialCharTok{+}\NormalTok{ batch, }\AttributeTok{data =}\NormalTok{ graft\_df)}
\FunctionTok{anova}\NormalTok{(rcbd.fit1)}
\end{Highlighting}
\end{Shaded}

\begin{verbatim}
Analysis of Variance Table

Response: strength
          Df Sum Sq Mean Sq F value   Pr(>F)   
pressure   3 178.17  59.390  8.1071 0.001916 **
batch      5 192.25  38.450  5.2487 0.005532 **
Residuals 15 109.89   7.326                    
---
Signif. codes:  0 '***' 0.001 '**' 0.01 '*' 0.05 '.' 0.1 ' ' 1
\end{verbatim}

The small p-value for \texttt{pressure} (0.0019) provides strong
evidence that extrusion pressure significantly affects graft strength
after accounting for batch differences.

\subsubsection{Model Adequacy Checks}\label{model-adequacy-checks}

The assumptions for an RCBD are the same as for a CRD. We perform the
same diagnostic checks.

\begin{Shaded}
\begin{Highlighting}[]
\NormalTok{rcbd.r1 }\OtherTok{\textless{}{-}} \FunctionTok{rstudent}\NormalTok{(rcbd.fit1)}
\NormalTok{rcbd.fitted1 }\OtherTok{\textless{}{-}} \FunctionTok{fitted.values}\NormalTok{(rcbd.fit1)}

\FunctionTok{qqnorm}\NormalTok{(rcbd.r1, }\AttributeTok{main =} \StringTok{"Normal Q{-}Q Plot"}\NormalTok{)}
\FunctionTok{qqline}\NormalTok{(rcbd.r1)}
\FunctionTok{plot}\NormalTok{(rcbd.fitted1, rcbd.r1, }\AttributeTok{ylab =} \StringTok{"Standardized residuals"}\NormalTok{, }
     \AttributeTok{xlab =} \StringTok{"Fitted values"}\NormalTok{, }\AttributeTok{main =} \StringTok{"Residuals vs. Fitted"}\NormalTok{)}
\FunctionTok{abline}\NormalTok{(}\AttributeTok{h =} \DecValTok{0}\NormalTok{)}
\FunctionTok{plot}\NormalTok{(graft\_df}\SpecialCharTok{$}\NormalTok{pressure, rcbd.r1, }\AttributeTok{ylab =} \StringTok{"Standardized residuals"}\NormalTok{, }
     \AttributeTok{xlab =} \StringTok{"Extrusion pressure"}\NormalTok{, }\AttributeTok{main =} \StringTok{"Residuals vs. Treatment"}\NormalTok{)}
\FunctionTok{abline}\NormalTok{(}\AttributeTok{h =} \DecValTok{0}\NormalTok{)}
\FunctionTok{plot}\NormalTok{(graft\_df}\SpecialCharTok{$}\NormalTok{batch, rcbd.r1, }\AttributeTok{ylab =} \StringTok{"Standardized residuals"}\NormalTok{, }
     \AttributeTok{xlab =} \StringTok{"Batches of raw material"}\NormalTok{, }\AttributeTok{main =} \StringTok{"Residuals vs. Block"}\NormalTok{)}
\FunctionTok{abline}\NormalTok{(}\AttributeTok{h =} \DecValTok{0}\NormalTok{)}
\end{Highlighting}
\end{Shaded}

\begin{figure}

\begin{minipage}{0.50\linewidth}

\begin{figure}[H]

{\centering \includegraphics{unit5-factor/crbd_files/figure-pdf/plot-diagnostics-rcbd-1.pdf}

}

\subcaption{Normal Q-Q Plot}

\end{figure}%

\end{minipage}%
%
\begin{minipage}{0.50\linewidth}

\begin{figure}[H]

{\centering \includegraphics{unit5-factor/crbd_files/figure-pdf/plot-diagnostics-rcbd-2.pdf}

}

\subcaption{Residuals vs.~Fitted Values}

\end{figure}%

\end{minipage}%
\newline
\begin{minipage}{0.50\linewidth}

\begin{figure}[H]

{\centering \includegraphics{unit5-factor/crbd_files/figure-pdf/plot-diagnostics-rcbd-3.pdf}

}

\subcaption{Residuals vs.~Treatment (Pressure)}

\end{figure}%

\end{minipage}%
%
\begin{minipage}{0.50\linewidth}

\begin{figure}[H]

{\centering \includegraphics{unit5-factor/crbd_files/figure-pdf/plot-diagnostics-rcbd-4.pdf}

}

\subcaption{Residuals vs.~Block (Batch)}

\end{figure}%

\end{minipage}%

\end{figure}%

\subsubsection{Pairwise Comparisons}\label{pairwise-comparisons-1}

Again, since the treatment factor (\texttt{pressure}) is significant, we
perform post-hoc tests.

\paragraph{Tukey's HSD Test}\label{tukeys-hsd-test-1}

Tukey's HSD compares all pairs of treatment levels while controlling the
family-wise error rate.

\begin{Shaded}
\begin{Highlighting}[]
\FunctionTok{TukeyHSD}\NormalTok{(rcbd.fit1, }\AttributeTok{which =} \StringTok{"pressure"}\NormalTok{)}
\end{Highlighting}
\end{Shaded}

\begin{verbatim}
  Tukey multiple comparisons of means
    95% family-wise confidence level

Fit: aov(formula = strength ~ pressure + batch, data = graft_df)

$pressure
               diff        lwr       upr     p adj
8700-8500 -1.133333  -5.637161  3.370495 0.8854831
8900-8500 -3.900000  -8.403828  0.603828 0.1013084
9100-8500 -7.050000 -11.553828 -2.546172 0.0020883
8900-8700 -2.766667  -7.270495  1.737161 0.3245644
9100-8700 -5.916667 -10.420495 -1.412839 0.0086667
9100-8900 -3.150000  -7.653828  1.353828 0.2257674
\end{verbatim}

\paragraph{Fisher's LSD Test}\label{fishers-lsd-test-1}

The \texttt{LSD.test()} function from the \texttt{agricolae} package
correctly handles the error structure of an RCBD.

\begin{Shaded}
\begin{Highlighting}[]
\DocumentationTok{\#\# install.packages("agricolae")}
\FunctionTok{library}\NormalTok{(agricolae)}

\NormalTok{out }\OtherTok{\textless{}{-}} \FunctionTok{LSD.test}\NormalTok{(rcbd.fit1, }\AttributeTok{trt =} \StringTok{"pressure"}\NormalTok{, }\AttributeTok{p.adj =} \StringTok{"none"}\NormalTok{, }\AttributeTok{group =} \ConstantTok{FALSE}\NormalTok{)}
\FunctionTok{print}\NormalTok{(out}\SpecialCharTok{$}\NormalTok{comparison)}
\end{Highlighting}
\end{Shaded}

\begin{verbatim}
            difference pvalue signif.        LCL       UCL
8500 - 8700   1.133333 0.4795         -2.1974047  4.464071
8500 - 8900   3.900000 0.0247       *  0.5692620  7.230738
8500 - 9100   7.050000 0.0004     ***  3.7192620 10.380738
8700 - 8900   2.766667 0.0970       . -0.5640714  6.097405
8700 - 9100   5.916667 0.0018      **  2.5859286  9.247405
8900 - 9100   3.150000 0.0621       . -0.1807380  6.480738
\end{verbatim}

\bookmarksetup{startatroot}

\chapter{Two-Factor Factorial Design}\label{two-factor-factorial-design}

\section{Battery Design Experiment}\label{battery-design-experiment}

This analysis explores data from a \textbf{two-factor factorial
experiment} designed to assess the lifespan of a battery. The experiment
investigates two factors: \textbf{material type} (with 3 levels) and
\textbf{operating temperature} (with 3 levels: 15°C, 70°C, and 125°C).
The primary goal is to understand not only how each factor individually
affects battery life but, more importantly, whether the effect of
temperature depends on the material type used. This combined effect is
known as an \textbf{interaction}.

\section{Data Setup and Preparation}\label{data-setup-and-preparation}

First, we organize the raw data into a structured \texttt{data.frame}.
This is a best practice in R that makes the data easier to manage and
the code more readable. We create columns for the response variable
\texttt{life} and the two factors, \texttt{material} and
\texttt{temperature}, ensuring they are treated as categorical variables
(factors) for the analysis.

\begin{Shaded}
\begin{Highlighting}[]
\DocumentationTok{\#\# Response variable: battery life}
\NormalTok{life }\OtherTok{\textless{}{-}} \FunctionTok{c}\NormalTok{(}\DecValTok{130}\NormalTok{,}\DecValTok{155}\NormalTok{,}\DecValTok{74}\NormalTok{,}\DecValTok{180}\NormalTok{,  }\DecValTok{34}\NormalTok{,}\DecValTok{40}\NormalTok{,}\DecValTok{80}\NormalTok{,}\DecValTok{75}\NormalTok{,   }\DecValTok{20}\NormalTok{,}\DecValTok{70}\NormalTok{,}\DecValTok{82}\NormalTok{,}\DecValTok{58}\NormalTok{,}
          \DecValTok{150}\NormalTok{,}\DecValTok{188}\NormalTok{,}\DecValTok{159}\NormalTok{,}\DecValTok{126}\NormalTok{, }\DecValTok{136}\NormalTok{,}\DecValTok{122}\NormalTok{,}\DecValTok{106}\NormalTok{,}\DecValTok{115}\NormalTok{, }\DecValTok{25}\NormalTok{,}\DecValTok{70}\NormalTok{,}\DecValTok{58}\NormalTok{,}\DecValTok{45}\NormalTok{,}
          \DecValTok{138}\NormalTok{,}\DecValTok{110}\NormalTok{,}\DecValTok{168}\NormalTok{,}\DecValTok{160}\NormalTok{, }\DecValTok{174}\NormalTok{,}\DecValTok{120}\NormalTok{,}\DecValTok{150}\NormalTok{,}\DecValTok{139}\NormalTok{, }\DecValTok{96}\NormalTok{,}\DecValTok{104}\NormalTok{,}\DecValTok{82}\NormalTok{,}\DecValTok{60}\NormalTok{)}

\DocumentationTok{\#\# Create the data frame}
\NormalTok{battery\_df }\OtherTok{\textless{}{-}} \FunctionTok{data.frame}\NormalTok{(}
  \AttributeTok{life =}\NormalTok{ life,}
  \AttributeTok{material =} \FunctionTok{factor}\NormalTok{(}\FunctionTok{rep}\NormalTok{(}\DecValTok{1}\SpecialCharTok{:}\DecValTok{3}\NormalTok{, }\AttributeTok{each =} \DecValTok{12}\NormalTok{)),}
  \AttributeTok{temperature =} \FunctionTok{factor}\NormalTok{(}\FunctionTok{rep}\NormalTok{(}\FunctionTok{rep}\NormalTok{(}\FunctionTok{c}\NormalTok{(}\DecValTok{15}\NormalTok{, }\DecValTok{70}\NormalTok{, }\DecValTok{125}\NormalTok{), }\AttributeTok{each =} \DecValTok{4}\NormalTok{), }\DecValTok{3}\NormalTok{))}
\NormalTok{)}

\DocumentationTok{\#\# Preview the data}
\NormalTok{battery\_df}
\end{Highlighting}
\end{Shaded}

\begin{verbatim}
   life material temperature
1   130        1          15
2   155        1          15
3    74        1          15
4   180        1          15
5    34        1          70
6    40        1          70
7    80        1          70
8    75        1          70
9    20        1         125
10   70        1         125
11   82        1         125
12   58        1         125
13  150        2          15
14  188        2          15
15  159        2          15
16  126        2          15
17  136        2          70
18  122        2          70
19  106        2          70
20  115        2          70
21   25        2         125
22   70        2         125
23   58        2         125
24   45        2         125
25  138        3          15
26  110        3          15
27  168        3          15
28  160        3          15
29  174        3          70
30  120        3          70
31  150        3          70
32  139        3          70
33   96        3         125
34  104        3         125
35   82        3         125
36   60        3         125
\end{verbatim}

\section{Exploratory Data Analysis and
Visualization}\label{exploratory-data-analysis-and-visualization}

Before fitting a formal model, we visualize the data to get an intuition
for the relationships between the factors and the response.

\section{Boxplots of Main Effects}\label{boxplots-of-main-effects}

Boxplots are excellent for examining the distribution of battery life
for each level of our factors independently. This gives us a preliminary
look at the \textbf{main effects}---the individual impact of material
type and temperature.

\begin{Shaded}
\begin{Highlighting}[]
\FunctionTok{library}\NormalTok{(ggplot2)}

\DocumentationTok{\#\# Boxplot for Material Type}
\FunctionTok{ggplot}\NormalTok{(battery\_df, }\FunctionTok{aes}\NormalTok{(}\AttributeTok{x =}\NormalTok{ material, }\AttributeTok{y =}\NormalTok{ life, }\AttributeTok{fill =}\NormalTok{ material)) }\SpecialCharTok{+}
  \FunctionTok{geom\_boxplot}\NormalTok{() }\SpecialCharTok{+}
  \FunctionTok{labs}\NormalTok{(}\AttributeTok{title =} \StringTok{"Battery Life by Material Type"}\NormalTok{, }\AttributeTok{x =} \StringTok{"Material Type"}\NormalTok{, }\AttributeTok{y =} \StringTok{"Life (hours)"}\NormalTok{) }\SpecialCharTok{+}
  \FunctionTok{theme\_minimal}\NormalTok{() }\SpecialCharTok{+}
  \FunctionTok{theme}\NormalTok{(}\AttributeTok{legend.position =} \StringTok{"none"}\NormalTok{)}
\DocumentationTok{\#\# Boxplot for Temperature}
\FunctionTok{ggplot}\NormalTok{(battery\_df, }\FunctionTok{aes}\NormalTok{(}\AttributeTok{x =}\NormalTok{ temperature, }\AttributeTok{y =}\NormalTok{ life, }\AttributeTok{fill =}\NormalTok{ temperature)) }\SpecialCharTok{+}
  \FunctionTok{geom\_boxplot}\NormalTok{() }\SpecialCharTok{+}
  \FunctionTok{labs}\NormalTok{(}\AttributeTok{title =} \StringTok{"Battery Life by Temperature"}\NormalTok{, }\AttributeTok{x =} \StringTok{"Temperature (°C)"}\NormalTok{, }\AttributeTok{y =} \StringTok{"Life (hours)"}\NormalTok{) }\SpecialCharTok{+}
  \FunctionTok{theme\_minimal}\NormalTok{() }\SpecialCharTok{+}
  \FunctionTok{theme}\NormalTok{(}\AttributeTok{legend.position =} \StringTok{"none"}\NormalTok{)}
\end{Highlighting}
\end{Shaded}

\begin{figure}

\begin{minipage}{0.50\linewidth}

\begin{figure}[H]

{\centering \includegraphics{unit6-factorial/factorial_files/figure-pdf/plot-boxplots-1.pdf}

}

\subcaption{Distribution of Battery Life by Material and Temperature.}

\end{figure}%

\end{minipage}%
%
\begin{minipage}{0.50\linewidth}

\begin{figure}[H]

{\centering \includegraphics{unit6-factorial/factorial_files/figure-pdf/plot-boxplots-2.pdf}

}

\subcaption{Distribution of Battery Life by Material and Temperature.}

\end{figure}%

\end{minipage}%

\end{figure}%

\section{Interaction Plot}\label{interaction-plot}

The most crucial plot for a factorial experiment is the
\textbf{interaction plot}. It displays the mean battery life for each
combination of material and temperature. If the lines are parallel, it
suggests there is no interaction. If the lines are not parallel (i.e.,
they cross or diverge), it indicates that the effect of temperature on
battery life is different for each material type, signaling a likely
interaction.

\begin{Shaded}
\begin{Highlighting}[]
\FunctionTok{ggplot}\NormalTok{(battery\_df, }\FunctionTok{aes}\NormalTok{(}\AttributeTok{x =}\NormalTok{ temperature, }\AttributeTok{y =}\NormalTok{ life, }\AttributeTok{group =}\NormalTok{ material, }\AttributeTok{color =}\NormalTok{ material)) }\SpecialCharTok{+}
  \FunctionTok{stat\_summary}\NormalTok{(}\AttributeTok{fun =}\NormalTok{ mean, }\AttributeTok{geom =} \StringTok{"line"}\NormalTok{, }\AttributeTok{size =} \DecValTok{1}\NormalTok{) }\SpecialCharTok{+}
  \FunctionTok{stat\_summary}\NormalTok{(}\AttributeTok{fun =}\NormalTok{ mean, }\AttributeTok{geom =} \StringTok{"point"}\NormalTok{, }\AttributeTok{size =} \DecValTok{3}\NormalTok{) }\SpecialCharTok{+}
  \FunctionTok{labs}\NormalTok{(}
    \AttributeTok{title =} \StringTok{"Interaction Plot: Material Type and Temperature"}\NormalTok{,}
    \AttributeTok{x =} \StringTok{"Temperature (°C)"}\NormalTok{,}
    \AttributeTok{y =} \StringTok{"Average Battery Life (hours)"}\NormalTok{,}
    \AttributeTok{color =} \StringTok{"Material Type"}
\NormalTok{  ) }\SpecialCharTok{+}
  \FunctionTok{theme\_minimal}\NormalTok{() }\SpecialCharTok{+}
  \FunctionTok{theme}\NormalTok{(}\AttributeTok{plot.title =} \FunctionTok{element\_text}\NormalTok{(}\AttributeTok{hjust =} \FloatTok{0.5}\NormalTok{))}
\end{Highlighting}
\end{Shaded}

\begin{figure}[H]

{\centering \includegraphics{unit6-factorial/factorial_files/figure-pdf/plot-interaction-1.pdf}

}

\caption{Interaction between Material Type and Temperature.}

\end{figure}%

The non-parallel lines in the plot strongly suggest that a significant
interaction effect is present. Specifically, the performance of Material
3 drops less dramatically with increasing temperature compared to
Materials 1 and 2.

\section{Model Fitting and Analysis of Variance
(ANOVA)}\label{model-fitting-and-analysis-of-variance-anova}

We now fit a linear model to formally test the significance of the main
effects and the interaction term. The model
\texttt{life\ \textasciitilde{}\ material\ *\ temperature} is shorthand
for
\texttt{life\ \textasciitilde{}\ material\ +\ temperature\ +\ material:temperature}.
We use a sum-to-zero contrast (\texttt{contr.sum}) for balanced
interpretation of the effects. The \textbf{ANOVA table} will tell us if
the variation caused by our factors is statistically significant
compared to the random variation in the data.

\begin{Shaded}
\begin{Highlighting}[]
\DocumentationTok{\#\# Fit the full factorial model}
\NormalTok{battery\_fit }\OtherTok{\textless{}{-}} \FunctionTok{lm}\NormalTok{(life }\SpecialCharTok{\textasciitilde{}}\NormalTok{ material }\SpecialCharTok{*}\NormalTok{ temperature, }
                  \AttributeTok{data =}\NormalTok{ battery\_df,}
                  \AttributeTok{contrasts =} \FunctionTok{list}\NormalTok{(}\AttributeTok{material =}\NormalTok{ contr.sum, }\AttributeTok{temperature =}\NormalTok{ contr.sum))}

\FunctionTok{summary}\NormalTok{(battery\_fit)}
\end{Highlighting}
\end{Shaded}

\begin{verbatim}

Call:
lm(formula = life ~ material * temperature, data = battery_df, 
    contrasts = list(material = contr.sum, temperature = contr.sum))

Residuals:
    Min      1Q  Median      3Q     Max 
-60.750 -14.625   1.375  17.938  45.250 

Coefficients:
                       Estimate Std. Error t value Pr(>|t|)    
(Intercept)             105.528      4.331  24.367  < 2e-16 ***
material1               -22.361      6.125  -3.651  0.00111 ** 
material2                 2.806      6.125   0.458  0.65057    
temperature1             39.306      6.125   6.418  7.1e-07 ***
temperature2              2.056      6.125   0.336  0.73975    
material1:temperature1   12.278      8.662   1.417  0.16778    
material2:temperature1    8.111      8.662   0.936  0.35735    
material1:temperature2  -27.972      8.662  -3.229  0.00325 ** 
material2:temperature2    9.361      8.662   1.081  0.28936    
---
Signif. codes:  0 '***' 0.001 '**' 0.01 '*' 0.05 '.' 0.1 ' ' 1

Residual standard error: 25.98 on 27 degrees of freedom
Multiple R-squared:  0.7652,    Adjusted R-squared:  0.6956 
F-statistic:    11 on 8 and 27 DF,  p-value: 9.426e-07
\end{verbatim}

\begin{Shaded}
\begin{Highlighting}[]
\DocumentationTok{\#\# Generate the ANOVA table}
\FunctionTok{anova}\NormalTok{(battery\_fit)}
\end{Highlighting}
\end{Shaded}

\begin{verbatim}
Analysis of Variance Table

Response: life
                     Df Sum Sq Mean Sq F value    Pr(>F)    
material              2  10684  5341.9  7.9114  0.001976 ** 
temperature           2  39119 19559.4 28.9677 1.909e-07 ***
material:temperature  4   9614  2403.4  3.5595  0.018611 *  
Residuals            27  18231   675.2                      
---
Signif. codes:  0 '***' 0.001 '**' 0.01 '*' 0.05 '.' 0.1 ' ' 1
\end{verbatim}

The ANOVA table shows very small p-values (\texttt{Pr(\textgreater{}F)})
for \texttt{material}, \texttt{temperature}, and, most importantly, the
\texttt{material:temperature} interaction. This confirms our visual
inspection: all effects are statistically significant. \textbf{Because
the interaction is significant, our interpretation should focus on the
interaction itself rather than the main effects in isolation.}

\section{Model Adequacy Checks}\label{model-adequacy-checks-1}

The validity of our ANOVA results depends on the model's residuals
meeting certain assumptions (normality, constant variance,
independence). We check these with diagnostic plots.

\begin{Shaded}
\begin{Highlighting}[]
\DocumentationTok{\#\# Extract standardized residuals and fitted values}
\NormalTok{battery\_fit\_diag }\OtherTok{\textless{}{-}} \FunctionTok{data.frame}\NormalTok{(}
  \AttributeTok{residuals =} \FunctionTok{rstandard}\NormalTok{(battery\_fit),}
  \AttributeTok{fitted =} \FunctionTok{fitted.values}\NormalTok{(battery\_fit)}
\NormalTok{)}

\DocumentationTok{\#\# Normal Q{-}Q Plot}
\NormalTok{p1 }\OtherTok{\textless{}{-}} \FunctionTok{ggplot}\NormalTok{(battery\_fit\_diag, }\FunctionTok{aes}\NormalTok{(}\AttributeTok{sample =}\NormalTok{ residuals)) }\SpecialCharTok{+}
  \FunctionTok{stat\_qq}\NormalTok{() }\SpecialCharTok{+}
  \FunctionTok{stat\_qq\_line}\NormalTok{() }\SpecialCharTok{+}
  \FunctionTok{labs}\NormalTok{(}\AttributeTok{title =} \StringTok{"Normal Q{-}Q Plot"}\NormalTok{, }\AttributeTok{x =} \StringTok{"Theoretical Quantiles"}\NormalTok{, }\AttributeTok{y =} \StringTok{"Standardized Residuals"}\NormalTok{) }\SpecialCharTok{+}
  \FunctionTok{theme\_minimal}\NormalTok{()}

\DocumentationTok{\#\# Residuals vs. Fitted Plot}
\NormalTok{p2 }\OtherTok{\textless{}{-}} \FunctionTok{ggplot}\NormalTok{(battery\_fit\_diag, }\FunctionTok{aes}\NormalTok{(}\AttributeTok{x =}\NormalTok{ fitted, }\AttributeTok{y =}\NormalTok{ residuals)) }\SpecialCharTok{+}
  \FunctionTok{geom\_point}\NormalTok{() }\SpecialCharTok{+}
  \FunctionTok{geom\_hline}\NormalTok{(}\AttributeTok{yintercept =} \DecValTok{0}\NormalTok{, }\AttributeTok{linetype =} \StringTok{"dashed"}\NormalTok{, }\AttributeTok{color =} \StringTok{"red"}\NormalTok{) }\SpecialCharTok{+}
  \FunctionTok{labs}\NormalTok{(}\AttributeTok{title =} \StringTok{"Residuals vs. Fitted Values"}\NormalTok{, }\AttributeTok{x =} \StringTok{"Fitted Values"}\NormalTok{, }\AttributeTok{y =} \StringTok{"Standardized Residuals"}\NormalTok{) }\SpecialCharTok{+}
  \FunctionTok{theme\_minimal}\NormalTok{()}

\NormalTok{p1 }
\NormalTok{p2}
\end{Highlighting}
\end{Shaded}

\begin{figure}

\begin{minipage}{0.50\linewidth}

\begin{figure}[H]

{\centering \includegraphics{unit6-factorial/factorial_files/figure-pdf/plot-diagnostics-1.pdf}

}

\subcaption{Diagnostic plots for the battery life model.}

\end{figure}%

\end{minipage}%
%
\begin{minipage}{0.50\linewidth}

\begin{figure}[H]

{\centering \includegraphics{unit6-factorial/factorial_files/figure-pdf/plot-diagnostics-2.pdf}

}

\subcaption{Diagnostic plots for the battery life model.}

\end{figure}%

\end{minipage}%

\end{figure}%

The Normal Q-Q plot shows the points falling roughly along the line,
suggesting the normality assumption is met. The Residuals vs.~Fitted
plot shows a random scatter of points around the zero line, indicating
that the variance is reasonably constant. The model assumptions appear
to be satisfied.

\section{Post-Hoc Analysis: Pairwise
Comparisons}\label{post-hoc-analysis-pairwise-comparisons}

Since the interaction is significant, we must compare the means of the
nine specific treatment combinations (3 materials × 3 temperatures).
Simply comparing the average effect of Material 1 vs.~Material 2 would
be misleading, as that difference depends on the temperature.

\section{Tukey's HSD Test}\label{tukeys-hsd-test-2}

\textbf{Tukey's Honest Significant Difference (HSD)} test is a post-hoc
test that compares all possible pairs of means while controlling the
family-wise error rate. We apply it to an \texttt{aov} model object. The
output for the \texttt{material:temperature} interaction shows which
specific combinations are significantly different from one another.

\begin{Shaded}
\begin{Highlighting}[]
\DocumentationTok{\#\# Fit the model using aov() for Tukey\textquotesingle{}s test}
\NormalTok{battery\_aov }\OtherTok{\textless{}{-}} \FunctionTok{aov}\NormalTok{(life }\SpecialCharTok{\textasciitilde{}}\NormalTok{ material }\SpecialCharTok{*}\NormalTok{ temperature, }\AttributeTok{data =}\NormalTok{ battery\_df)}

\DocumentationTok{\#\# Perform Tukey\textquotesingle{}s HSD test}
\FunctionTok{TukeyHSD}\NormalTok{(battery\_aov)}
\end{Highlighting}
\end{Shaded}

\begin{verbatim}
  Tukey multiple comparisons of means
    95% family-wise confidence level

Fit: aov(formula = life ~ material * temperature, data = battery_df)

$material
        diff       lwr      upr     p adj
2-1 25.16667 -1.135677 51.46901 0.0627571
3-1 41.91667 15.614323 68.21901 0.0014162
3-2 16.75000 -9.552344 43.05234 0.2717815

$temperature
            diff        lwr       upr     p adj
70-15  -37.25000  -63.55234 -10.94766 0.0043788
125-15 -80.66667 -106.96901 -54.36432 0.0000001
125-70 -43.41667  -69.71901 -17.11432 0.0009787

$`material:temperature`
               diff         lwr        upr     p adj
2:15-1:15     21.00  -40.823184  82.823184 0.9616404
3:15-1:15      9.25  -52.573184  71.073184 0.9998527
1:70-1:15    -77.50 -139.323184 -15.676816 0.0065212
2:70-1:15    -15.00  -76.823184  46.823184 0.9953182
3:70-1:15     11.00  -50.823184  72.823184 0.9994703
1:125-1:15   -77.25 -139.073184 -15.426816 0.0067471
2:125-1:15   -85.25 -147.073184 -23.426816 0.0022351
3:125-1:15   -49.25 -111.073184  12.573184 0.2016535
3:15-2:15    -11.75  -73.573184  50.073184 0.9991463
1:70-2:15    -98.50 -160.323184 -36.676816 0.0003449
2:70-2:15    -36.00  -97.823184  25.823184 0.5819453
3:70-2:15    -10.00  -71.823184  51.823184 0.9997369
1:125-2:15   -98.25 -160.073184 -36.426816 0.0003574
2:125-2:15  -106.25 -168.073184 -44.426816 0.0001152
3:125-2:15   -70.25 -132.073184  -8.426816 0.0172076
1:70-3:15    -86.75 -148.573184 -24.926816 0.0018119
2:70-3:15    -24.25  -86.073184  37.573184 0.9165175
3:70-3:15      1.75  -60.073184  63.573184 1.0000000
1:125-3:15   -86.50 -148.323184 -24.676816 0.0018765
2:125-3:15   -94.50 -156.323184 -32.676816 0.0006078
3:125-3:15   -58.50 -120.323184   3.323184 0.0742711
2:70-1:70     62.50    0.676816 124.323184 0.0460388
3:70-1:70     88.50   26.676816 150.323184 0.0014173
1:125-1:70     0.25  -61.573184  62.073184 1.0000000
2:125-1:70    -7.75  -69.573184  54.073184 0.9999614
3:125-1:70    28.25  -33.573184  90.073184 0.8281938
3:70-2:70     26.00  -35.823184  87.823184 0.8822881
1:125-2:70   -62.25 -124.073184  -0.426816 0.0474675
2:125-2:70   -70.25 -132.073184  -8.426816 0.0172076
3:125-2:70   -34.25  -96.073184  27.573184 0.6420441
1:125-3:70   -88.25 -150.073184 -26.426816 0.0014679
2:125-3:70   -96.25 -158.073184 -34.426816 0.0004744
3:125-3:70   -60.25 -122.073184   1.573184 0.0604247
2:125-1:125   -8.00  -69.823184  53.823184 0.9999508
3:125-1:125   28.00  -33.823184  89.823184 0.8347331
3:125-2:125   36.00  -25.823184  97.823184 0.5819453
\end{verbatim}

\section{Fisher's LSD Method}\label{fishers-lsd-method}

The \textbf{Fisher's Least Significant Difference (LSD)} method is
another option for pairwise comparisons. To test the interaction means,
we must specify both factors in the \texttt{trt} argument.

\begin{Shaded}
\begin{Highlighting}[]
\FunctionTok{library}\NormalTok{(agricolae)}

\DocumentationTok{\#\# Perform LSD test on the interaction term}
\NormalTok{lsd\_results }\OtherTok{\textless{}{-}} \FunctionTok{LSD.test}\NormalTok{(battery\_aov, }\AttributeTok{trt =} \FunctionTok{c}\NormalTok{(}\StringTok{"material"}\NormalTok{, }\StringTok{"temperature"}\NormalTok{),}
                        \AttributeTok{p.adj =} \StringTok{"none"}\NormalTok{, }\AttributeTok{group =} \ConstantTok{FALSE}\NormalTok{)}

\DocumentationTok{\#\# Print the comparison table}
\FunctionTok{print}\NormalTok{(lsd\_results}\SpecialCharTok{$}\NormalTok{comparison)}
\end{Highlighting}
\end{Shaded}

\begin{verbatim}
              difference pvalue signif.         LCL        UCL
1:125 - 1:15      -77.25 0.0003     *** -114.950479 -39.549521
1:125 - 1:70        0.25 0.9892          -37.450479  37.950479
1:125 - 2:125       8.00 0.6667          -29.700479  45.700479
1:125 - 2:15      -98.25 0.0000     *** -135.950479 -60.549521
1:125 - 2:70      -62.25 0.0022      **  -99.950479 -24.549521
1:125 - 3:125     -28.00 0.1392          -65.700479   9.700479
1:125 - 3:15      -86.50 0.0001     *** -124.200479 -48.799521
1:125 - 3:70      -88.25 0.0001     *** -125.950479 -50.549521
1:15 - 1:70        77.50 0.0002     ***   39.799521 115.200479
1:15 - 2:125       85.25 0.0001     ***   47.549521 122.950479
1:15 - 2:15       -21.00 0.2631          -58.700479  16.700479
1:15 - 2:70        15.00 0.4214          -22.700479  52.700479
1:15 - 3:125       49.25 0.0124       *   11.549521  86.950479
1:15 - 3:15        -9.25 0.6187          -46.950479  28.450479
1:15 - 3:70       -11.00 0.5544          -48.700479  26.700479
1:70 - 2:125        7.75 0.6765          -29.950479  45.450479
1:70 - 2:15       -98.50 0.0000     *** -136.200479 -60.799521
1:70 - 2:70       -62.50 0.0021      ** -100.200479 -24.799521
1:70 - 3:125      -28.25 0.1358          -65.950479   9.450479
1:70 - 3:15       -86.75 0.0001     *** -124.450479 -49.049521
1:70 - 3:70       -88.50 0.0000     *** -126.200479 -50.799521
2:125 - 2:15     -106.25 0.0000     *** -143.950479 -68.549521
2:125 - 2:70      -70.25 0.0007     *** -107.950479 -32.549521
2:125 - 3:125     -36.00 0.0605       .  -73.700479   1.700479
2:125 - 3:15      -94.50 0.0000     *** -132.200479 -56.799521
2:125 - 3:70      -96.25 0.0000     *** -133.950479 -58.549521
2:15 - 2:70        36.00 0.0605       .   -1.700479  73.700479
2:15 - 3:125       70.25 0.0007     ***   32.549521 107.950479
2:15 - 3:15        11.75 0.5279          -25.950479  49.450479
2:15 - 3:70        10.00 0.5907          -27.700479  47.700479
2:70 - 3:125       34.25 0.0732       .   -3.450479  71.950479
2:70 - 3:15       -24.25 0.1980          -61.950479  13.450479
2:70 - 3:70       -26.00 0.1685          -63.700479  11.700479
3:125 - 3:15      -58.50 0.0036      **  -96.200479 -20.799521
3:125 - 3:70      -60.25 0.0029      **  -97.950479 -22.549521
3:15 - 3:70        -1.75 0.9248          -39.450479  35.950479
\end{verbatim}

The results from both Tukey's HSD and Fisher's LSD provide detailed
p-values for comparing pairs of treatment combinations, allowing us to
make specific conclusions, such as ``at 125°C, Material 3 has a
significantly longer life than Materials 1 and 2.''


\backmatter


\end{document}
